\chapter{Classical Canonical Ensemble}

In many physical situations, a system cannot be regarded as completely isolated, but rather as being in thermal contact with a large reservoir that can exchange energy with it. While the total energy of the combined system (the system plus its environment) is conserved, the energy of the subsystem of interest fluctuates around an equilibrium value determined by the temperature of the reservoir.

The statistical description appropriate for such conditions is the \textit{canonical ensemble}.
In this ensemble, the macroscopic state of the system is specified by fixing the temperature \(T\), the volume \(V\), and the number of particles \(N\), while the energy is allowed to vary according to a well-defined probability distribution.
The fundamental quantity that characterizes the canonical ensemble is the \textit{partition function}, which encodes the statistical weight of all accessible microstates.

From the partition function, all thermodynamic quantities—such as the internal energy, entropy, and Helmholtz free energy—can be derived. The canonical ensemble thus provides a natural bridge between microscopic mechanics and macroscopic thermodynamics, allowing us to compute equilibrium properties of systems in contact with a thermal bath.

\section{System in a Thermal Bath}

Let us consider a system \(\mathcal{S}\) in thermal contact with a large reservoir \(\mathcal{E}\), which we may refer to as the \textit{environment} or \textit{thermal bath}.
From a macroscopic point of view, the system is characterized by fixed values of temperature, volume, and number of particles \((T, V, N)\).
Microscopically, however, its state is determined by a point in its phase space,
\[
    \mathcal{M}_{\mathcal{S}} = \{ (q_i, p_i) \}, \quad i = 1, \dots, N,
\]
where each pair \((q_i, p_i)\) specifies the coordinates and momenta of the particles that constitute the system.

At thermal equilibrium, both the system and the reservoir share the same temperature,
\[
    T_1 = T_2 = T,
\]
and are free to exchange energy through microscopic interactions.
Nevertheless, the total energy of the composite system, which we may call the \textit{universe} \(\mathcal{U} = \mathcal{S} + \mathcal{E}\), remains constant:
\[
    E = E_1 + E_2 = \text{const}.
\]
Therefore, while the subsystem \(\mathcal{S}\) and its environment fluctuate in energy, the complete system \(\mathcal{U}\) as a whole is described by a \textbf{microcanonical ensemble}, since its total energy is fixed.

The goal is to determine the probability distribution describing the microscopic states of the subsystem \(\mathcal{S}\), when it is in contact with the thermal bath.
Starting from the microcanonical distribution of the total system, we write
\[
    \rho_{mc}(q_i^{(1)}, p_i^{(1)}, q_i^{(2)}, p_i^{(2)}) = \frac{1}{\omega(E)} \, \delta\!\left(\mathcal{H}(q_i^{(1)}, p_i^{(1)}, q_i^{(2)}, p_i^{(2)}) - E\right),
\]
where \(\omega(E)\) is the density of states of the entire universe at total energy \(E\).
To obtain the effective probability density for the subsystem, we integrate out the environmental degrees of freedom:
\[
    \rho_c^{(1)}(q_i^{(1)}, p_i^{(1)}) = \int_{\mathcal{M}_2} \mathrm{d}\Omega_2 \, \rho_{mc}(q_i^{(1)}, p_i^{(1)}, q_i^{(2)}, p_i^{(2)}).
\]
By recalling the expression for the microcanonical probability distribution, we get
\[
    \rho_c^{(1)} = \frac{1}{\omega(E)} \int_{\mathcal{M}_2} \mathrm{d}\Omega_2 \, \delta(E_1 + E_2 - E) = \frac{1}{\omega(E)} \, \omega_2(E - E_1),
\]
where \(\omega_2(E_2)\) denotes the density of states of the reservoir evaluated at energy \(E_2 = E - E_1\).
The probability of finding the subsystem in a state of energy \(E_1\) is therefore proportional to the number of accessible microstates of the environment compatible with energy \(E_2 = E - E_1\).

To proceed, it is convenient to express the density of states of the environment in terms of its entropy:
\[
    S_2(E_2) = k_B \ln \omega_2(E_2).
\]
Since the environment is assumed to be much larger than the subsystem, its energy variations \(\Delta E_2 = -E_1\) are extremely small compared with its total energy.
This allows us to expand \(S_2(E_2)\) in a Taylor series around the point \(E_2 = E\):
\[
    S_2(E - E_1) \approx S_2(E) - E_1 \left(\frac{\partial S_2}{\partial E_2}\right)_{E_2 = E} + \frac{E_1^2}{2} \left(\frac{\partial^2 S_2}{\partial E_2^2}\right)_{E_2 = E} + \dots
\]
For a large reservoir, the higher-order terms can be neglected, since its temperature is essentially unaffected by the small energy exchanges.
From thermodynamics we know that
\[
    \frac{\partial S}{\partial E} = \frac{1}{T},\, \text{ from }\, \mathrm{d}E = T \mathrm{d}S -p \mathrm{d}V + \mu \mathrm{d}N,
\]
so that
\[
    S_2(E - E_1) \approx S_2(E) - \frac{E_1}{T}.
\]
By exponentiating this relation, we recover the corresponding expression for the density of states:
\[
    \omega_2(E - E_1) = e^{S_2(E - E_1)/k_B} \approx e^{S_2(E)/k_B} e^{-E_1 / (k_B T)}.
\]
The first factor \(e^{S_2(E)/k_B}\) is constant because the total energy \(E\) of the universe is fixed.
Consequently, the probability of finding the subsystem in a microstate of energy \(E_1\) is proportional to
\[
    \rho_c^{(1)} \propto e^{-\beta E_1}, \qquad \beta = \frac{1}{k_B T}.
\]

This exponential dependence on the subsystem energy defines the canonical ensemble.
Physically, it reflects the fact that configurations of the subsystem with higher energy correspond to fewer available microstates for the environment, and are thus less probable.
The term with the opposite sign in the exponent, \(e^{+\beta E_1}\), would imply an increasing probability for higher energies and would make normalization impossible, so only the decaying exponential form is physically acceptable.

Hence, the canonical probability distribution for a system in thermal equilibrium with a heat bath is given by
\begin{equation}
    \rho_c(q_i, p_i) = \frac{1}{Z} e^{-\beta \mathcal{H}(q_i, p_i)},
    \label{eq:canonical_distribution}
\end{equation}
where the normalization factor \(Z\) is known as the \textbf{canonical partition function}.
It is obtained by requiring that the total probability over phase space be one:
\begin{equation}
    Z = \int_{\mathcal{M}_{\mathcal{S}}} \mathrm{d}\Omega \, e^{-\beta \mathcal{H}(q_i, p_i)}.
    \label{eq:canonical_partition_function}
\end{equation}
For a system of \(N\) particles in \(d\) dimensions this becomes\footnote{Note the measure of the phase space: we have incorporated the \textit{indistinguishability factor} \(\xi_N\): \(\xi_N = N!\) for indistinguishable particles, and \(\xi_N = 1\) for distinguishable ones. Also, the factor \(h^{dN}\) ensures the correct dimensionality and connects classical and quantum descriptions. Now we have properly defined a general measure on phase space.}
\[
    Z = \frac{1}{\xi_N} \int \prod_{i=1}^{N} \frac{\mathrm{d}^d q_i \, \mathrm{d}^d p_i}{h^d} \, e^{-\beta \mathcal{H}(q_i, p_i)} = \frac{1}{\xi_N} \int_0^{\infty} \mathrm{d}E \, e^{-\beta E} \, \omega(E).
\]
The partition function plays a central role in statistical mechanics, as it contains all the thermodynamic information about the system.
Once \(Z\) is known, quantities such as the internal energy, entropy, and free energy can be derived from it through standard thermodynamic relations.

Finally, let us define the notion of an average quantity within this ensemble.
\begin{definition}{Canonical average.}
    If \(f(q_i, p_i)\) is any observable function defined on the phase space of the system, its average over the canonical ensemble is given by
    \begin{equation}
        \langle f \rangle_c = \int_{\mathcal{M}_{\mathcal{S}}} \mathrm{d}\Omega \, \rho_c(q_i, p_i) f(q_i, p_i) = \frac{1}{Z} \int_{\mathcal{M}_{\mathcal{S}}} \mathrm{d}\Omega \, e^{-\beta \mathcal{H}(q_i, p_i)} f(q_i, p_i).
        \label{eq:canonical_average}
    \end{equation}
    This operation, often called the \textit{canonical average}, provides the bridge between the microscopic description of the system and the macroscopic quantities that can be experimentally measured.
\end{definition}
In equilibrium statistical mechanics, macroscopic observables are nothing but ensemble averages of microscopic functions evaluated with the canonical distribution.

\section{Partition Function and Thermodynamic Quantities}

Recalling the expression found for the partition function, we can now analyze its physical dependencies:
\begin{equation}
    Z = \int_{\textcolor{blue}{\mathcal{M}_{\mathcal{S}}}} \mathrm{d}\Omega \,
    e^{-\textcolor{green}{\beta} \textcolor{orange}{\mathcal{H}}}
    = Z(\textcolor{blue}{V}, \textcolor{green}{T}, \textcolor{orange}{N}),
    \label{eq:Helmholtz_partition_function}
\end{equation}
that is, the partition function depends on the thermodynamic variables characterizing the canonical ensemble: the volume \(V\), the temperature \(T\), and the number of particles \(N\).
In this sense, \(Z\) plays the role of a thermodynamic potential from which all equilibrium properties can be derived.

A particularly useful relation expresses the Helmholtz free energy in terms of the partition function, providing a bridge between thermodynamics and statistical mechanics:
\[
    F(T, V, N) = E - T S = -\frac{1}{\beta} \log Z.
\]
The logarithm is essential in this definition, since it restores the correct \textit{extensivity} of the thermodynamic potential. Indeed, while the partition function \(Z\) scales exponentially with the system size (\(Z \propto V^N\)), its logarithm grows linearly with \(V\), \(N\), and \(T\):
\[
    \log Z \propto N \log V \quad \Rightarrow \quad F \propto V,\, N,\, T.
\]
Hence, by taking the logarithm we recover a quantity that behaves properly under a rescaling of the system. Let's demonstrate the relation between \(F\) and \(Z\).

From the definition of \(F\) we can write:
\[
    Z = e^{-\beta F} = \int_{\mathcal{M}_{\mathcal{S}}} \mathrm{d}\Omega \, e^{-\beta \mathcal{H}}.
\]
This identity can be rearranged as:
\[
    \int_{\mathcal{M}_{\mathcal{S}}} \mathrm{d}\Omega \, e^{\beta (F - \mathcal{H})} = 1.
\]
Differentiating both sides with respect to \(\beta\) gives:
\[
    \frac{\partial}{\partial \beta} \int_{\mathcal{M}_{\mathcal{S}}}
    \mathrm{d}\Omega \, e^{\beta (F - \mathcal{H})} = 0,
\]
from which it follows:
\[
    \int_{\mathcal{M}_{\mathcal{S}}} \mathrm{d}\Omega \,
    e^{\beta (F - \mathcal{H})}
    \left[(F - \mathcal{H}) + \beta \frac{\partial F}{\partial \beta}\right] = 0.
\]
Since the integrand is weighted by the canonical probability density
\[
    \frac{e^{-\beta \mathcal{H}}}{Z},
\]
we can rewrite the previous expression in the form:\footnote{We have also used \(\beta \frac{\partial}{\partial \beta} = - T \frac{\partial}{\partial T}\).}
\[
    F \int_{\mathcal{M}_{\mathcal{S}}} \mathrm{d}\Omega \, \frac{e^{-\beta \mathcal{H}}}{Z}
    = \int_{\mathcal{M}_{\mathcal{S}}} \mathrm{d}\Omega \, \frac{e^{-\beta \mathcal{H}}}{Z} \, \mathcal{H}
    + T \frac{\partial F}{\partial T} \int_{\mathcal{M}_{\mathcal{S}}} \mathrm{d}\Omega \, \frac{e^{-\beta \mathcal{H}}}{Z}.
\]
Recognizing the expression for the canonical average, we obtain:
\[
    F = \langle \mathcal{H} \rangle_c + T \frac{\partial F}{\partial T}.
\]
Finally, recalling that \(\langle \mathcal{H} \rangle_c = E\) and \(S=-\frac{\partial F}{\partial T}\), we recover the thermodynamic definition of the Helmholtz free energy:
\[
    F = E - T S.
\]

\begin{remark}
    The Helmholtz free energy \(F(T,V,N)\) is therefore the natural thermodynamic potential of the canonical ensemble. Its differential form,
    \[
        \mathrm{d}F = -S\,\mathrm{d}T - p\,\mathrm{d}V + \mu\,\mathrm{d}N,
    \]
    directly provides the equilibrium relations between macroscopic observables and serves as the starting point for the derivation of all other state functions.
\end{remark}

Another useful relation connects the internal energy to the partition function:
\begin{equation}
    E = \langle \mathcal{H} \rangle_c
    = \int_{\mathcal{M}_{\mathcal{S}}} \mathrm{d}\Omega \, \frac{e^{-\beta \mathcal{H}}}{Z} \, \mathcal{H}
    = \frac{1}{Z} \left( -\frac{\partial Z}{\partial \beta} \right)
    = -\frac{\partial}{\partial \beta} \log Z.
    \label{eq:internal_energy_partition_function}
\end{equation}
This expression highlights how the mean energy of the system can be derived directly from the temperature dependence of the partition function.
Once the partition function \(Z\) is known, all macroscopic thermodynamic quantities follow systematically:
the Helmholtz free energy from \(F = -k_B T \log Z\),
the internal energy from \(E = -\partial_\beta \log Z\),
and finally the entropy from the thermodynamic identity \(S = (E - F)/T\).
In summary,
\[
    Z \;\Rightarrow\; F,\, E,\, S,
\]
showing that the knowledge of the partition function completely determines the equilibrium thermodynamics.

\paragraph{Entropy.} Let us recall the result obtained for the microcanonical ensemble, where the statistical and thermodynamic definitions of entropy coincide,
\[
    S_{mc} = -k_B \langle \log \rho_{mc} \rangle_{mc} = S_{th}.
\]
We now aim to demonstrate the corresponding relation in the canonical ensemble:
\[
    \begin{aligned}
        S_{c} & = -k_B \langle \log \rho_{c} \rangle_{c} = -k_B \int_{\mathcal{M}_{\mathcal{S}}} \mathrm{d}\Omega \, \rho_c \log \rho_c                                                    \\
              & = -k_B \int_{\mathcal{M}_{\mathcal{S}}} \mathrm{d}\Omega \, \rho_c \left(\log(e^{-\beta \mathcal{H}}) - \log(Z)\right)                                                     \\
              & = k_B \int_{\mathcal{M}_{\mathcal{S}}} \mathrm{d}\Omega \, \rho_c \left(\beta \mathcal{H}\right) + k_B \log(Z) \int_{\mathcal{M}_{\mathcal{S}}} \mathrm{d}\Omega \, \rho_c \\
              & =\frac{k_B}{k_B T} \langle \mathcal{H} \rangle_c + k_B \log(Z)                                                                                                             \\
              & = \frac{E-F}{T} = S_{th} = S_c,
    \end{aligned}
\]
practically proving \textit{Boltzmann's universal formula}. Hence, in the thermodynamic limit we have
\[
    s_{mc} = s_c = s_{th}.
\]

\subsection{Distinguishable and Indistinguishable Systems}

Consider a system composed of multiple independent and distinguishable subsystems.
If we denote by \(A\) and \(B\) two such subsystems, the total system can be expressed as:
\[
    \mathcal{S} = A \cup B,
    \quad \mathcal{H} = \mathcal{H}_A + \mathcal{H}_B,
    \quad \mathcal{M} = \mathcal{M}_A \otimes \mathcal{M}_B,
    \quad \mathrm{d}\Omega = \mathrm{d}\Omega_A \, \mathrm{d}\Omega_B.
\]
The statistical independence of the two parts implies that the total partition function factorizes as:
\[
    Z = \int_{\mathcal{M}_A \otimes \mathcal{M}_B}
    \mathrm{d}\Omega_A \, \mathrm{d}\Omega_B \,
    e^{-\beta (\mathcal{H}_A + \mathcal{H}_B)}
    = Z_A Z_B.
\]
More generally, for a system composed of \(N\) independent, distinguishable subsystems, we obtain the product rule:
\[
    Z = \prod_{i=1}^{N} Z_i.
\]
This property reflects the fact that, for independent subsystems, the total probability measure in phase space factorizes, and so does the corresponding statistical weight.

\begin{example}[Identical particles]
    Consider a system of \(N\) non-interacting, non-relativistic, identical but \textit{distinguishable} free particles in one dimension.
    The Hamiltonian is additive:
    \[
        \mathcal{H} = \sum_{i=1}^{N} \frac{p_i^2}{2m},
    \]
    and the total partition function follows immediately:
    \[
        Z^{(\mathrm{dist})} = \int e^{-\beta \mathcal{H}} \, \mathrm{d}\Omega
        = (Z_1)^N,
    \]
    where \(Z_1\) is the single-particle partition function.

    However, if the particles are \textit{indistinguishable}, the situation changes.
    Since any permutation of the \(N\) particles represents the same microscopic state, we must divide by the number of equivalent configurations, namely \(N!\).
    The correct expression for the canonical partition function is therefore:
    \[
        Z^{(\mathrm{ind})} = \frac{(Z_1)^N}{N!}.
    \]
    This correction accounts for the overcounting of identical configurations in phase space and ensures the proper extensivity of thermodynamic quantities such as entropy and free energy.
\end{example}

\section{Generalized Equipartition Theorem}

The equipartition theorem provides a fundamental link between the microscopic degrees of freedom of a system and its macroscopic thermal properties. In its generalized form, it states that each parameter \(\xi_i\), coordinate or its conjugate momentum for \(i=1,\dots,2dN\), contribute to the mean energy according to their appearance in the Hamiltonian. This result extends the classical formulation beyond the simple quadratic case, offering a deeper understanding of how thermal energy is distributed among different types of degrees of freedom.

\begin{theorem}[Generalized Equipartition Theorem]
    For a canonical ensemble described by the Hamiltonian \(\mathcal{H}(\{\xi_i\})\), the following relation holds for each generalized coordinate or momentum \(\xi_j \in [a,\,b]\):
    \begin{equation}
        k_B T = \left\langle \xi_j \frac{\partial \mathcal{H}}{\partial \xi_j} \right\rangle _c,
        \label{eq:generalized_equipartition}
    \end{equation}
    provided that the boundary term vanishes:
    \begin{equation}
        \xi_j e^{-\beta \mathcal{H}} \Big|_{\xi_j=a}^{\xi_j=b} = 0,
        \label{eq:equipartition_boundary_condition}
    \end{equation}
    where \(a\) and \(b\) are typically extended to \(\pm\infty\).
\end{theorem}


\begin{proof}
    Consider the canonical measure on phase space:\footnote{Since we consider a dimensionless measure and we include the \(\xi_N\) factor in \(\mathrm{d} \Omega\), we will consider these contributions to have been absorbed in the definition of the generalized variables \(\xi_i\) (it also helps to avoid confusion between the ind. factor \(\xi_N\) with the generalized variables \(\xi_i\) in the computation).}
    \[
        \mathrm{d}\Omega \, e^{-\beta \mathcal{H}} = \left( \prod_{i=1}^{2dN} \mathrm{d}\xi_i \right) e^{-\beta \mathcal{H}}.
    \]
    We can isolate the differential with respect to one variable \(\xi_j\) and differentiate the product \(\xi_j e^{-\beta \mathcal{H}}\) to obtain
    \[
        \mathrm{d}\Omega \, e^{-\beta \mathcal{H}}
        = \left( \prod_{i \neq j} \mathrm{d}\xi_i \right) \mathrm{d}\xi_j \, e^{-\beta \mathcal{H}}
        = \left( \prod_{i \neq j} \mathrm{d}\xi_i \right)
        \left[
            \mathrm{d}\!\left( \xi_j e^{-\beta \mathcal{H}} \right)
            - \xi_j e^{-\beta \mathcal{H}} \left( -\beta \frac{\partial \mathcal{H}}{\partial \xi_j} \right) \mathrm{d}\xi_j
            \right],
    \]
    which simplifies to
    \[
        \mathrm{d}\Omega \, e^{-\beta \mathcal{H}}
        = \left( \prod_{i \neq j} \mathrm{d}\xi_i \right) \mathrm{d}\!\left( \xi_j e^{-\beta \mathcal{H}} \right)
        + \mathrm{d}\Omega \, \beta \, \xi_j \frac{\partial \mathcal{H}}{\partial \xi_j} e^{-\beta \mathcal{H}}.
    \]
    Using the normalization condition of the canonical ensemble,
    \[
        1 = \frac{1}{Z} \int \mathrm{d}\Omega \, e^{-\beta \mathcal{H}},
    \]
    and substituting the expression above, we get:
    \[
        1 = \frac{1}{Z} \left[
            \int \left( \prod_{i \neq j} \mathrm{d}\xi_i \right)
            \int_a^b \mathrm{d}\!\left( \xi_j e^{-\beta \mathcal{H}} \right)
            + \beta \int \mathrm{d}\Omega \, \xi_j \frac{\partial \mathcal{H}}{\partial \xi_j} e^{-\beta \mathcal{H}}
            \right].
    \]
    Requiring the boundary term to vanish,
    \[
        \xi_j e^{-\beta \mathcal{H}} \Big|_{\xi_j=a}^{\xi_j=b} = 0,
    \]
    with \(a,b \to \pm\infty\), the first term disappears and we obtain:
    \[
        \frac{1}{\beta} = k_B T = \left\langle \xi_j \frac{\partial \mathcal{H}}{\partial \xi_j} \right\rangle _c.
    \]
\end{proof}

\begin{corollary}[Classical Equipartition Theorem]
    If the Hamiltonian depends quadratically on one of its canonical variables, namely
    \[
        \mathcal{H} = A\,\xi_j^2 + \tilde{\mathcal{H}}(\xi_{i \neq j}),
    \]
    with \(A\) constant and \(\xi_j \in (-\infty, \infty)\), the boundary term
    \[
        \xi_j e^{-\beta \mathcal{H}} \Big|_{-\infty}^{\infty} \sim \xi_j e^{-\beta A \xi_j^2} \Big|_{-\infty}^{\infty} = 0
    \]
    automatically vanishes,\footnote{The integrand is Gaussian and centered at \(\xi_j = 0\), so the product decays exponentially at infinity.}
    and the generalized equipartition theorem yields:
    \[
        \left\langle \xi_j \frac{\partial \mathcal{H}}{\partial \xi_j} \right\rangle _c
        = \left\langle 2A\,\xi_j^2 \right\rangle _c = k_B T.
    \]
    Hence, the mean energy associated with each quadratic degree of freedom is
    \begin{equation}
        E_{\xi_j} = \langle A\,\xi_j^2 \rangle_c = \frac{1}{2}k_B T.
        \label{eq:classical_equipartition}
    \end{equation}
    Therefore, every quadratic canonical variable, whether a coordinate or a conjugate momentum, contributes an average energy of \(\tfrac{1}{2}k_B T\) to the total internal energy of the system.
\end{corollary}

The equipartition theorem thus provides a direct and general rule for distributing the thermal energy among the accessible microscopic degrees of freedom.
Its consequences become particularly clear when applied to systems with quadratic Hamiltonians, as illustrated in the following examples.

\begin{example}[Ideal Gas]
    For a gas of \(N\) non-interacting, non-relativistic particles in \(d\) spatial dimensions,
    the Hamiltonian depends quadratically only on the \(dN\) momentum components:
    \[
        \mathcal{H} = \sum_{i=1}^{N} \frac{|p_i|^2}{2m}.
    \]
    Each quadratic term contributes an average energy of \(\tfrac{1}{2}k_B T\), hence:
    \[
        E = \frac{dN}{2}k_B T.
    \]
    This result shows that the internal energy of a classical ideal gas is entirely kinetic and
    depends only on the number of degrees of freedom, not on the nature of the particles or their interactions.
\end{example}

\begin{example}[Harmonic Oscillators]
    For a system of \(N\) independent harmonic oscillators in \(d\) dimensions,
    the Hamiltonian is quadratic in both momenta and coordinates:
    \[
        \mathcal{H} = \sum_{i=1}^{N} \left( \frac{|p_i|^2}{2m} + \frac{1}{2}m\omega^2 |q_i|^2 \right).
    \]
    Since each oscillator contributes two quadratic terms (one kinetic, one potential),
    the total internal energy becomes:
    \[
        E = dN k_B T.
    \]
    Thus, the equipartition theorem immediately provides the temperature dependence of the mean energy
    and, consequently, of the heat capacity for harmonic systems.
\end{example}