\chapter{Classical Canonical Ensamble}

\section*{Canonical System in Thermal Bath}

At equilibrium, the temperature of the system and the environment is equal:
\[
    T_1 = T_2 = T.
\]
The macroscopic state of the system is characterized by fixed values of:
\[
    T, \quad V, \quad N.
\]
The system is described by the phase space:
\[
    \mathcal{S} = \{ (q_i, p_i) \},
\]
where energy can be exchanged with the environment, but the total energy is conserved:
\[
    E = E_1 + E_2 = \text{const}.
\]
Thus, the universe (system + environment) is described by a microcanonical ensemble.

To obtain the canonical probability distribution for the system alone, we integrate out the degrees of freedom of the environment:
\[
    P_c(q_i, p_i) = \frac{1}{Z_N} e^{-\beta H(q_i, p_i)},
\]
where the partition function is defined as:
\[
    Z_N[T, V] = \int \dd \Gamma \, e^{-\beta H}.
\]
Alternatively, using the density of states:
\[
    Z_N = \int \dd E \, \omega(E) \, e^{-\beta E}.
\]

\subsection*{Remarks}

For multiple distinguishable and independent species \(A, B, \dots\), the partition function factorizes:
\[
    Z_T = \int \dd \Gamma_A \dd \Gamma_B \cdots \, e^{-\beta(H_A + H_B + \cdots)} = Z_{NA} Z_{NB} \cdots.
\]
\subsection*{Canonical Averages}

The canonical average of an observable \(f(q_i, p_i)\) is given by:
\[
    \langle f \rangle_c = \int \dd \Gamma \, P_c(q_i, p_i) \, f(q_i, p_i) = \frac{1}{Z_N} \int \dd \Gamma \, e^{-\beta H(q_i, p_i)} f(q_i, p_i).
\]

\section*{Thermodynamic Quantities}

We define the Helmholtz free energy:
\[
    F(T, V, N) = -k_B T \log Z_N \quad \Rightarrow \quad Z_N = e^{-\beta F}.
\]
The internal energy is:
\[
    E = \langle H \rangle_c = -\frac{\partial}{\partial \beta} \log Z_N.
\]
The entropy is:
\[
    S_c = -k_B \langle \log P_c \rangle_c.
\]
Using Boltzmann's universal formula and thermodynamic identities, we obtain:
\[
    S_c = \frac{E - F}{T} = -\left( \frac{\partial F}{\partial T} \right)_{V,N}.
\]
Hence, in the thermodynamic limit:
\[
    S_c = S_{mc} = S_{th}.
\]

\section*{Equipartition Theorem}

Let \(x_j \in [a, b]\) be one of the canonical coordinates or momenta. Suppose the Hamiltonian \(H\) depends quadratically on \(x_j\), and the following condition holds:

\[
    \left\langle x_j \frac{\partial H}{\partial x_j} \right\rangle = k_B T.
\]

\subsection*{Corollary (Standard Equipartition Theorem)}

If a coordinate appears quadratically in the Hamiltonian, then it contributes to the internal energy with an additive term of:
\[
    \frac{1}{2} k_B T.
\]

\subsection*{Examples}

\begin{itemize}
    \item \textbf{Free non-relativistic gas in \(d\)-dimensions:} The Hamiltonian depends quadratically only on the \(dN\) components of the momenta. Hence:
          \[
              E = \frac{dN}{2} k_B T.
          \]
    \item \textbf{Gas of harmonic oscillators in \(d\)-dimensions:} The Hamiltonian depends quadratically on both the \(dN\) components of momenta and positions. Hence:
          \[
              E = dN k_B T.
          \]
\end{itemize}
Other applications will be explored in the exercises.
\end{document}