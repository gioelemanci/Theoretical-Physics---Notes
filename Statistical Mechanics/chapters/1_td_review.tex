\chapter{Thermodynamics Review}

We begin our exploration of statistical mechanics with a review of the fundamental concepts of thermodynamics. Thermodynamics is the study of energy, heat, work, and the macroscopic properties of systems in equilibrium. It provides the foundation for understanding how microscopic interactions give rise to macroscopic phenomena.

\section*{Thermodynamic Systems and States}
A thermodynamic system is defined as a macroscopic region in which we can study the behavior of matter and energy. The boundaries of this system can be real or imaginary, and they separate the system from its surroundings. The surroundings are everything outside the system that can interact with it.

Thermodynamic systems can be classified into three main types:
\begin{itemize}
    \item \textbf{Isolated systems:} These systems do not exchange matter or energy with their surroundings. An example is a thermos bottle that keeps its contents insulated from the external environment.
    \item \textbf{Closed systems:} These systems can exchange energy (in the form of heat or work) but not matter with their surroundings. A common example is a sealed container of gas that can expand or contract but does not allow gas to enter or leave.
    \item \textbf{Open systems:} These systems can exchange both energy and matter with their surroundings. An example is a boiling pot of water, where heat is transferred to the water from the stove, and water vapor escapes into the air.
\end{itemize}

The state of a thermodynamic system is described by its \textbf{macroscopic properties}, such as temperature, pressure, volume, and internal energy. These properties can change during thermodynamic processes, which are the transformations that occur within the system, but we will analize only systems already at equilibrium.

We can divide thermodynamic variables into two categories of \textit{conjugate}\footnote{Conjugate variables are pairs of variables that are related to each other in a specific way, often through a mathematical relationship, we will soon see examples of this.} quantities:

\begin{itemize}
    \item \textbf{Intensive variables}, which do not depend on the size or extent of the system.
    \item \textbf{Extensive variables}, which depend on the size or extent of the system.
\end{itemize}

\begin{table}[H]
    \centering
    \renewcommand{\arraystretch}{1.3} % spaziatura verticale
    \begin{tabular}{llcll}
        \toprule
                     & \textbf{Extensive} &                       &              & \textbf{Intensive} \\
        \midrule
        $E$          & Energy             &                       & --           &                    \\
        $S$          & Entropy            & $\longleftrightarrow$ & $T$          & Temperature        \\
        $V$          & Volume             & $\longleftrightarrow$ & $p$          & Pressure           \\
        $N$          & Particles          & $\longleftrightarrow$ & $\mu$        & Chem. potential    \\
        $\mathbf{P}$ & Polarization       & $\longleftrightarrow$ & $\mathbf{E}$ & Electric field     \\
        $\mathbf{M}$ & Magnetization      & $\longleftrightarrow$ & $\mathbf{B}$ & Magnetic field     \\
        \bottomrule
    \end{tabular}
\end{table}

These variables are related by \textbf{state equations}, which express how the properties of a system are interdependent. For example, the ideal gas law relates pressure, volume, and temperature for an ideal gas:
\[
    pV = nRT,
\]
where \(n\) is the number of moles of gas, \(R\) is the ideal gas constant.

\section{Laws of Thermodynamics}

We now summarize the four fundamental laws of thermodynamics, which govern the behavior of thermodynamic systems.

\subsection{Zeroth Law of Thermodynamics}

If two systems \(A\) and \(B\) are described by their set of thermodynamic variables \(\mathcal{M}_A\) and \(\mathcal{M}_B\) respectively, and they are both at equilibrium, we can use only a subset of \(\mathcal{M}_A \times \mathcal{M}_B\): \(a \times b \) (\(a \in \mathcal{M}_A \text{ and } b \in \mathcal{M}_B\)), which reduces the expression of the equilibrium to a function of these variables: \(\mathcal{F} (a, b) = 0\).

Thanks to this law, we can introduce the concept of temperature as a property that determines whether two systems are in thermal equilibrium. To do so we know that:
\begin{enumerate}[label=(\roman*)]
    \item \(A \sim A\) always holds (reflexivity).
    \item If \(A \sim B\) then \(B \sim A\) (symmetry).
    \item If \(A \sim B\) and \(B \sim C\) then \(A \sim C\) (transitivity).
\end{enumerate}
Then we want \(\mathcal{F}(a,b)\) to respect these properties, so we can write:
\[
    \mathcal{F}(a,b) = f(a) - f(b) = 0 \quad \Longrightarrow \quad f(a) = f(b) = T.
\]

\subsection{First Law of Thermodynamics}
The First Law of Thermodynamics is a statement of the \textbf{conservation of energy}. It states that the change in internal energy of a system is given by the heat exchanged with the environment minus the work done by the system on its surroundings, plus the energy associated with the change in the number of particles (if applicable):
\begin{equation}
    \mathrm{d}E = \delta \mathcal{Q} - \delta \mathcal{L} + \mu \mathrm{d}N.
\end{equation}

This law is valid for any generic transformation, and it tells us how one can operate on the energy of a system. The terms \(\delta \mathcal{Q}\) and \(\delta \mathcal{L}\) are not exact differentials, meaning that they depend on the path taken during the transformation, while \(E\) and \(N\) are exact differentials, meaning that they depend only on the initial and final states of the system. In general:
\[
    \oint \delta \mathcal{Q} \neq 0, \quad \oint \delta \mathcal{L} \neq 0, \quad \oint \mathrm{d}E = 0, \quad \oint \mathrm{d}N = 0.
\]

\subsection{Second Law of Thermodynamics}
The Second Law of Thermodynamics states that natural processes tend to move towards a state of maximum \textbf{entropy}. This law introduces the concept of irreversibility in thermodynamic processes and implies that heat cannot spontaneously flow from a colder body to a hotter body.
Mathematically, the Second Law can be expressed as:
\begin{equation}
    \mathrm{d}S \geq \frac{\delta \mathcal{Q}}{T},
\end{equation}
where the equality holds just for reversible processes.

If we combine the First and Second Laws for reversible processes, we can derive an important relation known as the \textbf{Gibbs relation} for a system with a variable number of particles:
\begin{equation}
    \mathrm{d}E = T \mathrm{d}S - p \mathrm{d}V + \mu \mathrm{d}N.
    \label{eq:gibbs_relation}
\end{equation}
which will be fundamental to derive the thermodynamic potentials.

\subsection{Third Law of Thermodynamics}
The Third Law of Thermodynamics states that as the temperature of a system approaches absolute zero (0 Kelvin), the entropy of a perfect crystal approaches a constant minimum value, which is typically taken to be zero. This law implies that it is impossible to reach absolute zero through any finite number of processes.

\section{Thermodynamic Potentials}

Thermodynamic potentials are functions that describe the state of a thermodynamic system and are particularly useful for analyzing systems under different constraints.

If we take the Gibbs relation \eqref{eq:gibbs_relation} it is clear that the natural variables of the internal energy \(E\) are \(S\), \(V\), and \(N\). Thus \(E = E(S,V,N)\) is a homogeneous function of degree 1 in its extensive variables, meaning that if we scale all extensive variables by a factor \(\lambda > 0\), the internal energy scales by the same factor:
\[
    E(\lambda S, \lambda V, \lambda N) = \lambda E(S,V,N).
\]
The only functions that satisfy this property are linear functions, so we can write:
\[
    E = TS - pV + \mu N.
\]
We obtained the expression of our first thermodynamic potential, the internal energy \(E\). Note that it is a function of conjugate variables.

From this expression is easy to derive dependencies between the variables, for example:
\[
    T = \left[ \frac{\partial E}{\partial S} \right]_{V,N}, \quad p = -\left[ \frac{\partial E}{\partial V} \right]_{S,N}, \quad \mu = \left[ \frac{\partial E}{\partial N} \right]_{S,V}.
\]

We can define other thermodynamic potentials by performing \textbf{Legendre transforms} on the internal energy to change its natural variables. The most common thermodynamic potentials (defined here for reversible processes) are:

\begin{description}[style=nextline]
    \item[Internal Energy $E(S,V,N)$]
          \begin{equation}
              E = TS - pV + \mu N
          \end{equation}
          \begin{equation}
              \mathrm{d}E = T\,\mathrm{d}S - p\,\mathrm{d}V + \mu\,\mathrm{d}N
          \end{equation}

    \item[Helmholtz Free Energy $F(T,V,N)$]
          \begin{equation}
              F = E - TS = -pV + \mu N
          \end{equation}
          \begin{equation}
              \mathrm{d}F = -S\,\mathrm{d}T - p\,\mathrm{d}V + \mu\,\mathrm{d}N
          \end{equation}
    \item[Gibbs Free Energy $G(T,p,N)$]
          \begin{equation}
              G = E - TS + pV = \mu N
          \end{equation}
          \begin{equation}
              \mathrm{d}G = -S\,\mathrm{d}T + V\,\mathrm{d}p + \mu\,\mathrm{d}N
          \end{equation}
    \item[Enthalpy $H(S,p,N)$]
          \begin{equation}
              H = E + pV = TS + \mu N
          \end{equation}
          \begin{equation}
              \mathrm{d}H = T\,\mathrm{d}S + V\,\mathrm{d}p + \mu\,\mathrm{d}N
          \end{equation}
    \item[Grand Potential $\Omega(T,V,\mu)$]
          \begin{equation}
              \Omega = E - TS - \mu N = -pV
          \end{equation}
          \begin{equation}
              \mathrm{d}\Omega = -S\,\mathrm{d}T - p\,\mathrm{d}V - N\,\mathrm{d}\mu
          \end{equation}
\end{description}

For general processes (not necessarily reversible) we can write the differentials of the thermodynamic potentials considering the inequalities from the Second Law: every equality becomes an inequality, (\(= \,\, \longrightarrow \,\, \leq\)).

\paragraph{Thermodynamic limit.} In the thermodynamic limit, where the number of particles \(N\) and the volume \(V\) of the system approach infinity while maintaining a constant density \(n = N/V\), the thermodynamic potentials become extensive quantities; this means that they scale linearly with the size of the system. This property is crucial for ensuring that the thermodynamic description remains valid and consistent for large systems, allowing us to apply thermodynamic principles to real-world scenarios where systems contain a vast number of particles. For simplicity, we will often work with a fixed number of particles, and later generalize the results to the thermodynamic limit.

\paragraph{Variational principle.} The thermodynamic potentials also satisfy a variational principle: for a system in equilibrium, the appropriate thermodynamic potential is minimized (or maximized) under the given constraints. Fixing a triplet of conjugate variables, the corresponding thermodynamic potential reaches an extremum at equilibrium. We can imagine the hessian matrix of the potential with respect to its natural variables, and we can see that it is positive definite (minimum) or negative definite (maximum).
\[
    \left[ \frac{\partial^2 F}{\partial T^2} \right]_V < 0, \quad \left[ \frac{\partial^2 F}{\partial V^2} \right]_T > 0, \quad \frac{\partial \mu}{\partial T} = \frac{\partial^2 F}{\partial T \partial N} \leq 0.
\]