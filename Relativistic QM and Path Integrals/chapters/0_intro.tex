\chapter*{Introduction}

The unification of Quantum Mechanics and Special Relativity represents one of the most significant milestones in theoretical physics. While non-relativistic quantum mechanics successfully describes atomic spectra and low-energy phenomena, it fails to respect the fundamental symmetries of spacetime imposed by Einstein's theory.

In this course, we reconstruct quantum theory from first principles to ensure Lorentz covariance. We will see that imposing relativistic constraints is not merely a formal correction; it leads to profound physical consequences, including the emergence of intrinsic spin, the necessity of antiparticles, and the CPT theorem. Furthermore, we will move beyond the canonical operator formalism to introduce the Path Integral approach, which provides the most transparent framework for understanding symmetries, gauge theories, and the deep connection between quantum field theory and statistical mechanics.

\subsection*{Symmetry as a Guiding Principle}

In theoretical physics, dynamics is often dictated by symmetry. Before writing down equations of motion, we must understand the mathematical language of symmetries: \textbf{Group Theory} (Chapter 1).

We will focus on Lie groups and Lie algebras, specifically the Lorentz group $SO(3,1)$ and the unitary groups $SU(N)$. A central insight of this course is that elementary particles are not just objects that "have" properties; they are defined mathematically as \textbf{irreducible representations} of the spacetime symmetry group. This geometric perspective allows us to classify particles by their spin and mass before we even discuss their dynamics.

\subsection*{Relativistic Wave Equations}

The Schrödinger equation is structurally non-relativistic, treating space and time on unequal footings (first-order in time, second-order in space). To restore covariance, we must construct new wave equations.

In the second chapter, we systematically derive the equations governing different spin sectors:
\begin{itemize}
    \item The \textbf{Klein-Gordon equation} for scalar particles (spin 0).
    \item The \textbf{Dirac equation} for fermions (spin $1/2$), which naturally explains the origin of the gyromagnetic factor $g=2$ and the existence of antimatter.
    \item The \textbf{Proca and Maxwell equations} for vector bosons (spin 1).
\end{itemize}
We will analyze the discrete symmetries (Parity, Time Reversal, Charge Conjugation) of these fields and confront the difficulties of the "single-particle" interpretation, which inevitably points towards the necessity of a many-body field theory.

\subsection*{The Path Integral Formalism}

While the canonical quantization method (imposing commutation relations) is rigorous, it can be cumbersome for relativistic systems. In the third chapter, we introduce Richard Feynman's \textbf{Path Integral} formulation.

This approach reimagines quantum mechanics not as the evolution of a state vector in Hilbert space, but as a sum over all possible classical histories, weighted by the action phase $e^{iS/\hbar}$. This formalism is particularly powerful for:
\begin{enumerate}
    \item \textbf{Perturbation Theory:} It allows for a systematic derivation of Feynman diagrams and correlation functions.
    \item \textbf{Symmetries:} It makes the preservation of symmetries (like gauge invariance) manifest in the measure of integration.
    \item \textbf{Fermions:} We will introduce \textbf{Grassmann numbers} (anticommuting variables) to extend the path integral to fermionic fields.
\end{enumerate}
Finally, by performing a "Wick rotation" to imaginary time ($t \to -i\tau$), we will demonstrate the isomorphism between the quantum path integral and the partition function of Statistical Mechanics, revealing the unity between quantum fluctuations and thermal fluctuations.

\vspace{1cm}
\textbf{Prerequisites:} \textit{Familiarity with the Lagrangian and Hamiltonian formalisms of Classical Mechanics, standard Quantum Mechanics, and the basic tenets of Special Relativity is required.}