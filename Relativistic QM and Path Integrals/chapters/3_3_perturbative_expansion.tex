\section{Perturbative Expansion}

The free theory corresponds to a gaussian path integral, which is exactly solvable. With interactions, one is often unable to compute exactly the path integral, and one must resort to some sort of approximation. The simplest one is the perturbative expansion around a free theory, which consists in expanding the solution in power series of the coupling constants that parametrize the interactions. If the couplings are small enough, the perturbative expansion might give a good approximation of the solution.

We describe the perturbative expansion taking as a guiding example the \textbf{anharmonic oscillator}
\[
    S[x] = \int \d{t} \left( \frac{m}{2} \dot{x}^2 - \frac{m\omega^2}{2} x^2 -\frac{g}{3!} x^3 - \frac{\lambda}{4!} x^4 \right),
\]
where if the coupling constants \(g\) and \(\lambda\) vanish, the theory is exactly solvable. Thus, one may try to include perturbatively the corrections that arise when \(g\) and \(\lambda\) are small enough. It is convenient to split the action as the sum of two terms, a free part \(S_0\) which is exactly solvable and an interacting part \(S_{int}\)
\[
    \begin{aligned}
        S[x]       & = S_0[x] + S_{int}[x],                                                       \\
        S_0[x]     & = \int \d{t} \left( \frac{m}{2} \dot{x}^2 - \frac{m\omega^2}{2} x^2 \right), \\
        S_{int}[x] & = \int \d{t} \left( -\frac{g}{3!} x^3 - \frac{\lambda}{4!} x^4 \right).
    \end{aligned}
\]
The perturbative expansion is easily generated in the path integral setup. Including a source term, one expands in a Taylor series the exponential of the interaction term
\[
    \begin{aligned}
        Z[J] & = \int \mathrm{D}x\, e^{\frac{i}{\hbar}\left(S[x]+\int \d{t}Jx\right)} = \int \mathrm{D}x\, e^{\frac{i}{\hbar}S_{int}[x]}e^{\frac{i}{\hbar}\left(S_{0}[x]+\int \d{t}Jx\right)} \\
             & = \int \mathrm{D}x\, \left[ \sum_{n=0}^{\infty} \frac{1}{n!} \left(\frac{i}{\hbar}S_{int}[x]\right)^n \right] e^{\frac{i}{\hbar}\left(S_0[x]+\int \d{t}J x\right)}.
    \end{aligned}
\]
We can keep terms until we are satisfied with the resolution (depending on the entity of \(g\) and \(\lambda\)). Written in the last form, one proceeds in computing it term by term with the use of the Wick’s theorem. It can be written also in the form
\[
    Z[J] = \langle e^{\frac{i}{\hbar} S_{int}[x]} \rangle_{U,0,J},
\]
where the subscripts \(U,\,0,\,J\) denote unnormalized averaging \((U)\) in the free theory \((0)\) with an arbitrary source \((J)\). This expression is sometimes called the \textbf{“Dyson formula”}. It generates the perturbative expansion which can be depicted with \textit{Feynman diagrams}, as we shall see.

An alternative way of writing the perturbative series is the following one
\[
    \begin{aligned}
        Z[J] & = \int \mathrm{D}x\, e^{\frac{i}{\hbar}S_{int}[x]}e^{\frac{i}{\hbar}\left(S_{0}[x]+\int \d{t}Jx\right)}                                                                                                                                                                               \\
             & = \exp\left( \frac{i}{\hbar} S_{int}\left[ \frac{\hbar}{i} \frac{\delta}{\delta J} \right] \right) \int \mathrm{D}x\, e^{\frac{i}{\hbar}\left(S_0[x]+\int \d{t}J x\right)} = \exp\left( \frac{i}{\hbar} S_{int}\left[ \frac{\hbar}{i} \frac{\delta}{\delta J} \right] \right) Z_0[J],
    \end{aligned}
\]
where \(Z_0[J]\) is the generating functional of the free theory. This expression is often more convenient to work with, since it reduces the problem to computing functional derivatives of the known free generating functional. It presents the solution as a (quite complicated) differential operator acting on the functional of the free theory \(Z_0[J]\). In particular, all vacuum diagrams are generated by
\[
    Z[0] = \int \mathrm{D}x\, e^{\frac{i}{\hbar}S[x]} = \exp\left( \frac{i}{\hbar} S_{int}\left[ \frac{\hbar}{i} \frac{\delta}{\delta J} \right] \right) Z_0[J] \Big\vert_{J=0}.
\]
The perturbative expansion can be represented using Feynman diagrams. These are constructed by expanding the interaction term in the path integral and applying Wick’s theorem to compute the correlation functions within the free theory. In these diagrams, vertices, represented as dots, correspond to the interaction potentials and involve a coupling constant multiplied by quantum variables. These variables are paired in all possible combinations using free propagators, which are graphically depicted as lines.

This construction is illustrated below through the example of vacuum diagrams for the anharmonic oscillator.

\subsection{Vacuum Diagrams}

As an example, we compute perturbatively the corrections to the ground state energy of the harmonic oscillator due to the \textbf{anharmonic potential} terms. It is often the case that one computes using the euclidean version of the theory and performs the inverse Wick rotation at the very end to obtain the final result in minkowskian time.

Thus, let us consider the euclidean generating functional and action for the \textbf{anharmonic oscillator}
\[
    \begin{aligned}
        Z_E[J] & = \int \mathrm{D}x\, e^{-\frac{1}{\hbar}\left(S_E[x]-\int \d{\tau} J x\right)},                                                                                          \\
        S_E[x] & = \lim_{\beta \to \infty} \int_{-\beta/2}^{\beta/2} \d{\tau} \left( \frac{m}{2} \dot{x}^2 + \frac{m\omega^2}{2} x^2 + \frac{g}{3!} x^3 + \frac{\lambda}{4!} x^4 \right).
    \end{aligned}
\]
We want to compute corrections to the ground state energy, which can be obtained from the vacuum diagrams, which correspond to \(Z_E[0]\). Using the perturbative expansion, we have
\[
    \begin{aligned}
        Z_E[0] = \int \mathrm{D}x\, e^{-\frac{1}{\hbar}S_{E}[x]} = \langle 1 \rangle_{U} = \lim_{\beta \to \infty} \bra{0} e^{-\beta \hat{H}} \ket{0} = \left\langle e^{-\frac{1}{\hbar} S_{E, int}[x]} \right\rangle_{U,0} = \lim_{\beta \to \infty} e^{-\beta E_0^{(0)}+ \Delta E_0},
    \end{aligned}
\]
where the exact energy \(E_0\) of the ground state \(\ket{0}\) of the anharmonic oscillators differs from the ground state energy of the harmonic oscillator \(E_0^{(0)}\) by the term \(\Delta E_0\) due to the anharmonic potential (here we have the contributions from interaction terms). The correction \(\Delta E_0\) can be computed perturbatively, considering the first non-vanishing corrections to exemplify the perturbative expansion with path integrals and the use of Feynman diagrams.

When we take \(\beta \to \infty\), we are projecting onto the ground state, since we assume that the spectrum of the Hamiltonian is bounded from below. Thus, the euclidean time evolution operator \(e^{-\beta \hat{H}}\) suppresses all contributions from excited states exponentially fast as \(\beta\) increases. Then we apply the perturbative expansion in the interaction term (as we did in Minkowskian time in the previous section), which gives us an unnormalized average in the free theory. Now we can recognize that this average corresponds to the vacuum amplitude of the free theory multiplied by corrections due to the interactions.

We can continue the computation for the interaction terms in the correlation function by considering separately the two terms (by setting each one to zero and effectively \textit{turning off} one interaction at a time). Let us look first at the case with \(g=0\) and focus on the first correction in \(\lambda\)
\[
    \begin{aligned}
        Z_E[0] & = \langle 1 \rangle_U = \left\langle \sum_{n=0}^{\infty} \frac{1}{n!} S_{E,\,int}[x]^n \right\rangle_{U,0} = \langle 1 \rangle_{U,0} -\frac{\lambda}{4!} \int_{-\beta/2}^{\beta/2} \d{\tau} \langle x^4(\tau) \rangle_{U,0} + \cdots                 \\
               & = \langle 1 \rangle_{U,0} \left[ 1 -\frac{\lambda}{4!} \int_{-\beta/2}^{\beta/2} \d{\tau} \langle x^4(\tau) \rangle_{U,0} +\cdots \right] = \langle 1 \rangle_{U,0} \left[ 1 -\frac{\lambda}{4!} \left( 3\times \feynEight \right) + \cdots \right].
    \end{aligned}
\]
In the last line, we have used Wick contractions to calculate normalized correlation functions in the free theory, and then introduced a graphical representation in terms of \textbf{Feynman diagrams}. In this graphical representation, a line denotes a propagator that joins two points in time, while vertices arising from the interactions are denoted by dots. The term we obtained contains just one vertex where four lines can enter or exit, corresponding to the power four of the dynamical variable \(x(\tau)\) associated to the interaction under consideration.

Recalling the euclidean propagator calculated in eq. \eqref{eq:euclidean_green_function_harmonic_oscillator}
\[
    G_E(\tau - \tau^{\prime}) = \langle x(\tau) x(\tau^{\prime}) \rangle_0 = \frac{1}{2\omega} e^{-\omega \vert \tau - \tau^{\prime} \vert} = \feynLine
\]
where this is the \textbf{Feynman Line} associated to the two point correlation function; we can compute the value of the previous diagram, considering that we have four fields at the same time \(\tau\). Thus, we have to consider all possible Wick contractions, which give three identical contributions (we have \(x(\tau)^4\) not \(x_1(\tau)x_2(\tau)\cdots\) so the fields are identical), each corresponding to a pair of propagators that start and end at the same time \(\tau\). Therefore, we have
\[
    DIAGRAMS \quad SUM
\]
So that we understand why we wrote \(3 \times \feynEight\) in the previous expression. Now, we can compute the value of the diagram \(\feynEight\) understanding it as a loop with two propagators that start and end at the same time:
\[
    \feynEight = \int_{-\beta/2}^{\beta/2} \d{\tau} \, G_E(\tau,\tau)^2 = \int_{-\beta/2}^{\beta/2} \d{\tau} \, \left( \frac{1}{2\omega} \right)^2 = \frac{\beta}{4\omega^2}.
\]
Thus, up to this order (first term) in perturbation theory, we have\footnote{This computation relies on the boundaries conditions we have set: with \(\beta \to  \infty\) we are projected on the ground state, and the propagator is computed with this precise Green function, while different boundary conditions would lead to different propagators and thus different results.}
\[
    Z_E[0] = \langle 1 \rangle_{U,0} \left[ 1 - \frac{\lambda}{4!} \left( 3 \times \frac{\beta}{4\omega^2} \right) + \cdots \right] = \langle 1 \rangle_{U,0} \exp\left( -\beta \frac{\lambda}{32\omega^2} + \cdots \right),
\]
so that we can read the correction to the ground state energy due to the quartic interaction
\[
    \Delta E_0 = \frac{1}{32} \frac{\lambda}{\omega^2}.
\]

Similarly, one may consider the case with \(g \neq 0\) and \(\lambda = 0\). The first non-vanishing correction arises from
\[
    \begin{aligned}
        Z_E[0] & = \langle 1 \rangle_U = \left\langle \left( 1- S_{E,\,int} + \frac{1}{2}S_{E,\,int}^2 + \cdots \right) \right\rangle_{U,0}                                                                                                                                                  \\
               & = \langle 1 \rangle_{U,0} + \frac{g}{3!} \int_{-\beta/2}^{\beta/2} \d{\tau} \langle x^3(\tau)\rangle_{U,0} + \frac{g^2}{2(3!)^2} \int_{-\beta/2}^{\beta/2} \d{\tau} \int_{-\beta/2}^{\beta/2} \d{\tau^{\prime}} \langle x^3(\tau) x^3(\tau^{\prime}) \rangle_{U,0} + \cdots \\
               & = \langle 1 \rangle_{U,0} \left[ 1 + 0 + \frac{g^2}{2(3!)^2}\left((3!)\times \feynSunset + (3^2)\times \feynDumbbell \right) + \cdots \right].
    \end{aligned}
\]
The first correction vanishes because there is no way to contract three fields at the same time in pairs (from Wick's theorem). The second correction is represented by a six-point correlation function, which gives two types of contributions: we have three fields at time \(\tau\) and three fields at time \(\tau^{\prime}\), which can be contracted in two different ways.
\[
    DIAGRAMS \quad SUM
\]
The frist one has two vertices and three propagators connecting them (in \(3\times 2\times 1\) possible permutations), while the second one has two vertices connected by a single propagator (the two vertices can be chosen from any of the present, so \(3\times 3\) possibilities), with each vertex having a loop attached to it. We can check the combinatorial factors by counting the number of Wick contractions that give rise to each diagram: \(6+9 = 15\), which is indeed the number of ways to contract six fields in pairs.

Thus we have found two types of diagrams: one is the \textbf{sunset diagram} \(\feynSunset\) where two vertices are connected by three propagators, while the other one is the \textbf{dumbbell diagram} \(\feynDumbbell\) where two vertices are connected by a single propagator and each vertex has a loop attached to it. The combinatorial factors arise from the number of Wick contractions that give rise to each diagram. We can compute their values as follows. For the sunset diagram, we have\footnote{We do not compute the limit for \(\beta \to \infty\) yet, since we are just interested in the value of the diagram itself. We will take the limit at the very end to extract the ground state energy correction after inserting these results into the expression for \(Z_E[0]\).}
\[
    \begin{aligned}
        \feynSunset & = \int_{-\beta/2}^{\beta/2} \d{\tau} \int_{-\beta/2}^{\beta/2} \d{\tau^{\prime}} \, G_E(\tau,\tau^{\prime})^3 = \int_{-\beta/2}^{\beta/2} \d{\tau} \int_{-\beta/2}^{\beta/2} \d{\tau^{\prime}} \, \left( \frac{1}{2\omega} e^{-\omega \vert \tau - \tau^{\prime} \vert} \right)^3 \\
                    & = \cdots                                                                                                                                                                                                                                                                          \\
                    & = \frac{\beta}{8\omega^3}\frac{2}{3\omega} = \frac{\beta}{12\omega^4},
    \end{aligned}
\]
while for the dumbbell diagram, we have
\[
    \begin{aligned}
        \feynDumbbell & = \int_{-\beta/2}^{\beta/2} \d{\tau} \int_{-\beta/2}^{\beta/2} \d{\tau^{\prime}} \, G_E(\tau,\tau) G_E(\tau - \tau^{\prime}) G_E(0)                                                                                                                                         \\
                      & = \cdots                                                                                                                                                                                                                                                                    \\
                      & = \int_{-\beta/2}^{\beta/2} \d{\tau} \int_{-\beta/2}^{\beta/2} \d{\tau^{\prime}} \, \left( \frac{1}{2\omega} \right)^2 \left( \frac{1}{2\omega} e^{-\omega \vert \tau - \tau^{\prime} \vert} \right) = \frac{\beta}{8\omega^3}\frac{2}{\omega} = \frac{\beta^2}{4\omega^4}.
    \end{aligned}
\]
Now, we can insert these results into the expression for \(Z_E[0]\) to obtain
\[
    Z_E[0] = \langle 1 \rangle_{U,0} \left[ 1 + \frac{g^2}{2(3!)^2} \left( (3!) \times \frac{\beta}{12\omega^4} + (3^2) \times \frac{\beta^2}{4\omega^4} \right) + \cdots \right] = \langle 1 \rangle_{U,0} \exp\left( \beta \frac{11}{8(3!)^2} \frac{g^2}{\omega^4} + \cdots \right),
\]
finding the entity of the correction to the ground state energy due to the cubic interaction
\[
    \Delta E_0 = - \frac{11}{288} \frac{g^2}{\omega^4}.
\]

\subsection{Other Correlators and Feynman Diagrams}
In a similar way, one computes the perturbative expansion of other correlation functions, considering that the vacuum diagrams correspond to unnormalized 0-point function, relating to the ground state energy. We start from the vacume, something happens during the evolution, and we return to the vacuum.

We can treat the two point correlation function in a similar way. We have two particles, one in the far past and one in the far future, which interacts in some way (for example the anharmonic potential seen before) during their evolution. The two point correlation function is given by
\[
    \begin{aligned}
        \langle x(t_1) x(t_2) \rangle & = \frac{1}{Z} \int \mathrm{D}x\, x(t_1) x(t_2) e^{\frac{i}{\hbar}S[x]} = \left\langle x(t_1) x(t_2) e^{\frac{i}{\hbar} S_{int}[x]} \right\rangle_{0}              \\
                                      & = \left\langle x(t_1) x(t_2) \left[ 1 + \frac{i}{\hbar} S_{int}[x] + \frac{1}{2!} \left( \frac{i}{\hbar} S_{int}[x] \right)^2 + \cdots \right] \right\rangle_{0}.
    \end{aligned}
\]
We can interpret as follows: we have two external points at times \(t_1\) and \(t_2\) where particles are created/annihilated, and we have to consider all possible interactions that may happen during their evolution. Using Wick contractions, we can compute the perturbative expansion of this correlation function in the free theory. For the case of a cubic interaction, \(S_{int} = -\frac{g}{3!}\int\d{t}x(t)^3\), it leads to the following diagrammatic expansion up to second order in perturbation theory:
\[
    DIAGRAMS
\]
its like studying a 8 point function but with 2 external points fixed.
The first diagram corresponds to the free propagator between times \(t_1\) and \(t_2\). The second diagram represents the first-order correction due to a single interaction vertex, where the two external points are connected through a loop. The third and fourth diagrams represent second-order corrections, involving two interaction vertices. The third diagram shows two vertices connected by three propagators, while the fourth diagram has two vertices connected by a single propagator with loops attached to each vertex. This exemplifies how Feynman diagrams arise naturally in the perturbative expansion of correlation functions in quantum field theory.

Let us describe graphically the corrections that must be computed for calculating perturbatively the 4-point function (in a QFT context, it is linked to the scattering of 2 incoming particles to 2 outgoing particles)
\[
    \langle x(t_1) x(t_2) x(t_3) x(t_4) \rangle
\]
where one may keep in mind that setting \(t_1,\,t_2 \to -\infty\) describes incoming states while \(t_3,\,t_4 \to +\infty\) describes outgoing states. We have
\[
    \begin{aligned}
        \langle x(t_1) x(t_2) x(t_3) x(t_4) \rangle & = \frac{1}{Z} \int \mathrm{D}x\, x(t_1) x(t_2) x(t_3) x(t_4) e^{\frac{i}{\hbar}S[x]} = \left\langle x(t_1) x(t_2) x(t_3) x(t_4) e^{\frac{i}{\hbar} S_{int}[x]} \right\rangle_{0} \\
                                                    & = \left\langle x(t_1) x(t_2) x(t_3) x(t_4) \left[ 1 + \frac{i}{\hbar} S_{int}[x] + \frac{1}{2!} \left( \frac{i}{\hbar} S_{int}[x] \right)^2 + \cdots \right] \right\rangle_{0}.
    \end{aligned}
\]

For the case of a cubic interaction, \(S_{int} = -\frac{g}{3!}\int\d{t}x(t)^3\), it leads to the following diagrammatic expansion which is obtained by the systematic use of Wick contractions:\TODO{Modify last diagram in first row, it needs 4 external points.}
\begin{figure}[H]
    \centering
    \includegraphics[width=0.8\textwidth]{img/diagrammatic_expansion_cubic_interaction.png}
    \caption{Diagrammatic expansion of the 4-point function for a cubic interaction up to second order in perturbation theory.}
    \label{fig:diagrammatic_expansion_cubic_interaction}
\end{figure}

One notices disconnected, connected, and 1PI diagrams, which play a significant role in QFT (1PI diagrams are those diagrams that remain connected after cutting any single internal line). This exemplifies the emergence of Feynman diagrams in describing the perturbative expansion.

Nobody knows what happens during the interaction, virtually anything can happen, but we can sum over all the possibilities using the path integral formalism and Wick’s theorem.

Here ends the treatment for the \textbf{bosonic path integral}: we saw that this formalism aplies naturally on the KG field and scalar fields in general, and we developed the perturbative expansion with Feynman diagrams. In the next section, we extend the path integral formalism to fermionic systems.

\section{Path Integrals for Fermions}

We now discuss how to extend the path integral method to fermionic systems. Fermions at the classical level can be described by \textbf{Grassmann variables}, also known as \textbf{anticommuting} numbers or fermionic variables. Grassmann variables allow us to define “classical” models whose quantization produces degrees of freedom that satisfy the Pauli exclusion principle.

The question is: how can we define a path integral for fermionic systems? The main difficulty arises from the anticommuting nature of fermionic variables, which makes the standard definition of the path integral inapplicable. There is no classical trajectory for fermions in the usual sense, since the ferionic nature of the particles vanishes in the classical limit \(\hbar \to 0\). To overcome this issue, we introduce Grassmann variables to describe fermionic degrees of freedom at the classical level.

We need to introduce a suitable Hamiltonian theory to deal with fermionic degrees of freedom, which has symmetric Poisson Brackets, in order to quantize them canonically obtaining a quantum theory based on \textit{anticommutiors}.
This is why we call it a \textbf{pseudoclassical model}: it is not a classical model in the usual sense, but it is a formal construction that allows us to quantize fermionic degrees of freedom, and leads to the correct quantum theory based on Grassmann variables.
\[
    \left\{ \psi,\,\overline{\psi} \right\}_{C} \longrightarrow \left\{ \hat{\psi},\,\hat{\psi}^\dagger \right\} = i \hbar \dots .
\]
Models with Grassmann variables are often called “pseudoclassical”, as the spin at the classical level is just a formal construction (the value of any finite spin vanishes for \(\hbar \to  0\), and thus cannot be measured classically). In the following, we first exemplify the use of Grassmann variables in mechanical models. The method extends to field theories as well, so that a Dirac field can be treated classically with Grassmann variables. Then, we develop canonical quantization for mechanical models containing Grassmann variables.

At last, we derive a path integral representation of the transition amplitude for fermionic systems starting from its operatorial expression and using a suitable definition of fermionic coherent states.

\subsection{Grassmann Algebras}

A \(n\)-dimensional Grassmann algebra \(\mathcal{G}_n\) is generated by a set of generators \(\theta_i\) with \(i=1,\,\ldots,\,n\), which satisfy the anticommutation relations
\[
    \left\{ \theta_i,\,\theta_j \right\} = 0, \quad \forall i,\,j = 1,\,\ldots,\,n.
\]
From these relations, it follows that \(\theta_i^2 = 0\) for all \(i\). The elements of the Grassmann algebra are linear combinations of products of the generators, with complex coefficients. This suggests already at the classical level the essence of the Pauli exclusion principle, according to which one cannot put two identical fermions in the same quantum state. Physicists often call these generators anticommuting numbers.

\paragraph{Functions.}

\paragraph{Derivatives.}

\paragraph{Integrals.}

\paragraph{Reality properties.}

\paragraph{Gaussian integrals.}

\subsection{Pseudoclassical Model and Canonical Quantization}

\subsection{Coherent States}

\subsection{Fermionic Path Integrals}

\paragraph{Correlation functions.}