\section{Dirac Equation}
Dirac found the correct equation to describe particles of spin \(\tfrac{1}{2}\) by looking for a relativistic wave equation that could admit a probabilistic interpretation and thus be consistent with the principles of quantum mechanics. The Klein-Gordon equation did not have such an interpretation. Although a probabilistic interpretation will not be possible in the presence of interactions (eventually, Dirac’s wave function must be treated as a classical field to be quantized again in the second quantization), it is useful to retrace the line of thinking that brought Dirac to the formulation of an equation of first order in time.

The relativistic relation between energy and momentum reads
\[
    p_\mu p^{\mu} = -m^2c^2 \iff E^2 = p^2 c^2 + m^2 c^4,
\]
thus, with the correspondence principle
\[
    E = cp^0 \to i \hbar \frac{\partial}{\partial t}, \quad \mathbf{p} \to -i \hbar \nabla \quad \implies p_\mu \to i \hbar \partial_\mu,
\]
which leads to the Klein-Gordon equation, second order in both time and space derivatives. As a consequence, the KG conserved current density is not positive definite, making difficult a probability density interpretation of the theory.

Dirac proposed an alternative equation, first order in both time and space derivatives, that could overcome these problems. He started from the ansatz
\begin{equation}
    E = c \mathbf{p} \cdot \bs{\alpha} + \beta m c^2,
    \label{eq:dirac_equation_alpha_beta}
\end{equation}
with \(\bs{\alpha} = (\alpha^1,\,\alpha^2,\,\alpha^3)\) and \(\beta\) hermitian matrices to be determined. Squaring both sides and imposing the relativistic relation, one finds the conditions
\[
    \begin{aligned}
        E^2 & = (c \mathbf{p} \cdot \bs{\alpha} + \beta m c^2)(c \mathbf{p} \cdot \bs{\alpha} + \beta m c^2)                                                                                                 \\
            & = c^2 (p^i \alpha^i)^2 + m^2 c^4 \beta^2 + mc^3 \left(p^i \alpha^i \beta + \beta p^j \alpha^j \right)                                                                                          \\
            & = m^2 c^4 \frac{1}{2} (\beta\beta + \beta\beta) + c^2 p^i p^j \alpha^i \alpha^j + m c^3 p^i \left(\alpha^i \beta + \beta \alpha^i \right)                                                      \\
            & = m^2 c^4 \frac{1}{2} \{\beta,\,\beta\} + c^2 p_i p^j \frac{1}{2}\left(\alpha^i \alpha^j + \alpha^j \alpha^i + \alpha^i \alpha^j - \alpha^j \alpha^i \right) + m c^3 p^i \{\alpha^i,\,\beta\},
    \end{aligned}
\]
where we have recognized the anticommutator \(\{\,,\,\}\) and decomposed the product of \(\alpha\) matrices into symmetric and antisymmetric parts. Since \(p^i p^j \to \hbar^2 \partial^i \partial^j\) is symmetric in the indices \(i\) and \(j\), the antisymmetric part of the product of \(\alpha\) matrices vanishes. Thus, we are able to recognize another commutator, before imposing the realativistic relation:
\[
    E^2 = m^2 c^4 \frac{1}{2}\{\beta,\,\beta\} + c^2 p^i p^j \frac{1}{2} \{\alpha^i,\,\alpha^j\} + m c^3 p^i \{\alpha^i,\,\beta\} = m^2 c^4 + c^2 p^2,
\]
hence the relations that the matrices must satisfy:
\begin{equation}
    \begin{aligned}
        \{\alpha^i,\,\alpha^j\} & = 2 \delta^{ij} \mathbb{I}, \\
        \{\alpha^i,\,\beta\}    & = 0,                        \\
        \{\beta,\,\beta\}       & = 2 \mathbb{I}.
    \end{aligned}
    \label{eq:clifford_relations}
\end{equation}
These relations define the \textbf{Clifford Algebra} and cannot be satisfied by numbers, as the first relation implies that \(\alpha^i\) cannot commute among themselves. The smallest matrices that can satisfy these relations (minimal solution) are \(4 \times 4\) \textbf{traceless} matrices, which can be constructed in terms of the Pauli matrices \(\sigma^i\):
\begin{equation}
    \alpha^i = \begin{pmatrix}
        0        & \sigma^i \\
        \sigma^i & 0
    \end{pmatrix}, \quad \beta = \begin{pmatrix}
        \mathbb{I} & 0           \\
        0          & -\mathbb{I}
    \end{pmatrix},
    \label{eq:dirac_representation}
\end{equation}
known as the \textbf{Dirac representation} of the Clifford algebra; let us remember that the Pauli matrices are
\[
    \sigma^1 = \begin{pmatrix}
        0 & 1 \\
        1 & 0
    \end{pmatrix}, \quad
    \sigma^2 = \begin{pmatrix}
        0 & -i \\
        i & 0
    \end{pmatrix}, \quad
    \sigma^3 = \begin{pmatrix}
        1 & 0  \\
        0 & -1
    \end{pmatrix},
\]
and they satisfy the relation \(\sigma^i \sigma^j = \delta^{ij} \mathbb{I} + i \epsilon^{ijk} \sigma^k\).

\begin{theorem}
    All four dimensional irreducible representations of the Clifford algebra are unitarily equivalent to the Dirac representation, and thus related by a change of basis; other non trivial representations can be constructed as direct sums of irreducible ones and are thus reducible.
\end{theorem}

With the correspondence principle, we can now quantize the ansatz for the energy and thus write the \textbf{Dirac equation} in Hamiltonian form:
\begin{equation}
    i \hbar \frac{\partial}{\partial t} \psi(\mathbf{x},\,t) = \left(-i \hbar c \bs{\alpha} \cdot \nabla + \beta m c^2 \right) \psi(\mathbf{x},\,t),
    \label{eq:dirac_equation_hamiltonian}
\end{equation}
where the operator acting on the right-hand side is the Dirac Hamiltonian, acting on the four-component spinor \(\psi(\mathbf{x},\,t)\); since \(\bs{\alpha}\) and \(\beta\) are hermitian matrices, the Hamiltonian is hermitian as well and the time evolution is unitary. The Dirac equation can be written in covariant form\TODO{Finish computation}
\[
    \begin{aligned}
        \frac{-\beta}{\hbar c} \left(i \hbar \frac{\partial}{\partial t} - \beta m c^2 \right) \psi  = \frac{\beta}{\hbar c} \left(-i \hbar c \bs{\alpha} \cdot \nabla + \beta m c^2 \right) \psi \\
        =  \dots,
    \end{aligned}
\]
we can identify components of a scalar product between four-vectors if we define the gamma matrices as
\begin{equation}
    \gamma^0 = -i\beta, \quad \gamma^i = -i \beta \alpha^i,
    \label{eq:gamma_matrices}
\end{equation}
so that the Dirac equation takes the form
\begin{equation}
    (i \hbar \gamma^{\mu} \partial_{\mu} + \mu^2) \psi(x) = 0,
    \label{eq:dirac_equation_covariant}
\end{equation}
with \(\mu = \frac{m c}{\hbar}\) the inverse Compton wavelength of the particle. The gamma matrices satisfy the clifford algebra relations
\[
    \left\{ \gamma^\mu, \gamma^\nu \right\} = 2 \eta^{\mu \nu},
\]
where \(\eta^{\mu \nu} = \text{diag}(-1,\,1,\,1,\,1)\)\TODO{Check if notation is mostly plus or minus.} is the Minkowski metric tensor. In the Dirac representation, the gamma matrices read
\[
    \gamma^0 = -i \begin{pmatrix}
        \mathbb{I} & 0           \\
        0          & -\mathbb{I}
    \end{pmatrix}, \quad \gamma^i = -i \begin{pmatrix}
        0         & \sigma^i \\
        -\sigma^i & 0
    \end{pmatrix},
\]
and we can adopt natural units and \textbf{Feynman slash notation}
\[
    \slashed{A} = \gamma^{\mu} A_{\mu} = - \gamma_{\mu} A^{\mu},
\]
so that \(\slashed{\partial} = \gamma^{\mu} \partial_{\mu}\) to write the Dirac equation in a more compact form:
\[
    \left(i \gamma^{\mu} \partial_{\mu} + \mu \right) \psi(x) = 0 \quad \Longleftrightarrow \quad (i \slashed{\partial} + m) \psi(x) = 0.
\]

\subsection{Continuity Equation}

It is immediate to derive an equation of continuity describing the conservation of a positive definite charge. Dirac tentatively identified the relative charge density, appropriately normalized, with a probability density. Let us see how to get the continuity equation algebraically. Using the hamiltonian form, we multiply eq. \eqref{eq:dirac_equation_hamiltonian} with \(\psi^\dagger\) on the left and subtract the hermitian-conjugated equation multiplied by \(\psi\) on the right, and obtain (remember that \(\bs{\alpha}\) and \(\beta\) are hermitian matrices)
\[
    \begin{aligned}
         & i \hbar \psi^{\dagger} \frac{\partial}{\partial t} \psi + i \hbar c \psi^{\dagger} \bs{\alpha} \cdot \nabla \psi - m c^2 \psi^{\dagger} \beta \psi        \\
         & - i \hbar \frac{\partial}{\partial t} \psi^{\dagger} \psi - i \hbar c \nabla \psi^{\dagger} \cdot \bs{\alpha} \psi + m c^2 \psi^{\dagger} \beta \psi = 0,
    \end{aligned}
\]
which can be rearranged, after simplifying the mass terms, as the continuity equation
\begin{equation}
    \frac{\partial \psi^{\dagger} \psi}{\partial t} + \nabla \cdot (c \psi^{\dagger} \bs{\alpha} \psi) = 0,
    \label{eq:dirac_continuity_equation}
\end{equation}
identifying the probability density and current density as
\[
    \rho = \psi^{\dagger} \psi, \quad \mathbf{J} = c \psi^{\dagger} \bs{\alpha} \psi.
\]
The probability density is positive definite, overcoming the problem of the Klein-Gordon equation. The current density can be interpreted as the flow of probability per unit area per unit time.

\subsection{Plane Wave Solutions}

The free equation admits plane wave solutions which contain the phase \(e^{ip_\mu x^{\mu}}\) for propagation in space-time and a polarization \(\omega(p)\) for the spin. Inserting in the Dirac equation \eqref{eq:dirac_equation_covariant} a plane wave ansatz of the form
\[
    \psi_p (x) \sim \omega(p) e^{i p_\mu x^{\mu}}, \quad \omega(p) = \begin{pmatrix}
        \omega_1(p) \\
        \omega_2(p) \\
        \omega_3(p) \\
        \omega_4(p)
    \end{pmatrix},
\]
with \(p^{\mu}\) arbitrary four-momentum, one finds conditions on the spinor \(\omega(p)\) and on the four-momentum \(p^{\mu}\) such that the Dirac equation is satisfied. Plugging the ansatz in the Dirac equation, one finds
\[
    \left(\slashed{\partial} + m\right)\psi_p (x) = \left(\gamma^{\mu} p_{\mu} + m \right) \omega(p) e^{i p_\mu x^{\mu}} = 0,
\]
where \(\partial_{\mu} \psi_p = \partial_\mu \left(\omega(p) e^{i p_\mu x^{\mu}}\right) = i p_\mu \omega(p) e^{i p_\mu x^{\mu}}\), hence we find
\[
    \left(i \gamma^{\mu} p_{\mu} + m \right) \omega(p) e^{i p_\mu x^{\mu}} = 0,
\]
which can be multiplied by \(\left(-i \gamma^{\nu} p_{\nu} + m \right)\) on the left to obtain
\[
    \left(\slashed{p}^2 + m^2 \right) \omega(p) e^{i p_\mu x^{\mu}} = \left(p_{\mu} p^{\mu} + m^2 \right) \omega(p) = 0,
\]
since it is easy to show that \(\slashed{p}^2 = p^2\):
\[
    \begin{aligned}
        \slashed{p}^2 & = \gamma^{\mu} \gamma^{\nu} p_{\mu} p_{\nu} = \frac{1}{2} \left(\gamma^{\mu} \gamma^{\nu} + \gamma^{\nu} \gamma^{\mu} + \gamma^{\mu} \gamma^{\nu} - \gamma^{\nu} \gamma^{\mu}\right) p_{\mu} p_{\nu} \\
                      & = \frac{1}{2} \{\gamma^{\mu},\,\gamma^{\nu}\} p_{\mu} p_{\nu} = \eta^{\mu \nu} p_{\mu} p_{\nu} = p^2,
    \end{aligned}
\]
since the antisymmetric part vanishes due to the symmetry of \(p_{\mu} p_{\nu}\). Thus, we find the mass-shell condition
\[
    p_{\mu} p^{\mu} + m^2 = 0 \quad \Longleftrightarrow \quad E^2 = p^2 + m^2,
\]
which is satisfied by both positive and negative energy solutions \(E = \pm \sqrt{p^2 + m^2}\), as for Klein-Gordon. The spinor \(\omega(p)\) must satisfy the equation
\begin{equation}
    \left(p_{\mu}p^{\mu} + m^2 \right) \omega(p) = 0.
    \label{eq:dirac_spinor_equation}
\end{equation}

\paragraph{Particle at rest.}
In order to understand the structure of the spinor \(\omega(p)\), let us first consider the case of a particle at rest, i.e., \(p^{\mu} = (E,\, 0,\, 0,\, 0)\). In this case, the Dirac equation \eqref{eq:dirac_equation_covariant} with the plane wave ansatz reduces to
\[
    \begin{aligned}
        \left(\gamma^0 p_0 + m \right) \omega(p) e^{-i p_0 x^0} = 0, \\
        \left(-i (-i \beta) E + m\right) \omega(p) e^{-i Et} = 0,    \\
        \beta E \omega(p) = m \omega(p),
    \end{aligned}
\]
and by multiplting by beta (\(\beta^2 = \mathbb{I}\)) we get
\[
    E \omega(p) = \beta m \omega(P) \implies E \omega(p) = \begin{pmatrix}
        m & 0 & 0  & 0  \\
        0 & m & 0  & 0  \\
        0 & 0 & -m & 0  \\
        0 & 0 & 0  & -m
    \end{pmatrix}
\]
which are the four solutions of the Dirac equation for a particle at rest. The first two solutions correspond to positive energy, while the last two correspond to negative energy. The spinor \(\omega(p)\) is thus a four-component object, with two independent components for each value of the energy:
\[
    \begin{aligned}
        E = m ) \quad \psi_1(x)  & = \begin{pmatrix}
                                         1 \\
                                         0 \\
                                         0 \\
                                         0
                                     \end{pmatrix} e^{-imt}, \quad &  & \psi_2(x) = \begin{pmatrix}
                                                                                        0 \\
                                                                                        1 \\
                                                                                        0 \\
                                                                                        0
                                                                                    \end{pmatrix} e^{-imt}, \\
        E = -m ) \quad \psi_3(x) & = \begin{pmatrix}
                                         0 \\
                                         0 \\
                                         1 \\
                                         0
                                     \end{pmatrix} e^{imt}, \quad  &  & \psi_4(x)  = \begin{pmatrix}
                                                                                         0 \\
                                                                                         0 \\
                                                                                         0 \\
                                                                                         1
                                                                                     \end{pmatrix} e^{imt}.
    \end{aligned}
\]
The general case with arbitrary momentum can be derived with similar calculations. Alternatively, they can be obtained from a Lorentz transformation applied to the solution above. To use the last method, it is necessary to study explicitly the covariance of the Dirac equation, which we postpone for a while. At this stage, Dirac seemed to have solved the issue regarding the probabilistic interpretation of relativistic quantum mechanics, but was confronted with the existence of apparently unphysical negative energy solutions. Nevertheless, he continued to explore the consequences of his equation, starting by looking at its non-relativistic limit.

\subsection{Pauli Equation: Non-Relativistic Limit}

To study the non-relativistic limit of the Dirac equation, we reinsert \(\hbar\) and \(c\). It is convenient to use the hamiltonian form
\[
    i \hbar \partial_t \psi(\mathbf{x},\,t) = \left( c \bs{\alpha} \cdot \mathbf{p} + \beta m c^2 \right) \psi(\mathbf{x},\,t),
\]
where \(\mathbf{p} = -i \hbar \nabla\). We separate the rest energy from the total energy by writing the spinor as
\begin{equation}
    \psi(\mathbf{x},\,t) = e^{-\tfrac{i}{\hbar} m c^2 t} \begin{pmatrix}
        \varphi(\mathbf{x},\,t) \\
        \chi(\mathbf{x},\,t)
    \end{pmatrix},
    \label{eq:dirac_spinor_split}
\end{equation}
factoring out an expected time dependence due to the rest energy \(E = mc^2\), and splitting the Dirac spinor into two two-component spinors \(\varphi\) and \(\chi\). Inserting this expression into the Dirac equation (and remembering the expressions for \(\alpha^i\) and \(\beta\) in \eqref{eq:dirac_representation}), we obtain two coupled equations for \(\varphi\) and \(\chi\):
\[
    i \hbar \frac{\partial}{\partial t} e^{-\tfrac{i}{\hbar} m c^2 t} \begin{pmatrix}
        \varphi(\mathbf{x},\,t) \\
        \chi(\mathbf{x},\,t)
    \end{pmatrix} = \left(- i \hbar c \bs{\alpha} \cdot \nabla + mc^2 \beta\right)e^{-\tfrac{i}{\hbar} m c^2 t} \begin{pmatrix}
        \varphi(\mathbf{x},\,t) \\
        \chi(\mathbf{x},\,t)
    \end{pmatrix},
\]
whch computing terms individually gives
\[
    \begin{aligned}
        i \hbar\partial_t \left(e^{-\tfrac{i}{\hbar} m c^2 t} \begin{pmatrix}
                                                                  \varphi(\mathbf{x},\,t) \\
                                                                  \chi(\mathbf{x},\,t)
                                                              \end{pmatrix}\right)                     = i \hbar e^{-\tfrac{i}{\hbar} m c^2 t} \left[-\frac{i}{\hbar} m c^2 \begin{pmatrix}
                                                                                                                                                                                    \varphi(\mathbf{x},\,t) \\
                                                                                                                                                                                    \chi(\mathbf{x},\,t)
                                                                                                                                                                                \end{pmatrix} + \begin{pmatrix}
                                                                                                                                                                                                    \partial_t \varphi(\mathbf{x},\,t) \\
                                                                                                                                                                                                    \partial_t \chi(\mathbf{x},\,t)
                                                                                                                                                                                                \end{pmatrix}\right],  \\
        -i \hbar c \bs{\alpha} \cdot \nabla e^{-\tfrac{i}{\hbar} m c^2 t} \begin{pmatrix}
                                                                              \varphi(\mathbf{x},\,t) \\
                                                                              \chi(\mathbf{x},\,t)
                                                                          \end{pmatrix} = -i \hbar c\, e^{-\tfrac{i}{\hbar} m c^2 t} \begin{pmatrix}
                                                                                                                                         0                                             & \bs{\sigma}\cdot\nabla\chi(\mathbf{x},\,t) \\
                                                                                                                                         \bs{\sigma}\cdot\nabla\varphi(\mathbf{x},\,t) & 0
                                                                                                                                     \end{pmatrix}, \\
        mc^2 \beta e^{-\tfrac{i}{\hbar} m c^2 t} \begin{pmatrix}
                                                     \varphi(\mathbf{x},\,t) \\
                                                     \chi(\mathbf{x},\,t)
                                                 \end{pmatrix} = mc^2 e^{-\tfrac{i}{\hbar} m c^2 t} \begin{pmatrix}
                                                                                                        \varphi(\mathbf{x},\,t) \\
                                                                                                        -\chi(\mathbf{x},\,t)
                                                                                                    \end{pmatrix}.
    \end{aligned}
\]
Finally we get
\[
    \left[-\frac{i}{\hbar} m c^2 \begin{pmatrix}
            \varphi(\mathbf{x},\,t) \\
            \chi(\mathbf{x},\,t)
        \end{pmatrix} + \begin{pmatrix}
            \partial_t \varphi(\mathbf{x},\,t) \\
            \partial_t \chi(\mathbf{x},\,t)
        \end{pmatrix}\right] = - \begin{pmatrix}
        0                                             & \bs{\sigma}\cdot\nabla\chi(\mathbf{x},\,t) \\
        \bs{\sigma}\cdot\nabla\varphi(\mathbf{x},\,t) & 0
    \end{pmatrix} -\frac{i}{\hbar} mc^2 \begin{pmatrix}
        \varphi(\mathbf{x},\,t) \\
        -\chi(\mathbf{x},\,t)
    \end{pmatrix}, \\
\]
which can be separated into the two equations
\[
    \begin{aligned}
        i \hbar \frac{\partial}{\partial t} \varphi(\mathbf{x},\,t) & = c \bs{\sigma} \cdot \mathbf{p} \chi(\mathbf{x},\,t),                                   \\
        i \hbar \frac{\partial}{\partial t} \chi(\mathbf{x},\,t)    & = c \bs{\sigma} \cdot \mathbf{p} \varphi(\mathbf{x},\,t) - 2 m c^2 \chi(\mathbf{x},\,t).
    \end{aligned}
\]
If we now take the non relativistic limit, where the kinetic energy is much smaller than the rest energy, we can neglect the time derivative in the second equation, obtaining
\[
    \chi(\mathbf{x},\,t) \simeq \frac{\bs{\sigma} \cdot \mathbf{p}}{2 m c} \varphi(\mathbf{x},\,t),
\]
which can be inserted into the first equation to obtain
\[
    i \hbar \frac{\partial}{\partial t} \varphi(\mathbf{x},\,t) = \frac{(\bs{\sigma} \cdot \mathbf{p})^2}{2 m} \varphi(\mathbf{x},\,t).
\]
Using the identity
\[
    (\bs{\sigma} \cdot \mathbf{p})^2 = \mathbf{p}^2 + i \epsilon^{ijk} \sigma^k p^i p^j = \mathbf{p}^2,
\]
from the algebra of the Pauli matrices \(\sigma^i \sigma^j = \delta^{ij} + i \epsilon^{ijk} \sigma^k\) and the null product of the symmetric momenta operators and the antisymmetric Levi-Civita symbol, we finally obtain the \textbf{Pauli equation} for a free particle:
\begin{equation}
    i \hbar \frac{\partial}{\partial t} \varphi(\mathbf{x},\,t) = \frac{\mathbf{p}^2}{2 m} \varphi(\mathbf{x},\,t).
    \label{eq:pauli_equation_free_particle}
\end{equation}
It is similar to the Schrödinger equation, but the wave function \(\varphi(\mathbf{x},\,t)\) is a two-component spinor, describing the two possible spin states of a spin-\(\tfrac{1}{2}\) particle.

\subsubsection{Electromagnetic Coupling}

This analysis can be extended to include the interaction with an external electromagnetic field, introduced via \textbf{minimal coupling}:
\[
    p^{\mu} \to \pi^{\mu} = p^{\mu} - \frac{e}{c} A^{\mu},
\]
where \(A^{\mu} = (\phi,\, \mathbf{A})\) is the four-potential of the electromagnetic field. The Hamiltonian in presence of an external electromagnetic field reads
\[
    H = \frac{\pi^2}{2m} = \frac{1}{2m}\left(p - \frac{e}{c}A\right)^2,
\]
thus we are substituting \(E \to E - e\phi\) and \(\mathbf{p} \to \mathbf{p} - \frac{e}{c} \mathbf{A}\).
Starting from the Dirac equation in \eqref{eq:dirac_equation_alpha_beta}\footnote{We can invert \(\mathbf{p} \cdot \bs{\alpha} \to \bs{\alpha}\cdot \mathbf{p}\) since they act on different spaces, hence they commute.} making the minimal substitution, we have
\[
    E - e\phi = c \bs{\alpha} \cdot \bs{\pi} + \beta m c^2 \implies E = c \bs{\alpha} \cdot \bs{\pi} + \beta m c^2 + e \phi,
\]
thus the Dirac equation in presence of an external electromagnetic field reads
\[
    i \hbar \frac{\partial}{\partial t} \psi(\mathbf{x},\,t) = \left(c \bs{\alpha} \cdot \bs{\pi} + \beta m c^2 + e \phi \right) \psi(\mathbf{x},\,t).
\]
If we repeat the previous analysis for the non-relativistic limit, splitting the spinor and factoring out the rest energy, we obtain
\[
    i \hbar \frac{\partial}{\partial t} \begin{pmatrix}
        \varphi(\mathbf{x},\,t) \\
        \chi(\mathbf{x},\,t)
    \end{pmatrix} + mc^2 \begin{pmatrix}
        \varphi(\mathbf{x},\,t) \\
        \chi(\mathbf{x},\,t)
    \end{pmatrix} = c \bs{\sigma} \cdot \bs{\pi} \begin{pmatrix}
        \chi(\mathbf{x},\,t) \\
        \varphi(\mathbf{x},\,t)
    \end{pmatrix} + mc^2 \begin{pmatrix}
        \varphi(\mathbf{x},\,t) \\
        -\chi(\mathbf{x},\,t)
    \end{pmatrix} + e \phi \begin{pmatrix}
        \varphi(\mathbf{x},\,t) \\
        \chi(\mathbf{x},\,t)
    \end{pmatrix}.
\]
This leads to the coupled equations
\[
    \begin{aligned}
        i \hbar \frac{\partial}{\partial t} \varphi(\mathbf{x},\,t) & = c \bs{\sigma} \cdot \bs{\pi} \chi(\mathbf{x},\,t) + e \phi \varphi(\mathbf{x},\,t),                                \\
        i \hbar \frac{\partial}{\partial t} \chi(\mathbf{x},\,t)    & = c \bs{\sigma} \cdot \bs{\pi} \varphi(\mathbf{x},\,t) - 2 m c^2 \chi(\mathbf{x},\,t) + e \phi \chi(\mathbf{x},\,t).
    \end{aligned}
\]
In the non-relativistic limit, we can neglect again the time derivative in the second equation, obtaining
\[
    \chi(\mathbf{x},\,t) \simeq \frac{\bs{\sigma} \cdot \bs{\pi}}{2 m c} \varphi(\mathbf{x},\,t),
\]
which can be inserted into the first equation to obtain
\[
    i \hbar \frac{\partial}{\partial t} \varphi(\mathbf{x},\,t) = \left(\frac{(\bs{\sigma} \cdot \bs{\pi})^2}{2 m} + e \phi \right) \varphi(\mathbf{x},\,t).
\]
Using the algebra of the Pauli matrices, we can expand \((\bs{\sigma} \cdot \bs{\pi})^2\) as
\[
    \begin{aligned}
        (\bs{\sigma} \cdot \bs{\pi})^2 & = \sigma^i \sigma^j \pi_i \pi_j = \left(\delta^{ij} + i \epsilon^{ijk} \sigma_k\right) \pi_i \pi_j                       \\
                                       & = \pi^2 + i \epsilon^{ijk} \sigma_k \pi_i \pi_j = \pi^2 + \frac{i}{2} \epsilon^{ijk} \sigma_k \left[\pi_i, \pi_j\right],
    \end{aligned}
\]\QUESTION{Is the index of sigma k lower? i think so but professor used upper index.}
since only the antisymmetric part of \(\pi_i \pi_j\) contributes to the last term. The commutator of the kinetic momenta can be computed as
\[
    \left[\pi_i, \pi_j\right] = \left[p_i - \frac{e}{c} A_i, p_j - \frac{e}{c} A_j\right] = -\frac{e}{c} \left([p_i, A_j] - [p_j, A_i]\right),
\]
which can be applied to a test function \(f(\mathbf{x})\) to obtain
\[
    \left[p_i,\,A_j\right] f(\mathbf{x}) = -i \hbar \partial_i (A_j f(\mathbf{x})) + A_j i \hbar \partial_i f(\mathbf{x}) = -i \hbar (\partial_i A_j) f(\mathbf{x}),
\]
so that the commutator \(\left[p_i,\,A_j\right] = -i \hbar \partial_i A_j\) and we can continue the computation of \(\left[\pi_i, \pi_j\right]\) as:
\[
    \left[\pi_i, \pi_j\right] = -\frac{e}{c} \left([p_i, A_j] - [p_j, A_i]\right) =  \frac{e i \hbar}{c} \left(\partial_i A_j - \partial_j A_i\right).
\]
If we now insert this commutator into the previous expression for \((\bs{\sigma} \cdot \bs{\pi})^2\), we obtain
\[
    (\bs{\sigma}\cdot \bs{\pi})^2 = \bs{\pi}^2 - \frac{e \hbar}{2 c} \sigma_k \epsilon^{ijk} (\partial_i A_j - \partial_j A_i),
\]
where we can recgnize the electromagnetic strength tensor components \(F_{ij} = \partial_i A_j - \partial_j A_i\), which can be related to the magnetic field as \(B^k = \frac{1}{2} \epsilon^{ijk} F_{ij}\), indeed
\[
    \epsilon^{ijk} F_{ij} = \epsilon^{ijk} (\partial_i A_j - \partial_j A_i) = B^k - \epsilon^{jik} \partial_j A_i = = B^k + \epsilon^{ijk} \partial_i A_j = 2 B^k.
\]
Thus, we finally obtain
\[
    (\bs{\sigma}\cdot \bs{\pi})^2 = \bs{\pi}^2 - \frac{e \hbar}{c} \bs{\sigma} \cdot \mathbf{B},
\]
and recognizing the \textbf{Pauli spin operator} \(\mathbf{S} = \frac{\hbar}{2} \bs{\sigma}\), we can write the non-relativistic limit of the Dirac equation (Pauli equation) in presence of an external electromagnetic field as
\begin{equation}
    i \hbar \frac{\partial}{\partial t} \varphi(\mathbf{x},\,t) = \left[\frac{\bs{\pi}^2}{2 m} + e \phi - \frac{e}{m c} \mathbf{S} \cdot \mathbf{B}\right] \varphi(\mathbf{x},\,t).
    \label{eq:pauli_equation_em}
\end{equation}

\paragraph{Gyromagnetic factor.}
The last term in the Pauli equation \eqref{eq:pauli_equation_em} describes the interaction between the intrinsic magnetic moment of the particle and the external magnetic field. The magnetic moment operator is defined as
\[
    \boldsymbol{\mu} = \frac{g e}{2 m c} \mathbf{S},
\]
where \(g\) is the \textbf{gyromagnetic factor}. Comparing with the Pauli equation, we find that the gyromagnetic factor for a Dirac particle is \(g = 2\). This result was one of the first triumphs of the Dirac equation, as it correctly predicted the gyromagnetic factor of the electron, which had been measured experimentally to be very close to 2.

Classically, when we have electromagnetic coupling, the interacting term of the Hamiltonian is
\[
    H = - \bs{\mu} \cdot \mathbf{B},
\]
where \(\bs{\mu} = \frac{g e}{2 m c} \mathbf{L}\) is the magnetic moment of the particle with orbital angular momentum \(\mathbf{L}\) and \(g\) expected to be 1. We can see that the intrinsic spin of the electron gives rise to a magnetic moment twice as large as that expected from classical considerations, but we can concile the results by noting that the spin is not a classical angular momentum; considering a constant magnetic field \(\mathbf{B} = B \hat{z}\), the vector potential can be chosen as \(\mathbf{A} = \frac{1}{2} \mathbf{B} \times \mathbf{r}\), so that the squared kinetic momentum reads
\[
    \begin{aligned}
        \bs{\pi}^2 & = \left(\mathbf{p} - \frac{e}{c} \mathbf{A}\right)^2 = \mathbf{p}^2 - \frac{e}{c} (\mathbf{p} \cdot \mathbf{A} + \mathbf{A} \cdot \mathbf{p}) + \frac{e^2}{c^2} \mathbf{A}^2,                                    \\
                   & = \mathbf{p}^2 + \frac{e^2}{c^2} \mathbf{A}^2 - \frac{e}{c} \left\{ \mathbf{p},\, \mathbf{A} \right\} = \mathbf{p}^2 + \frac{e^2}{c^2} \mathbf{A}^2 - \frac{2e}{c} \mathbf{p} \cdot \mathbf{A}                   \\
                   & = \mathbf{p}^2 + \frac{e^2}{c^2} \mathbf{A}^2 - \frac{e}{ c} \mathbf{B} \cdot \left(\mathbf{r} \times \mathbf{p}\right) = \mathbf{p}^2 + \frac{e^2}{c^2} \mathbf{A}^2 - \frac{e}{c} \mathbf{B} \cdot \mathbf{L},
    \end{aligned}
\]
where we have used the fact that \(\nabla \cdot \mathbf{A} = 0\) for this choice of gauge, so that \(\mathbf{p} \cdot \mathbf{A} = \mathbf{A} \cdot \mathbf{p}\) and in the end the identity of the triple product \(\mathbf{a} \cdot (\mathbf{b} \times \mathbf{c}) = \mathbf{b} \cdot (\mathbf{c} \times \mathbf{a})\). Thus, inserting this back into the Pauli equation \eqref{eq:pauli_equation_em} gives
\[
    i \hbar \frac{\partial}{\partial t} \varphi(\mathbf{x},\,t) = \left[\frac{\mathbf{p}^2}{2 m} + \frac{e^2}{2 m c^2} \mathbf{A}^2 + e \phi - \frac{e}{2 m c} (\mathbf{L} + 2 \mathbf{S}) \cdot \mathbf{B}\right] \varphi(\mathbf{x},\,t),
\]
where we can see that both the orbital angular momentum and the intrinsic spin contribute to the magnetic moment, with gyromagnetic factors \(g_L = 1\) and \(g_S = 2\) respectively.

\subsection{Angular Momentum and Spin}

As seen from the non-relativistic limit, the Dirac spinor describes a particle of spin \(\tfrac12\), such as the electron. The spin operator acts on the two components of the wave function \(\psi\), and is proportional to the Pauli matrices \(\mathbf{S} = \frac{\hbar}{2}\bs{\sigma}\). This suggests that the full spin operator acting on the four-component Dirac spinor is given (in natural units) in the Dirac representation by
\[
    \mathbf{S} = \frac{1}{2} \bm{\Sigma} = \frac{1}{2} \begin{pmatrix}
        \bs{\sigma} & 0           \\
        0           & \bs{\sigma}
    \end{pmatrix}, \text{ acting on } \psi = \begin{pmatrix}
        \varphi \\
        \chi
    \end{pmatrix}.
\]
The \(\Sigma^i\) matrices can be expressed in terms of the Pauli matrices as
\[
    \Sigma^i = -\frac{i}{2} \epsilon^{ijk} \alpha^j \alpha^k, \quad \text{where} \quad \alpha^i = \begin{pmatrix}
        0        & \sigma^i \\
        \sigma^i & 0
    \end{pmatrix}.
\]
Indeed we can compute \(\Sigma^1\) as
\[
    \Sigma^1 = -\frac{i}{2} \left(\epsilon^{123} \alpha^2 \alpha^3 + \epsilon^{132} \alpha^3 \alpha^2\right) = -\frac{i}{2} \left(\alpha^2 \alpha^3 - \alpha^3 \alpha^2\right),
\]
since \(\epsilon^{123} = 1\) and \(\epsilon^{132} = -1\). Now we can use the fact that the \(\alpha^i\) matrices anticommute, \(\{\alpha^i,\, \alpha^j\} = 0\), to write
\[
    \Sigma^1 = -\frac{i}{2} \left(2 \alpha^2 \alpha^3\right) = -i \alpha^2 \alpha^3 = \begin{pmatrix}
        \sigma^1 & 0        \\
        0        & \sigma^1
    \end{pmatrix},
\]
since from the Pauli algebra \(\sigma^2 \sigma^3 = i \sigma^1\).

The orbital angular momentum operator is defined as usual as the operatoral version of the classical expression
\[
    \mathbf{L} = \mathbf{r}\times \mathbf{p} \longrightarrow L^i = \epsilon^{ijk} x^J p^k,
\]
while the total angular momentum operator is given by
\[
    \mathbf{J} = \mathbf{L} + \mathbf{S} \quad \Longrightarrow \quad J^i = L^i + S^i,
\]
which is conserved under rotational symmetry. It is possible to verify that the total angular momentum operator commutes with the Dirac Hamiltonian (appearing in r.h.s. of equation \eqref{eq:dirac_equation_hamiltonian}) in the free partile case:
\[
    \left[H_D,\, J^i\right] = 0, \quad \text{where} \quad H_D = p^i \alpha^i + \beta m.
\]
This can be shown by computing the commutators \(\left[H_D,\, L^i\right]\) and \(\left[H_D,\, S^i\right]\) separately, using the commutation relations of the angular momentum operators and the properties of the \(\alpha^i\) and \(\beta\) matrices. We will start from the orbital part:
\[
    \begin{aligned}
        \left[H_D,\, L^i\right] & = \left[\alpha^l p^l + \beta m,\, \epsilon^{ijk} x^j p^k\right] = \alpha^l \left[p^l,\, x^j \right] \epsilon^{ijk} p^k + \left[\beta m,\, \epsilon^{ijk} x^j p^k\right], \\
                                & = \alpha^l \left(-i \delta^{lj}\right) \epsilon^{ijk} p^k = -i \epsilon^{ijk} \alpha^j p^k,
    \end{aligned}
\]
since in the first step we used the null commutators \(\left[p^l,\, p^k\right] = 0\) and \(\left[\beta m,\, x^j\right] = 0 = \left[\beta m,\, p^k\right]\). Now we can compute the spin part:
\[
    \left[H_D,\, S^i\right] = \left[\alpha^l p^l + \beta m,\, -\tfrac{i}{4} \epsilon^{ijk} \alpha^j \alpha^k\right] = - \frac{i}{4} \epsilon^{ijk} p^l \left[\alpha^l,\, \alpha^j \alpha^k\right] - \frac{i}{4} \epsilon^{ijk} \left[\beta m,\, \alpha^j \alpha^k\right],
\]
where the second term is zero since \(\beta\) commutes with the \(\alpha^i\) matrices and the first term is the only commutator to compute since \(p^i\) commutes with \(\alpha^j\). Using the expression \(\left[A,\,BC\right] = \left\{A,\,B\right\}C - B\left\{A,\,C\right\}\) we can write
\[
    \begin{aligned}
        \left[H_D,\, S^i\right] & = - \frac{i}{4} \epsilon^{ijk} p^l \left( \left\{\alpha^l,\,\alpha^j\right\}\alpha^k - \alpha^j \left\{\alpha^l,\, \alpha^k\right\} \right) \\
                                & = - \frac{i}{4} \epsilon^{ijk} p^l \left( 2 \delta^{lj} \alpha^k - 2 \delta^{lk} \alpha^j \right)                                           \\
                                & = - \frac{i}{2} \epsilon^{ijk} \left(p^j \alpha^k - p^k \alpha^j\right) = + i \epsilon^{ijk} \alpha^j p^k,
    \end{aligned}
\]
since the two terms are antisymmetric in \(j\) and \(k\). Finally, summing the two contributions we obtain
\[
    \left[H_D,\, J^i\right] = \left[H_D,\, L^i\right] + \left[H_D,\, S^i\right] = -i \epsilon^{ijk} \alpha^j p^k + i \epsilon^{ijk} \alpha^j p^k = 0,
\]
showing that the total angular momentum is conserved in the Dirac theory.

\subsection{Idrogen Atom and Dirac Equation}

A crucial test for the Dirac equation was to check its predictions for the quantized energy levels of the hydrogen atom. The problem is exactly solvable. Nevertheless, it is illuminating to study perturbatively the solution for the energy levels and compare it with the non-relativistic solution of the Schrödinger equation. The energies obtained from the Schrödinger, Klein-Gordon, and Dirac equations are
\[
    \begin{aligned}
        E^{(S)}_{n,\,l}  & = - \frac{m_e \alpha^2}{2 n^2}, \quad \alpha = \frac{e^2}{4 \pi \epsilon_0 \hbar c} \simeq \frac{1}{137},                       \\
        E^{(KG)}_{n,\,l} & = m_e \left[1 - \frac{\alpha^2}{2n^2} - \frac{\alpha^4}{n^4} \left( \frac{n}{2l + 1} -\frac{3}{8} \right) + O(\alpha^6)\right], \\
        E^{(D)}_{n,\,j}  & = m_e \left[1 - \frac{\alpha^2}{2n^2} - \frac{\alpha^4}{n^4} \left( \frac{n}{2j + 1} -\frac{3}{8} \right) + O(\alpha^6)\right],
    \end{aligned}
\]
where \(n = 1,\,2,\,\dots \infty\) is the principal quantum number, \(l>0\) is the orbital angular momentum quantum number, and \(j = l \pm \tfrac{1}{2}\) is the total angular momentum quantum number. The Schrödinger non-relativistic result has the main quantum number \(n\) and degeneration in \(l = 0,\,1,\,\dots n-1\) (as \(n-l\) must be a strictly positive integer, \(n-l > 0\)). The degeneration\footnote{There is an additional degeneration in the magnetic quantum number \(m\), common to all three cases.} in \(l\) is broken by relativistic effects (“fine structure” effects), but the Klein-Gordon prediction is in contradiction with the experimental results (seen in the Pashen spectroscopic series, the spectral lines in the infrared due to transitions to level \(n=3\) from higher levels): \(2l + 1\) is an odd integer, but that number is experimentally measured to be even. The prediction from the Dirac equation gives instead a result compatible with experiments since now \(2j + 1\) is even.\footnote{Additional effects exist but are smaller. The most important ones are the hyperfine structure, due to the interaction of the electron with the magnetic moment of the nucleus, and the “Lamb shift”, which breaks the degeneracy in \(j\), due to quantum corrections obtainable by using the Dirac field as a QFT.}

\subsection{Properties of Gamma Matrices}

The original Dirac equation was written in terms of the \(\alpha^i\) and \(\beta\) matrices, but it is often more convenient to use the set of four gamma matrices defined as
\[
    \gamma^0 = -i \beta, \quad \gamma^i = -i \beta \alpha^i,
\]
which are \(4\times 4\) traceless matrices\footnote{We can reconstruct \(\alpha^i\) with \(\alpha^i = \gamma^i \gamma^0 = -i \beta \alpha^i (-i \beta) = - \beta \alpha^i \beta = \beta^2 \alpha^i = \alpha^i\).} satisfying the \textbf{Clifford algebra}
\[
    \left\{\gamma^{\mu},\, \gamma^{\nu}\right\} = 2 \eta^{\mu \nu},
\]
with
\[
    \begin{aligned}
        \left(\gamma^0\right)^{\dagger} = - \gamma^0 \textit{ antihermitian}, \\
        \left(\gamma^i\right)^{\dagger} = + \gamma^i \textit{ hermitian}.
    \end{aligned}
\]
We can compact the last two properties as
\[
    \left(\gamma^{\mu}\right)^{\dagger} = \gamma^0 \gamma^{\mu} \gamma^0 = -\beta \gamma^{\mu} \beta,
\]
which can be verified separately for \(\mu = 0\) and \(\mu = i\).

We can now prove that they are traceless, \(\mathrm{Tr}(\gamma^{\mu}) = 0\). We will prove it only for \(\gamma^1\), the other cases being similar. Using the anticommutation relations we have \((\gamma^{\mu})^2 = \mathbb{I}\) for \(\mu = 0,\,1,\,2,\,3\). Thus,
\[
    \mathrm{Tr}(\gamma^1) = \mathrm{Tr}(\gamma^1 \gamma^2 \gamma^2) = \mathrm{Tr}(\gamma^2 \gamma^1 \gamma^2),
\]
where in the last step we used the cyclicity of the trace. If we instead use the anticommutation relation \(\{\gamma^1,\, \gamma^2\} = 0\), we have
\[
    \Tr(\gamma^2 \gamma^1 \gamma^2) = - \Tr(\gamma^1 \gamma^2 \gamma^2) = - \Tr(\gamma^1),
\]
so that \(\Tr(\gamma^1) = - \Tr(\gamma^1) \implies \Tr(\gamma^1) = 0\).

\paragraph{Chirality matrix.}
We have to introduce one last matrix, the \textbf{chirality matrix} or \textbf{gamma five}:
\begin{equation}
    \gamma^5 = i \gamma^0 \gamma^1 \gamma^2 \gamma^3,
    \label{eq:chirality_matrix}
\end{equation}
which has the properties
\[
    \begin{aligned}
        \left(\gamma^5\right)^{\dagger}         & = \gamma^5       &  & \text{ hermiticity},                        \\
        \left(\gamma^5\right)^2                 & = \mathbb{I}     &  & \text{ idempotency},                        \\
        \left\{\gamma^5,\, \gamma^{\mu}\right\} & = 0, \forall \mu &  & \text{ anticommutation with } \gamma^{\mu}, \\
        \Tr(\gamma^5)                           & = 0              &  & \text{ tracelessness}.
    \end{aligned}
\]
In the Dirac representation, the chirality matrix takes the form
\[
    \gamma^5 = \begin{pmatrix}
        0           & -\mathbb{I} \\
        -\mathbb{I} & 0
    \end{pmatrix}.
\]

\begin{example}
    Let's verify the anticommutation relation between \(\gamma^5\) and \(\gamma^{3}\):
    \[
        \begin{aligned}
            \left\{\gamma^5, \gamma^{3}\right\} & = i \left(\gamma^0 \gamma^1 \gamma^2 \gamma^3 \gamma^3 + \gamma^3 \gamma^0 \gamma^1 \gamma^2 \gamma^3\right) \\
                                                & = i \left(\gamma^0 \gamma^1 \gamma^2 + (-\gamma^0 \gamma^1 \gamma^2)\right) = 0,
        \end{aligned}
    \]
    since \((\gamma^3)^2 = \mathbb{I}\) and \(\gamma^3\) anticommutes with \(\gamma^0\), \(\gamma^1\), and \(\gamma^2\), so we changed the sign three times in the second term.
\end{example}

\paragraph{Chiral projectors.}
We can use the chirality matrix to define the \textbf{chiral projectors}
\begin{equation}
    P_{R} = \frac{1 + \gamma^5}{2}, \quad P_{L} = \frac{1 - \gamma^5}{2},
    \label{eq:chiral_projectors}
\end{equation}
which satisfy the properties of projectors:
\[
    P_R + P_L = \mathbb{I}, \quad P_L^2 = P_L, \quad P_R^2 = P_R.
\]
These projectors can be used to decompose a Dirac spinor into its right-handed and left-handed components:
\[
    \psi = \psi_R + \psi_L, \quad \text{where} \quad \psi_R = P_R \psi, \quad \psi_L = P_L \psi.
\]
These two components have definite chirality, with \(\gamma^5 \psi_R = + \psi_R\) and \(\gamma^5 \psi_L = - \psi_L\), and they live in two-dimensional subspaces of the four-dimensional spinor space: they are \textbf{Weyl spinors}, which transform under the \((\tfrac{1}{2},\,0)\) and \((0,\,\tfrac{1}{2})\) representations of the Lorentz group independently:
\[
    \psi \in \left(\tfrac{1}{2},\,0\right) \oplus \left(0,\,\tfrac{1}{2}\right) \quad \Longrightarrow \quad \psi_R \in \left(\tfrac{1}{2},\,0\right), \quad \psi_L \in \left(0,\,\tfrac{1}{2}\right).
\]

Gamma matrices are four dimensional operators acting on Dirac spinors, which are elements of a four-dimensional complex vector space \(\psi \in \mathbb{C}^4\). They provide a representation of the Clifford algebra associated with the Minkowski spacetime metric, and they are essential in formulating the Dirac equation and describing the behavior of spin-\(\tfrac{1}{2}\) particles in relativistic quantum mechanics.

It is useful to introduce a complete basis for these linear operators acting on \(\mathbb{C}^4\). Such a basis is given by the set of \(4 \times 4 = 16\) linearly independent matrices:
\[
    \left\{ \mathbb{I},\, \gamma^\mu,\, \Sigma^{\mu \nu}, \gamma^{\mu}\gamma^{5},\, \gamma^{5} \right\},
\]
where \(\Sigma^{\mu \nu} = \frac{-i}{4} [\gamma^\mu, \gamma^\nu]\) (with \(\mu<\nu\)) and we can count \(1 + 4 + 6 + 4 + 1 = 16\) independent matrices. We can also generalize this complete basis to span any even dimensional space of gamma matrices, i.e. \(d = 2k\) with \(k \in \mathbb{N}\). In this case, the dimension of the space of linear operators acting on spinors is \(2^{2k} = 4^k\), and a complete basis is given by
\[
    \left\{ \mathbb{I},\, \gamma^{\mu},\, \gamma^{\mu \nu},\, \gamma^{\mu \nu \rho},\, \dots,\, \gamma^{\mu_1 \mu_2 \dots \mu_{2k}} \right\}, \quad \gamma^{\mu_1 \mu_2 \dots \mu_n} = \frac{1}{n!} \sum_{\sigma \in S_n} \text{sgn}(\sigma) \gamma^{\mu_{\sigma(1)}} \gamma^{\mu_{\sigma(2)}} \dots \gamma^{\mu_{\sigma(n)}},
\]
where the totally antisymmetric products of gamma matrices are defined as above, summing over all permutations \(\sigma\) of \(n\) indices with the appropriate sign.

\begin{remark}
    \(\gamma^{\mu_1 \mu_2 \dots \mu_n}\), the totally antisymmetric product of \(n\) gamma matrices, works in odd dimensions as well, but in that case the set of gamma matrices is not complete, as there is an additional matrix (the analog of \(\gamma^5\) in four dimensions) that commutes with all gamma matrices and can be used to construct additional linearly independent matrices. There is then degeneracy in the representation of the Clifford algebra in odd dimensions. Since we cannot define chirality in odd dimensions, this means that Weyl spinors cannot be defined in odd dimensions.
\end{remark}

\subsection{Covariance Formulation}

The Dirac equation, derived from relativistic considerations, is consistent with relativistic invariance. To prove it explicitly, it is necessary to show that the equation is invariant in form under a change of inertial frame of reference as generated by a proper and orthochronous Lorentz transformation.

Recall that by Lorentz invariance, one generically refers to the transformations that are continuously connected to the identity, leaving out the discrete transformations of parity \(P\) and time reversal \(T\), which are treated separately. Thus, we need to construct the precise transformation of the Dirac spinor \(\psi(x)\) under a Lorentz transformation \(\Lambda\). One may conjecture it to be linear and of the form
\[
    \psi(x) \xrightarrow{\Lambda} \psi'(x') = S(\Lambda) \psi(x),
\]
so that a Lorentz transformation acts as
\[
    \begin{aligned}
        \left(\gamma^{\mu} \partial_\mu + m \right) = 0 & \quad \iff \quad \left(\gamma^{\mu} \partial_\mu^{\prime} + m \right) \psi^{\prime} (x^{\prime}) = 0, \\
        x^{\mu}                                         & \quad \iff \quad x^{\prime \mu} = \Lambda^{\mu}_{\ \nu} x^{\nu},                                      \\
        \partial_{\mu}                                  & \quad \iff \quad \partial_{\mu}^{\prime} = \left(\Lambda^{-1}\right)_{\mu}^{\ \nu} \partial_{\nu},    \\
        \psi(x)                                         & \quad \iff \quad \psi^{\prime}(x^{\prime}) = S(\Lambda) \psi(x).
    \end{aligned}
\]
Relating the second reference frame to the first one through the transformation of coordinates, we have
\[
    \left(\gamma^{\mu} \partial_\mu^{\prime} + m \right) \psi^{\prime} (x^{\prime}) = 0 \longrightarrow (\Lambda) \left(\gamma^{\mu} \Lambda_{\mu}^{\ \nu} \partial_\nu + m \right) S(\Lambda) \psi(x) = 0,
\]
which can be rewritten after multiplying by \(S^{-1}(\Lambda)\) from the left as
\[
    \left( S^{-1}(\Lambda) \gamma^{\mu} S(\Lambda) \Lambda_{\mu}^{\ \nu} \partial_\nu + m \right) \psi(x) = 0.
\]
Comparing with the original Dirac equation, we see that they are equivalent if \(S(\Lambda)\) satisfies the relation
\[
    S^{-1}(\Lambda) \gamma^{\mu} S(\Lambda) \Lambda_{\mu}^{\ \nu} = \gamma^{\nu},
\]
or equivalently, after multiplying with \(\Lambda^{\rho}_{\ \nu}\), observing that \(\Lambda^{\rho}_{\ \nu} \Lambda_{\mu}^{\ \nu} = \delta^\rho_{\ \mu}\), and renaming indices,\footnote{Note that \(\Lambda_\mu^{\ \nu}\) acts on vectors with lower indices: it is obtained by raising/lowering indices on \(\Lambda^{\mu}_{\ \nu}\) so that it corresponds to the matrix \(\eta \Lambda \eta^{-1} = (\Lambda^{-1})^T\). The last relation follows from the defining property \(\Lambda^T \eta \Lambda = \eta\). Then, one may check that \(\Lambda^{\rho}_{\ \nu} \Lambda_{\mu}^{\ \nu} = \left[\Lambda (\eta \Lambda \eta^{-1})^T\right]^{\rho}_{\ \mu} = \left[\Lambda \Lambda^{-1}\right]^{\rho}_{\ \mu} = \delta^{\rho}_{\ \mu}\).}
\begin{equation}
    S^{-1}(\Lambda) \gamma^{\mu} S(\Lambda) = \Lambda^{\mu}_{\ \nu} \gamma^{\nu}.
    \label{eq:lorentz_transformation_spinor}
\end{equation}
This equation defines the spinorial representation of the Lorentz group. We have to show that such a representation exists. Thus we have to consider infinitesimal Lorentz transformations of the form
\[
    \Lambda^{\mu}_{\ \nu} = \delta^{\mu}_{\ \nu} + \omega^{\mu}_{\ \nu}, \quad \text{with } \omega_{\mu \nu} = - \omega_{\nu \mu},
\]
where the antisymmetry of \(\omega_{\mu \nu}\) follows from the defining property of Lorentz transformations \(\Lambda^T \eta \Lambda = \eta\). We can write the infinitesimal transformation of \(S(\Lambda)\) as
\[
    S(\Lambda) = \mathbb{I} + \frac{i}{2} \omega_{\mu \nu} \Sigma^{\mu \nu}, \quad \text{with } \Sigma^{\mu \nu} = \frac{-i}{4} [\gamma^{\mu},\, \gamma^{\nu}] = -\Sigma^{\nu \mu},
\]
where \(\Sigma^{\mu \nu}\) are the six independent generators of the spinorial representation of the Lorentz group. Now we can plug these infinitesimal transformations into equation \eqref{eq:lorentz_transformation_spinor} and verify that it holds to first order in \(\omega\):
\[
    \begin{aligned}
        S^{-1}(\Lambda) \gamma^{\rho} S(\Lambda)                                                                                                                                & = \Lambda^{\rho}_{\ \sigma} \gamma^{\sigma},                                                                     \\
        \left(\mathbb{I} - \frac{i}{2} \omega_{\mu \nu} \Sigma^{\mu \nu}\right) \gamma^{\rho} \left(\mathbb{I} + \frac{i}{2} \omega_{\alpha \beta} \Sigma^{\alpha \beta}\right) & = \left(\delta^{\rho}_{\ \sigma} + \omega^{\rho}_{\ \sigma} \right) \gamma^{\sigma},                             \\
        \frac{i}{2} \omega_{\alpha \beta} [\gamma^{\rho},\, \Sigma^{\alpha \beta}] + O(\omega^2)                                                                                & = \eta^{\mu \rho} \omega_{\mu \sigma} \gamma^{\sigma} = \omega_{\alpha \beta} \eta^{\alpha \rho} \gamma^{\beta}, \\
        \frac{i}{2} \omega _{\alpha \beta}[\gamma^{\rho},\, \Sigma^{\alpha \beta}]                                                                                              & = \omega_{\alpha \beta} \frac{\eta^{\alpha \rho} \gamma^{\beta} - \eta^{\beta \rho} \gamma^{\alpha}}{2},
    \end{aligned}
\]
where in the second step we expanded to first order in \(\omega\), in the second step we renamed (\(\mu \leftrightarrow \alpha\) and \(\sigma \leftrightarrow \beta\)) and rearranged indices and in the third step we used the antisymmetry of \(\omega_{\mu \nu}\) to rewrite the r.h.s. Finally, we have found that the relation \eqref{eq:lorentz_transformation_spinor} holds if the following commutation relation is satisfied:
\begin{equation}
    \left[ \Sigma^{\alpha \beta},\, \gamma^\rho \right] = i \left( \eta^{\alpha \rho} \gamma^{\beta} - \eta^{\beta \rho} \gamma^{\alpha} \right),
    \label{eq:commutation_Sigma_gamma}
\end{equation}
which can be verified using the definition of \(\Sigma^{\alpha \beta} = \frac{-i}{4} [\gamma^{\alpha},\, \gamma^{\beta}]\) in terms of gamma matrices and the Clifford algebra. Thus, we have shown that the spinorial representation of the Lorentz group exists.

To convince ourselves further, we may compute \eqref{eq:commutation_Sigma_gamma} explicitly for some values of the indices. For example, let us take \(\alpha = 1\), \(\beta = 2\), and \(\rho = 2\):
\[
    \begin{aligned}
        \left[ \Sigma^{12},\, \gamma^2 \right]                         & = i \left( \eta^{1 2} \gamma^{2} - \eta^{2 2} \gamma^{1} \right),                                                                                              \\
        \left[ \Sigma^{12},\, \gamma^2 \right]                         & = -\frac{i}{4} \left[ [\gamma^{1},\, \gamma^{2}],\, \gamma^2 \right] = -\frac{i}{4} \left[ \gamma^1 \gamma^2 - \gamma^2 \gamma^1,\, \gamma^2 \right]           \\
                                                                       & = -\frac{i}{4} \left( \gamma^1 \gamma^2 \gamma^2 - \gamma^2 \gamma^1 \gamma^2 - \gamma^2 \gamma^1 \gamma^2 + \gamma^2 \gamma^2 \gamma^1 \right) = -i \gamma^1, \\
        i \left( \eta^{1 2} \gamma^{2} - \eta^{2 2} \gamma^{1} \right) & = i \left( 0 - (+1) \gamma^1 \right) = -i \gamma^1,
    \end{aligned}
\]
as expected; for the other combinations of indices, the verification is similar. Finally, let us remark that this same result can be obtained with the use of the gamma matrix properties and Clifford algebra presented in the previous section:
\[
    \begin{aligned}
        \left[ \Sigma^{\mu \nu},\, \gamma^{\rho}\right] & = -\frac{i}{4} \left[ [\gamma^{\mu},\, \gamma^{\nu}],\, \gamma^{\rho} \right] = -\frac{i}{4} \left( \left[\gamma^{\mu} \gamma^{\nu},\, \gamma^{\rho}\right] - \left[\gamma^{\nu} \gamma^{\mu} ,\, \gamma^{\rho}\right]\right)                                           \\
                                                        & \qquad [AB,\, C] = A\{B,\, C\} - \{A,\, C\}B                                                                                                                                                                                                                            \\
                                                        & = -\frac{i}{4} \left( \gamma^{\mu} \left\{\gamma^{\nu},\,\gamma^{\rho}\right\} - \left\{\gamma^{\mu},\,\gamma^{\rho}\right\} \gamma^{\nu} - \gamma^{\nu} \left\{\gamma^{\mu},\,\gamma^{\rho}\right\} + \left\{\gamma^{\nu},\,\gamma^{\rho}\right\} \gamma^{\mu} \right) \\
                                                        & = -\frac{i}{4} \left( 2 \eta^{\nu \rho} \gamma^{\mu} - 2 \eta^{\mu \rho} \gamma^{\nu} - 2 \eta^{\mu \rho} \gamma^{\nu} + 2 \eta^{\nu \rho} \gamma^{\mu} \right)                                                                                                         \\
                                                        & = -\frac{i}{2} \left( 2 \eta^{\nu \rho} \gamma^{\mu} - 2 \eta^{\mu \rho} \gamma^{\nu} \right)                                                                                                                                                                           \\
                                                        & = i \left( \eta^{\mu \rho} \gamma^{\nu} - \eta^{\nu \rho} \gamma^{\mu} \right).
    \end{aligned}
\]

Now that we know that the spinorial representation of the Lorentz group exists, we can construct finite transformations by exponentiating the infinitesimal ones. Let us take the infinitesimal transformation of \(S(\Lambda)\)
\[
    S(\Lambda) = \mathbb{I} + \frac{i}{2} \omega_{\mu \nu} \Sigma^{\mu \nu},
\]
Thus, the finite Lorentz transformation of a Dirac spinor is given by repeated application of the infinitesimal transformation, leading to the exponential form
\begin{equation}
    S(\Lambda) = \exp{\frac{i}{2} \omega_{\mu \nu} \Sigma^{\mu \nu}} = \exp{\frac{1}{4} \omega_{\mu \nu} \gamma^{\mu} \gamma^{\nu}},
    \label{eq:finite_lorentz_transformation_spinor}
\end{equation}
from antisymmetry of \(\omega_{\mu \nu}\).

\begin{example}[Rotation around axis]
    Transformations with \(\omega_{12} = -\omega_{21} = \varphi\) and other vanishing produce rotations around the \(z\)-axis. If the parameter \(\varphi\) is finite then
    \[
        \omega^{\mu}_{\ \nu} = \begin{pmatrix}
            0 & 0        & 0       & 0 \\
            0 & 0        & \varphi & 0 \\
            0 & -\varphi & 0       & 0 \\
            0 & 0        & 0       & 0
        \end{pmatrix}, \implies \Lambda^{\mu}_{\ \nu} = (e^{\omega})^{\mu}_{\ \nu} = \begin{pmatrix}
            1 & 0              & 0             & 0 \\
            0 & \cos{\varphi}  & \sin{\varphi} & 0 \\
            0 & -\sin{\varphi} & \cos{\varphi} & 0 \\
            0 & 0              & 0             & 1
        \end{pmatrix}.
    \]
    We obtained this result by exponentiating the matrix \(\omega^{\mu}_{\ \nu}\): note that even powers of \(\omega\) have positive sign in the central diagonal block, while odd powers have alternating signs (\(e^{\omega} = e^{\varphi J}\) with \(J = \begin{pmatrix}
        0  & 1 \\
        -1 & 0
    \end{pmatrix}\) the generator of rotations around \(\hat{z}\) and \(J^2 = -\mathbb{I}\), \(J^3 = - J \) and finally \(J^4 = \mathbb{I}\)). If we take the Taylor expansion of the exponential we find
    \[
        \begin{aligned}
            e^{\varphi J} & = \sum_{n=0}^{\infty} \frac{(\varphi J)^n}{n!} = \mathbb{I} \left[1 + \frac{\varphi^2}{2!} J^2 + \frac{\varphi^4}{4!} J^4 + \dots \right] + J \left[\varphi + \frac{\varphi^3}{3!} J^2 + \frac{\varphi^5}{5!} J^4 + \dots \right] \\
                          & = \cos{\varphi} \mathbb{I} + \sin{\varphi} J.
        \end{aligned}
    \]
    Thus the result obtained by exponentiating the matrix of Lie parameters is the rotation matrix (even and odd powers in the central diagonal block of \(\omega^{\mu}_\nu\)).

    So we are looking at rotations around \(\hat{z}\). Let's find an expression for \(S(\Lambda)\): using the spinorial representation found in \eqref{eq:finite_lorentz_transformation_spinor} we have
    \[
        \begin{aligned}
            S(\Lambda) & = \exp{\frac{i}{2} \omega_{\mu \nu} \Sigma^{\mu \nu}} = \exp{\frac{1}{4} \omega_{\mu \nu} \gamma^{\mu} \gamma^{\nu}} = \exp{\frac{1}{4}\omega_{12}\gamma^1 \gamma^2 + \frac{1}{4}\omega_{21}\gamma^2 \gamma^1} \\
                       & = \exp{\frac{\varphi}{2} \gamma^1 \gamma^2} = \exp{\frac{\varphi}{2} \begin{pmatrix}
                                                                                                      \sigma^3 & 0        \\
                                                                                                      0        & \sigma^3
                                                                                                  \end{pmatrix}} = \begin{pmatrix}
                                                                                                                   e^{i \frac{\varphi}{2}} & 0                        & 0                       & 0                       \\
                                                                                                                   0                       & e^{i \frac{\varphi}{2} } & 0                       & 0                       \\
                                                                                                                   0                       & 0                        & e^{i \frac{\varphi}{2}} & 0                       \\
                                                                                                                   0                       & 0                        & 0                       & e^{i \frac{\varphi}{2}}
                                                                                                               \end{pmatrix},
        \end{aligned}
    \]
    where we used the Dirac representation of the gamma matrices and stopped the Taylor expansion of \(\exp{\begin{pmatrix}
            \sigma^3 & 0        \\
            0        & \sigma^3
        \end{pmatrix}}\) at the first order since \((\sigma^3)^2 = \mathbb{I}\).

    The transformation is immediately recognized to be unitary, \(S^{\dagger} (\Lambda) = S^{-1}(\Lambda)\). It is also clear that it is a spinorial transformation\footnote{The \(\omega\) matrices are a four vector representation, here acting on a spinor.}, which is \textbf{double valued}: the rotation with \(\psi=2\pi\) (that coincides with the identity on vectors) is represented by \(-1\) on the spinors; it is thus necessary to make a rotation of \(4\pi\) to get back the identity. We should have a double cover of our \(\mathrm{SO}(3)\) representation.

    The rotation of an angle \(\phi\) around a generic axis \(\hat{\mathbf{n}}\) is represented by
    \[
        S(\Lambda) = \begin{pmatrix}
            \exp{i \frac{\phi}{2} \hat{\mathbf{n}} \cdot \bs{\sigma}} & 0                                                         \\
            0                                                         & \exp{i \frac{\phi}{2} \hat{\mathbf{n}} \cdot \bs{\sigma}}
        \end{pmatrix},
    \]
    which is recognized to be unitary as well. Now going back to the non relativistic limit  and splitting the representation as in \eqref{eq:dirac_spinor_split}
    \[
        \psi(x) = e^{-\frac{i}{\hbar}mc^2 t} \begin{pmatrix}
            \varphi(x) \\
            \chi(x)
        \end{pmatrix},
    \]
    we can see that the \(4D\) spinor representation reduces to two copies of the \(2D\) spinor representation of \(\mathrm{SU}(2)\) acting on non relativistic Pauli spinors, with the spin operator \(\mathbf{S} = \frac{1}{2} \bs{\sigma}\) acting on each \(2D\) subspace:
    \[
        \mathbf{S} = \frac{1}{2} \bs{\Sigma} = \frac{1}{2} \begin{pmatrix}
            \bs{\sigma} & 0           \\
            0           & \bs{\sigma}
        \end{pmatrix}.
    \]
\end{example}

\begin{example}[Boost along axis]
    We could imagine a pure boost along the \(x\)-axis, which would be represented by
    \[
        \omega_{0i} = - \omega_{i0} = \eta
    \]
    so that the finite transformation matrix is
    \[
        \omega^{\mu}_{\ \nu} = \begin{pmatrix}
            0     & -\eta & 0 & 0 \\
            -\eta & 0     & 0 & 0 \\
            0     & 0     & 0 & 0 \\
            0     & 0     & 0 & 0
        \end{pmatrix}, \implies \Lambda^{\mu}_{\ \nu} = (e^{\omega})^{\mu}_{\ \nu} = \begin{pmatrix}
            \cosh{\eta}  & -\sinh{\eta} & 0 & 0 \\
            -\sinh{\eta} & \cosh{\eta}  & 0 & 0 \\
            0            & 0            & 1 & 0 \\
            0            & 0            & 0 & 1
        \end{pmatrix},
    \]
    where we obtained this result by exponentiating the matrix \(\omega^{\mu}_{\ \nu}\): note that even powers of \(\omega\) have positive sign in the first diagonal block, while odd powers have always positive sign as well ( \(e^{\omega} = e^{\eta K}\) with \(K = \begin{pmatrix}
        0  & -1 \\
        -1 & 0
    \end{pmatrix}\) the generator of boosts along \(\hat{x}\) and \(K^2 = \mathbb{I}\), \(K^3 = K\) and finally \(K^4 = \mathbb{I}\)). If we take the Taylor expansion of the exponential we find
    \[
        \begin{aligned}
            e^{\eta K} & = \sum_{n=0}^{\infty} \frac{(\eta K)^n}{n!} = \mathbb{I} \left[1 + \frac{\eta^2}{2!} K^2 + \frac{\eta^4}{4!} K^4 + \dots \right] + K \left[\eta + \frac{\eta^3}{3!} K^2 + \frac{\eta^5}{5!} K^4 + \dots \right] \\
                       & = \mathbb{I}\cosh{\eta} + K\sinh{\eta}.
        \end{aligned}
    \]
    Thus the result obtained by exponentiating the matrix of Lie parameters is the boost matrix (even and odd powers in the first diagonal block of \(\omega^{\mu}_\nu\)).

    This is almost the same case as before, but now space and time coordinates are mixing, it's like a rotation of an imaginary angle (rotating in the first diagonal block of \( \omega\)). There is not anymore the alternating signs in the expansion, but we recognize the expansion of the hyperbolic functions \(\cosh\) and \(\sinh\); if we then use the relation
    \[
        \gamma^2 + \gamma^2 \beta^2 = 1 \implies \gamma = \cosh{\eta}, \quad \gamma \beta = \sinh{\eta} \implies \beta = \tanh{\eta},
    \]
    then we can relate the parameter \(\eta\) to the velocity of the boost \(\beta = \frac{v}{c}\) and the Lorentz factor \(\gamma = \left(1-\beta^2\right)^{-1/2}\) to obtain the usual form of the Lorentz boost along \(\hat{x}\):
    \[
        \Lambda^{\mu}_{\ \nu} = \begin{pmatrix}
            \gamma        & -\beta \gamma & 0 & 0 \\
            -\beta \gamma & \gamma        & 0 & 0 \\
            0             & 0             & 1 & 0 \\
            0             & 0             & 0 & 1
        \end{pmatrix}.
    \]
    If we call \(\eta\) the \textit{rapidity}, it is additive for boosts along the same direction, while velocities are added in a more complicated way. For a spinor reresentation we have to find \(S(\Lambda)\):
    \[
        \begin{aligned}
            S(\Lambda) & = \exp{\frac{i}{2} \omega_{\mu \nu} \Sigma^{\mu \nu}} = \exp{\frac{1}{4} \omega_{\mu \nu} \gamma^{\mu} \gamma^{\nu}} = \exp{\frac{1}{4}\omega_{01}\gamma^0 \gamma^1 + \frac{1}{4}\omega_{10}\gamma^1 \gamma^0} \\
                       & = \exp{\frac{\eta}{2} \gamma^0 \gamma^1} = \exp{\frac{\eta}{2} \begin{pmatrix}
                                                                                                0        & \sigma^1 \\
                                                                                                \sigma^1 & 0
                                                                                            \end{pmatrix}} = \exp{\frac{\eta}{2} \alpha^1},
        \end{aligned}
    \]
    where, if we now Taylor expand the expression, we find even powers giving the identity, while odd powers give \(\alpha^1\). Thus in the end we can recognize the hyperbolic functions again:
    \[
        \begin{aligned}
            \exp{\frac{\eta}{2} \alpha^1} & = \sum_{n=0}^{\infty} \frac{\left(\frac{\eta}{2} \alpha^1\right)^n}{n!} = \mathbb{I} \left[1 + \frac{1}{2!}\left(\frac{\eta}{2}\right)^2 + \frac{1}{4!}\left(\frac{\eta}{2}\right)^4 + \dots \right]           \\
                                          & + \alpha^1 \left[-\frac{\eta}{2} + \frac{1}{3!}\left(-\frac{\eta}{2}\right)^3 + \frac{1}{5!}\left(-\frac{\eta}{2}\right)^5 + \dots \right] = \mathbb{I}\cosh{\frac{\eta}{2}} - \alpha^1 \sinh{\frac{\eta}{2}},
        \end{aligned}
    \]
    to find the final expression for the spinorial representation of a boost along \(\hat{x}\):
    \[
        S(\Lambda) = \mathbb{I}\cosh{\frac{\eta}{2}} - \alpha^1 \sinh{\frac{\eta}{2}} = \begin{pmatrix}
            \cosh{\frac{\eta}{2}}  & 0                     & -\sinh{\frac{\eta}{2}} & 0                     \\
            0                      & \cosh{\frac{\eta}{2}} & 0                      & \sinh{\frac{\eta}{2}} \\
            -\sinh{\frac{\eta}{2}} & 0                     & \cosh{\frac{\eta}{2}}  & 0                     \\
            0                      & \sinh{\frac{\eta}{2}} & 0                      & \cosh{\frac{\eta}{2}}
        \end{pmatrix}.
    \]

    Note that this transformation is not unitary, but satisfies \(S^{\dagger} (\Lambda) = S(\Lambda)\): \(\alpha^1\) is hermitian and computing further, we find the whole representation to be hermitian.
\end{example}

\subsubsection{Plane Waves}

We want to construct the plane wave solution wiuth the spinorial representation. The previous boost transformation can be written and generalized as follows to find the general plane wave solutions of the Dirac equation for a particle with arbitrary momentum \(\mathbf{p}\). Starting from the shell condition for a free particle
\[
    E^2 - \mathbf{p}^2 = m^2,
\]
and considering that the four momentum can be written as
\[
    p^{\mu} = (E,\, \mathbf{p})= (m\gamma,\, m \bs{\beta} \gamma), \quad \text{with } p^{\prime\,\mu} = (m,\, \bs{0})
\]
obtained from the rest frame, one can use hyperbolic trigonometric identities (plus the previous results \(\gamma = \cosh(\eta)\) and \(\beta \gamma = \sinh(\eta)\)) to find
\[
    \begin{aligned}
        \tanh{\frac{\eta}{2}} & = \frac{\sinh{\eta}}{1 + \cosh{\eta}} = \frac{\beta \gamma}{1 + \gamma} = \frac{\vert \mathbf{p} \vert}{E + m}, \\
        \cosh{\frac{\eta}{2}} & = \sqrt{\frac{1}{2} + \frac{1}{2}\cosh(\eta)} = \sqrt{\frac{E + m}{2m}},
    \end{aligned}
\]
and if we plug them in the previous expression for the Lorentz boost transformation along \(\hat{x}\) we can find the following representation:
\[
    S(\Lambda) = \mathbb{I}\cosh{\frac{\eta}{2}} - \alpha^1 \sinh{\frac{\eta}{2}} = \cosh{\frac{\eta}{2}} \left( \mathbb{I} - \alpha^1 \tanh{\frac{\eta}{2}} \right) =
\]
then generalize to a boost in an arbitrary direction \(\hat{v} = \frac{\mathbf{v}}{\vert \mathbf{v} \vert }\) by substituting \(\alpha^1\) with
\[
    \alpha^1 \to \frac{\bs{\alpha}^1\cdot \mathbf{v}}{\vert \mathbf{v} \vert } = \frac{\bs{\alpha}\cdot\bs{\beta}}{\vert \bs{\beta}\vert}
\]
and finally change the direction of the boost \(\eta \to -\eta\)  so that by acting on a spinor at rest we get the spinor moving with velocity \(\mathbf{v}\) (and momentum \(\mathbf{p}\)). The final transformation takes the form
\[
    S(\Lambda) = \sqrt{\frac{E + m}{2m}} \left( \mathbb{I} + \frac{\bs{\alpha}^1 \cdot \mathbf{p}}{E + m} \right).
\]
and applied to the spinors
\[
    \psi_1 = \begin{pmatrix}
        1 \\
        0 \\
        0 \\
        0
    \end{pmatrix} e^{-imt}, \quad \psi_2 = \begin{pmatrix}
        0 \\
        1 \\
        0 \\
        0
    \end{pmatrix} e^{-imt}, \quad \psi_3 = \begin{pmatrix}
        0 \\
        0 \\
        1 \\
        0
    \end{pmatrix} e^{+imt}, \quad \psi_4 = \begin{pmatrix}
        0 \\
        0 \\
        0 \\
        1
    \end{pmatrix} e^{+imt},
\]
produce the general plane wave solutions of the Dirac equation. We obtain the positive energy solutions (the columns of the matrix \(S(\Lambda)\) times the plane wave).

    [pdf...]

\subsubsection{Pseudo Unitarity}

The spinorial representation in (151) is not unitary, \(S^{\dagger}(\Lambda) \neq S^{-1}(\Lambda)\), as seen on the Lorentz boost. This is understandable in the light of a theorem according to which unitary irreducible representations of compact groups are finite-dimensional, while those of non-compact groups are infinite-dimensional. Lorentz’s group is non-compact because of the boosts. However, the spinorial representations are pseudo-unitary in the sense that
\[
    S^{\dagger}(\Lambda) = \beta S^{-1}(\Lambda)\beta
\]
Computations.... (pdf and sofi)

\paragraph{Fermionic Bilinears.}
We start from the Dirac spinor and we want to find the \textbf{Dirac conjugate} of the spinor:
\[
    \bar{\psi} (x) = \psi^{\dagger}(x) \beta,
\]
which transforms as
\[
    \bar{\psi}(x) \to \bar{\psi^{\prime}}(x^{\prime}) = S^{-1}(\Lambda)\bar{\psi}(x).
\]
Let's show this result:
\[
    pdf
\]
By looking at the transformation rules, one can notice that the bilinear \( \bar{\psi}(x) \psi(x)\) form a scalar:
\[
    \bar{\psi}(x) \psi(x) \to \bar{\psi^{\prime} }(x^{\prime} ) \psi^{\prime} (x^{\prime} ) = \bar{\psi}(x) S(\Lambda) S^{-1}(\Lambda) \psi(x) = \bar{\psi}(x) \psi(x).
\]
If we take \(\psi^{\dagger}\psi\) it's not a scalar, but it's the time component of the four vector which appeared in the continuity equation
The quantity \(\psi^{\dagger}\psi\) instead is not a scalar but identifies the time component of the four-vector \(J^{\mu} = ...\)  which is the current that appears in the continuity equation (79). It is written in a manifestly covariant form as
\[
    J^{\mu} = i \bar{\psi} \gamma^{\mu} \psi
\]
transformation law [pdf]
The quantities \(\bar{\psi}\psi\) and \(\bar{\psi} \gamma^{\mu} \psi\) are examples of fermionic bilinears, quantities that furnish useful expressions for describing the physical properties of the spin 1/2 relativistic particle. Quite generally, using the basis of the spinor space \(\Gamma^A = (\mathbb{I},\,\gamma^{\mu},\, \Sigma^{\mu \nu},\, \gamma^{\mu} \gamma^5,\, \gamma^5)\), one may define fermionic bilinears of the form
\[
    \bar{\psi} \Gamma^A \psi
\]
which transform as scalar, vector, antisymmetric tensor of rank 2, pseudovector, and pseu-doscalar, respectively. We have already discussed the first two cases. For the pseudoscalar (neglecting for notational simplicity the dependence on the spacetime point), we find
\[
    pdf
\]
that indeed we recognize to be a scalar under proper and orthochronous Lorentz transformations (the adjective “pseudo” refers to a different behavior under spatial reflection, i.e., under a parity transformation). As the last example, we consider the antisymmetric tensor
\[
    pdf
\]
where we have used...

------------------------------------

\subsection{Discrete Symmetries}

In addition to the Lorentz transformations connected to the identity, one can prove the invariance of the free Dirac equation under discrete transformations such as \textbf{parity} \(P\) (also known as spatial reflection), \textbf{time reversal} \(T\), and \textbf{charge conjugation} \(C\), which exchanges particles with antiparticles.

\paragraph{Parity.}
Our definition of the parity
\[
    \begin{aligned}
        \mathbf{x} \xrightarrow{P} \mathbf{x}^{\prime} = - \mathbf{x}, \\
        t \xrightarrow{P} t^{\prime} = t,
    \end{aligned}
\]
which in tensorial notation becomes:
\[
    pdf
\]
This is a discrete operation with \(\det P^{\mu}_{\ \nu} = -1\). It belongs to the Lorentz group O(3, 1) but is not connected to the identity. Together with the identity, it forms a subgroup isomorphic to \(\mathbb{Z}_2 = \{1,\,-1\}\). Invariance under parity can be studied by conjecturing an appropriate linear transformation of the spinor
\[
    pdf
\]
generated by a suitable matrix \(\mathcal{P} \). Requiring invariance in form of the Dirac equation
\[
    requiring \, invariance \, pdf and paper
\]
or equivalently
\[
    pdf
\]
A matrix \(\mathcal{P}\) that commutes with \(\gamma^0\) and anticommutes with \(\gamma^i\) is \(\gamma^0\) itself, or equivalently, \(\beta = i \gamma^0\). Thus, one may choose \(\mathcal{P} = \eta_P \beta\)  with \(\eta_P\) a phase fixed by requiring that \(\mathcal{P}^4\) coincides with the identity on fermions (so that the possible choices are \(\eta_P = (\pm 1,\,\pm i)\)). For simplicity we choose \(\eta_P=1\), and use the parity transformations
\[
    pdf
\]
From these basic rules, one deduces the transformations of the \textit{fermionic bilinears}
\[
    \begin{aligned}
        \bar{\psi}(x)\psi(x) & \xrightarrow{P} \bar{\psi}^{\prime} (x^{\prime}) \psi^{\prime} (x^{\prime}) = \bar{\psi}(x)\psi(x) \quad \text{scalar} \\
        2                    & \xrightarrow{P}  \quad \text{pseudo-scalar}                                                                            \\
        3                    & \xrightarrow{P}  \quad \text{(polar) vector}                                                                           \\
        4                    & \xrightarrow{P}  \quad \text{(axial) vector}                                                                           \\
        5                    & \xrightarrow{P}  \quad \text{tensor}.
    \end{aligned}
\]

\paragraph{Chiral properties of spinors.}
Dirac spinors have four components, but under the group \(\mathrm{SO}^+(3,1)\) we know there this is  a reducible representation. We could introduce the projector operators
\[
    pdf
\]
in order to separate the Dirac spinors in their \textit{left and right handed components}
\[
    pdf
\]
They constitute the two irreducible spin \(\tfrac12\) representations of the Lorentz group contained inside the Dirac spinor. The irreducibility follows from the fact that the Lorentz generators \(\Sigma^{\mu \nu}\) commute with the projectors \(P_L\) and \(P_R\). Then, also finite transformations commute with the projectors. For example, considering \(P_L\) one calculates
\[
    pdf
\]
knowing that \(\Sigma^{\mu \nu}\) commutes with the projectors since \(\gamma^5\) commutes with an even number of gamma matrices, and likewise for \(P_R\). The interpretation of this commutativity is that operating with an infinitesimal Lorentz rotation on a chiral spinor of given chirality produces a chiral spinor of the same chirality.
    [... pdf ...]
Now, let us consider parity. Including parity, the Dirac spinor is not reducible anymore. Parity transforms left-handed spinors into right-handed ones and vice versa. Recalling the form of the parity transformation of a Dirac spinor, \(\psi\xrightarrow{P}\psi^{\prime}=\beta \psi\), one finds that a left-handed spinor is transformed into a right-handed one
\[
    pdf
\]
Both chiralities are needed to realize parity, as parity exchanges two opposite chiralities. Additional remarks: the representations of the Lorentz group are constructed systematically using the fact that its Lie algebra can be written in terms of two commuting [...]
Chirality is a Lorentz invariant concept.

\paragraph{Chiral representation.}
When dealing with chiral fermions, it is often useful to employ a different representation of
the gamma matrices, called the chiral representation. A chiral representation is identified by
the fact that the chiral matrix \(\gamma^5\) is diagonal. A chiral representation is given by
\[
    \gamma matrices pdf
\]
where is obtained from the Dirac representation by a similarity transformation (a change of basis) generated by a unitary matrix U \(\gamma^{\mu}_{\text{chiral}} = U \gamma^{\mu}_\text{Dirac} U^{-1}\) and
\[
    pdf
\]
In the chiral representation, the Lorentz generators are given by
\[
    pdf
\]
The block-diagonal form of the Lorentz generators makes evident that they act independently on the chiral parts of a Dirac spinor
\[
    pdf
\]
where the two-component chiral spinors (Weyl spinors) are identified by
\[
    pdf
\]

\paragraph{Helicity.}
For massless fermions, chirality is correlated to the concept of helicity, which we indicate by h. It is defined as the projection of the spin along the direction of motion
\[
    h = \frac{\mathbf{S} \cdot \mathbf{p}}{\vert \mathbf{p} \vert}
\]
Note that helicity is a Lorentz invariant concept only for massless particles. Let us consider a massless, left-handed fermion \(\psi_L = \frac{1-\gamma^5}{2}\psi\), which satisfies \(\gamma^5 \psi_L = -\psi_L\). Its Dirac equation in momentum space (i.e., after a Fourier transform) reads
\[
    pdf | paper
\]
with the mass-shell condition \(p_\mu p^{\mu} = 0\). Considering motion along the positive \(z\) direction, then \(p^0 = p^3\) and \(p^1 = p^2 = 0\) (for mass-shell it has the momentum on the \(z\) axis and the energy equal to it), so that
\[
    0 = (\gamma^0 p_0 + \gamma^3 p_3) \psi_L(p) = p^0(\gamma^3 - \gamma^0) \psi_L(p) \longrightarrow \gamma^0 \psi_L(p) = \gamma^3 \psi_L (p).
\]
Now, the spin operator
\[
    pdf
\]
has a component along the \(z\) axis given by
\[
    pdf
\]
and measures the helicity \(h\). One computes it as follows
\[
    \begin{aligned}
        h \psi_L (p) & =   \\
                     & = .
    \end{aligned}
\]
Thus, \(\psi_L\) describes a particle of helicity \(h=\tfrac12\). Its antiparticle is described by the charge conjugated field \(\psi_L,c\) that is right-handed and has helicity \(h=-\tfrac12\). This statement will become clear after the discussion on charge conjugation.

\paragraph{Time reversal}
Our definition of the time reversal
\[
    \begin{aligned}
        \mathbf{x} \xrightarrow{P} \mathbf{x}^{\prime} = \mathbf{x}, \\
        t \xrightarrow{P} t^{\prime} = -t,
    \end{aligned}
\]
which in tensorial notation becomes:
\[
    pdf
\]
It is a discrete symmetry with \(\det T^{\mu \nu} = -1\). It belongs to \(\mathrm{O}(3,1)\) but is not connected to the identity. Together with the identity it forms a subgroup isomorphic to \(\mathbb{Z}_2 = \{1,-1\}\). The way time reversal acts on spinors can be found by conjecturing a suitable anti-linear transformation on the spinor
\[
    pdf
\]
generated by a matrix \(\mathcal{T}\). The complex conjugate is suggested by the non-relativistic limit that links the Dirac equation to the Schr¨odinger equation. The Schr¨oedinger equation is known to have a time-reversal symmetry that acts by transforming the wave function to its complex conjugate one. Thus, requiring invariance implies the equivalence
\[
    \begin{aligned}
        pdf \\
        pdf
    \end{aligned}
\]
Comparing the latter with the complex conjugate of the former, namely \((\gamma^{\mu *}\partial_\mu + m)\psi^*(x) = 0\), one finds
\[
    pdf
\]
The last equality is obtained using the explicit Dirac representation of gamma matrices (77).
One needs to find a matrix \(\mathcal{T}\) that commutes with \(\gamma^0\) and \(\gamma^2\) and anticommutes with \(\gamma^1\) and \(\gamma^3\). This matrix is proportional to \(\gamma^1\gamma^3\). Adding an arbitrary phase \(\eta_T\) one finds
\[
    pdf
\]
For simplicity, one can set \(\eta_T = 1\). Note that on spinors \(T^2 = -1\) and \(T^4=1\).

\paragraph{Hole theory.}
To overcome the problem of negative energy solutions, Dirac developed the theory of holes, abandoning the single-particle interpretation of his wave equation and predicting the existence of antiparticles. He supposed that the vacuum state, defined as the state with the lowest energy, consists of a configuration in which all the negative energy levels are occupied by electrons (the “Dirac sea”): Pauli’s exclusion principle guarantees that no more electrons can be added to the negative energy levels. This vacuum state has, by definition, vanishing energy and charge
\[
    E_{\text{(vacuum)}} = 0, \quad Q_{\text{(vacuum)}} = 0.
\]
The state with one physical electron consists of an occupied positive energy level on top of the filled Dirac sea
\[
    pdf
\]
It has a charge \(e<0\) (by convention) and cannot jump to a negative energy level because the negative energy levels are all occupied, and the Pauli principle forbids the jump: the configuration is stable. In addition, one can also imagine a configuration in which a negative energy level lacks its electron: this is a hole in the Dirac sea. It is equivalent to a configuration in which a particle with positive energy and charge \(-e\) is present on top of the vacuum. In fact, filling the hole with an electron with negative energy \(-E_p\) and charge \(e\) gives back the vacuum state with vanishing energy and charge:
\[
    pdf
\]
These considerations led Dirac to predict the existence of the positron, the antiparticle of the electron. Moreover, it appears possible to imagine the phenomenon of pair creation: a photon that interacts with the vacuum can transfer its energy to an electron with negative energy sitting in the Dirac sea and brings it up to positive energy, thus creating an electron and a hole, i.e., an electron/positron pair. This interpretation provides useful physical intuition, though it is not directly applicable to bosonic systems (as Pauli’s principle is not valid for bosons). The correct realization of these ideas is implemented in QFT, both for fermions and bosons.

\paragraph{Charge conjugation.}

The Dirac equation can be coupled to electromagnetism with the minimal substitution \(p_\mu \to p_\mu -eA_\mu\). It takes the form
\[
    pdf
\]
and describes particles with charge e and antiparticles with same mass but opposite charge \(-e\), as suggested by the hole theory of Dirac. It should be possible to describe the same physics in terms of a Dirac equation for the antiparticles, identifying the original particles as anti-antiparticles.
The new equation must take the form
\[
    pdf
\]
where \(\psi_c\) denotes the charge conjugation of \(\psi\). The existence of a discrete transformation that links \(\psi\) to \(\psi_c\) is expected on physical ground, as it describes the same physical situation.
This transformation is called charge conjugation. It exchanges particles and antiparticles. To identify it, one proceeds as follows. One compares eq. (219) with the complex conjugate of (218), which becomes
\[
    pdf
\]
[...]
[sofi]

\(\mathcal{C}\) anticommutes with \(\gamma^0\) and \(\gamma^2\) and commutes with \(\gamma^1\) and \(\gamma^3\); then we may take
\[
    \mathcal{C} = \gamma^0\gamma^2
\]
Note that C is antisymmetric (\(\mathcal{C}^T = -\mathcal{C}\)) and coincides with its inverse (\(\mathcal{C}^{-1} = \mathcal{C}\)). Inserting an arbitrary phase \(\eta_C\), one finds for the charge conjugation transformation of the Dirac spinor
\[
    \psi \to \psi_c = \eta_C \mathcal{A} \psi^* = \dots
\]
here written in two equivalent ways. The arbitrary phase is usually set to 1 for simplicity. What we have described is not a true symmetry if one keeps the background \(A_\mu\) fixed. To achieve invariance, one should transform the background as well \(A_\mu \xrightarrow{C} A_\mu^c = -A_\mu\). One names this particular symmetry as “background symmetry”: it relates solutions in a given background to solutions in a transformed background. It becomes a true symmetry when also the field \(A_\mu\) is treated as a dynamical field, subject to its own equations of motion and to the transformation for \(A_\mu\) given above. This is the charge conjugation symmetry of QED.
Finally, let us show that the charge conjugation of a left-handed spinor is right-handed and vice versa: considering that a left-handed spinor and its Dirac conjugate satisfy
\[
    \psi_L = P_L \psi_L \to \overline{\psi_L} = \overline{\psi_L} P_R
\]
one finds by a direct computation that \(\psi_L,c\) is right-handed
\[
    pdf
\]
where we have used the Dirac basis \(\mathcal{C} = \gamma^0\gamma^2\) and \(gamma^{5,T} = gamma^5\), i.e. \(P_R^T = P_R\).
    [sofi dappertutto]

\paragraph{CPT.} Although the discrete symmetries C, P, and T of the free theory can be broken by interactions (notably, by the weak interaction), the CPT combination is found to be always valid for theories which are Lorentz invariant (i.e., invariant under \(\mathrm{SO}^+(3,1)\)). The theorem that proves this statement is known as the “CPT theorem” and will not be proved in generality on these notes. In the case of a Dirac fermion, the CPT transformation takes the form
\[
    \begin{aligned}
        x^{\mu} & \to x^{\prime \mu} = - x^{\mu},                         \\
        \psi(x) & \to \psi_{CPT}(x^{\prime}) = \eta_{CPT}gamma^5 \psi(x),
    \end{aligned}
\]
with \(\eta_{cpt}\) an arbitrary phase and one verifies quite easily the invariance of the free Dirac equation. [sofi]

\subsection{Action}
The action is of great value to study symmetries, interactions, and equations of motion. It is also the starting point for quantization, either canonical or through path integrals. To identify an action for the Dirac equation, one ensures Lorentz invariance by taking a scalar lagrangian density. The latter is constructed using the Dirac field \(\psi\) and its Dirac conjugate \(\overline{\psi} = \psi^{\dagger} \beta = \psi^{\dagger} i \gamma^0\), which has the property of transforming in such a way to make the product \(\overline{\psi} \psi\) a scalar. Then, one recognizes that a suitable action is given by
\[
    S[\psi,\,\overline{\psi}] = \int \mathrm{d}^4 x\,\mathcal{L}, \quad \mathcal{L} = -\overline{\psi}(\gamma^{\mu} \partial_\mu + m)\psi
\]
It is a Lorentz scalar, and varying \(\overline{\psi}\) and \(\psi\) independently, one finds that the least action principle indeed produces the Dirac equation and its conjugated one
\[
    \begin{aligned}
        \delta \overline{\psi}) & \quad (\gamma^{\mu} \partial_\mu + m)\psi(x) = 0, \\
        \delta \psi)            & \quad pdf
    \end{aligned}
\]

\paragraph{Symmetries.}
The symmetries under the Lorentz group have already been described. The symmetries under space-time translations are verified by taking the spinor \(\psi(x)\) transforming as a scalar (\(\psi(x) \to \psi^{\prime}(x^{\prime}) = \psi(x)\) under \(x^{\mu} \to x^{\prime \mu} = x^{\mu} + a^{\mu}\) with \(a^{\mu}\) constant). The related Noether current gives the energy-momentum tensor.
Let us consider in more detail the internal symmetry generated by phase transformations
\[
    \begin{aligned}
        pdf
    \end{aligned}
\]
forming the group \(\mathrm{U}(1)\). It is immediate to check that the action (232) is invariant. The infinitesimal version reads
\[
    pdf
\]
and extending \(\alpha\) to an arbitrary function \(\alpha(x)\) we compute the variation of the action
\[
    pdf
\]
which verifies again the \(\mathrm{U}(1)\) symmetry (for constant \(\alpha\)), obtaining at the same time the Noether current
\[
    pdf
\]
which is conserved on-shell (i.e., using the equations of motion: \(\partial_{\mu}J^{\mu}=0\)). As already noticed, the conserved charge density is positive definite
\[
    pdf
\]
and led Dirac to interpret it as a probability density. In second quantization, it is reinterpreted as the symmetry related to the fermionic number (its charge will count the number of particles minus the number of antiparticles), and in that context, eq. (238) becomes an operator that is no longer positive definite. In the coupling to electromagnetism, it is related to the electric charge.
More generally, a collection of N Dirac fermions with the same Dirac mass is invariant under the group \(\mathrm{U}(N) = \mathrm{U}(1)\times \mathrm{SU}(N) \). To see this, let us consider N fermions \(\psi^i\) transforming in the fundamental representation of \(\mathrm{U}(N)\), the N representation,
\[
    pdf
\]
The Dirac conjugates \(\psi^i\) (that contains the complex conjugate fields) transform in the antifundamental representation of \(\mathrm{U}(N)\), the N representation, which can be written as
\[
    pdf
\]
Then, the lagrangian
\[
    pdf
\]
is manifestly invariant. The corresponding Noether currents
\[
    pdf
\]
are conserved \(\partial_\mu J^{\mu,\,a}=0\). They are derived by considering infinitesimal transformations \(U = e^{i \alpha^a T^a} = 1 + i \alpha^a T^a\) and extending the infinitesimal Lie parameters \(\alpha^a\) to be arbitrary functions, as usual.
If the mass vanishes, the internal symmetry becomes larger, and given by the group \(\mathrm{U}(2N)\), as left and right-handed fermions transform independently. This fact may be better appreciated and proved keeping in mind the properties of chiral fermions and their charge conjugation.

    [sofi computations diocane]



\paragraph{Chiral fermions.}
Often, one analyzes the action rather than the equations of motion to derive general properties of the system. Thus, it is interesting to study the action written in terms of the irreducible chiral components \(\psi_L\) and \(\psi_R\) and their Dirac conjugates. It takes the form
\[
    pdf
\]
which is verified by recalling the properties of the projectors \(\psi_{L/R} \), and in particular
\[
    \begin{aligned}
        pdf
    \end{aligned}
\]
This form of the action shows that the Dirac mass term m cannot be present for chiral fermions (i.e., in models where one keeps only \(\psi_L\) by setting \(\psi_R = 0\), or more generally where left-handed fermions are coupled differently to other particles than right-handed fermions). Recall that the Dirac mass term had the property of being invariant under the \(\mathrm{U}(1)\) phase transformations given in equation ???\footnote{In chiral models where parity is not conserved, there may be several right-handed and left-handed fermions with different charges and even different in numbers. The fermions entering the standard model are, in fact, chiral, in the sense that left-hand fermions have couplings different from their right-handed partners (i.e., different charges). They cannot have Dirac masses, which would not be gauge invariant: the transformation laws of \(\psi_L\) under the standard model symmetries (\(\mathrm{SU}(3)\times \mathrm{SU}(2)\times \mathrm{U}(1)\)) are different from those of \(\psi_R\). The Dirac masses of the standard model emerge as a consequence of the Higgs mechanism for the spontaneous breaking of the \(\mathrm{SU}(2)\times \mathrm{U}(1)\) gauge symmetry.}

[Majorana mass and Majorana fermions]


\paragraph{Dirac and Majorana masses in scalar theories.}
By definition, a Majorana fermion is described by a spinor field that satisfies a reality condition of the type \(\mu_c (x)=\mu(x)\), often interpret by saying that particles and antiparticles coincide. It describes an electrically uncharged fermion. Indeed the transformation (234) is no longer a symmetry: it cannot be applied to \(\mu\) as it does not respect the constraint \(\mu_c = \mu\), and the corresponding conserved charge no longer exists. A Majorana fermion possesses half the degrees of freedom of a Dirac fermion.
To better understand the physical meaning of Dirac and Majorana mass, it is useful to describe an analogy with scalar particles. As a complex scalar field can be thought of as the combination of two real scalars with the same mass, similarly, a Dirac fermion can be considered as composed of two Majorana fermions with identical masses.
The analog of a Majorana fermion is a real scalar field \(\psi\), which satisfies \(\psi^* = \psi\) and a Klein-Gordon equation with mass \(\mu\), derivable from the lagrangian
\[
    pdf
\]
Two free real scalar fields \(\psi_1\) and \(\psi_2\) with different masses \(\mu_1\) and \(\mu_2\), are described by the lagrangian
\[
    pdf
\]
If the masses are identical, \(\mu_1 = \mu_2 = m\), the model acquires a \(\mathrm{SO}(2)\) symmetry that mixes the fields \(\psi_1\) and \(\psi_2\), and the lagrangian becomes
\[
    pdf
\]
The term \(\psi_1^2 + \psi_2^2 \) is \(\mathrm{SO}(2)\) invariant, as is the kinetic term. The lagrangian can be written in terms of a complex field \(\phi\) defined by
\[
    pdf
\]
with \(\psi_1\) and \(\psi_2\) the real and imaginary part of \(\phi\), respectively. In this basis, the lagrangian (252) takes the form
\[
    pdf.
\]
The symmetry \(\mathrm{SO}(2) = \mathrm{U}(1)\) becomes \(\phi^{\prime} = e^{i \alpha} \phi\) and \(\phi^{\prime *} = e^{-i \alpha} \phi^*\) , and the corresponding charge is often called “bosonic number” (it acts as the electric charge in a coupling to electromagnetism).

    [...]

\paragraph{Green functions and propagator.}
