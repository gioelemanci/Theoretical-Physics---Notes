\section{Dirac Equation}

\subsection{Continuity Equations}

\subsection{Plane Wave Solutions}

\subsection{Pauli Equation: Non-Relativistic Limit}

\subsubsection{Electromagnetic Coupling}

\paragraph{Gyromagnetic factor.}

\subsection{Angular Momentum and Spin}

\subsection{Idrogen Atom and Dirac Equation}

\subsection{Properties of Gamma Matrices}

\subsection{Covariance Formulation}

[...]

Let's take the exponential form of \(\Lambda\) and \(S(\Lambda)\) matrices and compute an infinitesimal transformation and verify that such matrices exist\dots
This is a representation of Lorentz' transformations



\begin{example}[Rotation around axis]\TODO{put title}
    Transformations with \(\omega_{12} = -\omega_{21} = \psi\) \dots
    \[
        pdf,
    \]
    the result pf exponentiating the matrix of lee parameters is the rotation matrix (even and odd powers if the central diagonal block of \(\omega^{\mu}_\nu\)). So we are looking at rotations around \(\hat{z}\). Let's find an expression for \(S(\Lambda)\): looking at the exponential definition with gamma matrices
    \[
        S(\Lambda)=\exp{\frac{1}{4}\omega_{12}\gamma^1 \gamma^2 + \frac{1}{4}\omega_{21}\gamma^2 \gamma^1} = \dots
    \]
    where we used the Dirac representation of the gamma matrices.

    The transformation is immediately recognized to be unitary, \(S^{\dagger} (\Lambda) = S^{-1}(\Lambda)\). It is also clear that it is a spinorial transformation\footnote{The \(\omega\) matrices were a four vector representation, here we are acting on a spinor.}, which is \textbf{double valued}: the rotation with \(\psi=2\pi\) (that coincides with the identity on vectors) is represented by  on the spinors. It is necessary to make a rotation of \(4\pi\) to get back the identity. We should have a double cover of our \(\mathrm{SO}(3)\) representation.

    The rotation of an angle \(\phi\) around a generic axis \(\hat{\mathbf{n}}\) is represented by
    \[
        S = \exp{i \frac{\phi}{2} \hat{\mathbf{n}} \cdot \bs{\sigma}}\dots
    \]
\end{example}

\begin{example}[Boost along axis]
    \(\omega_{0i} = - \omega_{i0} = \eta\), so
    \[
        pdf
    \]
    Almost as before but now space and time are mixing, it's like a rotation of an imaginary angle (rotating in the first diagonal block of \( \omega\)). There is not anymore the alternating signs in the expansion, but we recognize the expansion of the hyperbolic functions \(\cosh\) and \(\sinh\):
    \[
        pdf
    \]
    If we call \(\eta\) the \textit{rapidity}, it is additive for boosts along the same direction, while velocities are added in a more complicated way. For a spinor rpresentation we have to find \(S(\Lambda)\):
    \[
        pdf
    \]
    in the expansion even powers give the identity, while odd ones gives \(\alpha^1\). Note that this transformation is not unitary, but satisfies \(S^{\dagger} (\Lambda) = S(\Lambda)\): \(\alpha^1\) is hermitian and computing we find the whole representation to be hermitian.
\end{example}

\paragraph{Plane Waves.}
We want to construct the plane wave solution wiuth the spinorial representation. The previous boost transformation can be written and generalized as follows to find the general plane wave solutions of the Dirac equation.
\[
    paper1
\]
Using hyperbolic trigonometric identities, and considering \((E, \mathbf{p})= (m\gamma, m \bs{\beta} \gamma)\), obtained from the rest frame value \((m, \bs{0})\), one finds
\[
    pdf
\]
and if we plug them in the previous expression for the Lorentz boost transf repr we find:
\[
    pdf
\]
then generalize to a boost in an arbitrary direction \(\frac{\mathbf{v}}{\vert \mathbf{v} \vert }\)  by substituting \(\alpha^1\) with \(\frac{\bs{\alpha}^1\cdot \mathbf{v}}{\vert \mathbf{v} \vert } = \frac{\bs{\alpha}\cdot\bs{\beta}}{\vert \bs{\beta}\vert}\), and finally change the direction of the boost \(\eta \to -\eta\)  so that by acting on a spinor at rest we get the spinor moving with velocity \(\mathbf{v}\) (and momentum \(\mathbf{p}\)). The final transformation takes the form
\[
    pdf
\]
and applied to the spinors (85) and (86) produce the general plane wave solutions of the Dirac equation. We obtain the positive energy solutions (the columns of the matrix \(S(\Lambda)\) times the plane wave). [pdf...]

\paragraph{Pseudo-Unitarity.}
The spinorial representation in (151) is not unitary, \(S^{\dagger}(\Lambda) \neq S^{-1}(\Lambda)\), as seen on the Lorentz boost. This is understandable in the light of a theorem according to which unitary irreducible representations of compact groups are finite-dimensional, while those of non-compact groups are infinite-dimensional. Lorentz’s group is non-compact because of the boosts. However, the spinorial representations are pseudo-unitary in the sense that
\[
    S^{\dagger}(\Lambda) = \beta S^{-1}(\Lambda)\beta
\]
Computations.... (pdf and sofi)

\paragraph{Fermionic Bilinears.}
We start from the Dirac spinor and we want to find the \textbf{Dirac conjugate} of the spinor:
\[
    \bar{\psi} (x) = \psi^{\dagger}(x) \beta,
\]
which transforms as
\[
    \bar{\psi}(x) \to \bar{\psi^{\prime}}(x^{\prime}) = S^{-1}(\Lambda)\bar{\psi}(x).
\]
Let's show this result:
\[
    pdf
\]
By looking at the transformation rules, one can notice that the bilinear \( \bar{\psi}(x) \psi(x)\) form a scalar:
\[
    \bar{\psi}(x) \psi(x) \to \bar{\psi^{\prime} }(x^{\prime} ) \psi^{\prime} (x^{\prime} ) = \bar{\psi}(x) S(\Lambda) S^{-1}(\Lambda) \psi(x) = \bar{\psi}(x) \psi(x).
\]
If we take \(\psi^{\dagger}\psi\) it's not a scalar, but it's the time component of the four vector which appeared in the continuity equation
The quantity \(\psi^{\dagger}\psi\) instead is not a scalar but identifies the time component of the four-vector \(J^{\mu} = ...\)  which is the current that appears in the continuity equation (79). It is written in a manifestly covariant form as
\[
    J^{\mu} = i \bar{\psi} \gamma^{\mu} \psi
\]
transformation law [pdf]
The quantities \(\bar{\psi}\psi\) and \(\bar{\psi} \gamma^{\mu} \psi\) are examples of fermionic bilinears, quantities that furnish useful expressions for describing the physical properties of the spin 1/2 relativistic particle. Quite generally, using the basis of the spinor space \(\Gamma^A = (\mathbb{I},\,\gamma^{\mu},\, \Sigma^{\mu \nu},\, \gamma^{\mu} \gamma^5,\, \gamma^5)\), one may define fermionic bilinears of the form
\[
    \bar{\psi} \Gamma^A \psi
\]
which transform as scalar, vector, antisymmetric tensor of rank 2, pseudovector, and pseu-doscalar, respectively. We have already discussed the first two cases. For the pseudoscalar (neglecting for notational simplicity the dependence on the spacetime point), we find
\[
    pdf
\]
that indeed we recognize to be a scalar under proper and orthochronous Lorentz transformations (the adjective “pseudo” refers to a different behavior under spatial reflection, i.e., under a parity transformation). As the last example, we consider the antisymmetric tensor
\[
    pdf
\]
where we have used...

------------------------------------

\subsection{Discrete Symmetries}

In addition to the Lorentz transformations connected to the identity, one can prove the invariance of the free Dirac equation under discrete transformations such as \textbf{parity} \(P\) (also known as spatial reflection), \textbf{time reversal} \(T\), and \textbf{charge conjugation} \(C\), which exchanges particles with antiparticles.

\paragraph{Parity.}
Our definition of the parity
\[
    \begin{aligned}
        \mathbf{x} \xrightarrow{P} \mathbf{x}^{\prime} = - \mathbf{x}, \\
        t \xrightarrow{P} t^{\prime} = t,
    \end{aligned}
\]
which in tensorial notation becomes:
\[
    pdf
\]
This is a discrete operation with \(\det P^{\mu}_{\ \nu} = -1\). It belongs to the Lorentz group O(3, 1) but is not connected to the identity. Together with the identity, it forms a subgroup isomorphic to \(\mathbb{Z}_2 = \{1,\,-1\}\). Invariance under parity can be studied by conjecturing an appropriate linear transformation of the spinor
\[
    pdf
\]
generated by a suitable matrix \(\mathcal{P} \). Requiring invariance in form of the Dirac equation
\[
    requiring \, invariance \, pdf and paper
\]
or equivalently
\[
    pdf
\]
A matrix \(\mathcal{P}\) that commutes with \(\gamma^0\) and anticommutes with \(\gamma^i\) is \(\gamma^0\) itself, or equivalently, \(\beta = i \gamma^0\). Thus, one may choose \(\mathcal{P} = \eta_P \beta\)  with \(\eta_P\) a phase fixed by requiring that \(\mathcal{P}^4\) coincides with the identity on fermions (so that the possible choices are \(\eta_P = (\pm 1,\,\pm i)\)). For simplicity we choose \(\eta_P=1\), and use the parity transformations
\[
    pdf
\]
From these basic rules, one deduces the transformations of the \textit{fermionic bilinears}
\[
    \begin{aligned}
        \bar{\psi}(x)\psi(x) & \xrightarrow{P} \bar{\psi}^{\prime} (x^{\prime}) \psi^{\prime} (x^{\prime}) = \bar{\psi}(x)\psi(x) \quad \text{scalar} \\
        2                    & \xrightarrow{P}  \quad \text{pseudo-scalar}                                                                            \\
        3                    & \xrightarrow{P}  \quad \text{(polar) vector}                                                                           \\
        4                    & \xrightarrow{P}  \quad \text{(axial) vector}                                                                           \\
        5                    & \xrightarrow{P}  \quad \text{tensor}.
    \end{aligned}
\]

\paragraph{Chiral properties of spinors.}
Dirac spinors have four components, but under the group \(\mathrm{SO}^+(3,1)\) we know there this is  a reducible representation. We could introduce the projector operators
\[
    pdf
\]
in order to separate the Dirac spinors in their \textit{left and right handed components}
\[
    pdf
\]
They constitute the two irreducible spin \(\tfrac12\) representations of the Lorentz group contained inside the Dirac spinor. The irreducibility follows from the fact that the Lorentz generators \(\Sigma^{\mu \nu}\) commute with the projectors \(P_L\) and \(P_R\). Then, also finite transformations commute with the projectors. For example, considering \(P_L\) one calculates
\[
    pdf
\]
knowing that \(\Sigma^{\mu \nu}\) commutes with the projectors since \(\gamma^5\) commutes with an even number of gamma matrices, and likewise for \(P_R\). The interpretation of this commutativity is that operating with an infinitesimal Lorentz rotation on a chiral spinor of given chirality produces a chiral spinor of the same chirality.
    [... pdf ...]
Now, let us consider parity. Including parity, the Dirac spinor is not reducible anymore. Parity transforms left-handed spinors into right-handed ones and vice versa. Recalling the form of the parity transformation of a Dirac spinor, \(\psi\xrightarrow{P}\psi^{\prime}=\beta \psi\), one finds that a left-handed spinor is transformed into a right-handed one
\[
    pdf
\]
Both chiralities are needed to realize parity, as parity exchanges two opposite chiralities. Additional remarks: the representations of the Lorentz group are constructed systematically using the fact that its Lie algebra can be written in terms of two commuting [...]
Chirality is a Lorentz invariant concept.

\paragraph{Chiral representation.}
When dealing with chiral fermions, it is often useful to employ a different representation of
the gamma matrices, called the chiral representation. A chiral representation is identified by
the fact that the chiral matrix \(\gamma^5\) is diagonal. A chiral representation is given by
\[
    \gamma matrices pdf
\]
where is obtained from the Dirac representation by a similarity transformation (a change of basis) generated by a unitary matrix U \(\gamma^{\mu}_{\text{chiral}} = U \gamma^{\mu}_\text{Dirac} U^{-1}\) and
\[
    pdf
\]
In the chiral representation, the Lorentz generators are given by
\[
    pdf
\]
The block-diagonal form of the Lorentz generators makes evident that they act independently on the chiral parts of a Dirac spinor
\[
    pdf
\]
where the two-component chiral spinors (Weyl spinors) are identified by
\[
    pdf
\]

\paragraph{Helicity.}
For massless fermions, chirality is correlated to the concept of helicity, which we indicate by h. It is defined as the projection of the spin along the direction of motion
\[
    h = \frac{\mathbf{S} \cdot \mathbf{p}}{\vert \mathbf{p} \vert}
\]
Note that helicity is a Lorentz invariant concept only for massless particles. Let us consider a massless, left-handed fermion \(\psi_L = \frac{1-\gamma^5}{2}\psi\), which satisfies \(\gamma^5 \psi_L = -\psi_L\). Its Dirac equation in momentum space (i.e., after a Fourier transform) reads
\[
    pdf | paper
\]
with the mass-shell condition \(p_\mu p^{\mu} = 0\). Considering motion along the positive \(z\) direction, then \(p^0 = p^3\) and \(p^1 = p^2 = 0\) (for mass-shell it has the momentum on the \(z\) axis and the energy equal to it), so that
\[
    0 = (\gamma^0 p_0 + \gamma^3 p_3) \psi_L(p) = p^0(\gamma^3 - \gamma^0) \psi_L(p) \longrightarrow \gamma^0 \psi_L(p) = \gamma^3 \psi_L (p).
\]
Now, the spin operator
\[
    pdf
\]
has a component along the \(z\) axis given by
\[
    pdf
\]
and measures the helicity \(h\). One computes it as follows
\[
    \begin{aligned}
        h \psi_L (p) & =   \\
                     & = .
    \end{aligned}
\]
Thus, \(\psi_L\) describes a particle of helicity \(h=\tfrac12\). Its antiparticle is described by the charge conjugated field \(\psi_L,c\) that is right-handed and has helicity \(h=-\tfrac12\). This statement will become clear after the discussion on charge conjugation.

\paragraph{Time reversal}
Our definition of the time reversal
\[
    \begin{aligned}
        \mathbf{x} \xrightarrow{P} \mathbf{x}^{\prime} = \mathbf{x}, \\
        t \xrightarrow{P} t^{\prime} = -t,
    \end{aligned}
\]
which in tensorial notation becomes:
\[
    pdf
\]
It is a discrete symmetry with \(\det T^{\mu \nu} = -1\). It belongs to \(\mathrm{O}(3,1)\) but is not connected to the identity. Together with the identity it forms a subgroup isomorphic to \(\mathbb{Z}_2 = \{1,-1\}\). The way time reversal acts on spinors can be found by conjecturing a suitable anti-linear transformation on the spinor
\[
    pdf
\]
generated by a matrix \(\mathcal{T}\). The complex conjugate is suggested by the non-relativistic limit that links the Dirac equation to the Schr¨odinger equation. The Schr¨oedinger equation is known to have a time-reversal symmetry that acts by transforming the wave function to its complex conjugate one. Thus, requiring invariance implies the equivalence
\[
    \begin{aligned}
        pdf \\
        pdf
    \end{aligned}
\]
Comparing the latter with the complex conjugate of the former, namely \((\gamma^{\mu *}\partial_\mu + m)\psi^*(x) = 0\), one finds
\[
    pdf
\]
The last equality is obtained using the explicit Dirac representation of gamma matrices (77).
One needs to find a matrix \(\mathcal{T}\) that commutes with \(\gamma^0\) and \(\gamma^2\) and anticommutes with \(\gamma^1\) and \(\gamma^3\). This matrix is proportional to \(\gamma^1\gamma^3\). Adding an arbitrary phase \(\eta_T\) one finds
\[
    pdf
\]
For simplicity, one can set \(\eta_T = 1\). Note that on spinors \(T^2 = -1\) and \(T^4=1\).

\paragraph{Hole theory.}
To overcome the problem of negative energy solutions, Dirac developed the theory of holes, abandoning the single-particle interpretation of his wave equation and predicting the existence of antiparticles. He supposed that the vacuum state, defined as the state with the lowest energy, consists of a configuration in which all the negative energy levels are occupied by electrons (the “Dirac sea”): Pauli’s exclusion principle guarantees that no more electrons can be added to the negative energy levels. This vacuum state has, by definition, vanishing energy and charge
\[
    E_{\text{(vacuum)}} = 0, \quad Q_{\text{(vacuum)}} = 0.
\]
The state with one physical electron consists of an occupied positive energy level on top of the filled Dirac sea
\[
    pdf
\]
It has a charge \(e<0\) (by convention) and cannot jump to a negative energy level because the negative energy levels are all occupied, and the Pauli principle forbids the jump: the configuration is stable. In addition, one can also imagine a configuration in which a negative energy level lacks its electron: this is a hole in the Dirac sea. It is equivalent to a configuration in which a particle with positive energy and charge \(-e\) is present on top of the vacuum. In fact, filling the hole with an electron with negative energy \(-E_p\) and charge \(e\) gives back the vacuum state with vanishing energy and charge:
\[
    pdf
\]
These considerations led Dirac to predict the existence of the positron, the antiparticle of the electron. Moreover, it appears possible to imagine the phenomenon of pair creation: a photon that interacts with the vacuum can transfer its energy to an electron with negative energy sitting in the Dirac sea and brings it up to positive energy, thus creating an electron and a hole, i.e., an electron/positron pair. This interpretation provides useful physical intuition, though it is not directly applicable to bosonic systems (as Pauli’s principle is not valid for bosons). The correct realization of these ideas is implemented in QFT, both for fermions and bosons.

\paragraph{Charge conjugation.}

The Dirac equation can be coupled to electromagnetism with the minimal substitution \(p_\mu \to p_\mu -eA_\mu\). It takes the form
\[
    pdf
\]
and describes particles with charge e and antiparticles with same mass but opposite charge \(-e\), as suggested by the hole theory of Dirac. It should be possible to describe the same physics in terms of a Dirac equation for the antiparticles, identifying the original particles as anti-antiparticles.
The new equation must take the form
\[
    pdf
\]
where \(\psi_c\) denotes the charge conjugation of \(\psi\). The existence of a discrete transformation that links \(\psi\) to \(\psi_c\) is expected on physical ground, as it describes the same physical situation.
This transformation is called charge conjugation. It exchanges particles and antiparticles. To identify it, one proceeds as follows. One compares eq. (219) with the complex conjugate of (218), which becomes
\[
    pdf
\]
[...]
[sofi]

\(\mathcal{C}\) anticommutes with \(\gamma^0\) and \(\gamma^2\) and commutes with \(\gamma^1\) and \(\gamma^3\); then we may take
\[
    \mathcal{C} = \gamma^0\gamma^2
\]
Note that C is antisymmetric (\(\mathcal{C}^T = -\mathcal{C}\)) and coincides with its inverse (\(\mathcal{C}^{-1} = \mathcal{C}\)). Inserting an arbitrary phase \(\eta_C\), one finds for the charge conjugation transformation of the Dirac spinor
\[
    \psi \to \psi_c = \eta_C \mathcal{A} \psi^* = \dots
\]
here written in two equivalent ways. The arbitrary phase is usually set to 1 for simplicity. What we have described is not a true symmetry if one keeps the background \(A_\mu\) fixed. To achieve invariance, one should transform the background as well \(A_\mu \xrightarrow{C} A_\mu^c = -A_\mu\). One names this particular symmetry as “background symmetry”: it relates solutions in a given background to solutions in a transformed background. It becomes a true symmetry when also the field \(A_\mu\) is treated as a dynamical field, subject to its own equations of motion and to the transformation for \(A_\mu\) given above. This is the charge conjugation symmetry of QED.
Finally, let us show that the charge conjugation of a left-handed spinor is right-handed and vice versa: considering that a left-handed spinor and its Dirac conjugate satisfy
\[
    \psi_L = P_L \psi_L \to \overline{\psi_L} = \overline{\psi_L} P_R
\]
one finds by a direct computation that \(\psi_L,c\) is right-handed
\[
    pdf
\]
where we have used the Dirac basis \(\mathcal{C} = \gamma^0\gamma^2\) and \(gamma^{5,T} = gamma^5\), i.e. \(P_R^T = P_R\).
    [sofi dappertutto]

\paragraph{CPT.} Although the discrete symmetries C, P, and T of the free theory can be broken by interactions (notably, by the weak interaction), the CPT combination is found to be always valid for theories which are Lorentz invariant (i.e., invariant under \(\mathrm{SO}^+(3,1)\)). The theorem that proves this statement is known as the “CPT theorem” and will not be proved in generality on these notes. In the case of a Dirac fermion, the CPT transformation takes the form
\[
    \begin{aligned}
        x^{\mu} & \to x^{\prime \mu} = - x^{\mu},                         \\
        \psi(x) & \to \psi_{CPT}(x^{\prime}) = \eta_{CPT}gamma^5 \psi(x),
    \end{aligned}
\]
with \(\eta_{cpt}\) an arbitrary phase and one verifies quite easily the invariance of the free Dirac equation. [sofi]

\subsection{Action}
The action is of great value to study symmetries, interactions, and equations of motion. It is also the starting point for quantization, either canonical or through path integrals. To identify an action for the Dirac equation, one ensures Lorentz invariance by taking a scalar lagrangian density. The latter is constructed using the Dirac field \(\psi\) and its Dirac conjugate \(\overline{\psi} = \psi^{\dagger} \beta = \psi^{\dagger} i \gamma^0\), which has the property of transforming in such a way to make the product \(\overline{\psi} \psi\) a scalar. Then, one recognizes that a suitable action is given by
\[
    S[\psi,\,\overline{\psi}] = \int \mathrm{d}^4 x\,\mathcal{L}, \quad \mathcal{L} = -\overline{\psi}(\gamma^{\mu} \partial_\mu + m)\psi
\]
It is a Lorentz scalar, and varying \(\overline{\psi}\) and \(\psi\) independently, one finds that the least action principle indeed produces the Dirac equation and its conjugated one
\[
    \begin{aligned}
        \delta \overline{\psi}) & \quad (\gamma^{\mu} \partial_\mu + m)\psi(x) = 0, \\
        \delta \psi)            & \quad pdf
    \end{aligned}
\]

\paragraph{Symmetries.}
The symmetries under the Lorentz group have already been described. The symmetries under space-time translations are verified by taking the spinor \(\psi(x)\) transforming as a scalar (\(\psi(x) \to \psi^{\prime}(x^{\prime}) = \psi(x)\) under \(x^{\mu} \to x^{\prime \mu} = x^{\mu} + a^{\mu}\) with \(a^{\mu}\) constant). The related Noether current gives the energy-momentum tensor.
Let us consider in more detail the internal symmetry generated by phase transformations
\[
    \begin{aligned}
        pdf
    \end{aligned}
\]
forming the group \(\mathrm{U}(1)\). It is immediate to check that the action (232) is invariant. The infinitesimal version reads
\[
    pdf
\]
and extending \(\alpha\) to an arbitrary function \(\alpha(x)\) we compute the variation of the action
\[
    pdf
\]
which verifies again the \(\mathrm{U}(1)\) symmetry (for constant \(\alpha\)), obtaining at the same time the Noether current
\[
    pdf
\]
which is conserved on-shell (i.e., using the equations of motion: \(\partial_{\mu}J^{\mu}=0\)). As already noticed, the conserved charge density is positive definite
\[
    pdf
\]
and led Dirac to interpret it as a probability density. In second quantization, it is reinterpreted as the symmetry related to the fermionic number (its charge will count the number of particles minus the number of antiparticles), and in that context, eq. (238) becomes an operator that is no longer positive definite. In the coupling to electromagnetism, it is related to the electric charge.
More generally, a collection of N Dirac fermions with the same Dirac mass is invariant under the group \(\mathrm{U}(N) = \mathrm{U}(1)\times \mathrm{SU}(N) \). To see this, let us consider N fermions \(\psi^i\) transforming in the fundamental representation of \(\mathrm{U}(N)\), the N representation,
\[
    pdf
\]
The Dirac conjugates \(\psi^i\) (that contains the complex conjugate fields) transform in the antifundamental representation of \(\mathrm{U}(N)\), the N representation, which can be written as
\[
    pdf
\]
Then, the lagrangian
\[
    pdf
\]
is manifestly invariant. The corresponding Noether currents
\[
    pdf
\]
are conserved \(\partial_\mu J^{\mu,\,a}=0\). They are derived by considering infinitesimal transformations \(U = e^{i \alpha^a T^a} = 1 + i \alpha^a T^a\) and extending the infinitesimal Lie parameters \(\alpha^a\) to be arbitrary functions, as usual.
If the mass vanishes, the internal symmetry becomes larger, and given by the group \(\mathrm{U}(2N)\), as left and right-handed fermions transform independently. This fact may be better appreciated and proved keeping in mind the properties of chiral fermions and their charge conjugation.

    [sofi computations diocane]



\paragraph{Chiral fermions.}
Often, one analyzes the action rather than the equations of motion to derive general properties of the system. Thus, it is interesting to study the action written in terms of the irreducible chiral components \(\psi_L\) and \(\psi_R\) and their Dirac conjugates. It takes the form
\[
    pdf
\]
which is verified by recalling the properties of the projectors \(\psi_{L/R} \), and in particular
\[
    \begin{aligned}
        pdf
    \end{aligned}
\]
This form of the action shows that the Dirac mass term m cannot be present for chiral fermions (i.e., in models where one keeps only \(\psi_L\) by setting \(\psi_R = 0\), or more generally where left-handed fermions are coupled differently to other particles than right-handed fermions). Recall that the Dirac mass term had the property of being invariant under the \(\mathrm{U}(1)\) phase transformations given in equation ???\footnote{In chiral models where parity is not conserved, there may be several right-handed and left-handed fermions with different charges and even different in numbers. The fermions entering the standard model are, in fact, chiral, in the sense that left-hand fermions have couplings different from their right-handed partners (i.e., different charges). They cannot have Dirac masses, which would not be gauge invariant: the transformation laws of \(\psi_L\) under the standard model symmetries (\(\mathrm{SU}(3)\times \mathrm{SU}(2)\times \mathrm{U}(1)\)) are different from those of \(\psi_R\). The Dirac masses of the standard model emerge as a consequence of the Higgs mechanism for the spontaneous breaking of the \(\mathrm{SU}(2)\times \mathrm{U}(1)\) gauge symmetry.}

[Majorana mass and Majorana fermions]


\paragraph{Dirac and Majorana masses in scalar theories.}
By definition, a Majorana fermion is described by a spinor field that satisfies a reality condition of the type \(\mu_c (x)=\mu(x)\), often interpret by saying that particles and antiparticles coincide. It describes an electrically uncharged fermion. Indeed the transformation (234) is no longer a symmetry: it cannot be applied to \(\mu\) as it does not respect the constraint \(\mu_c = \mu\), and the corresponding conserved charge no longer exists. A Majorana fermion possesses half the degrees of freedom of a Dirac fermion.
To better understand the physical meaning of Dirac and Majorana mass, it is useful to describe an analogy with scalar particles. As a complex scalar field can be thought of as the combination of two real scalars with the same mass, similarly, a Dirac fermion can be considered as composed of two Majorana fermions with identical masses.
The analog of a Majorana fermion is a real scalar field \(\psi\), which satisfies \(\psi^* = \psi\) and a Klein-Gordon equation with mass \(\mu\), derivable from the lagrangian
\[
    pdf
\]
Two free real scalar fields \(\psi_1\) and \(\psi_2\) with different masses \(\mu_1\) and \(\mu_2\), are described by the lagrangian
\[
    pdf
\]
If the masses are identical, \(\mu_1 = \mu_2 = m\), the model acquires a \(\mathrm{SO}(2)\) symmetry that mixes the fields \(\psi_1\) and \(\psi_2\), and the lagrangian becomes
\[
    pdf
\]
The term \(\psi_1^2 + \psi_2^2 \) is \(\mathrm{SO}(2)\) invariant, as is the kinetic term. The lagrangian can be written in terms of a complex field \(\phi\) defined by
\[
    pdf
\]
with \(\psi_1\) and \(\psi_2\) the real and imaginary part of \(\phi\), respectively. In this basis, the lagrangian (252) takes the form
\[
    pdf.
\]
The symmetry \(\mathrm{SO}(2) = \mathrm{U}(1)\) becomes \(\phi^{\prime} = e^{i \alpha} \phi\) and \(\phi^{\prime *} = e^{-i \alpha} \phi^*\) , and the corresponding charge is often called “bosonic number” (it acts as the electric charge in a coupling to electromagnetism).

    [...]

\paragraph{Green functions and propagator.}
