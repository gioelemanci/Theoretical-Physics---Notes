\section{Correlation Functions}

Correlation functions are quantities used to describe several physical observables in the quantum theory. They are also useful to develop the perturbative expansion around the solvable gaussian path integral that corresponds to a “free” theory (we include the harmonic oscillator in this class, as it is recognized as the quantum mechanical equivalent of a free Klein-Gordon QFT).

Correlation functions are \textit{normalized averages of the product of n dynamical variables, evaluated at different times and weighted by} \(e^{\frac{i}{\hbar}S}\). In our one-dimensional example, the normalized “\(n\)-point correlation function” is defined by
\begin{equation}
    \langle x(t_1)x(t_2)\cdots x(t_n) \rangle = \frac{1}{Z} \int \mathrm{D} x \, x(t_1)x(t_2)\cdots x(t_n) e^{\frac{i}{\hbar}S[x]},
    \label{eq:n_point_correlation_function}
\end{equation}
where \(Z = \int \mathrm{D} x \, e^{\frac{i}{\hbar}S[x]}\) is the normalization factor, also known as the \textbf{partition function}, providing \(\langle 1 \rangle = 1\) (by definition).

Of particular importance is the 2-point correlation function \(\langle x(t_1) x(t_2) \rangle\), often called the propagator. It is understood that correlation functions depend implicitly on the boundary conditions that specify the initial and final states. Very often, especially in quantum field theory, one chooses the initial and final states to be the vacuum state (the state with lowest energy) and considers an infinite propagation time (which also implements the projection onto the ground state). We have considered amplitudes between positions eigenstates, but one can insert any desired state as a boundary state when deffining the matrix elements of the evolution operator.

It is often useful to introduce a generating functional \(Z[J]\) defined by
\begin{equation}
    Z[J] = \int \mathrm{D} x \, e^{\frac{i}{\hbar} \left(S[x] + \int \d{t} J(t) x(t)\right)},
    \label{eq:generating_functional}
\end{equation}
where \(J(t)\) is an auxiliary external function, known as a \textbf{source}.
Visually, when we compute the equations of motion, we are adding a source that "pushes" the particle around
\[
    E(x) = J(x),
\]
substantially being a mathematical trick to compute all correlation functions by functional differentiation with respect to the source \(J(t)\).

Let's consider a differentation with respect to the source at time \(t_1\):
\[
    \frac{\delta}{\delta J(t_1)} Z[J] = \int \mathrm{D} x \, x(t_1) e^{\frac{i}{\hbar} \left(S[x] + \int \d{t} J(t) x(t)\right)},
\]
where we have to pay attention to the functional derivative, which acts as a delta function
\begin{equation}
    \frac{\delta}{\delta J(t_1)} \int \d{t} J(t) x(t) = x(t_1).
    \label{eq:functional_derivative}
\end{equation}
Continuing this process iterating the functional differentiation \(n\) times, we obtain
\[
    \frac{\delta}{\delta J(t_n)} \cdots \frac{\delta}{\delta J(t_2)} \frac{\delta}{\delta J(t_1)} \to \left(\frac{i}{\hbar}\right)^n x(t_1)x(t_2)\cdots x(t_n),
\]
and then if we apply this to the generating functional and set the source to zero at the end, we obtain
\[
    \langle x(t_1)x(t_2)\cdots x(t_n) \rangle = \left. \frac{1}{Z} \left(\frac{\hbar}{i}\right)^n \frac{\delta}{\delta J(t_n)} \cdots \frac{\delta}{\delta J(t_2)} \frac{\delta}{\delta J(t_1)} Z[J] \right|_{J=0}.
\]
Alternatively, if we expand the generating functional in powers of the source \(J(t)\), we can read off all correlation functions of the theory summed over \(n\) as coefficients
\[
    Z[J] = \sum_{n=0}^{\infty} \frac{1}{n!} \left(\frac{i}{\hbar}\right)^n \int \d{t_1} \d{t_2} \cdots \d{t_n} \langle x(t_1)x(t_2)\cdots x(t_n) \rangle_U J(t_1) J(t_2) \cdots J(t_n),
\]
where this n-point correlation function is non normalized (i.e., without the factor \(1/Z\), as indicated by the subscript \(U\)). Thus our definition of path intergrals is given by the normalized n point correlation functions
\[
    \langle x(t_1)x(t_2)\cdots x(t_n) \rangle = \frac{1}{Z} \int \mathrm{D} x \, x(t_1)x(t_2)\cdots x(t_n) e^{\frac{i}{\hbar}S[x]},
\]
which can be computed by functional differentiation of the generating functional with respect to the source \(J(t)\). It is also easier to compute the correlation functions differetiating the generating functional expanded in powers of the source \(J(t)\).

\subsection{Comparison with Canonical Quantization}

It is useful to compare with the corresponding definition of correlation functions given in canonical quantization. We have employed the \textbf{Schrödinger picture} to evaluate the transition amplitude. In this picture operators are time-independent, and states acquire the time dependence by the Schrödinger equation. To state the equivalent definition of the n-point correlation function, given the n times which we have to order
\[
    t_1,\,t_2,\, \dots,\, t_n \xrightarrow{T} \ t_{T(1)} ,\, t_{T(2)} ,\, \dots,\, t_{T(n)},
\]
respecting the order \(t_{T(1)} \leq t_{T(2)} \leq \cdots \leq t_{T(n)}\), implemented using the time ordering permutations \(T\), we can define the \textbf{time-ordered n-point correlation function} as
\[
    \begin{aligned}
        \langle x(t_1)x(t_2)\cdots x(t_n) \rangle = \frac{1}{Z} \bra{x_f} e^{-\frac{i}{\hbar}\hat{H}(t_f - t_{T(n)})} \hat{x} e^{-\frac{i}{\hbar}\hat{H}(t_{T(n)} - t_{T(n-1)})} \cdots \\
        \cdots e^{-\frac{i}{\hbar}\hat{H}(t_{T(2)} - t_{T(1)})} \hat{x} e^{-\frac{i}{\hbar}\hat{H}(t_{T(1)} - t_i)} \ket{x_i},
    \end{aligned}
\]
where the normalization factor \(Z = \bra{x_f} e^{-\frac{i}{\hbar}\hat{H}(t_f - t_i)} \ket{x_i}\) is the transition amplitude. The time ordering is necessary because operators at different times do not commute in general. One can verify that this definition coincides with the one given in the path integral formalism.

We have to change paradigm to the \textbf{Heisenberg picture} to see the equivalence more clearly. In this picture, states are time-independent, and operators acquire the time dependence through the Heisenberg equation of motion
\[
    i \hbar \frac{\mathrm{d}}{\mathrm{d} t} \hat{O}_H (t) = [\hat{O}_H (t),\, \hat{H}], \quad \Longrightarrow \quad \hat{O}_H (t) = e^{\frac{i}{\hbar}\hat{H} t} \hat{O} e^{-\frac{i}{\hbar}\hat{H} t},
\]
for a time independent hamiltonian \(\hat{H}\). In particular, the position operator in the Heisenberg picture is given by
\[
    \hat{x}_H (t) = e^{\frac{i}{\hbar}\hat{H} t} \hat{x} e^{-\frac{i}{\hbar}\hat{H} t}, \quad \hat{x}_H(t) \ket{x, t} = x \ket{x, t}.
\]
Then, the n-point correlation function can be rewritten as
\[
    \langle x(t_1)x(t_2)\cdots x(t_n) \rangle = \frac{1}{Z} \bra{x_f,\,t_f} T \left[ \hat{x}_H (t_1) \hat{x}_H (t_2) \cdots \hat{x}_H (t_n) \right] \ket{x_i,\,t_i},
\]
where \(T\) is the time-ordering operator that rearranges the operators in order of increasing time arguments from right to left. This expression is evidently equivalent to the previous one, and coincides with the path integral definition of the n-point correlation function.

The simplest boundary conditions to consider for correlation functions are given by setting the initial and final states equal to states at infinite past and future times, i.e., \(\ket{x_i,\, t_i} = \ket{x_i,\,-\infty}\) and \(\ket{x_f,\, t_f} = \ket{x_f,\, +\infty}\). In this case, one usually assumes that the system is in its ground state at \(t=-\infty\), and that the initial and final states coincide with the ground state \(\ket{0}\) of the system. In the Schrödinger picture we were more general, since the boundary states were arbitrary position eigenstates computed in the integral.

\subsection{Digression on Gaussian Integrals}

Gaussian integrals are integrals of exponential functions with quadratic exponents. They are very important in physics, as they appear in the evaluation of path integrals for free theories, and in perturbation theory around free theories. Here we review some basic results on gaussian integrals that will be useful in the following.

The simplest gaussian integral is the one-dimensional integral
\[
    \int_{-\infty}^{+\infty}  \frac{\mathrm{d} \phi}{(2\pi)^{\tfrac{1}{2}}}\, e^{-\frac{1}{2} K \phi^2} = \frac{1}{\sqrt{K}}, \quad \text{for } K \in \mathbb{R}^+.
\]
If we include a linear term in the exponent, we can complete the square writing \(-\frac{1}{2}K \phi^2 + J \phi = -\frac{1}{2}K \left(\phi - \frac{J}{K}\right)^2 + \frac{J^2}{2K}\) and changing the measure \(\phi \to \phi^{\prime} = \phi - \frac{J}{K}\) to obtain
\[
    \int_{-\infty}^{+\infty} \frac{\mathrm{d} \phi}{(2\pi)^{\tfrac{1}{2}}} \, e^{-\frac{1}{2} K \phi^2 + J \phi} = \frac{1}{\sqrt{K}} e^{\frac{J^2}{2K}}, \quad \text{for } K \in \mathbb{R}^+.
\]
We used the letters \(K\) and \(J\) to indicate a “kinetic” term and a “source”, in analogy with the notation used in path integrals. The integral can be generalized to \(n\) dimensions as
\[
    \int \frac{\mathrm{d}^n \phi}{(2\pi)^{\tfrac{n}{2}}} \, e^{-\frac{1}{2} \phi^i K_{ij} \phi^j} = \det(K_{ij})^{-\frac{1}{2}}
\]
and if we include a linear term in the exponent, we have
\[
    \int \frac{\mathrm{d}^n \phi}{(2\pi)^{\tfrac{n}{2}}} \, e^{-\frac{1}{2} \phi^i K_{ij} \phi^j + J_i \phi^i} = \det(K_{ij})^{-\frac{1}{2}} e^{\frac{1}{2} J_i (K^{-1})^{ij} J_j},
\]
where \(K_{ij}\) is a symmetric\footnote{Since \(\phi^i \phi^j\) is also symmetric in \(i \leftrightarrow j\).} positive definite matrix, so it is diagonalizable with positive eigenvalues, and \((K^{-1})^{ij} = G^{ij}\) is its inverse. Repeated indices are summed over from 1 to \(n\) (Einstein summation convention).

The first integral is immediate if \(K_{ij}\) is diagonal and valid in full generality since \(K_{ij}\) is diagonalizable by an orthogonal transformation, which leaves the measure invariant. The last integral is obtained again by square completion. These gaussian integrals are suitable for euclidean path integrals. Moreover, in a hypercondensed notation (to be explained shortly), path integrals look very much like ordinary integrals. Of course, the definition of determinants for infinite dimensional matrices is delicate and requires a regularization procedure.

By analytical extension, one obtains gaussian integrals suitable for quantum mechanics
\[
    \int \frac{\mathrm{d}^n \phi}{(-2\pi i)^{\tfrac{n}{2}}} \, e^{\frac{i}{2} \phi^i K_{ij} \phi^j + i J_i \phi^i} = \det(K_{ij})^{-\frac{1}{2}} e^{-\frac{i}{2} J_i G^{ij} J_j}, \quad \text{for } \Im(K_{ij}) > 0,
\]
where the condition on the imaginary part of \(K_{ij}\) ensures convergence of the integral (and again \(K_{ij}G^{jl}=\delta_i^{\ l}\)). This condition is usually satisfied in physical applications by adding a small imaginary part to the kinetic term, known as “\(i\epsilon\) prescription” or \textbf{“Feynman prescription”}: we replace
\[
    K_{ij} \to K_{ij} + i \epsilon \delta_{ij}
\]
with \(\epsilon > 0\) infinitesimal, which ensures a gaussian dumping for \(\vert \phi \vert \to \infty \):
\[
    e^{i \phi^i K_{ij} \phi^j} \to e^{i \phi^i (K_{ij} + i \epsilon \delta_{ij}) \phi^j} = e^{i \phi^i K_{ij} \phi^j} e^{-\epsilon \vert \phi \vert^2} \xrightarrow[\vert \phi \vert \to \infty]{} 0.
\]

In a hypercondensed notation, to be explained shortly, these formulae give the formal solution of path integrals of free theories (meaning theories with quadratic actions, in this context) without gauge invariances, in either quantum mechanics or quantum field theory. Gauge invariance would produce a vanishing \(\det(K_{ij})\), and one must apply a gauge fixing procedure to obtain a finite answer.

\subsection{Hypercondensed Notation and Generating Functionals}

To proceed swiftly, it is useful to introduce a hypercondensed notation. It allows us to treat path integrals, including those for field theories, formally as ordinary integrals. The hypercondensed notation is defined by lumping together discrete and continuous indices into a single index, so that a variable \(\phi_i\) can be used as a shorthand notation for the position \(x(t)\) of the particle, identifying
\[
    x(t) \to \phi^i \implies \begin{dcases}
        x \to \phi, \\
        t \to i.
    \end{dcases}
\]

Similarly, for fields, as the vector quadripotential \(A_\mu(x^{\nu})\), the hypercondensed notation is obtained by denoting
\[
    A_\mu (x^{\nu}) \to \phi^i \implies \begin{dcases}
        A_\mu \to \phi, \\
        \mu,\, x^{\nu} \to i.
    \end{dcases}
\]
where now the index i contains a discrete part (the discrete index \(\mu = 0, 1, 2, 3\)) and a continuous part (the spacetime coordinates \(x^{\nu} = (x^0,\,x^1,\,x^2,\,x^3) \in \mathbb{R}^4\)). Indices may be lowered and raised with a metric, so that one could also write \(\phi^i \phi_i = \phi^i g_{ij} \phi^j\), as in the following examples:
\[
    \begin{aligned}
        \phi_i \phi^i & = \int \d{t} x(t)x(t) = \int \d{t} \int \d{t^{\prime}} x(t)\delta(t-t^{\prime})x(t^{\prime}),                               \\
        \phi_i \phi^i & = \int \mathrm{d}^4 x A_\mu(x) A^\mu(x) = \int \mathrm{d}^4 x \mathrm{d}^4 y A_\mu(x) \eta^{\mu\nu} \delta^4(x-y) A_\nu(y),
    \end{aligned}
\]
where in the last step we have explicited the precence of a metric, given by the identity matrix in many cases, though one may consider more general situations (as we did in the second case). Repeated indices are understood to be summed over (the Einstein summation convention). In the first case, the summation includes an integration over time, while in the second case it includes a summation over the discrete index \(\mu\) and an integration over spacetime.

One must pay attention to simple-looking expressions, as they include integrations or infinite sums, and might not converge. With such a notation at hand, we are ready to review quickly the definition of correlation functions, introduce generating functionals, and present gaussian path integration formulae. We will also describe the Wick’s theorem, which gives a simple way of computing all correlation functions in a free theory in terms of the 2-point function only (the propagator).

The path integrals in \eqref{eq:Path_integral_configuration_space}, reported here for convenience
\[
    A = \int \mathrm{D} x(t) \, e^{\frac{i}{\hbar}S[x(t)]} = \lim_{N \to \infty} \int \left(\prod_{k=1}^{N-1} \d{x_k}\right) \left(\frac{m}{2\pi i \hbar \epsilon}\right)^{\frac{N}{2}} e^{\frac{i\epsilon}{\hbar}\sum_{k=1}^N \left[\frac{m}{2}\frac{(x_k - x_{k-1})^2}{\epsilon^2} - V(x_{k-1})\right]} = Z,
\]
after denoting the variables in a hypercondensed notation by \(\phi^i\), can be written as
\begin{equation}
    Z = \int \mathrm{D} \phi \, e^{\frac{i}{\hbar} S[\phi]}
    \label{eq:partition_function_hypercondensed}
\end{equation}
and the n point correlation function (in eq. \eqref{eq:n_point_correlation_function}) reads
\begin{equation}
    \langle \phi^{i_1} \phi^{i_2} \cdots \phi^{i_n} \rangle = \frac{1}{Z[0]} \int \mathrm{D} \phi \, \phi^{i_1} \phi^{i_2} \cdots \phi^{i_n} e^{\frac{i}{\hbar} S[\phi]}.
    \label{eq:n_point_correlation_function_hypercondensed}
\end{equation}
The generating functional can be written as
\begin{equation}
    Z[J] = \int \mathrm{D} \phi \, e^{\frac{i}{\hbar} S[\phi] + J_i \phi^i},
    \label{eq:generating_functional_hypercondensed}
\end{equation}
and generates all correlation functions by differentiation (in hypercondensed notation, functional derivatives look like usual derivatives, but we keep using the symbol \(\delta\) of functional derivative)
\[
    \langle \phi^{i_1} \phi^{i_2} \cdots \phi^{i_n} \rangle = \left. \frac{1}{Z} \left(\frac{\hbar}{i}\right)^n \frac{\delta}{\delta J_{i_n}} \cdots \frac{\delta}{\delta J_{i_2}} \frac{\delta}{\delta J_{i_1}} Z[J] \right|_{J=0}.
\]

We can now define the \textbf{generating functional of connected correlation functions} \(W[J]\) by
\begin{equation}
    Z[J] = e^{\frac{i}{\hbar} W[J]}, \quad \implies \quad W[J] = - i \hbar \ln(Z[J]).
    \label{eq:generating_functional_connected}
\end{equation}
One can prove that it generates “connected” correlation functions by differentiation
\[
    \langle \phi^{i_1} \phi^{i_2} \cdots \phi^{i_n} \rangle_{c} = \left. \left(\frac{\hbar}{i}\right)^{n-1} \frac{\delta}{\delta J_{i_n}} \cdots \frac{\delta}{\delta J_{i_2}} \frac{\delta}{\delta J_{i_1}} W[J] \right|_{J=0}.
\]
We will check this statement and its meaning in the free theories (next section). It is also useful to define the \textbf{effective action} \(\Gamma[\varphi]\) as the Legendre transform of \(W[J]\)
\begin{equation}
    \Gamma[\varphi] = \min_{J} \{W[J] - J_i \varphi^i\}, \quad \text{where } \varphi^i = \frac{\delta W[J]}{\delta J_i}.
    \label{eq:effective_action}
\end{equation}
which is considered as a classical action that includes all quantum corrections. It generates the so-called one-particle irreducible (1PI) correlation functions, though we will not investigate further this particular property. The minimum in \(J\) is obtained at \(\varphi^i = \frac{\delta W[J]}{\delta J_i}\), furnishing a relation \(\varphi^i = \varphi^i(J)\) that must be inverted to obtain \(J_i = J_i(\varphi)\) and inserted back into the right-hand side of \eqref{eq:effective_action} to obtain the effective action indeed as a functional of the variable \(\varphi^i\) only. The last two functionals, \(W [J]\) and \(\Gamma[\varphi]\), find their main applications in quantum field theory. Equivalent definitions can be given for euclidean path integrals.

\subsection{Free Theories}

It is useful to study free theories, here meaning theories that have a quadratic action. They provide a simple application of the previous formulae, giving at the same time additional intuition. A free theory is described by a quadratic action
\[
    S[\phi] = - \frac{1}{2} \phi^i K_{ij} \phi^j,
\]
which produces the linear equations of motion \(K_{ij} \phi^j = 0\). We assume \(K_{ij}\) invertible, which translates to the fact that there are no gauge symmetries in our model.

As an example, consider the harmonic oscillator, whose action is
\[
    \begin{aligned}
        S[x] & = \int_{-\infty}^{\infty}\d{t} \left(\frac{\dot{x}^2}{2} - \frac{\omega^2 x^2}{2}\right) = -\frac{1}{2} \int_{-\infty}^{\infty} \d{t} \, x(t) \left(\frac{\mathrm{d}^2}{\mathrm{d} t^2} + \omega^2 \right) x(t) \\
             & = -\frac{1}{2} \int \d{t} \int \d{t^{\prime}} \, x(t) \left[ \left(\frac{\mathrm{d}^2}{\mathrm{d} t^2} + \omega^2 \right) \delta(t - t^{\prime}) \right] x(t^{\prime})                                          \\
             & = -\frac{1}{2} \int \mathrm{d}t \mathrm{d}t^{\prime} \, x(t) K(t, t^{\prime}) x(t^{\prime}),
    \end{aligned}
\]
where we integrated by parts and introduced the Dirac delta function \(\delta(t-t^{\prime})\) to expose the “kinetic matrix” \(K(t,t^{\prime}) = \left(\frac{\mathrm{d}^2}{\mathrm{d} t^2} + \omega^2 \right) \delta(t - t^{\prime})\), thus we can write the action in hypercondensed notation as
\[
    S[\phi] = - \frac{1}{2} \phi^i K_{ij} \phi^j,
\]
general for any free theory. Denoting \(\mathrm{D}\phi = \frac{\mathrm{d}^n \phi}{(-i2\pi)^{\tfrac{n}{2}}}\), setting \(\hbar = 1\) for simplicity, and using the gaussian result in eq.(57), one calculates formally the path integral with sources
\[
    Z[J] = \int \mathrm{D} \phi \, e^{\frac{i}{2} \phi^i K_{ij} \phi^j + i J_i \phi^i} = \det(K_{ij})^{-\frac{1}{2}} e^{-\frac{i}{2} J_i (K^{-1})^{ij} J_j},
\]
where the inverse kinetic matrix \((K^{-1})^{ij} = G^{ij}\) is the \textbf{Green's function} of the theory, satisfying
\[
    K_{ij} G^{jk} = \delta_i^k.
\]
We can now compute all correlation functions by differentiating with respect to the source \(J_i\)
\[
    \langle \phi^{i_1} \phi^{i_2} \cdots \phi^{i_n} \rangle = \left. \frac{1}{Z} \left(i\right)^n \frac{\delta}{\delta J_{i_n}} \cdots \frac{\delta}{\delta J_{i_2}} \frac{\delta}{\delta J_{i_1}} Z[J] \right|_{J=0},
\]
obtaining
\[
    \begin{aligned}
        \langle 1 \rangle                     & = 1,                                    \\
        \langle \phi^{i_1} \rangle            & = 0,                                    \\
        \langle \phi^{i_1} \phi^{i_2} \rangle & = -i(K^{-1})^{i_1 i_2} = -iG^{i_1 i_2}.
    \end{aligned}
\]
The first one is a consequence of the normalization, the second one reflects the symmetry \(\phi^i \to -\phi^i\), and the third one is known as the propagator, which we find proportional to the inverse of the kinetic matrix \(K_{ij}\).

Continuing with the calculation of higher point functions, we see that all correlation functions with an odd number of points vanish, again a consequence of the symmetry \(\phi^i \to -\phi^i\). Those ones with an even number \(n\) factorize into a sum of \((n-1)!!\) terms, given by the product of the 2-point functions which connect any two points in all possible ways. This fact is known as the Wick’s theorem.

\begin{theorem}[Wick’s theorem]
    \dots
\end{theorem}

For example, the 4-point correlation function is given by
\[
    \langle \phi^{i_1} \phi^{i_2} \phi^{i_3} \phi^{i_4} \rangle = \langle \phi^1 \phi^2 \rangle \langle \phi^3 \phi^4 \rangle + \langle \phi^1 \phi^3 \rangle \langle \phi^2 \phi^4 \rangle + \langle \phi^1 \phi^4 \rangle \langle \phi^2 \phi^3 \rangle,
\]
that indeed contains the sum of \(3!!\) terms. This correlation function is not connected, as each term is the product of correlation functions of lower order (it disconnects into products of lower order correlation functions). This is true for all higher point correlation functions of the free theory. The generating functional of connected correlation functions \(W[J]\) is obtained from eq. (69) using the definition (64)
\[
    W[J] = \frac{1}{2} J_i G_{ij} J_j - \Lambda, \quad \Lambda = -\frac{i}{2} \ln(\det(K_{ij})) = -\frac{i}{2}\Tr (\ln(K_{ij})),
\]
where the constant \(\Lambda\) is an infinite constant that can be regularized and interpreted as the vacuum energy of the free theory. One verifies that it generates a 2-point correlation functions that is connected.\footnote{It also generates a 0-point function, given by \(-\Lambda\), that can be shown to be connected as well.}

Let us also calculate the effective action. The minimum in \(J\) of eq. (66) is achieved for
\[
    \varphi^i = \frac{\delta W}{\delta J_i} = G^{ij} J_j \implies J_i = K_{ij} \varphi^j,
\]
so that the effective action reads
\[
    \Gamma[\varphi] = W[J] - J_i \varphi^i = -\frac{1}{2} \varphi^i K_{ij} \varphi^j + \Lambda,
\]
We see that for a free theory, the effective action \(\Gamma[\phi]\) reproduces the original action \(S[\phi]\) with an additive constant \(-\Lambda\), which could be interpreted as minus a vacuum energy, which is of quantum origin. The latter can be disregarded if gravitational interactions are neglected. In general, the effective action is considered as a classical action that contains the effects of quantization in its couplings (and thus, the effective actions should not be quantized again). Reinserting \(\hbar\) by a simple rescaling, we collect here the formulae for a free (gaussian) theory
\[
    \begin{aligned}
        S[\phi]      & = - \frac{1}{2} \phi^i K_{ij} \phi^j,                                                                 \\
        Z[J]         & = \det(K_{ij})^{-\frac{1}{2}} e^{-\frac{i}{2\hbar} J_i G^{ij} J_j},                                   \\
        W[J]         & = \frac{1}{2} J_i G^{ij} J_j - \hbar\Lambda, \quad \Lambda = -\frac{i}{2} \ln(\det(K_{ij})),          \\
        \Gamma[\phi] & = -\frac{1}{2} \varphi^i K_{ij} \varphi^j - \hbar\Lambda = S[\varphi] + \hbar [\text{ corrections }],
    \end{aligned}
\]
with a propagator given by the two point correlation function
\[
    \langle \phi^{i} \phi^{j} \rangle = -i \hbar (K^{-1})^{ij} = -i \hbar G^{ij}.
\]
\subsubsection{Harmonic Oscillator}

Let us work out in more explicit terms the case of a harmonic oscillator with unit mass and frequency \(\omega\). The action is
\[
    S[x] = \int_{-\infty}^{\infty} \d{t} \left(\frac{\dot{x}^2}{2} - \frac{\omega^2 x^2}{2}\right), \quad Z[J] = \int \mathrm{D} x \, e^{i S[x] + i \int \d{t} J(t) x(t)},
\]
where we have computed the path integral to obtain \(Z[J]\). We repeat the deduction without using the hypercondensed notation. We consider an infinite propagation time and a transition amplitude between the ground state, classically achieved for \(x=0\). The action in the exponent can be manipulated with an integration by parts without producing boundary terms (imposing that \(x(t)\) is in its classical vacuum at initial and final times gives a vanishing boundary term, another justification will be given later on when treating the euclidean version of the problem). Thus, the action takes the form we already computed
\[
    S[x] = -\frac{1}{2} \int_{-\infty}^{\infty} \d{t} \, x(t) \left(\frac{\mathrm{d}^2}{\mathrm{d} t^2} + \omega^2 \right) \delta(t-t^{\prime}) x(t) = \int \int \d{t} \d{t^{\prime}} \, x(t) K(t, t^{\prime}) x(t^{\prime}),
\]
where \(K(t, t^{\prime}) = \left(\frac{\mathrm{d}^2}{\mathrm{d} t^2} + \omega^2 \right) \delta(t - t^{\prime})\) is the differential “kinetic” operator of the harmonic oscillator. We can compute the two point correlation function (the propagator) by
\[
    \langle x(t) x(t^{\prime}) \rangle = \left. \frac{1}{Z} \left(i\right)^2 \frac{\delta}{\delta J(t^{\prime})} \frac{\delta}{\delta J(t)} Z[J] \right|_{J=0} = -i (K^{-1})(t, t^{\prime}) = -i G(t, t^{\prime}),
\]
where the inverse of \(K\) is recognized as the Green function of the differential operator, and it can be conveniently written in a Fourier transform
\[
    G(t,t^{\prime}) = \int_{-\infty}^{\infty} \frac{\mathrm{d}p}{2\pi} \, \frac{e^{-ip(t-t^{\prime})}}{-p^2 +\omega^2},
\]
which can be verified to satisfy the equation
\[
    \int \d{t^{\prime \prime}} K(t, t^{\prime \prime}) G(t^{\prime \prime}, t^{\prime}) = \delta(t - t^{\prime}).
\]
that in a hypercondensed notation would have been written as \(K_{ij} G^{jl} = \delta_i^l\). Adding the Feynman \(i\epsilon\) prescription\footnote{Mathematically, it is seen to arise from requiring that the path integral has a small damping factor \(e^{-\tfrac{i}{\hbar} \int \d{t} x^2(t)}\), obtained by the shift \(\omega^2 \to \omega^2 - i \epsilon\) with \(\epsilon \to 0^+\). Recall the comments made under eq. (57).} for specifying how to integrate around the poles \(p=\pm \omega\), one computes
\[
    G(t,t^{\prime}) = \int_{-\infty}^{\infty} \frac{\mathrm{d}p}{2\pi} \, \frac{e^{-ip(t-t^{\prime})}}{-p^2 +\omega^2 - i \epsilon} = -\frac{i}{2\omega} \left[ \theta(t-t^{\prime}) e^{-i\omega (t-t^{\prime})} + \theta(t^{\prime}-t) e^{i\omega (t-t^{\prime})} \right] = -\frac{i}{2\omega} e^{-i\omega \vert t-t^{\prime} \vert}.
\]

\subsubsection{Klein Gordon Field}

The quantum field theory of a free Klein-Gordon scalar field can be viewed as a higher-dimensional analog of the harmonic oscillator. The action of a real scalar field \(\phi(x)\) is given by
\[
    S[\phi] = \int \mathrm{d}^4 x \left( -\frac{1}{2} \partial_\mu \phi \partial^\mu \phi - \frac{1}{2} m^2 \phi^2 \right),
\]
where \(m\) is the mass of the scalar particle. Integrating by parts and neglecting boundary terms, the action can be rewritten as
\[
    S[\phi] = -\frac{1}{2} \int \mathrm{d}^4 x \, \phi(x) \left( \Box + m^2 \right) \phi(x) = -\frac{1}{2} \int \mathrm{d}^4 x \mathrm{d}^4 y \, \phi(x) K(x,y) \phi(y),
\]
where \(K(x,y) = \left( \Box + m^2 \right) \delta^4(x-y)\) is the kinetic operator of the Klein-Gordon field, and \(\Box = \partial_\mu \partial^\mu\) is the d'Alembertian operator.

We can compute the path integral with sources
\[
    Z[J] = \int \mathrm{D} \phi \, e^{i S[\phi] + i \int \mathrm{d}^4 x \, J(x) \phi(x)} = N e^{\frac{i}{2} \int \mathrm{d}^4 x \mathrm{d}^4 y \, J(x) G(x,y) J(y)},
\]
where \(N = \det(K_{ij})^{-\frac{1}{2}}\) is a normalization constant. The inverse of the kinetic operator \(K\) is the Green's function \(G(x,y)\) of the Klein-Gordon operator.

We are in \((3+1)\) spacetime dimensions, but if we imagine to be in a \((0+1)\) spacetime dimension, meaning that there is only one time coordinate and no spatial coordinates, the Klein-Gordon field theory reduces to the harmonic oscillator treated previously. Indeed, in this case the action reduces to
\[
    S[\phi] = \int \mathrm{d}t \left( \frac{1}{2} \dot{\phi}^2 - \frac{1}{2} m^2 \phi^2 \right),
\]
which is the action of a harmonic oscillator with frequency \(\omega = m\).

If we compute the exponential in the path integral, we obtain
\[
    \int \mathrm{d}^4 z \, K(x,z) G(z,y) = \delta^4(x-y),
\]
where we have indeed \(\left(\Box_x + m^2 \right) G(x-y) = \delta^4(x-y)\) as the usual equation for the Green's function of the Klein-Gordon operator.

We can now compute the two-point correlation function (the propagator) by
\[
    \langle \phi(x) \phi(y) \rangle = \left. \frac{1}{Z} \left(i\right)^2 \frac{\delta}{\delta J(y)} \frac{\delta}{\delta J(x)} Z[J] \right|_{J=0} = -i G(x,y),
\]
where the Green's function can be conveniently written in a Fourier transform
\[
    G(x,y) = \int \frac{\mathrm{d}^4 p}{(2\pi)^4} \, \frac{e^{-ip_\mu (x^{\mu} - y^{\mu})}}{p^2 + m^2 - i \epsilon},
\]
where \(\mathrm{d}^4 p = \d{p_0}\mathrm{d}^3 \mathbf{p}\) and \(p^2 = p_\mu p^{\mu} = -(p_0)^2 + \mathbf{p}^2\). There are poles corresponding to the solutions of the mass shell condition, \(p_0 = \sqrt{\mathbf{p}^2 + m^2}\) (positive energies) and \(p_0 = -\sqrt{\mathbf{p}^2 + m^2}\) (negative energies). The Feynman \(i \epsilon\) prescription sends positive energies forward in time and negative energies backward in time. It corresponds to the physical interpretation of particles and antiparticles with positive energies and always propagating forward in time. Setting \(E_p = \sqrt{\mathbf{p}^2 + m^2}\) one finds
\[
    \begin{aligned}
        \langle \phi(x) \phi(y) \rangle & =       \\
                                        & = \dots
    \end{aligned}
\]

\subsubsection{Harmonic Oscillator in Euclidean Time}

The statistical partition function in the limit of vanishing temperature (\(\Theta \to  0\)), corresponding to an infinite euclidean propagation time (\(\beta = \frac{1}{k \Theta} \to  \infty\)), takes a simple form
\[
    Z_E = \Tr e^{-\beta \hat{H}} = \sum_{n=0}^{\infty} \bra{n} e^{-\beta \hat{H}} \ket{n} = e^{-\beta E_0} + \text{ subleading terms as } \beta \to \infty.
\]
This is true even in the presence of a source \(J\) if one assumes that the source is nonvanishing for a finite interval of time only: the remaining infinite time is sufficient to project the operator \(e^{-\beta \hat{H}}\) onto the ground state. This allows to rewrite the generating functional \(Z[J]\) in the euclidean case in a simpler way, justifying the dropping of boundary terms in the integration by parts in the classical action. The statistical partition function is obtained by using periodic boundary conditions, and for large \(\beta\) one gets the projection onto the ground state
\[
    Z_E[J] = \int \mathrm{D} x \, e^{-S_E[x] + \int_{-\tfrac{\beta}{2}}^{\tfrac{\beta}{2}} \d{\tau} J(\tau) x(\tau)} \xrightarrow{\beta \to  \infty} \lim_{\beta \to \infty} e^{- \beta E_0(J)},
\]
where the euclidean action is
\[
    S_E[x] = \int_{-\infty}^{\infty} \d{\tau} \left( \frac{1}{2} \left(\frac{\mathrm{d} x}{\mathrm{d} \tau}\right)^2 + \frac{1}{2} \omega^2 x^2 \right).
\]
where \(E_0(J)\) is the ground state energy in the presence of the source \(J\). We can now repeat the previous calculation in the present context. We integrate by parts without encountering boundary terms, as the paths are closed, and the path integral is strictly gaussian
\[
    Z_E[J] = \dots
\]
from which we get the Green's function in euclidean time, which can be computed as
\[
    G_E(\tau - \tau^{\prime}) = \int_{-\infty}^{\infty} \frac{\mathrm{d}p_E}{2\pi} \, \frac{e^{-ip_E(\tau - \tau^{\prime})}}{p_E^2 + \omega^2} = \frac{1}{2\omega} e^{-\omega \vert \tau - \tau^{\prime} \vert}.
\]
Using it for the computation of the two-point correlation function
\[
    \langle x(\tau) x(\tau^{\prime}) \rangle = G_E(\tau - \tau^{\prime}) = \int \frac{\mathrm{d}p_E}{2\pi} \, \frac{e^{-ip_E(\tau - \tau^{\prime})}}{p_E^2 + \omega^2}
\]
We now verify again the relation between quantum mechanics and statistical mechanics, realized by the analytic continuation in time, the Wick rotation. The inverse Wick rotation implies \(\tau = t_E \to i t_M = it\) and \(p_E \to -ip_M = -ip\), with the latter arising from the requirement that the correct Fourier transform is kept during the analytic deformation. Thus the two-point correlation function becomes
\[
    \langle x(\tau) x(\tau^{\prime}) \rangle \to \langle x(t) x(t^{\prime}) \rangle = -i G_M (t-t^{\prime}) = -i \int \frac{\mathrm{d}p_M}{2\pi} \, \frac{e^{-ip_M(t-t^{\prime})}}{-p_M^2 + \omega^2},
\]
which is the Feynman propagator of the harmonic oscillator computed previously
\[
    \frac{1}{2 \omega} e^{-\omega \vert \tau - \tau^{\prime} \vert} \to -\frac{i}{2\omega} e^{-i\omega \vert t - t^{\prime} \vert}.
\]
We recognize that the Feynman propagator is the unique analytical extension of the euclidean two-point function. All other Green functions, such as the retarded or advanced ones, correspond to different boundary conditions implemented with different prescriptions for performing the integration around the poles. They cannot be Wick rotated, as one would encounter poles in the analytic continuation.

This is the end for free path integrals and bosonic theories. We can now proceed to interacting theories with perturbative expansions, and later on to fermionic theories.
