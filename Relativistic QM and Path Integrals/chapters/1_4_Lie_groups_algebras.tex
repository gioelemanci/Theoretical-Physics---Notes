\section{Lie Groups and Lie Algebras}

A Lie group is, by definition, a group whose elements depend continuously on some parameters. By studying the infinitesimal group transformations, i.e., those transformations that differ slightly from the identity, one obtains the so-called Lie algebra of the group, an algebra that summarizes essential information about the group. In particular, the Lie algebra captures the non-abelian structure of the group. To introduce these topics, we first study some of the simplest yet most commonly used groups in physics and then list general properties and definitions.

\subsection{SO(2)}

Consider the familiar group of rotations in two-dimensional Euclidean space, the group SO(2) of real orthogonal $2 \times 2$ matrices with determinant equal to 1. These matrices generate the transformations of a vector
\[
    \vec{x} \rightarrow \vec{x}' = R \vec{x}
    \tag{88}
\]
or in tensor notation $x'^{i} = R^{i}_{\ j} x^{j}$ with $i,j = 1,2$. This is the defining (or vector) representation. The rotations that mix the two components of the vector $\vec{x} = (x, y) = (x^1, x^2)$ depend on an angle $\theta$ and can be written as
\[
    R(\theta) =
    \begin{pmatrix}
        \cos(\theta)  & \sin(\theta) \\
        -\sin(\theta) & \cos(\theta)
    \end{pmatrix}
    = 1 + \theta
    \begin{pmatrix}
        0  & 1 \\
        -1 & 0
    \end{pmatrix} + \dots
    \tag{89}
\]
where the matrix $T$ is the operator that "generates" the infinitesimal part of the transformation
\[
    T = \frac{i}{2}
    \begin{pmatrix}
        0  & 1 \\
        -1 & 0
    \end{pmatrix}
    \tag{90}
\]

The imaginary unit $i$ in (89) is conventional but allows us to present the generator $T$ as a Hermitian matrix (whose eigenvalues are real).

The group is abelian; its elements commute, $R(\theta_1) R(\theta_2) = R(\theta_2) R(\theta_1)$, and obviously
\[
    [T, T] = 0
    \tag{91}
\]
where $[.,.]$ denotes the commutator ($[A,B] = AB - BA$). This is called the Lie algebra of SO(2). In general, the Lie algebra of a group is generated by the commutators of its infinitesimal generators.

Finite transformations can be obtained by iterating infinitesimal transformations. If the parameter $\theta$ is not infinitesimal, consider $\theta$ with $n$ large enough to make it infinitesimal. Then, one can write
\[
    [R(\theta)]^n \approx (1 + \theta T)^n \rightarrow e^{i \theta T} = \cos(\theta) + i T \sin(\theta)
    \tag{92}
\]
which reproduces the finite transformation in (89). The notation $e^{i \theta T}$, which contains the infinitesimal generator $T$ and the continuous Lie parameter $\theta$ of the group, is the exponential representation of the elements of the group SO(2). It generalizes to arbitrary Lie groups.

Here, we have obtained the Lie algebra of the group SO(2) by considering the defining representation of SO(2), which is enough to recognize the abstract SO(2) Lie algebra. Then, one can study the various representations of the SO(2) Lie algebra in terms of other matrices and classify inequivalent representations.

Note that by defining the complex number $z = x + i y$, the SO(2) transformation of $(x, y)$ takes the form of U(1) a phase transformation
\[
    z' = x' + i y' = (x \cos(\theta) + y \sin(\theta)) + i (-x \sin(\theta) + y \cos(\theta)) = (\cos(\theta) - i \sin(\theta))(x + i y) = e^{-i \theta} z
    \tag{93}
\]
The groups SO(2) and U(1) are equivalent, $SO(2) \cong U(1)$.

\subsection*{SO(3)}

Consider now the group of rotations in three-dimensional space, the group $SO(3)$ of real orthogonal $3 \times 3$ matrices with determinant equal to 1. These matrices generate transformations of a three-dimensional vector $\vec{x} \rightarrow \vec{x}' = R \vec{x}$.

Consider the rotations around the three Cartesian axes with coordinates $(x, y, z) = (x^1, x^2, x^3)$
\[
    R_x(\theta_x) =
    \begin{pmatrix}
        1 & 0               & 0              \\
        0 & \cos(\theta_x)  & \sin(\theta_x) \\
        0 & -\sin(\theta_x) & \cos(\theta_x)
    \end{pmatrix}
    = 1 + \theta_x
    \begin{pmatrix}
        0 & 0  & 0 \\
        0 & 0  & 1 \\
        0 & -1 & 0
    \end{pmatrix}
    + \dots
    \tag{94}
\]

\[
    R_y(\theta_y) =
    \begin{pmatrix}
        \cos(\theta_y) & 0 & -\sin(\theta_y) \\
        0              & 1 & 0               \\
        \sin(\theta_y) & 0 & \cos(\theta_y)
    \end{pmatrix}
    = 1 + \theta_y
    \begin{pmatrix}
        0 & 0 & -1 \\
        0 & 0 & 0  \\
        1 & 0 & 0
    \end{pmatrix}
    + \dots
    \tag{95}
\]

\[
    R_z(\theta_z) =
    \begin{pmatrix}
        \cos(\theta_z)  & \sin(\theta_z) & 0 \\
        -\sin(\theta_z) & \cos(\theta_z) & 0 \\
        0               & 0              & 1
    \end{pmatrix}
    = 1 + \theta_z
    \begin{pmatrix}
        0  & 1 & 0 \\
        -1 & 0 & 0 \\
        0  & 0 & 0
    \end{pmatrix}
    + \dots
    \tag{96}
\]
so that the generators $T^i$ of the infinitesimal transformations are given by
\[
    T^1 =
    \begin{pmatrix}
        0 & 0  & 0 \\
        0 & 0  & i \\
        0 & -i & 0
    \end{pmatrix}, \quad
    T^2 =
    \begin{pmatrix}
        0 & 0 & -i \\
        0 & 0 & 0  \\
        i & 0 & 0
    \end{pmatrix}, \quad
    T^3 =
    \begin{pmatrix}
        0  & i & 0 \\
        -i & 0 & 0 \\
        0  & 0 & 0
    \end{pmatrix}
    \tag{97}
\]

The corresponding Lie algebra is easily computed by calculating the commutators of the matrices just identified
\[
    [T^i, T^j] = i \epsilon^{ijk} T^k
    \tag{98}
\]

The right-hand side is not zero, indicating that the group is non-abelian (the group elements do not commute). The constants $\epsilon^{ijk}$ are called the structure constants of the $SO(3)$ group because they encode the non-abelian structure of the group. A finite element of the group can be parameterized in exponential form as
\[
    R(\vec{\theta}) = e^{i \theta^i T^i}
    \tag{99}
\]
where $\theta^i$ are the independent parameters of the group (a rotation of angle $\theta = \sqrt{\vec{\theta} \cdot \vec{\theta}}$ around the axis of the unit vector $\hat{n} = \vec{\theta}/\theta$).

To understand the role of the Lie algebra, let us study the product
\[
    R(\vec{\alpha}) R(\vec{\beta}) R^{-1}(\vec{\alpha}) R(\vec{\beta})
    \tag{100}
\]
that would be the identity of an abelian group. For infinitesimal parameters and working at the linear order in both $\vec{\alpha}$ and $\vec{\beta}$, one finds
\[
    R(\vec{\alpha}) R(\vec{\beta}) R^{-1}(\vec{\alpha}) R(\vec{\beta}) = 1 + \alpha^i \beta^j [T^i, T^j] + \dots
    \tag{101}
\]
which is nonvanishing for the non-abelian group $SO(3)$: the Lie algebra captures the non-commutative structure of the Lie group. In addition, one understands that the result must correspond to an infinitesimal group transformation, just like the left-hand side, so that the commutator $[T^i, T^j]$ must be proportional to a generator, as indeed verified in (98).

We have obtained the Lie algebra using the defining representation, and now we can consider it as the abstract Lie algebra of the group $SO(3)$ and study its different irreducible representations, as done for the representations of the group. From the representations of the group studied previously, one obtains the corresponding representations of the associated Lie algebra. Conversely, exponentiating the matrices of a representation of the Lie algebra yields finite transformations that provide a representation of the group\footnote{Except for possible topological obstructions that might prevent the representation from being truly single-valued. This situation is exemplified by the spinor representations of $SO(3)$, which, as we will see later, are true (i.e., single-valued) representations of the $SU(2)$ group only.}.

Let us comment on the $SO(3)$ Lie algebra and relate it to known topics studied in quantum mechanics. In equation (98), we recognize the algebra of the quantum angular momentum operator. Renaming $T^i \rightarrow L^i$, we recognize the familiar algebra of the angular momentum (in units of $\hbar = 1$)
\[
    [L^i, L^j] = i \epsilon^{ijk} L^k
    \tag{102}
\]

The study of its irreducible unitary representations is solved explicitly using the methods of quantum mechanics: the known result is that these irreducible representations are those given by the spherical harmonics $|l, m\rangle \sim Y_{lm}$, which for fixed $l$ form a basis of the spin-$l$ representation. It is $(2l + 1)$-dimensional, as for fixed $l$ the possible values of $m$ are $2l + 1$:
\[
    Y_{lm} = [R^{(l)}(\theta)]_{lm'} Y_{lm'}, \quad l \text{ fixed}, \quad m, m' \in [-l, -l+1, \dots, 0, \dots, l]
    \tag{103}
\]

In the case of spinorial representations (i.e., with half-integer spin, i.e., with $l \rightarrow j$ and $j$ half-integer), a rotation by $2\pi$ (which for $SO(3)$ coincides with the identity) is represented by the matrix $-1$, and thus we speak of a 2-valued representation (one needs to rotate by another $2\pi$ to get back to the identity). As we will see, these spinorial representations are true representations of the $SU(2)$ group, which has the same Lie algebra as $SO(3)$ and therefore has the same local structure but different global properties.

\subsection*{U(1)}

Consider the group $U(1) = \{ e^{i\theta} \mid \theta \in [0, 2\pi] \}$, the group of phases defined via its defining representation. For infinitesimal transformations
\[
    e^{i\theta} = 1 + i\theta + \dots
    \tag{107}
\]
the infinitesimal generator is given by $T = 1$ (which we can think of as a $1 \times 1$ matrix), which produces the Abelian Lie algebra of the $U(1)$ group given by the commutator
\[
    [T, T] = 0
    \tag{108}
\]

In the charge $q$ representation, where the element $e^{i\theta}$ is represented by $e^{iq\theta}$, the infinitesimal generator is represented by $T = q$ and satisfies the same Lie algebra (108). Therefore, we can think of the Lie algebra $[T, T] = 0$ as the abstract Lie algebra corresponding to the $U(1)$ group, which is represented by different matrices in different representations. Since the irreducible representations of the $U(1)$ group are all one-dimensional, all these matrices are $1 \times 1$ matrices and thus are simply numbers. In the charge $q$ representation, the generator of $U(1)$ is represented by $T = q$. It is also common to use the notation $Q$ (which often denotes a charge) instead of $T$ for the generator of the $U(1)$ group. The groups $U(1)$ and $SO(2)$ identify the same Abelian Lie group, as already described.

\subsection*{SU(2)}

Let's now analyze the group $SU(2)$, the group of $2 \times 2$ unitary matrices with unit determinant:
\[
    SU(2) = \{ U \text{ complex matrices } 2 \times 2 \mid U^\dagger = U^{-1}, \ \det U = 1 \}
    \tag{109}
\]
We can write the matrices that differ infinitesimally from the identity matrix as
\[
    U = 1 + i T, \quad T \ll 1
    \tag{110}
\]

Now, the requirement that $U^\dagger = 1 - i T^\dagger$ coincides with $U^{-1} = 1 - i T$ implies that the matrices $T$ must be Hermitian:
\[
    T^\dagger = T
    \tag{111}
\]
while the requirement for unit determinant, $\det U = 1 + i \operatorname{tr} T = 1$, implies that these matrices must be traceless:
\[
    \operatorname{tr} T = 0
    \tag{112}
\]

A basis of Hermitian traceless $2 \times 2$ matrices is given by the Pauli matrices:
\[
    \sigma_1 =
    \begin{pmatrix}
        0 & 1 \\
        1 & 0
    \end{pmatrix}, \quad
    \sigma_2 =
    \begin{pmatrix}
        0 & -i \\
        i & 0
    \end{pmatrix}, \quad
    \sigma_3 =
    \begin{pmatrix}
        1 & 0  \\
        0 & -1
    \end{pmatrix}
    \tag{113}
\]
so we can express an arbitrary matrix $T$ as a linear combination of the $\sigma^a$:
\[
    T = \theta^a T^a, \quad T^a = \frac{1}{2} \sigma^a, \quad a = 1,2,3
    \tag{114}
\]

The normalization has been chosen to satisfy
\[
    \operatorname{Tr}(T^a T^b) = \frac{1}{2} \delta^{ab}
    \tag{115}
\]
With this normalization, the infinitesimal generators $T^a = \frac{1}{2} \sigma^a$ give rise to the following $SU(2)$ Lie algebra:
\[
    [T^a, T^b] = i \epsilon^{abc} T^c
    \tag{116}
\]
which is recognized to coincide with the Lie algebra of $SO(3)$. This shows that locally they are similar (they have the same structure constants), although globally there are differences: using the language of differential geometry, we can say that the group $SU(2)$ is a double cover of the group $SO(3)$. This difference is seen explicitly in the defining representation of $SU(2)$ (the spin-$\frac{1}{2}$ or $\mathbf{2}$ representation). A finite rotation is obtained by exponentiating infinitesimal transformations to make them finite:
\[
    U(\vec{\theta}) = \exp(i \theta^a T^a)
    \tag{117}
\]
In particular, a finite rotation around the $z$-axis is obtained by choosing $\theta^3 = \theta$ and $\theta^1 = \theta^2 = 0$, to find a matrix $U_3(\theta)$ given by
\[
    U_3(\theta) = e^{i \theta T^3} = \sum_{n=0}^\infty \frac{(i \theta T^3)^n}{n!}
    \tag{118}
\]
Setting $\theta = 2\pi$ gives the transformation
\[
    U_3(\theta = 2\pi) = -1
    \tag{119}
\]
which does not coincide with the identity in $SU(2)$. The identity transformation is obtained only for $\theta = 4\pi$. As known from quantum mechanics, all irreducible unitary representations of $SU(2)$ are characterized by a quantum number $j$ that can be either an integer or a half-integer. They are of dimension $2j + 1$.

\textbf{Historical Note:} Pauli introduced the matrices in (113) to describe the electron's spin, defining the spin operator $\vec{S} = \frac{1}{2} \vec{\sigma}$, which acts on a two-component wave function (spinor).

\subsection*{SU(3)}

The same analysis performed to extract the infinitesimal generators of $SU(2)$ applies also to the general $SU(N)$ group, whose generators are then seen to be traceless, Hermitian, $N \times N$ matrices. There are $N^2 - 1$ of such matrices, so that there are $N^2 - 1$ independent Lie parameters for the group $SU(N)$. In particular, the eight infinitesimal generators of $SU(3)$ in the fundamental representation are given by the Gell-Mann matrices $\lambda^a$, which form a basis of Hermitian $3 \times 3$ traceless matrices (generalizing the Pauli matrices $\sigma^a$ for $SU(2)$):
\[
    T^a = \frac{1}{2} \lambda^a, \quad a = 1, \dots, 8
    \tag{120}
\]
These matrices are normalized so that
\[
    \operatorname{Tr}(T^a T^b) = \frac{1}{2} \delta^{ab}
    \tag{122}
\]
just as was done for $SU(2)$, see eq. (115). An arbitrary element of the $SU(3)$ group in the fundamental representation is thus described by $3 \times 3$ matrices of the form
\[
    U(\theta) = \exp(i \theta^a T^a)
\]
where $\theta^a$ with $a = 1, \dots, 8$ are the eight parameters of the group. By calculating the Lie algebra, one finds the structure constants $f^{abc}$ that correspond to the $SU(3)$ group:
\[
    [T^a, T^b] = i f^{abc} T^c
    \tag{123}
\]
They are antisymmetric and given by:
\[
    f^{123} = 1, \quad
    f^{147} = -f^{156} = f^{246} = f^{257} = f^{345} = -f^{367} = \frac{1}{2}, \quad
    f^{458} = f^{678} = \frac{\sqrt{3}}{2}
    \tag{124}
\]
while all other $f^{abc}$ not related to these by permuting indices are zero. As an exercise, try to verify some of these numbers. This group has important applications in the description of color associated with strong interactions and in the quark model that classifies the hadrons composed of the three lightest flavors of quarks (up, down, strange).

\subsection*{General Case}

We summarize for arbitrary Lie groups what was illustrated above through examples. A Lie group is, by definition, a group of transformations that depend continuously on some parameters. By studying the infinitesimal transformations of the group, i.e., transformations that differ only slightly from the identity, we recognize the generators, operators that "generate" the infinitesimal transformations. They identify the so-called Lie algebra of the group, which summarizes information about the group.

In general, an element $g(\theta)$ of a Lie group $G$ (or, more precisely, of the component connected to the identity) can be parametrized in the following way:
\[
    g(\theta) = e^{i \theta^a T^a}, \quad a = 1, \dots, \dim G
    \tag{125}
\]
where the parameters $\theta^a$ are real numbers that parametrize the various elements of the group. They are chosen so that for $\theta^a = 0$ one gets the identity $g = 1$. The operators $T^a$ are the generators of the group. Considering the group as a group of $N \times N$ matrices for some $N$ (for example, the defining representation), the generators are also $N \times N$ matrices. They generate infinitesimal transformations when $\theta^a \ll 1$. Simply expand the exponential function in a Taylor series and keep the lowest order terms:
\[
    g(\theta) = 1 + i \theta^a T^a + \dots
    \tag{126}
\]

By studying the relations that capture the composition properties of the group using infinitesimal transformations (which are generically non-commutative), one obtains the Lie algebra of the group $G$:
\[
    [T^a, T^b] = i f^{ab}_{\ \ c} T^c
    \tag{127}
\]

The constants $f^{ab}_{\ \ c}$ are called structure constants of the group and characterize it. Groups with the same Lie algebra may only differ in their topology but are locally similar. It is useful to mention the Jacobi identities:
\[
    f^{ab}_{\ \ d} f^{de}_{\ \ c} + f^{bc}_{\ \ d} f^{da}_{\ \ e} + f^{ca}_{\ \ d} f^{db}_{\ \ e} = 0
    \tag{128}
\]
which are quadratic relations satisfied by the structure constants and emerge as a consequence of the operator identities:
\[
    [[T^a, T^b], T^c] + [[T^b, T^c], T^a] + [[T^c, T^a], T^b] = 0
    \tag{129}
\]

The structure constants can be used to define the adjoint representation $T^{(A)}$ of the Lie algebra, given by the formula:
\[
    (T^{(A)})^a_{\ bc} = -i f^{abc}
    \tag{130}
\]

It is verified to be a representation of the Lie algebra thanks to the Jacobi identities. It is a real representation because the structure constants are real numbers, and it is a representation of dimension equal to the dimension of the group, since the indices $a, b, c = 1, \dots, \dim G$.

Finally, it is useful to mention the Baker-Campbell-Hausdorff formula for the product of exponentials of two linear operators $A$ and $B$:
\[
    e^A e^B = e^{A + B + \frac{1}{2}[A, B] + \frac{1}{12}[A, [A, B]] - \frac{1}{12}[B, [A, B]] + \dots}
    \tag{131}
\]
where the dots indicate higher-order terms, always expressible in terms of commutators. This formula shows that the knowledge of the Lie algebra is sufficient to reconstruct the (generally non-commutative) product of the elements of the corresponding Lie group.

To summarize, let us list and review some of the main definitions and properties of Lie algebras:

\begin{enumerate}
    \item $g = \exp(i \theta^a T^a) \in G$, \quad $a = 1, \dots, \dim G$
    \item $[T^a, T^b] = i f^{ab}_{\ \ c} T^c$
    \item $\operatorname{tr}(T^a T^b) = \kappa^{ab}$ \quad (generators in the fundamental representation)
    \item $[[T^a, T^b], T^c] + [[T^b, T^c], T^a] + [[T^c, T^a], T^b] = 0 \Rightarrow f^{ab}_{\ \ d} f^{dc}_{\ \ e} + f^{bc}_{\ \ d} f^{da}_{\ \ e} + f^{ca}_{\ \ d} f^{db}_{\ \ e} = 0$
    \item $f^{abc} = f^{ab}_{\ \ d} \kappa^{dc}$ \quad (completely antisymmetric tensor)
\end{enumerate}

Point (i) describes the exponential parametrization of an arbitrary element of the group that is connected to the identity. The index $a$ takes as many values as the dimensions of the group. An element of the group is parametrized by the parameters $\theta^a$ with $a = 1, \dots, \dim G$.

Point (ii) corresponds to the Lie algebra satisfied by the infinitesimal generators $T^a$. The constants $f^{ab}_{\ \ c}$ are called structure constants and characterize the group $G$. They are antisymmetric on indices $a$ and $b$.

Point (iii) identifies (the inverse of) a metric $\kappa^{ab}$ called the "Killing metric". This metric is positive-definite only for compact and simple Lie groups, such as $SU(N)$ or $SO(N)$. Being positive, it is often normalized to the Kronecker delta: $\kappa^{ab} = \delta^{ab}$.

Point (iv) amounts to the so-called "Jacobi identities" satisfied by the structure constants. They can be used to construct the adjoint representation of the Lie algebra. Denoting by $(T^{(A)})^a_{\ bc}$ the matrix elements of the generators of the adjoint representation $T^{(A)}$, we have $(T^{(A)})^a_{\ bc} = -i f^{abc}$. The Jacobi identities imply that this is a representation. It is real and of dimension equal to the dimension of the group since the indices $a, b, c = 1, \dots, \dim G$. By exponentiation, it gives rise to a representation of the group.

In point (v), the Killing metric is used to raise an index of the structure constants. Then, $f^{abc}$ are completely antisymmetric in all indices: antisymmetry in the indices $a$ and $b$ is obvious from (ii), while antisymmetry in the indices $b$ and $c$ is deduced by taking the trace of the Jacobi identities in (iv) and using (ii) and (iii).

Finally, we conclude with the statement of a theorem which we shall not prove: The unitary irreducible representations of compact groups are finite-dimensional, while the unitary representations of non-compact groups must be infinite-dimensional.

Thus, compact groups such as $SO(N)$ and $SU(N)$ have unitary finite-dimensional irreps. Non-compact groups, such as the Lorentz group $SO(3,1)$ and the Poincaré group $ISO(3,1)$, have unitary representations that must be infinite-dimensional. For applications in relativistic field theory, it is useful to have some knowledge of:

\begin{itemize}
    \item[(i)] the finite-dimensional representations of the Lorentz group. They are not unitary and are used to label the quantum fields that define a given relativistic QFT,
    \item[(ii)] the unitary representations of the Poincaré group, which are infinite-dimensional and are realized in the Hilbert space of quantum field theories via unitary operators.
\end{itemize}