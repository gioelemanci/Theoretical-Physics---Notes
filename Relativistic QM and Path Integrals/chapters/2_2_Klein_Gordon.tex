\section{Spin 0: Klein-Gordon equation}

As discussed previously, the Schrödinger equation arises from the quantization of a \textit{non-relativistic} particle.
In complete analogy, the \textbf{Klein–Gordon equation} can be obtained by applying the same quantization procedure to a \textit{relativistic} particle (this is often referred to as \textit{first quantization} of the relativistic theory).

However, unlike the Schrödinger equation, the Klein–Gordon equation does not allow for a consistent \textbf{probabilistic interpretation} in terms of a positive-definite probability density.
This limitation signals that the equation, while formally correct as a relativistic wave equation, cannot be the final description of a single-particle quantum system.

A consistent framework is recovered by interpreting the Klein–Gordon \textit{wave function} as a \textbf{classical field} and quantizing this field itself.
In this approach—known as \textbf{second quantization} or \textbf{field quantization}—the field is promoted to an operator acting on a Fock space, describing states with an arbitrary number of identical particles and antiparticles.
Historically, this formalism first appeared in the quantization of the electromagnetic field, and it naturally extends to the Klein–Gordon field.

In the language of quantum field theory, the Klein–Gordon field thus represents a system of \textbf{spin-0 particles} and their corresponding antiparticles.
Nevertheless, even if we restrict ourselves to the first-quantized formulation, the Klein–Gordon equation remains of fundamental importance, as it encapsulates the essential features of the quantum mechanics of relativistic scalar particles and provides the starting point for the construction of quantum field theory.

\subsection{Derivation of the Klein-Gordon equation}

How can we obtain a relativistic version of the wave equation?
A natural starting point is to repeat the same procedure used for the Schrödinger equation, but now employing the \textbf{relativistic energy–momentum relation}.
For a free relativistic particle of mass \( m \), the four-momentum \( p^\mu = (p^0, p^1, p^2, p^3) = (E, \mathbf{p}) \) satisfies the \textbf{mass-shell condition}
\[
    p^\mu p_\mu = -m^2 c^2 \quad \Rightarrow \quad E^2 = \mathbf{p}^2 c^2 + m^2 c^4.
\]

One could then try to promote this classical relation to a wave equation by using the correspondence
\[
    E \rightarrow i\hbar \frac{\partial}{\partial t},
    \qquad
    \mathbf{p} \rightarrow -i\hbar \nabla.
\]
If we apply this substitution directly to the relativistic energy expression \( E = \sqrt{\mathbf{p}^2 c^2 + m^2 c^4} \), we obtain
\[
    i\hbar \frac{\partial}{\partial t} \psi(\mathbf{x},t)
    = \sqrt{-\hbar^2 c^2 \nabla^2 + m^2 c^4} \, \psi(\mathbf{x},t).
\]
However, this equation involves the \textit{square root of a differential operator}, whose precise meaning is not well defined.
Such an operator would introduce \textbf{non-local effects}, implying that distant points in space could directly influence each other—an undesirable and physically obscure feature.
For this reason, this form was soon abandoned.

A more elegant and mathematically tractable approach was proposed independently by \textbf{Oskar Klein} and \textbf{Walter Gordon}.
Instead of working with the square-root form, they started from the \textit{quadratic} energy–momentum relation
\[
    E^2 = \mathbf{p}^2 c^2 + m^2 c^4,
\]
and replaced \( E \) and \( \mathbf{p} \) by their corresponding operators.
This leads to the \textbf{Klein–Gordon equation}:
\begin{equation}
    \left( -\frac{1}{c^2} \frac{\partial^2}{\partial t^2} + \nabla^2 - \frac{m^2 c^2}{\hbar^2} \right) \psi(\mathbf{x},t) = 0.
\end{equation}
In a more compact and manifestly relativistic form, it can be written as
\begin{equation}
    (\Box - m^2) \, \phi(x) = 0,
\end{equation}
where the \textbf{d'Alembertian operator} is defined as
\[
    \Box = \partial^\mu \partial_\mu = -\partial_0^2 + \nabla^2.
\]

From now on, we adopt \textbf{natural units} with \( \hbar = c = 1 \), so that the equation takes the simple form
\[
    (\Box - m^2)\phi = 0,
\]
and the mass \( m \) becomes dimensionless, unless otherwise specified.

\subsection{Plane wave solutions}

The Klein–Gordon equation was constructed precisely to ensure that it admits plane-wave solutions satisfying the correct relativistic dispersion relation between energy and momentum:
\[
    (\Box - m^2)\,\phi(x) = 0.
\]

To find such solutions, let us assume a \textbf{plane wave ansatz} of the form
\begin{equation}
    \phi_p(x) = e^{i p_\mu x^\mu},
\end{equation}
where \( p_\mu \) is the four-momentum of the particle.
Substituting this expression into the Klein–Gordon equation gives
\[
    (\partial^\mu \partial_\mu - m^2)\, e^{i p_\nu x^\nu}
    = ((ip^\mu)(ip_\mu) - m^2)\, e^{i p_\nu x^\nu}
    = -(p^\mu p_\mu + m^2)\, e^{i p_\nu x^\nu} = 0.
\]
This condition is satisfied if the four-momentum \( p^\mu \) lies on the \textbf{mass shell}, that is,
\[
    p^\mu p_\mu = -m^2
    \quad \Rightarrow \quad
    (p^0)^2 = \mathbf{p}^2 + m^2
    \quad \Rightarrow \quad
    p^0 = \pm E_p, \qquad
    E_p = \sqrt{\mathbf{p}^2 + m^2}.
\]

Hence, the Klein–Gordon equation admits both \textbf{positive-} and \textbf{negative-energy solutions}.
The positive-energy modes correspond to
\[
    \phi(x) = e^{-i E_p t + i \mathbf{p} \cdot \mathbf{x}},
\]
while the negative-energy modes are given by
\[
    \phi(x) = e^{i E_p t - i \mathbf{p} \cdot \mathbf{x}}.
\]

At first sight, the existence of solutions with \( p^0 = -E_p \) seems problematic: if negative-energy states exist, the system could in principle decay indefinitely into lower and lower energy levels, making the theory \textbf{unstable}.
In the framework of relativistic quantum field theory, however, these modes acquire a consistent interpretation: they correspond to \textbf{antiparticles} carrying positive energy but opposite charge or quantum numbers.

The most general solution of the Klein–Gordon equation can therefore be expressed as a linear superposition of all plane waves with both signs of energy:
\begin{equation}
    \phi(x) = \int \frac{\mathrm{d}^3 p}{(2\pi)^3 2E_p}
    \left[
        a(\mathbf{p})\, e^{-i E_p t + i \mathbf{p} \cdot \mathbf{x}}
        + b^*(\mathbf{p})\, e^{i E_p t - i \mathbf{p} \cdot \mathbf{x}}
        \right],
\end{equation}
and its complex conjugate as
\[
    \phi^*(x) = \int \frac{\mathrm{d}^3 p}{(2\pi)^3 2E_p}
    \left[
        a^*(\mathbf{p})\, e^{i E_p t - i \mathbf{p} \cdot \mathbf{x}}
        + b(\mathbf{p})\, e^{-i E_p t + i \mathbf{p} \cdot \mathbf{x}}
        \right].
\]
Here, \( a(\mathbf{p}) \) and \( b(\mathbf{p}) \) are \textbf{Fourier coefficients} that determine the relative weight of each momentum mode.
The normalization factor \( 2E_p \) is conventional and chosen so that the coefficients transform as Lorentz scalars.
Finally, for a \textbf{real scalar field}, where \( \phi^* = \phi \), the coefficients of positive- and negative-energy modes coincide:
\[
    a(\mathbf{p}) = b(\mathbf{p}).
\]

\subsection{Continuity equation}

From the Klein–Gordon equation one can derive a continuity equation, although the associated conserved quantity cannot be interpreted as a probability density. Let us examine this in detail.

A straightforward way to obtain the continuity equation is to multiply the Klein–Gordon equation by the complex conjugate field $\phi^*$ and subtract from it the complex conjugate equation multiplied by $\phi$. One finds:
\[
    \begin{aligned}
        \phi^* (\Box - m^2) \phi - \phi (\Box - m^2) \phi^* & = \phi^*\Box\phi - m^2 \phi\phi^* - \phi\Box\phi^* + m^2 \phi\phi^* = \phi^* \partial_\mu \partial^{\mu} \phi - \phi \partial_\nu \partial^{\nu} \phi^*                                                        \\
                                                            & = \partial_\mu \left(\phi^* \partial^{\mu} \phi\right) - \partial_\mu \phi^* \partial^{\mu} \phi - \left(\partial_\nu \left(\phi^* \partial^{\nu} \phi\right) - \partial_\nu \phi^* \partial^{\nu} \phi\right) \\
                                                            & =\partial_\mu \left( \phi^* \partial^\mu \phi - \phi \partial^\mu \phi^* \right) = 0.
    \end{aligned}
\]
We can thus define the four-current
\begin{equation}
    J^\mu = \frac{1}{2 i m} \big( \phi^* \partial^\mu \phi - \phi \partial^\mu \phi^* \big),
\end{equation}
which satisfies the conservation law $\partial_\mu J^\mu = 0$.
The normalization factor is chosen so that $J^\mu$ is real and, in the nonrelativistic limit, reproduces the form of the probability current in the Schrödinger theory.

The temporal component reads
\begin{equation}
    J^0 = \frac{1}{2 i m} \big( \phi^* \partial^0 \phi - \phi \partial^0 \phi^* \big),
\end{equation}
which, while real, is not positive definite. Indeed, both $\phi$ and its time derivative $\partial_0 \phi$ can be freely specified as initial conditions, since the Klein–Gordon equation is second order in time. Therefore, $J^0$ can take either positive or negative values depending on these data. Evaluating $J^0$ for plane-wave solutions explicitly yields
\begin{equation}
    J^0 = \pm \frac{E_p}{m},
\end{equation}
confirming that its sign depends on the energy branch chosen.

We thus conclude that the Klein–Gordon equation cannot sustain a probabilistic interpretation. This difficulty motivated Dirac to search for a different relativistic wave equation that would preserve a positive-definite probability density. He succeeded, but it later became clear that all relativistic wave equations should be reinterpreted as classical field equations to be quantized anew, describing particles of mass $m$ as the quanta of those fields—an idea reminiscent of Einstein’s interpretation of electromagnetic waves in the photoelectric effect.
Historically, this field interpretation was first successfully applied by Yukawa in 1935, who employed the Klein–Gordon field to model nuclear interactions mediated by short-range forces.

\subsection{Yukawa potential}

Let us now consider the Klein–Gordon equation in the presence of a static, point-like source:
\[
    (\Box - m^2)\phi(x) = g\, \delta^3(\mathbf{x}),
\]
where the source is located at the origin of the coordinate system and $g$ characterizes the strength of its coupling to the Klein–Gordon field.
Since the source is static, we can look for a time-independent solution, in which case the equation reduces to
\begin{equation}
    (\nabla^2 - m^2)\phi(\mathbf{x}) = g\, \delta^3(\mathbf{x}).
    \label{eq:Klein_Gordon_static_source}
\end{equation}

This equation can be solved by Fourier transform, yielding the so-called \textit{Yukawa potential}:
\begin{equation}
    \phi(r) = \frac{g}{4\pi} \frac{e^{-m r}}{r},
\end{equation}
where the exponential factor introduces an effective cut-off in the interaction range.

To derive this result, we start by expressing the field as a Fourier transform
\[
    \phi(\mathbf{x}) = \int \frac{\mathrm{d}^3 k}{(2\pi)^3}\, e^{i \mathbf{k} \cdot \mathbf{x}} \tilde{\phi}(\mathbf{k}),
\]
and recall that the Fourier transform of the Dirac delta distribution is
\[
    \delta^3(\mathbf{x}) = \int \frac{\mathrm{d}^3 k}{(2\pi)^3}\, e^{i \mathbf{k} \cdot \mathbf{x}}.
\]
Substituting these expressions into the equation gives
\[
    \int \frac{\mathrm{d}^3 k}{(2\pi)^3}\, (-k^2 -m^2) e^{i \mathbf{k} \cdot \mathbf{x}} \tilde{\phi}(\mathbf{k}) = g \int \frac{\mathrm{d}^3 k}{(2\pi)^3}\, e^{i \mathbf{k} \cdot \mathbf{x}},
\]
\[
    \tilde{\phi}(\mathbf{k}) = \frac{-g}{k^2 + m^2}.
\]

The inverse transform can then be evaluated explicitly in spherical coordinates (\(r = \sqrt{|\mathbf{x}|^2}\)):
\[
    \begin{aligned}
        \phi(\mathbf{x}) & = \int \frac{\mathrm{d}^3 k}{(2\pi)^3}\, e^{i \mathbf{k} \cdot \mathbf{x}} \frac{-g}{k^2 + m^2}                                            \\
                         & = -\frac{g}{(2\pi)^3} \int_0^{\infty} \d{k} \frac{k^2}{k^2 + m^2} \int_{-1}^{1} \d{\cos \theta} e^{ikr \cos \theta} \int_0^{2\pi} \d{\psi} \\
                         & = -\frac{g}{(2\pi)^2} \int_0^{\infty} \d{k} \frac{k^2}{k^2 + m^2} \left[ \frac{e^{ikr} - e^{-ikr}}{ikr} \right],
    \end{aligned}
\]
which is the integral of an even function, since the integrand can be expressed in terms of \(2 \sin(kr)\). Moreover, by splitting the integral into the terms involving \(k e^{ikr}\) and \(k e^{-ikr}\), we observe that the latter, when integrated from \(0 \to \infty\), is identical to the former integrated from \(-\infty \to 0\). Therefore, the integration domain can be extended from \(0 \to \infty\) to \(-\infty \to \infty\) writing
\[
    \phi(\mathbf{x}) = -\frac{g}{(2\pi)^2 ir} \int_{-\infty}^{\infty} \d{k} \frac{k}{k^2 + m^2}e^{ikr}.
\]

\begin{figure}[H]
    \begin{minipage}{0.5\textwidth}
        It is convenient to study this expression in the complex plane: by closing the contour with an arc \(C\) in the upper half-plane, we can integrate on a closed curve and use \textbf{Cauchy’s residue theorem}:
        \[
            \frac{k}{k^2 + m^2}e^{ikr} = \frac{k}{(k+im)(k-im)}e^{ikr},
        \]
        which has two poles \(\bar{k} = \pm im = k^{\pm}\).
    \end{minipage}
    \hfill
    \begin{minipage}{0.45\textwidth}
        \centering
        \begin{tikzpicture}[scale=0.9]
            \draw[->] (-2.5,0) -- (2.5,0) node[right] {$\Re(k)$};
            \draw[->] (0,-0.5) -- (0,2.5) node[above] {$\Im(k)$};

            \draw[thick,->,green] (-2.2,0) -- (-1,0);
            \draw[thick,->,green] (-1,0) -- (1,0) node[below] {$I$};
            \draw[thick,green] (1,0) -- (2.2,0);

            \draw[thick,orange,
                postaction={decorate},
                decoration={markings, mark=at position 0.4 with {\arrow{<}}}
            ] (-2.2,0) arc[start angle=180,end angle=0,radius=2.2] node[above right] {$C$};

            \fill[red] (0,0.8) circle (2pt) node[right] {$k^+=im$};
            \fill[black] (-2.2,0) circle (1pt) node[below] {\small $-(L \to \infty)$};
            \fill[black] (2.2,0) circle (1pt) node[below] {\small $L \to  \infty $};
        \end{tikzpicture}
    \end{minipage}
\end{figure}

Hence, we obtain:
\[
    \begin{aligned}
        \phi(\mathbf{x}) & = -\frac{g}{(2\pi)^2 ir} (2\pi i)\sum_{i} \mathrm{Res} f(k) \Big|_{k^+}  \\
                         & = -\frac{g}{2\pi r} \lim_{k \to im} \frac{k e^{ikr}}{(k+im)(k-im)}(k-im) \\
                         & = -\frac{g}{2\pi r} \frac{im e^{-mr}}{2im} = -\frac{g}{4\pi} e^{-mr}.
    \end{aligned}
\]

This potential describes an attractive interaction between sources of the same sign. Its range is finite and characterized by
\[
    \lambda \sim \frac{1}{m},
\]
corresponding to the \textbf{Compton wavelength} of a particle of mass $m$. Hence, the Yukawa potential models \textit{short-range forces}, such as the nuclear force.
One can verify that \(\phi(\mathbf{x}) = -\tfrac{g}{4\pi} e^{-mr}\) satisfies equation \eqref{eq:Klein_Gordon_static_source} by direct substitution.

\paragraph{Feynman diagrams.} A pictorial representation of the interaction between two scalar charges $g_1$ and $g_2$ mediated by the exchange of a Klein–Gordon quantum can be represented by the following Feynman-like diagram:

\begin{figure}[H]
    \begin{minipage}{0.45\textwidth}
        \centering
        \begin{tikzpicture}[scale=1.3]
            % Worldlines
            \draw[thick] (-2,1) -- (2,1);   % (g1)
            \draw[thick] (-2,-1) -- (2,-1); % (g2)

            % Dashed interaction line (the exchanged field)
            \draw[dashed, thick] (0,1) -- (0,-1);

            % Labels
            \node at (-2.3,1) {$g_1$};
            \node at (-2.3,-1) {$g_2$};
            \node at (0.3,0) {$\phi$};
        \end{tikzpicture}
    \end{minipage}
    \hfill
    \begin{minipage}{0.5\textwidth}
        This diagram illustrates the exchange of a virtual scalar quantum between the worldlines of two point particles carrying charges $g_1$ and $g_2$.
        The resulting interaction potential is
        \begin{equation}
            V(r) = -\frac{g_1 g_2}{4\pi} \frac{e^{-m r}}{r},
        \end{equation}
        which is attractive for charges of the same sign.
    \end{minipage}
\end{figure}
The potential has a characteristic range
\[
    R \sim \frac{1}{m},
\]
and therefore describes short-range forces.

In 1935, Yukawa proposed that nuclear interactions arise from the exchange of a massive scalar particle, later identified as the \textit{meson}.
By estimating a typical range $R \sim 1$~fm (comparable to the proton radius), one obtains a mass
\[
    m \sim 197~\text{MeV},
\]
in remarkable agreement with the mass of the neutral pion, $m_{\pi^0} \approx 135~\text{MeV}$, discovered later in cosmic-ray experiments.

\subsection{Green functions and the propagator}

The Green functions of the KG equation are relevant for a quantum interpretation of the KG field (we refrain from calling it the KG wave function, as the probabilistic interpretation is untenable). A particular Green function $G(x - y)$ is associated with the so-called propagator, which is interpreted as the amplitude for propagating a quantum of the field from a spacetime point $y$ to another point $x$. The Green function $G(x)$ is defined as the solution of the KG equation in the presence of a pointlike and instantaneous source of unit charge, which, for simplicity, is located at the origin of the coordinate system ($y = 0$). Mathematically, it is defined to satisfy the equation
\begin{equation}
    (-\Box + m^2) G(x) = \delta^4(x).
\end{equation}

Knowing the Green function $G(x)$, one can represent a solution of the non-homogeneous KG equation
\begin{equation}
    (-\Box + m^2) \phi(x) = J(x),
\end{equation}
where $J(x)$ is an arbitrary function (a source), by
\begin{equation}
    \phi(x) = \phi_0(x) + \int d^4 y \, G(x - y) J(y),
\end{equation}
with $\phi_0(x)$ a solution of the associated homogeneous equation. This statement is verified inserting (38) in (37), and using the property (36).

In general, the Green function is not unique for hyperbolic differential equations, but it depends on the boundary conditions chosen at infinity. In the correct quantum interpretation, the causal conditions devised by Feynman and Stueckelberg are used. They allow us to interpret the negative energy solutions as related to antiparticles. These boundary conditions require to propagate forward in time the positive frequencies generated by the source $J(x)$, and back in time the remaining negative frequencies. In a Fourier transform, the solution is written as
\begin{equation}
    G(x) = \int \frac{d^4 p}{(2\pi)^4} \frac{e^{i p \cdot x}}{p^2 + m^2 - i\epsilon},
\end{equation}
where $\epsilon \rightarrow 0^+$ is a positive infinitesimal parameter that implements the boundary conditions stated above (the Feynman-Stueckelberg causal prescriptions). In a particle interpretation, the Green function describes the propagation of ``real particles'' as well as the effects of ``virtual particles'', all identified with the quanta of the scalar field.