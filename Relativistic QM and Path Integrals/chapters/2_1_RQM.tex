\chapter[Relativistic Quantum Mechanics]{Relativistic Quantum\\Mechanics}

The \textbf{Schr\"odinger equation} is a wave equation for the quantum mechanics of non-relativistic particles. The attempts to generalize it to the relativistic case led historically to the discovery of many different relativistic wave equations (Klein--Gordon, Dirac, Proca--Maxwell, etc.). It soon became clear that all of these wave equations for relativistic particles had some interpretative problems:
\begin{itemize}
    \item[(i)] some did not admit a probabilistic interpretation,
    \item[(ii)] all of them admitted solutions with negative energy.
\end{itemize}
These equations are often called ``\textbf{first quantized}'' equations, as they are obtained by quantizing the mechanics of a single relativistic particle.

To solve those problems, eventually, one had to reinterpret them as equations for \textit{classical fields} (just like Maxwell's equations) that should be quantized anew (hence the name of ``\textbf{second quantization}'' given to the quantum theory of fields). All of the interpretative problems can be solved consistently within the framework of quantum field theory: the quantum fields are seen to describe an arbitrary number of indistinguishable particles (the quanta of the field, like the photons for the electromagnetic field). The relativistic equations mentioned above remain valid, but reinterpreted as equations satisfied by quantum field operators.

The main reason for the interpretative problems of the first quantized equations lies in the fact that relativity allows particles to be created and destroyed by physical processes. It would not be consistent to fix the number of particles and require that number to be conserved. Indeed, let us recall that relativity assigns the energy \(E = mc^2\) to a particle of mass \(m\) at rest. In the limit \(c \to \infty\), which formally describes the nonrelativistic limit, it would take infinite energy to create a particle. Non-relativistic quantum mechanics can be developed consistently to conserve the number of particles, which is linked to the conservation of probability for those particles to exist somewhere in space. In relativistic quantum mechanics it is impossible to do so: certain processes that carry enough energy may allow the creation of new particles, as observed in nature. This explains the \textit{failure to have a probabilistic interpretation} of the quantum mechanics of a single particle in the relativistic regime.

The other problem, the \textit{presence of negative energy states}, was eventually turned into a prediction: the existence of antiparticles: every particle should have a corresponding antiparticle with the same mass and opposite charges. The existence of antiparticles was experimentally confirmed with the discovery of the positron in 1932, validating the theoretical framework.

Given that the methods of second quantization (alias quantum field theory or QFT) is the natural mathematical framework to study the above properties, why review the historical development? There are many justifications to do so. One reason is that the historical development \textit{clarifies the physical ideas leading to more formal constructions}, such as QFT. A second motivation is that one finds many situations that can be dealt with -- often more simply -- in the context of relativistic quantum mechanics without the need to turn to more elaborate methods. This happens, for example, if one considers those cases where pair creation is suppressed and the single-particle approximation is applicable. More generally, first-quantized methods, which nowadays go under the name of the \textit{worldline formalism}, are often used as efficient tools to study the scattering of relativistic particles. As a final motivation, one may recall that first-quantized methods for relativistic particles are pedagogically useful for approaching string theory, a model for quantum gravity where particles are generalized to strings. The reason is that string theory has been mostly developed in first-quantization.

The different relativistic wave equations mentioned above correspond to the quantum mechanics of particles with different spin \(s\). There is also a difference if the particle is massive (\(m\neq 0\)) or massless (\(m=0\)) if the spin is \(s>0\). The simplest relativistic equation is the Klein--Gordon equation, that describes scalar particles, i.e., particles of spin \(s=0\). It takes into account the correct relativistic relation between energy and momentum, and thus it contains the essence of all relativistic wave equations (like negative energy solutions that signal the need for antiparticles).

The correct wave equation for a relativistic particle depends crucially on the value of the spin \(s\), some standard names are as follows:
\begin{itemize}
    \item spin 0 $\to$ Klein--Gordon equation
    \item spin $\tfrac{1}{2}$ $\to$ Dirac equation
    \item spin 1 (m $\neq$ 0) $\to$ Proca equation
    \item spin 1 (m = 0) $\to$ (free) Maxwell equations
    \item spin $\tfrac{3}{2}$ $\to$ Rarita--Schwinger equation
    \item spin 2 $\to$ Fierz--Pauli equations (or linearized Einstein eq. fSr m = 0).
    \item spin $s>2$ $\to$ Fierz--Pauli eqs. (for m $\neq$ 0) and Fronsdal eqs. (for m = 0).
\end{itemize}

We have anticipated that relativistic particles are classified by their mass \(m\) and spin \(s\), where the value of the spin indicates that there are only \(2s+1\) independent physical components of the wave function, describing the possible polarizations of the spin vector along a chosen axSs. That is true unless \(m=0\), in which case the wave function describes only two physical components, those with maximum and minimum helicity (helicity is the projection of the spin along the direction of motion). The reduction of the number of degrees of freedom is mathematically achieved by the emergence of gauge symmetries satisfied by the corresponding wave equations, as we shall see in the examples of spin 1 and 2.

The classification just described is due to Wigner, who in 1939 studied the \textit{unitary irreducible representations} of the Poincar\'e group. The Poincar\'e group is, by definition, the group of symmetries of relativistic theories, symmetries that must be realized by unStary operators in the Hilbert space of the particle. Different particles have different realizations (i.e., representations) of the symmetry group and \textbf{Wigner's theorem} describes the possible different unitary representations that are allowed by group theory. As anticipated above, a physical way of understanding Wigner's classification is to recall that for a massive particle of spin \(s\), one may always find a reference frame where the particle is at rest. Then, its spin is observed to have the \(2s + 1\) physical projections along the \(z\)-axis, as familiar from quantum mechanics. Thus, we understand that massive particles of spin \(s\) must have \(2s+1\) physical polarizations. On the other hand, a rest frame does not exist if the particle is massless: the particle must travel with the speed of light in any frame. Choosing the direction of motion as the axis where to measure the spin, one finds that only two values of the helicity \(h = \pm s\) are possible. Other helicities are not needed, as they would never mix with the previous ones under Poincar\'e transformations (they could be considered as belonging to different particles, which may or may not exist in a given model. On the contrary, the discrete CPT symmetry requires both helicities \(\pm s\) to be present).

In these notes, after a brief review of the Schr\"odinger equation, we discuss the main properties of the Klein--Gordon and Dirac equations, treated as first quantized wave equations for particles of spin \(0\) and \(\tfrac{1}{2}\), and then briefly comment on other relativistic free wave equations.

Our main conventions for special relativity are reported in appendix \ref{app:relativistic_notation}, and we use the standard definitions for Lorentz and Poincar\'e groups as in the lecture notes.

\section{Schrödinger equation}

Crucial milestones in the discovery of quantum mechanics are:

\begin{itemize}
    \item (1900) the introduction of Planck’s constant $h$ to describe the spectrum of black-body radiation;
    \item (1905) Einstein’s use of $h$ to explain the photoelectric effect, by interpreting light as composed of quanta (photons) of energy $E = h\nu$;
    \item (1913) Bohr’s atomic model, where the electron energy levels are quantized as $E_n \sim \frac{1}{n^2}$;
    \item (1923) de Broglie’s hypothesis extending Einstein’s idea to matter, suggesting that particles with momentum $p$ exhibit wave-like properties with wavelength $\lambda = \frac{h}{p}$.
\end{itemize}

At that time, it was still unclear which fundamental laws governed the quantum behavior of subatomic particles. However, de Broglie’s assumption provided a natural explanation for Bohr’s quantized energy levels: the allowed orbits could be interpreted as those for which an integer number of electron wavelengths fits exactly along the circular trajectory around the nucleus.

\paragraph{De Broglie’s wave hypothesis.}
Inspired by special relativity, de Broglie proposed to associate a periodic wave to every material particle. A periodic wave function characterized by
\begin{itemize}
    \item $\nu = \frac{1}{T}$, the \textbf{frequency} (periodicity in time),
    \item $|\mathbf{k}| = \frac{1}{\lambda}$, the \textbf{wavenumber} (periodicity in space),
\end{itemize}
has the mathematical form of a plane wave:
\[
    \psi(\mathbf{x},t) \sim e^{2\pi i(\mathbf{k}\cdot\mathbf{x} - \nu t)}.
\]
This is the simplest possible representation of a wave extending through space and time, with phase $\Phi = 2\pi(\mathbf{k}\cdot\mathbf{x} - \nu t)$.


\paragraph{Invariant formulation of the de Broglie wave.}
De Broglie assumed that the phase $\Phi$ must be invariant under Lorentz transformations. Since the spacetime coordinates form a four-vector $x^\mu = (ct, \mathbf{x})$, the quantities $\nu/c$ and $\mathbf{k}$ must also combine into a four-vector:
\[
    k^\mu = \left(\frac{\nu}{c}, \mathbf{k}\right).
\]
If the phase $2\pi k_\mu x^\mu$ is invariant, then both $x^\mu$ and $k^\mu$ must transform in the same way under Lorentz transformations.
At the same time, special relativity tells us that the energy and momentum of a particle form the four-vector
\[
    p^\mu = \left(\frac{E}{c}, \mathbf{p}\right).
\]
In the case of photons, Einstein’s relation $E = h\nu$ is already known, and the photon momentum satisfies $\mathbf{p} = h\mathbf{k}$.
It was therefore natural for de Broglie to extend this proportionality to all particles by postulating that
\[
    p^\mu = h k^\mu,
\]
with the same universal constant $h$ linking the two four-vectors.
From this, the familiar relations follow immediately:
\[
    E = h\nu, \qquad \mathbf{p} = h\mathbf{k}.
\]
The second expression implies that any material particle with momentum $p$ has an associated wavelength
\[
    \lambda = \frac{h}{p}.
\]
This was the first step toward the concept of matter waves.

The scalar (Lorentz-invariant) contraction between the four-position and the four-wavevector is
\[
    x_\mu k^\mu = \mathbf{x}\cdot\mathbf{k} - \nu t.
\]
Thus, the plane wave associated with a free particle of definite energy and momentum can be written in a manifestly invariant form as
\begin{equation}
    \psi(\mathbf{x},t) \sim e^{2\pi i x_\mu k^\mu}
    = e^{2\pi i(\mathbf{k}\cdot\mathbf{x} - \nu t)}
    = e^{i(\mathbf{p} \cdot \mathbf{x} - Et)/\hbar}.
    \label{eq:debroglie_wave_function}
\end{equation}

In this expression, $\mathbf{p}\cdot\mathbf{x} - Et/c^2$ is the standard relativistic scalar product between the four-momentum and the four-position.
The exponential phase encodes the oscillatory nature of the particle’s quantum state and ensures that the function transforms covariantly under Lorentz transformations.
It will be this very function --- representing a free particle of definite energy and momentum --- that Schrödinger later used as a starting point to search for the differential equation governing its evolution.

\paragraph{From de Broglie’s hypothesis to Schrödinger’s equation.}
At this point, Schrödinger asked a fundamental question: \textit{what kind of equation does a de Broglie wave satisfy?}
He initially attempted to find such an equation in a fully relativistic form, aiming to describe the electron in the hydrogen atom. However, the resulting energy spectrum did not match the experimental results. Schrödinger then turned to the non-relativistic limit, which provided the correct predictions.
(Today, we understand that the discrepancies in the relativistic case were due to neglecting the electron’s spin, whose effects compensate for those missing corrections.)

For a free non-relativistic particle, the classical relation between energy and momentum is
\[
    E = \frac{p^2}{2m}.
\]
If we consider the de Broglie plane wave
\[
    \psi(\mathbf{x},t) = e^{i(\mathbf{p}\cdot\mathbf{x} - Et)/\hbar},
\]
we can check directly which differential equation it satisfies.
By acting with the operators \( i\hbar \frac{\partial}{\partial t} \) and \( -i\hbar\nabla \) on the wave function, we find
\[
    i\hbar \frac{\partial}{\partial t} \psi = E \psi,
    \qquad
    -i\hbar\nabla \psi = \mathbf{p} \psi.
\]
Combining these relations with the classical energy expression \( E = \frac{p^2}{2m} \), we obtain
\[
    i\hbar \frac{\partial}{\partial t} \psi(\mathbf{x},t)
    = \frac{1}{2m}(-i\hbar\nabla)^2 \psi(\mathbf{x},t)
    = -\frac{\hbar^2}{2m} \nabla^2 \psi(\mathbf{x},t).
\]
Thus, the de Broglie plane wave automatically satisfies the differential equation
\begin{equation}
    i\hbar \frac{\partial}{\partial t} \psi(\mathbf{x},t)
    = -\frac{\hbar^2}{2m} \nabla^2 \psi(\mathbf{x},t),
    \label{eq:free_Schrödinger_nonrel}
\end{equation}
which is precisely the \textbf{free Schrödinger equation} for a particle of mass \(m\).
In other words, Schrödinger’s equation is the dynamical law that has de Broglie’s plane waves as its elementary solutions.

\paragraph{Quantization prescription.}
The above reasoning suggests a simple and general rule for constructing a wave equation starting from a classical mechanical model of a particle.
Given a classical expression for the energy as a function of momentum (and possibly position), such as \(E = \frac{p^2}{2m}\), one can obtain the corresponding quantum equation by making the substitutions
\[
    E \;\longrightarrow\; i\hbar\,\frac{\partial}{\partial t},
    \qquad
    \mathbf{p} \;\longrightarrow\; -i\hbar\nabla,
\]
and letting these differential operators act on a wave function \(\psi(\mathbf{x},t)\).

This procedure --- which converts classical observables into operators acting on wave functions --- defines the \textbf{quantization prescription}.
When applied to the classical Hamiltonian \(H = \frac{p^2}{2m}\), it directly produces the Schrödinger equation for a free particle.

Schrödinger soon extended this idea to include interactions.
For example, by considering a charged particle in the electrostatic potential of a nucleus, he replaced the classical energy \(E = \frac{p^2}{2m} + V(\mathbf{x})\) and obtained
\[
    i\hbar \frac{\partial}{\partial t} \psi(\mathbf{x},t)
    = \left[-\frac{\hbar^2}{2m}\nabla^2 + V(\mathbf{x})\right]\psi(\mathbf{x},t),
\]
successfully reproducing Bohr’s energy levels for the hydrogen atom.
This achievement marked the beginning of modern quantum mechanics.

\paragraph{General form.}
Although it was first derived for a single non-relativistic particle, Schrödinger’s equation can be written in a general and abstract form:
\begin{equation}
    i\hbar \frac{\partial}{\partial t} \ket{\psi(\mathbf{x},t)}
    = \hat{H} \ket{\psi(\mathbf{x},t)},
    \label{eq:Schrödinger_general}
\end{equation}
where \(\hat{H}\) is the \textbf{Hamiltonian operator} of the system.

In this formulation, the equation becomes a universal dynamical law for all quantum systems, not limited to point particles.
The specific form of \(\hat{H}\) encodes the physical content of the problem — kinetic energy, external fields, and interactions — while the wave function \(\ket{\psi}\) encapsulates the probabilistic state of the system.
This single equation thus unifies the dynamics of all non-relativistic quantum phenomena under a common mathematical structure.

\subsection{Conservation of probability}

When a non-relativistic particle is described by a normalizable wave function $\psi(\mathbf{x},t)$, the quantity
\[
    \rho(\mathbf{x},t) = |\psi(\mathbf{x},t)|^2
\]
is interpreted as the \textbf{probability density} of finding the particle at position $\mathbf{x}$ at time $t$.
(The plane wave considered earlier is not normalizable in infinite space, so in practice one works with \emph{wave packets}, i.e., superpositions of plane waves that are localized in space.)

A crucial property of the Schrödinger equation is that this probability density obeys a \textbf{continuity equation}, expressing the conservation of total probability:
\begin{equation}
    \frac{\partial \rho}{\partial t} + \nabla \cdot \mathbf{J} = 0,
    \label{eq:continuity_probability}
\end{equation}
where the \textbf{probability current density} is given by
\begin{equation}
    \mathbf{J} = \frac{\hbar}{2mi} \left( \psi^* \nabla \psi - \psi \nabla \psi^* \right).
    \label{eq:Schrödinger_probability_current}
\end{equation}
This equation is directly analogous to the continuity equation in fluid dynamics or electrodynamics: the change in probability within a given volume is compensated by the flux of probability through its boundary.

Equation \eqref{eq:continuity_probability} expresses the \textbf{conservation of probability}, meaning that at any instant the particle must exist somewhere in space — it cannot be spontaneously created or annihilated in the non-relativistic theory. This can be understood by examining the non-relativistic limit of the relativistic energy–momentum relation:
\[
    E = \sqrt{p^2 c^2 + m^2 c^4}
    = mc^2 \sqrt{1 + \frac{p^2}{m^2 c^2}}
    \approx mc^2 + \frac{p^2}{2m} + \dots
\]
In the limit \( c \to \infty \), the rest energy term \( mc^2 \) becomes infinitely large, meaning that creating or destroying a massive particle would require an infinite amount of energy. Hence, the number of particles is effectively conserved.

Integrating \eqref{eq:continuity_probability} over all space shows explicitly that the \textbf{total probability}
\[
    P = \int \mathrm{d}^3\mathbf{x} \, \rho(\mathbf{x},t),
\]
is constant in time. Indeed,
\[
    \frac{\mathrm{d} P}{\mathrm{d} t}
    = \int \! \mathrm{d}^3\mathbf{x} \, \frac{\partial \rho}{\partial t}
    = - \int \! \mathrm{d}^3\mathbf{x} \, \nabla \cdot \mathbf{J} = - \oint_{\partial V} \mathbf{J} \cdot \mathrm{d}\mathbf{S}
    = 0,
\]
where the last equality holds if $\mathbf{J}$ vanishes sufficiently fast at infinity (\(V \to \infty\)).
Therefore, the Schrödinger dynamics preserves the normalization of the wave function:
\[
    \int |\psi(\mathbf{x},t)|^2 \, \mathrm{d}^3\mathbf{x} = 1.
\]