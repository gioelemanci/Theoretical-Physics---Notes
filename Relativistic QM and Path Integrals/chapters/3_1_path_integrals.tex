\chapter{Path Integrals}

Quantization can be introduced in two equivalent ways:
- operator formalism (canonical quantization, Hilbert space, linear operators, etc ..)
- path integrals (functional integrals).

Path integrals were introduced in quantum mechanics by Feynman in 1948, but until about 1970 they did not meet with much success, and the operatorial methods of canonical quantization were still the most widespread. In 1970, the success of gauge theories in developing the Standard Model of particle physics gave a strong impulse to path integral methods.

Quantization of gauge theories is much more clear and elegant if performed with path integrals. Furthermore, path integrals indicate a way of relating a quantum field theory in D spacetime dimensions (\(D-1\) spaces and 1 time) to the statistical mechanics of a system in D space dimensions. This link has given rise to a way of thinking and defining field theories using statistical mechanics and renormalization group ideas, as introduced by Wilson and others (lattice theories). Nowadays,it is convenient to master both methods: according to the problem at hand, one may find one formalism more convenient than the other, even though they are supposed to be equivalent.

\paragraph{Two-slit experiment.} To introduce path integrals, let us follow Feynman and consider the \textit{two-slit experiment} for the electron. The standard treatment used to explain the behavior of an electron which passes through the two slits of a barrier and creates a figure of interference on a screen employs the wave nature of the electron together with the Huygens principle for calculating the interference pattern from the elementary waves that originate from the slits. Feynman proposes an \textit{alternative description}. He suggests to keep thinking of the electron as a particle that however can accomplish both trajectories, each one with a certain “amplitude”.
The \textbf{total amplitude} \(A_{tot}\) is defined as the sum of the single amplitudes, and its square is related to the probability that the electron is revealed at a given point on the screen. Moreover, the \textbf{elementary amplitude} for each possible trajectory is related in a simple way to the classical action evaluated on the trajectory itself: Feynman, inspired by previous considerations of Dirac, associates to each trajectory an amplitude of unit norm (so that all trajectories “weigh” democratically the same way) and with phase equal to the value of the action \(S\) in units of \(\hbar\).

Thus we can write
\[
    A_{tot} = A(c_1) + A(c_2) + \dots + A(c_n), \quad A(c_n) = e^{\frac{i}{\hbar}S(c_n)}.
\]
thus we linked the total amplitude, which let us study probabilities in our experiments, to a sum of single aplitudes with the same module but different phases
\[
    A_{tot} = \sum_{n} e^{\frac{i}{\hbar}S(c_n)}, \quad \implies \quad P \propto |A_{tot}|^2,
\]
and defined by the action (the integral of the lagrangian along the path, for a free particle we have just the kinetic contribution)
\[
    S[q] = \int_0^{T} \d{t} \frac{1}{2}m \dot{q}^2.
\]

So for a path where we assume \(D \gg d\) and simplifying the problem with a constant velocity along the two trajectories,\TODO{Insert drawing of slit.} we can write the action for each path as
\[
    \begin{aligned}
        S(c_1) & = \int_0^T \d{t} \frac{1}{2}m \dot{q}^2 = \int_0^T \d{t} \frac{1}{2}m \left(\frac{D}{T}\right)^2 = \frac{m}{2} \frac{D^2}{T},      \\
        S(c_2) & = \int_0^T \d{t} \frac{1}{2}m \dot{q}^2 = \int_0^T \d{t} \frac{1}{2}m \left(\frac{D+d}{T}\right)^2 = \frac{m}{2} \frac{(D+d)^2}{T} \\
               & = \frac{mD^2}{2T} + \frac{md^2}{2T} + \frac{mDd}{T} \sim \frac{mD^2}{2T} + \frac{md^2}{2T} + pd + O(d^2) = S(c_1) + pd + O(d^2),
    \end{aligned}
\]
where in the last term \(p = \frac{mD}{T}\) is the momentum of the electron (we have simplified the term \(\propto d^2\) since \(d \to 0\)). Therefore we can now study the total amplitude
\[
    A_{tot} = e^{\frac{i}{\hbar}S(c_1)} + e^{\frac{i}{\hbar}S(c_2)} = e^{\frac{i}{\hbar}S(c_1)} \left(1 + e^{\frac{i}{\hbar} pd} \right) = A(c_1) \left(1 + e^{\frac{i}{\hbar} pd} \right),
\]
which is associated to a total probability proportional to
\[
    P \propto |A_{tot}|^2 = |A(c_1)|^2 \left|1 + e^{\frac{i}{\hbar} pd} \right|^2 = \left|1 + e^{\frac{i}{\hbar} pd} \right|^2,
\]
since \(|A(c_1)|^2=1\). It is easy to notice that the maximum probability, associated to the maximum amplitude, of revealing the electron on the screen is obtained when the two contributions add up constructively, thus
\[
    e^{\frac{i}{\hbar} pd} = 1, \quad \implies \quad \frac{pd}{\hbar} = 2\pi n , \text{ with } n \in \mathbb{N},
\]
One can interpret this condition as defining a wavelength \(\lambda = \frac{h}{p}\) so that when \(d\) contains an integer number of times such wavelengths there is constructive interference. The de Broglie relation is obtained by this rudimentary “path integral” and suggests that it contains the essential elements of quantum mechanics.

The number of slits can be increased, as well as the number of intermediate screens, to have the particle performing all possible paths from the initial point to the final point of observation, thus creating a path integral for the total amplitude.
\begin{figure}[H]
    \begin{minipage}{0.4\textwidth}
        \centering
        \includegraphics[width=0.75\textwidth]{img/rudimental_path_integral.png}
        \label{fig:rudimental_path_integral}
    \end{minipage}
    \hfill
    \begin{minipage}{0.55\textwidth}
        \caption{Scheme for a rudimental and discretized path integral. The electron can follow all possible paths from the source to the screen, passing ideally through multiple intermediate screens. Each path contributes with an amplitude of unit norm and phase given by the action evaluated on the path itself. The total amplitude is obtained by summing all contributions.}
    \end{minipage}
\end{figure}

The action is used in an essential way
\[
    S[q(t)] = \int \d{t} \mathrm{L} (q(t),\,\dot{q}(t)),
\]
where the classic path is the one that minimizes the action
\[
    \delta S = 0, \quad \implies \quad \frac{\partial \mathrm{L}}{\partial q} - \frac{\mathrm{d}}{\mathrm{d} t} \frac{\partial \mathrm{L}}{\partial \dot{q}} = 0.
\]
In quantum mechanics there is a crucial difference: the transition amplitude is obtained by using the action \(S[q]\) \textbf{for any possible path}
\[
    A = \sum_{c_n} e^{\frac{i}{\hbar}S(c_n)} \equiv \int \mathrm{D} q(t) \, e^{\frac{i}{\hbar}S[q(t)]}.
\]

Here we introduce the final notation of the \textbf{path integral} or \textbf{functional integral}: $S[q]$ is a functional, i.e., a function of the functions $q(t)$, that indicate the possible ``paths'' of the system, and the symbol $\int \mathrm{D}q$ indicates the formal integration over the space of paths $\{q(t)\}$. Various mathematical subtleties on how to define exactly the path integration are still open. Nevertheless, path integrals have become one of the main tools to study quantum systems.

In this formulation, the classic limit is intuitive: macroscopic systems have large values of the action $S$ in units of $\hbar$, the quantum of action. Small variations of a path
\[
    q(t) \to q(t) + \delta q(t),
\]
induce a variation of the action
\[
    S[q] \to S[q + \delta q] = S[q] + \delta S[q],
\]
which translates in phase variations $i\frac{1}{\hbar} \delta S[q]$ much bigger with respect to $i\pi$ (recall that for such a phase $e^{i\pi} = -1$) and the amplitudes of nearby paths cancel by destructive interference. This happens except at the point in which the action has a minimum, $\delta S = 0$, which identifies the classic trajectory. Trajectories close to the classical one have amplitudes that add up coherently since the phase does not vary: the functional integral, in the end, is dominated by the classic path.

\section{Canonical Quantization}

Canonical quantization is constructed starting from the hamiltonian formulation of a classical system. It is obtained by lifting its phase space coordinates, the generalized coordinates \(x^i\) and conjugate momenta \(p_i\), to linear operators \(\hat{x}^i\) and \(\hat{p}_i\) that act on a linear space endowed with a positive definite norm, the Hilbert space of physical states \(\mathcal{H}\). The basic operators must satisfy commutation relations required to be equal \(i\hbar\) times the value of the corresponding classical Poisson brackets
\begin{equation}
    \begin{aligned}
        \left[\hat{x}^i,\, \hat{x}^j \right] & = \left[\hat{p}_i,\, \hat{p}_j \right] = 0, \\
        \left[\hat{x}^i,\, \hat{p}_j \right] & = i \hbar \delta^i_j.
    \end{aligned}
    \label{eq:Quantization_commutation_relations}
\end{equation}
All classical observables \(A(x,p)\), which are functions on phase space, become linear operators \(\hat{A} (\hat{x} ,\hat{p})\) acting on the Hilbert space \(\mathcal{H}\).

The most important example is given by the hamiltonian function \(H(x,p)\), which upon quantization becomes the hamiltonian operator \(\hat{H} (\hat{x} ,\hat{p})\). The latter generates the time evolution of any state \(\ket{\psi} \in \mathcal{H}\) through the Schrodinger equation
\begin{equation}
    i\hbar \frac{\partial}{\partial t} \ket{\psi} = \hat{H} \ket{\psi}.
    \label{eq:Schrödinger_equation}
\end{equation}
The corresponding solution is a time-dependent state \(\ket{\psi}(t)\) that describes the evolution of the quantum system. This setup is known as the Schrodinger picture of quantum mechanics. It is a formal quantization procedure that becomes operative once one finds an irreducible unitary representation of the operator algebra in eq. \eqref{eq:Quantization_commutation_relations}.

A mathematical result, known as the Stonevon Neumann theorem, states that in quantum mechanics all irreducible representations of \eqref{eq:Quantization_commutation_relations} are unitarily equivalent, so that there is a unique procedure of quantizing a classical system.\footnote{Up to the problem of resolving ordering ambiguities, often present when one tries to relate the classical hamiltonian \(H(x,p)\) to its quantum counterpart \(\hat{H} (\hat{x} ,\hat{p})\).}
Historically, this theorem made it clear that the Schrodinger formulation of quantum mechanics was equivalent to the one proposed by Heisenberg with its matrix mechanics (known as the Heisenberg picture).

Let us consider, more specifically, the motion of a nonrelativistic particle in one dimension in the presence of an external potential \(V(x)\). The classical dynamics is fixed by the action
\[
    S[x] = \int \d{t} \left( \frac{m}{2} \dot{x}^2 - V(x) \right).
\]
The quantum theory is recognized by first developing the hamiltonian formulation, defined in the phase space with the Poisson bracket structure and phase space action
\[
    S[x,p] = \int \d{t} \left( p \dot{x} - H(x,p) \right), \quad H(x,p) = \frac{p^2}{2m} + V(x),
\]
with \(\{x,\,p\}=1\) and \(\{x,\,x\}=\{p,\,p\}=0\).

Now, as prescribed by canonical quantization, the phase space coordinates \(x\) and \(p\) become foundamental operators \(\hat{x}\) and \(\hat{p}\) acting on the Hilbert space of physical states \(\mathcal{H}\), satisfying the commutation relations presented by the algebra in eq. \eqref{eq:Quantization_commutation_relations} (in this setting there is only one coordinate and one momentum). The hamiltonian function \(H(x,p)\) becomes the hamiltonian operator \(\hat{H} (\hat{x} ,\hat{p})\) acting on \(\mathcal{H}\), responsible for the time evolution of any state \(\ket{\psi} \in \mathcal{H}\) through the Schrödinger equation \eqref{eq:Schrödinger_equation}.

Using the coordinate representation, obtained by considering the eigenstates |x〉 of the position operator \(\hat{x}\), that satisfy \(\hat{x}\ket{x} = x\ket{x}\) with real eigenvalues \(x\), and projecting the various states of the Hilbert space onto them to identify the wave functions, one finds the familiar way of realizing quantum mechanics as wave mechanics
\[
    \begin{aligned}
        \ket{\psi} & \longrightarrow \psi(x) = \bra{x}\ket{\psi},                                                                                                          \\
        \hat{x}    & \longrightarrow x, \quad \bra{x}\hat{x}\ket{x^{\prime}} = x \delta(x-x^{\prime}),                                                                     \\
        \hat{p}    & \longrightarrow -i\hbar \frac{\partial}{\partial x}, \quad \bra{x}\hat{p}\ket{x^{\prime}} = -i\hbar \frac{\partial}{\partial x} \delta(x-x^{\prime}), \\
        \hat{H}    & \longrightarrow -\frac{\hbar^2}{2m} \frac{\partial^2}{\partial x^2} + V(x).
    \end{aligned}
\]
Thus the Schrodinger equation \eqref{eq:Schrödinger_equation} becomes the familiar differential equation for the wave function \(\psi(x,t)\)
\[
    i\hbar \frac{\partial}{\partial t} \psi(x,t) = \left(-\frac{\hbar^2}{2m} \frac{\partial^2}{\partial x^2} + V(x)\right) \psi(x,t).
\]

Returning to the Dirac bra and ket notation, let us consider the solution of the Schrödinger equation. Given a ket \(\ket{\psi_i}\) that describes the system at initial time \(t_i\), the solution of the Schrödinger equation for time-independent hamiltonians can be written as
\[
    \ket{\psi(t)} = e^{-\frac{i}{\hbar} \hat{H} (t - t_i)} \ket{\psi_i},
\]
satisfying boundary condition \(\ket{\psi(t_i)} = \ket{\psi_i}\). The operator \(e^{-\frac{i}{\hbar} \hat{H} (t - t_i)}\) is called the evolution operator. The transition amplitude between an initial state \(\ket{\psi_i}\) at time \(t_i\) and a final state \(\ket{\psi_f}\) at time \(t_f\) is defined as
\[
    A = \bra{\psi_f} e^{-\frac{i}{\hbar} \hat{H} (t_f - t_i)} \ket{\psi_i} = \bra{\psi_f}\ket{\psi_i},
\]
where the matrix element of the evolution operator are shown to be those transition amplitudes indeed. The transition amplitude contains all the physical information about the quantum system, as it allows one to compute probabilities and expectation values of physical observables.

\section{Path Integrals in Phase Space}

To derive a path integral expression for the transition amplitudes, we start by inserting twice the identity operator \(\mathbb{I}\), expressed using the eigenstates of the position operator
\[
    \mathbb{I} = \int \d{x} \ket{x} \bra{x}, \quad \text{with } \bra{x}\ket{x^{\prime}} = \delta(x-x^{\prime}),
\]
then we can rewrite the transition amplitude derived previously by insterting two of this identityoperators
\[
    \begin{aligned}
        A & = \bra{\psi_f} e^{-\frac{i}{\hbar} \hat{H} (t_f - t_i)} \ket{\psi_i} = \bra{\psi_f} \mathbb{I} e^{-\frac{i}{\hbar} \hat{H} (t_f - t_i)} \mathbb{I} \ket{\psi_i} \\
          & = \int \d{x_i} \int \d{x_f} \bra{\psi_f}\ket{x_f} \bra{x_f} e^{-\frac{i}{\hbar} \hat{H} (t_f - t_i)} \ket{x_i} \bra{x_i}\ket{\psi_i}                            \\
          & = \int \d{x_i} \int \d{x_f} \psi_f^*(x_f) A(x_i,\,x_f,\,T) \psi_i(x_i),
    \end{aligned}
\]
where \(\psi_i(x_i) = \bra{x}\ket{\psi_i}\) and \(\psi_f(x_f) = \bra{x}\ket{\psi_f}\) are the wave functions for the initial and final
states. This rewriting shows that it is enough to consider the matrix element of the evolution operator between position eigenstates
\[
    A(x_i,\,x_f,\,T) = \bra{x_f} e^{-\frac{i}{\hbar} \hat{H} T} \ket{x_i}
\]
where \(T = t_f - t_i\) is the total propagation time. It satisfies the Schrodinger equation
\[
    i\hbar \frac{\partial}{\partial T} A(x_i,\,x_f,\,T) = \bra{x_f} \hat{H}(\hat{x},\,\hat{p}) e^{-\frac{i}{\hbar} \hat{H} T} \ket{x_i} = \hat{H}(x_f,\,\hat{p}_f= -i\hbar \partial_{x_f}) A(x_i,\,x_f,\,T),
\]
with initial condition \(A(x_i,\,x_f,\,0) = \delta(x_i - x_f)\).

We are going to consider quantum hamiltonians of a particle interacting with a generic potential \(\hat{V}(\hat{x})\), of the familiar form \(\hat{H}(\hat{x} ,\,\hat{p}) = \frac{\hat{p}^2}{2m} + \hat{V}(\hat{x})\). The derivation can be extended to more general hamiltonians, but this case is enough to illustrate the main ideas.

We can now proceed to derive the path integral formalism as follows. One can split the transition amplitude \(A(x_i,\,x_f,\,T)\) in the product of \(N\) factors, and insert the completeness relation \(N-1\) times in between the factors
\[
    \begin{aligned}
        A(x_i,\,x_f,\,T) & = \bra{x_f} e^{-\frac{i}{\hbar} \hat{H} T} \ket{x_i} = \bra{x_f} \left(e^{-\frac{i}{\hbar} \hat{H} \frac{T}{N}}\right)^N \ket{x_i}                                                                                    \\
                         & = \bra{x_f} e^{-\frac{i}{\hbar} \hat{H} \epsilon} \mathbb{I} e^{-\frac{i}{\hbar} \hat{H} \epsilon} \mathbb{I} \cdots e^{-\frac{i}{\hbar} \hat{H} \epsilon} \mathbb{I} e^{-\frac{i}{\hbar} \hat{H} \epsilon} \ket{x_i} \\
                         & = \int \left(\prod_{k=1}^{N-1} \d{x_k}\right) \bra{x_f} e^{-\frac{i\epsilon}{\hbar}\hat{H}} \ket{x_{N-1}} \bra{x_{N-1}} e^{-\frac{i\epsilon}{\hbar}\hat{H}} \ket{x_{N-2}} \cdots                                      \\
                         & = \int \left(\prod_{k=1}^{N-1} \d{x_k}\right) \prod_{k=1}^{N} \bra{x_k} e^{-\frac{i\epsilon}{\hbar}\hat{H}} \ket{x_{k-1}},
    \end{aligned}
\]
where for convenience we have denoted \(x_0 = x_i\), \(x_N = x_f\), and \(\epsilon = \frac{T}{N}\). To evaluate this expression better, it is convenient to use the completeness relation \(N\) more times, now expressed in terms of the momentum eigenstates
\[
    \mathbb{I} = \int \frac{\d{p}}{2\pi \hbar} \ket{p} \bra{p}, \quad \text{with } \bra{p}\ket{p^{\prime}} = 2\pi\hbar \delta(p-p^{\prime}),
\]
to obtain
\[
    \begin{aligned}
        A(x_i,\,x_f,\,T) & = \int \left(\prod_{k=1}^{N-1} \d{x_k}\right) \prod_{k=1}^{N} \bra{x_k} \mathbb{I} e^{-\frac{i\epsilon}{\hbar}\hat{H}} \ket{x_{k-1}}                                                                         \\
                         & = \int \left(\prod_{k=1}^{N-1} \d{x_k}\right) \prod_{k=1}^{N} \int \frac{\mathrm{d} p_k}{2\pi \hbar}\bra{x_k}\ket{p_k} \bra{p_k} e^{-\frac{i\epsilon}{\hbar}\hat{H}} \ket{x_{k-1}}                           \\
                         & = \int \left(\prod_{k=1}^{N-1} \d{x_k}\right) \left(\prod_{k=1}^{N} \frac{\mathrm{d} p_k}{2\pi \hbar}\right) \prod_{k=1}^{N} \bra{x_k}\ket{p_k} \bra{p_k} e^{-\frac{i\epsilon}{\hbar}\hat{H}} \ket{x_{k-1}},
    \end{aligned}
\]
which is an exact expression. Note that there is one more integration over momenta than integrations over coordinates, a consequence of choosing coordinate eigenstates as initial and final states in the transition amplitude. Now, one can manipulate this expression further by making approximations that are valid in the limit \(N \to \infty\) (i.e., \(\epsilon \to  0\)). The crucial point is the evaluation of the following matrix element in this limit, where we can Taylor expand the exponential in \(\epsilon\) to obtain
\[
    \begin{aligned}
        \bra{p} e^{-\frac{i\epsilon}{\hbar}\hat{H}(\hat{x},\,\hat{p})}\ket{x} & = \bra{p} \left(\mathbb{I} - \frac{i \epsilon}{\hbar} \hat{H}(\hat{x},\,\hat{p}) + \cdots \right) \ket{x}
                                                                              & = \bra{p} \left(\mathbb{I} - \frac{i \epsilon}{\hbar} H(x,\,p) + \cdots \right) \ket{x}                   \\
                                                                              & = \bra{p}\ket{x} e^{-\frac{i \epsilon}{\hbar} H(x,\,p)},
    \end{aligned}
\]
thus recovering the exponential form with the eigenvalues of the hamiltonian, letting us extract it from the matrix element. These approximations are all valid in the limit of small \(\epsilon\). The substitution \(\bra{p}\hat{H}(\hat{x},\,\hat{p})\ket{x} = \bra{p}\ket{x}H(x,p)\) follows from the simple structure of the considered hamiltonian, that allows one to act with the momentum operator on the left, and with the position operator on the right, to have the operators replaced by the corresponding eigenvalues. Notice that there is no need for commuting operators inside the hamiltonian, because of the simplicity of the hamiltonian
we have considered.

The final result is that all operators are simply replaced by eigenvalues. This way the quantum hamiltonian \(\hat{H}(\hat{x} ,\, \hat{p})\) gets replaced by the classical function \(H(x,p) = \frac{p^2}{2m} + V(x)\). There exists a mathematically rigorous proof that these manipulations are correct for a wide class of physically interesting potentials \(V(x)\) (the “Trotter formula”). We do not review these subtleties, as the physically intuitive derivation given above is enough for our purposes.

and remembering that the wave functions of the momentum eigenstates (the plane waves) are normalized as
\[
    \bra{x}\ket{p} = e^{\frac{i}{\hbar}px}, \quad \bra{p}\ket{x} = (\bra{x}\ket{p})^* = e^{-\frac{i}{\hbar}px},
\]
following from the normalizations chosen with the two completeness relations, one obtains
\[
    \bra{x_k}\ket{p_k}\bra{p_k}e^{-\frac{i\epsilon}{\hbar}\hat{H}}\ket{x_{k-1}} = e^{\frac{i}{\hbar}p_k x_k} e^{-\frac{i\epsilon}{\hbar} H(x_{k-1},\,p_k)} e^{-\frac{i}{\hbar}p_k x_{k-1}} = e^{\frac{i\epsilon}{\hbar} \left[p_k \frac{x_k - x_{k-1}}{\epsilon} - H(x_{k-1},\,p_k)\right]}.
\]
up to terms that vanish for \(\epsilon \to 0\). This expression can now be inserted in the last expression for the elements of \(A(x_i,\,x_f,\,T)\); at this stage, the transition amplitude does not contain any more operators, bras and kets, containing just integrations, though a big number of them, of ordinary functions
\[
    A(x_i,\,x_f,\,T) = \lim_{N \to \infty} \int \left(\prod_{k=1}^{N-1} \d{x_k}\right) \left(\prod_{k=1}^{N} \frac{\mathrm{d} p_k}{2\pi \hbar}\right)e^{\frac{i\epsilon}{\hbar}\sum_{k=1}^N \left[p_k \frac{x_k - x_{k-1}}{\epsilon} - H(x_{k-1},\,p_k)\right] } = \int \mathrm{D} x \mathrm{D} p \, e^{\frac{i}{\hbar}S[x,\,p]}.
\]
This is the \textbf{path integral in phase space}. One recognizes in the exponent a discretization of the classical phase space action
\[
    S[x,\,p] = \int_{t_i}^{t_f} \d{t} \left(p\dot{x} - H(x,\,p)\right),\quad \implies \quad \sum_{k=1}^{N} \epsilon\left( p_k \frac{x_k - x_{k-1}}{\epsilon} - H(x_{k-1},\,p_k)\right),
\]
where \(t_f - t_i = T = N\epsilon\) is the total propagation time, with the paths in phase space discretized as
\[
    \begin{aligned}
        x(t) & \longrightarrow x_k = x(t_i + k \epsilon), \\
        p(t) & \longrightarrow p_k = p(t_i + k \epsilon).
    \end{aligned}
\]
The last way of writing the amplitude is symbolic but suggestive: it indicates the sum over all paths in phase space weighted by the exponential of \(\frac{i}{\hbar}\) times the classical action. It depends implicitely on the boundary conditions assigned to the paths \(x(t)\). We can compact the notation by writing
\begin{equation}
    A(x_i,\,x_f,\,T) = \int \mathrm{D} x(t) \mathrm{D} p(t) \, e^{\frac{i}{\hbar}S[x(t),\,p(t)]},
    \label{eq:Path_integral_phase_space}
\end{equation}
where the symbol \(\int \mathrm{D} x(t) \mathrm{D} p(t)\) indicates the formal integration over the space of paths in phase space \(\{x(t),\,p(t)\}\). Various mathematical subtleties on how to define exactly the path integration are still open. Nevertheless, path integrals have become one of the main tools to study quantum systems.

\section{Path Integrals in Configuration Space}

The path integral in configurations space is easily derived by integrating over the momenta. The dependence on momenta in the exponent of \(A\) is at most quadratic and can be eliminated by gaussian integration: if we consider the previous expression for the transition amplitude in phase space
\[
    A(x_i,\,x_f,\,T) = \lim_{N \to \infty} \int \left(\prod_{k=1}^{N-1} \d{x_k}\right) \left(\prod_{k=1}^{N} \frac{\mathrm{d} p_k}{2\pi \hbar}\right)e^{\frac{i\epsilon}{\hbar}\sum_{k=1}^N \left[p_k \frac{x_k - x_{k-1}}{\epsilon} - \frac{p_k^2}{2m} - V(x_{k-1})    \right] }
\]
we can explicitly perform the gaussian integrations over the momenta \(p_k\) with
\[
    \int_{-\infty}^{\infty} \d{p} \, e^{-\frac{\alpha}{2} p^2} = \sqrt{\frac{2\pi}{\alpha}},
\]
which is valid for \(\alpha >0\) and can be generalized by \textbf{square completion} to exponents of the type \(-\frac{\alpha}{2}p^2 + \beta p\) as follows
\[
    \begin{aligned}
        \int_{-\infty}^{\infty} \d{p} \, e^{-\frac{\alpha}{2} p^2 + \beta p} & = \int_{-\infty}^{\infty} \d{p} \, e^{-\frac{\alpha}{2} \left(p^2 - \frac{2\beta}{\alpha} p\right)}  = \int_{-\infty}^{\infty} \d{p} \, e^{-\frac{\alpha}{2} \left[\left(p - \frac{\beta}{\alpha}\right)^2 - \left(\frac{\beta}{\alpha}\right)^2\right]} \\
                                                                             & = e^{\frac{\beta^2}{2\alpha}} \int_{-\infty}^{\infty} \d{p} \, e^{-\frac{\alpha}{2} \left(p - \frac{\beta}{\alpha}\right)^2} = e^{\frac{\beta^2}{2\alpha}} \sqrt{\frac{2\pi}{\alpha}},
    \end{aligned}
\]
performed by shifting the integration variable \(p \to p - \frac{\beta}{\alpha}\) (the differential do not change). Now, extending analytically this expression, we can include complex values of \(\alpha\) (see section \ref{sec:gaussian_integrals} for details). Note that the final exponential is the original exponential inside the integral with argument evaluated at the minimum in \(p\) (saddle point).

Returning to the path integral, and considering the hamiltonian \(H(x, p) = \frac{p^2}{2m} + V(x)\), one completes the squares
\[
    \begin{aligned}
        A & = \lim_{N \to \infty} \int \left(\prod_{k=1}^{N-1} \d{x_k}\right) \frac{1}{(2\pi\hbar)^N} \prod_{k=1}^{N} \int \d{p_k} \, e^{\frac{i\epsilon}{\hbar} \sum_{k=1}^N \left[p_k \frac{x_k - x_{k-1}}{\epsilon} - \frac{p_k^2}{2m} - V(x_{k-1})\right]} \\
          & \quad \begin{dcases}
                      \alpha = \frac{i \epsilon}{2m \hbar}              & \longrightarrow \left(\frac{2 \pi \hbar m}{i \epsilon}\right)^\frac{N}{2};                                         \\
                      \beta = \frac{i}{\hbar}\left(x_k - x_{k-1}\right) & \longrightarrow e^{\frac{\beta^2}{2\alpha}} = e^{\frac{im}{2 \hbar \epsilon}(x_k - x_{k-1})^2} \text{ inside sum}; \\
                      \frac{i\epsilon}{\hbar}V(x_{k-1})                 & \longrightarrow \text{remains unchanged inside sum};
                  \end{dcases}                                                                       \\
          & = \lim_{N \to \infty} \int \left(\prod_{k=1}^{N-1} \d{x_k}\right) \left(\frac{m}{2\pi i \hbar \epsilon}\right)^{\frac{N}{2}} e^{\frac{i\epsilon}{\hbar}\sum_{k=1}^N \left[\frac{m}{2}\frac{(x_k - x_{k-1})^2}{\epsilon^2} - V(x_{k-1})\right]}.
    \end{aligned}
\]
where we have implicitly performed the gaussian integrations \(N\) times over the momenta. Thus is now possible to write the transition amplitude as
\[
    A = \lim_{N \to \infty} \int \left(\prod_{k=1}^{N-1} \d{x_k}\right) \left(\frac{m}{2\pi i \hbar \epsilon}\right)^{\frac{N}{2}} e^{\frac{i\epsilon}{\hbar}\sum_{k=1}^N \left[\frac{m}{2}\frac{(x_k - x_{k-1})^2}{\epsilon^2} - V(x_{k-1})\right]} = \int \mathrm{D} x(t) \, e^{\frac{i}{\hbar}S[x(t)]}.
\]
Our exponent is complex, so the function keeps oscillating; by analytic continuation we can use this results also in the complex case.
This is the path integral in configuration space. It contains in the exponent the configuration space action suitably discretized
\[
    S[x] = \int_{t_i}^{t_f} \d{t} \left( \frac{m}{2} \dot{x}^2 - V(x) \right) \longrightarrow \sum_{k=1}^N \epsilon \left[\frac{m}{2}\frac{(x_k - x_{k-1})^2}{\epsilon^2} - V(x_{k-1})\right].
\]
Again, the last way of writing the path integral in \eqref{eq:Path_integral_configuration_space} is symbolic and indicates the formal sum over paths in configuration space, weighted by the exponential of \(i\hbar\) times the classical action. The space of paths is given by the space of functions \(x(t)\) with boundary values \(x(t_i)=x_i\) and \(x(t_f) = x_f\). It is an infinite dimensional space. How to perform concretely the path integral over this functional space is defined precisely by the discretization, that approximates a function \(x(t)\) by its \(N+1\) values \(x_k = x(t_i + \epsilon k)\) at \(k = 0, 1, 2, ..., N\), as shown in fig. \ref{fig:rudimental_path_integral}.

Thus, we have constructed the path integral that computes quantum mechanical amplitudes
\begin{equation}
    A = \int \mathrm{D} x(t) \, e^{\frac{i}{\hbar}S[x(t)]},
    \label{eq:Path_integral_configuration_space}
\end{equation}
by the sum of all histories of paths weighted by a phase given by the action, with all paths contributing.

\subsection{Free Particle}

For a free particle (\(V(x)=0\)) one may use repeatedly gaussian integration and calculate from eq. \eqref{eq:Path_integral_configuration_space} the exact transition amplitude\TODO{Add image.} as
\[
    \begin{aligned}
        A(x_i,\,x_f,\,T) & = \lim_{N \to \infty} \int \left(\prod_{k=1}^{N-1} \d{x_k}\right) \left(\frac{m}{2\pi i \hbar \epsilon}\right)^{\frac{N}{2}} e^{\frac{i\epsilon}{\hbar}\sum_{k=1}^N \left[\frac{m}{2}\frac{(x_k - x_{k-1})^2}{\epsilon^2}\right]} \\
                         & = \lim_{N \to \infty} \left(\frac{m}{2\pi i \hbar \epsilon}\right)^{\frac{N}{2}} \left(\frac{2\pi i \hbar \epsilon}{m}\right)^{\frac{N-1}{2}} \frac{1}{\sqrt{N}} e^{\frac{i}{\hbar} \frac{m (x_f - x_i)^2}{2T}}                   \\
                         & = \sqrt{\frac{m}{2 \pi i \hbar T}} e^{\frac{i}{\hbar} \frac{m (x_f - x_i)^2}{2T}} = \bra{x_f} e^{-\frac{i}{\hbar} \hat{H}T} \ket{x_i}.
    \end{aligned}
\]
It satisfies the free Schrodinger equation
\[
    i\hbar \frac{\partial}{\partial T} A(x_i,\,x_f,\,T) = - \frac{\hbar^2}{2m}\frac{\partial^2}{\partial x_f^2} A(x_i,\,x_f,\,T),
\]
with initial condition \(A(x_i,\,x_i,\,0) = \delta(x_f - x_i)\), since we can compute
\[
    - \frac{\hbar^2}{2m}\frac{\partial^2}{\partial x_f^2} \sqrt{\frac{m}{2 \pi i \hbar T}} e^{\frac{i}{\hbar} \frac{m (x_f - x_i)^2}{2T}} = \frac{m}{2T^2} (x_f - x_i)^2 A(x_i,\,x_f,\,T) = i\hbar \frac{\partial}{\partial T} A(x_i,\,x_f,\,T).
\]
This way, the path integral has produced a solution of the Schrodinger equation. The result is very suggestive: up to a prefactor, the solution is given by the exponential of \(\tfrac{i}{\hbar}\) times the classical action evaluated on the classical path \(S[x]\), i.e., the path that satisfies the classical equations of motion.
The free particle case is also quite special: the exact final result is valid for any \(N\), and there is no need to take the limit \(N \to  \infty\). The case \(N=1\), which carries no integration at all, is already exact.

A formal but useful way of calculating gaussian path integrals is achieved by working directly in the continuum limit. One does not need the precise definition of the path integral measure but uses only its formal properties, in particular, its translational invariance. The calculation is formal in the sense that one assumes properties of the path integral measure (that eventually must be proven by an explicit regularization and construction, as the one given earlier). The calculation goes as follows. The action is \(S[x]= \int_0^T \d{t} \frac{m}{2}\dot{x}^2\), and the classical equations of motion with the boundary conditions are solved by
\[
    x_{cl} (t) = x_i + \frac{x_f - x_i}{T}t.
\]
One can represent a generic path \(x(t)\) as the classical path \(x_{cl}(t)\) plus quantum fluctuations \(q(t)\)
\[
    x(t) = x_{cl}(t) + q(t),
\]
where the fluctuations \(q(t)\) must vanish at \(t=0\) and \(t=T\) to preserve the boundary conditions. One may interpret \(x_{cl}(t)\) as the origin in the space of functions. Then, one computes the path integral as follows
\[
    \begin{aligned}
        A(x_i,\,x_f,\,T) & = \int \mathrm{D} x \, e^{\frac{i}{\hbar}S[x]} = \int \mathrm{D} (x_{cl} + q) \, e^{\frac{i}{\hbar} S[x_{cl} + q]}                                                \\
                         & = e^{\frac{i}{\hbar} S[x_{cl}]} \int \mathrm{D} q \, e^{\frac{i}{\hbar} S[q]} = N e^{\frac{i}{\hbar} S[x_{cl}]} = N e^{\frac{i}{\hbar}\frac{m(x_{f-x_i})^2}{2T}},
    \end{aligned}
\]
where we have used different inputs to solve it:
\begin{itemize}
    \item The translational invariance of the path integral measure has been used in the form \(\mathrm{D}x = \mathrm{D}(x_{cl}+q) = \mathrm{D}q\).
    \item The normalization factor \(N = \int \mathrm{D} q\, e^ {\frac{i}{\hbar} S[q]}\) is undetermined by this method, but it is a constant that does not depend on \(x_i\) and \(x_f\). Very often, its precise value is not needed, but one can fix it by requiring that the final result satisfies the Schrodinger equation, finding \(N = \sqrt{\frac{m}{2\pi i \hbar T}}\).
    \item The action splits into a classical part plus a quantum part:
          \[
              S[x_{cl} + q] = S[x_{cl}] + S[q] + \text{(linear terms in } q\text{)}.
          \]
          There is no linear term in \(q(t)\) in the action because the function \(x_{cl}(t)\) solves the classical equations of motion: for quadratic actions one has \(S[x_{cl} + q] = S[x_{cl}] + S[q]\); we can integrate non quadratic terms by parts indeed:
          \[
              \int \d{t} \, \dot{x}_{cl} \dot{q} = \left. \dot{x}_{cl} q \right|_0^T - \int \d{t} \, \ddot{x}_{cl} q = 0,
          \]
          where the first term vanishes because \(q(0) = q(T) = 0\), while the second term vanishes because \(\ddot{x}_{cl} = 0\) for the free particle (it solves the classical equations of motion).
\end{itemize}
This method of calculation is very powerful and can be extended to more general cases, as we will see later.

\subsection{Euclidean Time and Statistical Mechanics}

Quantum mechanics can be related to statistical mechanics by an analytic continuation. We introduce this relation by considering the free particle, with
\[
    A(x_i,\,x_f,\,T) = \int \mathrm{D}x \, e^{\frac{i}{\hbar} S[x]} = \sqrt{\frac{m}{2 \pi i \hbar T}} e^{\frac{i}{\hbar} \frac{m (x_f - x_i)^2}{2T}}.
\]
Continuing analytically the time parameter to purely imaginary values by \(T \to - i \beta\) with real \(\beta\), and setting \(\hbar = 1\), the free Schrodinger equation turns into the \textbf{heat equation} (or diffusion equation)
\[
    \frac{\partial}{\partial \beta} A = \frac{1}{2m} \frac{\partial^2}{\partial x_f^2} A.
\]
Its fundamental solution, i.e. the solution with boundary condition \(A \xrightarrow{\beta \to 0} \delta(x_f - x_i)\), is given by
\[
    A = \sqrt{\frac{m}{2 \pi \beta}} e^{-\frac{m(x_f - x_i)^2}{2\beta}},
\]
and can be obtained from the original expression by the same analytic continuation. This continuation is called “\textbf{Wick rotation}”, see figure below.
\begin{figure}[H]
    \begin{minipage}{0.3\textwidth}
        \centering
        \includegraphics[width=0.75\textwidth]{img/wick-rotation.png}
        \label{fig:wick_rotation}
    \end{minipage}
    \hfill
    \begin{minipage}{0.65\textwidth}
        The Wick rotation can be performed directly on the path integral to obtain euclidean path integrals. Analytically continuing the time variable \(t \to -i \tau\) , one finds that the action with “minkowskian” time (i.e. with a real time \(t\)) turns into an “euclidean” action \(S_E\) defined by\QUESTION{Why integration limits change without imaginary unit?}
        \[
            i S[x] = i \int_{0}^{T} \d{t} \frac{m}{2} \dot{x}^2 \xrightarrow[\d{t} \to -i \d{\tau}]{T \to -i \beta} - S_E[x] = - \int_{0}^{\beta} \d{\tau} \frac{m}{2} \dot{x}^2.
        \]
    \end{minipage}
\end{figure}
In the euclidean action we have defined \(\dot{x} = \frac{\mathrm{d}}{\mathrm{d} \tau} x\), with \(\tau\) usually called “euclidean time” (or imaginary time). The euclidean action thus defined is positive definite and it appears in the path integral as follows
\begin{equation}
    \bra{x_f} e^{-\beta \hat{H}} \ket{x_i} = \int \mathrm{D} x \, e^{-S_E[x]},
    \label{eq:heat_kernel_path_integral}
\end{equation}
where the crucial difference with respect to the original path integral is \textit{the absence of the imaginary unit} \(i\) \textit{in the exponent}. The operator on the left-hand side is called the “heat kernel” and the path integrals computes its matrix elements. For a free theory, the path integral is truly gaussian, with exponential damping rather than increasingly rapid phase oscillations (now the phase is real). In this form, it coincides with the integral functional introduced by Wiener in the 1920's to study brownian motion and the heat equation (that explains why eq. \eqref{eq:heat_kernel_path_integral} is called the heat kernel).

Such euclidean path integrals are useful in \textbf{statistical mechanics}, where \(\beta\) is related to the absolute temperature \(\Theta\) as \(\beta = \frac{1}{k \Theta}\), where \(k\) is the Boltzmann’s constant. To see this, let us consider the trace of the evolution operator
\[
    e^{-\frac{i}{\hbar} \hat{H} T}, \quad e^{-\frac{i}{\hbar} \hat{H} T} \ket{n} = e^{-\frac{i}{\hbar} E_n T} \ket{n},
\]
where we have written it using energy eigenstates (labeled by \(n\) if the spectrum is discrete), or equivalently we could use position eigenstates (which are continuous and labeled by \(x\)). Since these are two complete bases of the Hilbert space, and its eigenvalues identify the diagonal elements, we can write the trace in either way:
\[
    Z \equiv \Tr(e^{-\frac{i}{\hbar}\hat{H}T}) = \sum_{n} e^{-\frac{i}{\hbar}E_n T} = \int \d{x} \bra{x} e^{-\frac{i}{\hbar} \hat{H} T} \ket{x}.
\]
This can be Wick rotated \(Z \to  Z_E\) (with \(T \to - i \beta\)) to obtain the statistical partition function \(Z_E\) of the quantum system with hamiltonian \(\hat{H}\). Setting \(\hbar=1\), it reads
\begin{equation}
    Z_E = \Tr(e^{-\beta \hat{H}}) = \sum_{n} e^{- \beta E_n} = \int \d{x} \bra{x} e^{-\beta \hat{H}} \ket{x}.
    \label{eq:statistical_partition_function}
\end{equation}
Again, it is immediate to find a path integral representation of the statistical partition function: one performs a Wick rotation of the path integral action, sets the initial state (at euclidean time \(\tau = 0\)) equal to the final state (at euclidean time \(\tau = \beta\)), and sums over all possible states, as indicated in \eqref{eq:statistical_partition_function}. The paths become closed, as \(x(\beta)=x(0)\), and the partition function becomes
\[
    Z_E = \Tr(e^{-\beta \hat{H}}) = \int_{PBC} \mathrm{D} x\, e^{-S_E[x]},
\]
where \(PBC\) stands for “periodic boundary conditions”, indicating the sum over all paths that close onto themselves in an euclidean time \(\beta\).
Introduced here for the free theory, the Wick rotation is supposed to be of more general value, relating quantum mechanics to statistical mechanics in the interacting case as well. Even if one is interested in the theory with a real time, nowadays, one often works in the euclidean version of the theory, where factors of the imaginary unit \(i\) are absent, and path integral convergence is more easily kept under control. Only at the very end one performs the inverse Wick rotation to read off the result for the theory in real time.

The Wick rotation procedure is better appreciated by first considering the usual time as corresponding to the real line of a complex plane. Then, defining
\[
    t_{\theta} = t e^{-i\theta},
\]
the usual real time appears at \(\theta=0\), while the euclidean time \(\tau\) appears at \(\theta = \frac{\pi}{2}\)  as \(t_{\frac{\pi}{2}} = -i \tau\). The analytical continuation of all physical quantities is achieved by continually increasing \(\theta\) form 0 to \(\frac{\pi}{2}\), a clockwise rotation of the real axis into the imaginary one. The generalized partition function
\[
    Z_{\theta} = \Tr(e^{-\frac{i}{\hbar} \hat{H} t_{\theta}})
\]
with a complex time \(t_{\theta} = t e^{-i\theta}\) with positive \(t\) has a damping factor for all \(0 < \theta < \frac{\pi}{2}\) and for all hamiltonians that are bounded from below (up to an inessential overall factor due to the value of the ground state energy, if that happens to be negative).

Similar considerations can be made for path integrals in minkowskian and euclidean times with other boundary conditions. Path integrals in euclidean times are mathematically better defined (one may develop a mathematically well-defined measure theory on the space of functions), at least for quadratic actions and perturbations thereof. Path integral with a minkowskian time are more delicate, and physicists usually use the argument of rapid phase oscillations to deduce that unwanted terms vanish. The Wick rotation suggests a way of defining the path integral in real time starting from the euclidean time one. These points of mathematical rigor are not needed for the applications that we are going to consider, and the derivation of path integrals described previously is enough for our purposes.

\subsection{Comments}

We have seen that the quantization of a classical system with action $S[x]$ is achieved by the path integral $\int Dx \, e^{\frac{i}{\hbar}S[x]}$ that computes the transition amplitude.

In the path integral formulation, the classical limit is intuitive: macroscopic systems have large values of action in $\hbar$ units. Macroscopically small variations of paths can still make the phase variations $\frac{\delta S[x]}{\hbar}$ much bigger than $\pi$, so that amplitudes of nearby paths cancel by destructive interference. This is true except for variations that make $\delta S[x] = 0$, which is the condition that identifies the classical path. Nearby paths have amplitudes that sum coherently with the classical one, and the path integral is dominated by the classical trajectory.

The notation $\int Dx$ is symbolic and indicates the formal integration over the space of functions $x(t)$. To make it precise, one has to regulate the functional space by making it finite-dimensional (``regularization''). Then one integrates over the regulated finite-dimensional space, and eventually takes the continuum limit by removing the regularization parameters. If this procedure is done with care, the limit exists and gives the correct transition amplitude. In the previous derivation, we have seen that the space of paths is regulated by approximating the functions $x(t)$ by their $N-1$ values computed at intermediate points, the $x_k$'s with $k=1, ..., N-1$. This makes the space of functions finite dimensional. The action is discretized and evaluated using the approximated functions. At this stage, the integration over the regulated functional space is well-defined. Eventually, one takes the continuum limit ($N \to \infty$): if the integration measure is chosen appropriately, this limit exists and gives a viable definition of the path integral.

We started from canonical quantization and derived the above discretized form of the space of functions. This regularization is often called \textbf{Time Slicing (TS)}. Vice versa, one can start directly with the path integral, regulate it suitably, and use it to construct the quantum theory (Feynman originally started this way). The path integral is used to evaluate a transition amplitude that is seen to satisfy a Schrödinger wave equation. This can be viewed as an alternative approach to quantization. In the regularization procedure of the path integral, one must make several choices, and they may produce different transition amplitudes. For example, in a TS regularization one may discretize the potential term $V(x(t))$ in the action to $V(x_k)$ or $V(x_{k-1})$ or $V(\frac{1}{2}(x_k + x_{k-1}))$. In the present case, this makes no difference, and one obtains the same continuum limit.

These ambiguities are the path integral counterparts of the \textit{ordering ambiguities} of canonical quantization. Ordering ambiguities arise when one must construct a quantum hamiltonian out of a classical one: as $\hat{x}$ and $\hat{p}$ do not commute, it may happen that one must choose an ordering to define the quantum hamiltonian. Different orderings produce different quantum hamiltonians, and thus different quantum theories. Examples of systems where ordering ambiguities arise are the case of a charged particle in a magnetic field and the motion of a particle in a curved space, where the respective hamiltonians are:
\[
    H = \frac{(\vec{p} - \frac{e}{c}\vec{A}(x))^2}{2m} \,, \qquad H = \frac{1}{2m}g^{ij}(x)p_i p_j \,.
\]

We have introduced path integrals by considering a single degree of freedom. Extension to a finite number of degrees of freedom is immediate, so that quantizing the motion of one or more particles in a finite dimensional space does not pose any new conceptual problem. For example, the motion of a nonrelativistic particle in $\mathbb{R}^3$ with cartesian coordinates $\vec{x}$, in the presence of a scalar potential $V(\vec{x})$, is quantized by the following discretized path integral:
\[
    \int Dx \, e^{\frac{i}{\hbar} S[x]} = \lim_{N \to \infty} \int \left( \prod_{k=1}^{N-1} d^3 x_k \right) \left( \frac{m}{2\pi i \hbar \epsilon} \right)^{\frac{3N}{2}} e^{\frac{i}{\hbar} \sum_{k=1}^N \epsilon \left[ \frac{m}{2} \frac{(\vec{x}_k - \vec{x}_{k-1})^2}{\epsilon^2} - V(\vec{x}_{k-1}) \right]},
\]
where, of course, the classical action is
\[
    S[x] = \int_0^T dt \left( \frac{m}{2} \dot{\vec{x}}^2 - V(\vec{x}) \right) = \lim_{N \to \infty} \sum_{k=1}^N \epsilon \left[ \frac{m}{2} \frac{(\vec{x}_k - \vec{x}_{k-1})^2}{\epsilon^2} - V(\vec{x}_{k-1}) \right] \,.
\]
Formally, one can also consider the case of an infinite number of degrees of freedom, as appropriate for a field theory. In this case, convergence is not guaranteed, and the removal of the regularization may lead to infinite results. In the class of theories called renormalizable, the infinities can be removed consistently by a \textit{renormalization} procedure that redefines the dynamical variables and the coupling constants, and allows to obtain finite results, at least at the level of perturbation theory.