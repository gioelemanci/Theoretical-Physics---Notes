\chapter{Path Integrals}

Quantization can be introduced in two equivalent ways:
- operator formalism (canonical quantization, Hilbert space, linear operators, etc ..)
- path integrals (functional integrals).

Path integrals were introduced in quantum mechanics by Feynman in 1948, but until about 1970 they did not meet with much success, and the operatorial methods of canonical quantization were still the most widespread. In 1970, the success of gauge theories in developing the Standard Model of particle physics gave a strong impulse to path integral methods. Quantization of gauge theories is much more clear and elegant if performed with path integrals. Furthermore, path integrals indicate a way of relating a quantum field theory in D spacetime dimensions (\(D-1\) spaces and 1 time) to the statistical mechanics of a system in D space dimensions. This link has given rise to a way of thinking and defining field theories using statistical mechanics and renormalization group ideas, as introduced by Wilson and others (lattice theories). Nowadays,it is convenient to master both methods: according to the problem at hand, one may find one formalism more convenient than the other, even though they are supposed to be equivalent. To introduce path integrals, let us follow Feynman and consider the two-slit experiment for the electron. The standard treatment used to explain the behavior of an electron which passes through the two slits of a barrier and creates a figure of interference on a screen employs the wave nature of the electron together with the Huygens principle for calculating the interference pattern from the elementary waves that originate from the slits. Feynman proposes an alternative description. He suggests to keep thinking of the electron as a particle that however can accomplish both trajectories, each one with a certain “amplitude”. The total amplitude Atot is defined as the sum of the single amplitudes, and its square is related to the probability that the electron is revealed at a given point on the screen. Moreover, the elementary amplitude for each possible trajectory is related in a simple way to the classical action evaluated on the trajectory itself: Feynman, inspired by previous considerations of Dirac, associates to each trajectory an amplitude of unit norm (so that all trajectories “weigh” democratically the same way) and with phase equal to the value of the action \(S\) in units of \(\hbar\).

Thus we can write
\[
    A_{tot} = A(c_1) + A(c_2) + \dots + A(c_n), \quad A(c_n) = e^{\frac{i}{\hbar}S(c_n)}.
\]
thus we linked the total amplitude, which let us study probabilities in our experiments, to a sum of single aplitudes with the same module (1) but different phases, dfined by the action
\[
    S[x] = \int_0^{T} \d{t} \frac{1}{2}m \dot{x}^2.
\]
So for a path where we assume \(D \gg d\)\TODO{Insert drawing of slit.} we can write the action for the path
\[
    \begin{aligned}
        S(c_1) & =                                         \\
        S(c_2) & = \frac{m}{2} \frac{(D+d)^2}{T} = \dots .
    \end{aligned}
\]
where \(p = \frac{mD}{T}\) is the momentum of the electron. Therefore we can no study the total amplitude
\[
    A_{tot} = e^{\frac{i}{\hbar}S(c_1)} + e^{\frac{i}{\hbar}S(c_2)} = \dots
\]

It is easy to notice that the maximum probability, associated to the maximum ampliutude, of revealiing the electron on the screen is
\[
    e^{\frac{i}{\hbar} pd} = 1, \quad \implies \quad \frac{pd}{\hbar} = 2\pi n , \text{ with } n \in \mathbb{N},
\]
One can interpret this condition as defining a wavelength \(\lambda = \frac{h}{p}\) so that when d contains an integer number of times such wavelengths there is constructive interference. The de Broglie relation is obtained by this rudimentary “path integral” and suggests that it contains the essential elements of quantum mechanics. The number of slits can be increased, as well as the number of intermediate screens, to have the particle performing all possible paths from the initial point to the final point of observation, thus creating a path integral for the total amplitude.\TODO{Insert drawing of multiple paths.}

The action is used in an essential way
\[
    S[q(t)] = \int \d{t} \mathrm{L} (q(t),\,\dot{q}(t)).
\]
The classic path is the one that minimizes the action
\[
    \delta S = 0, \quad \implies \quad
\]
In quantum mechanics, the transition amplitude is obtained by using the action \(S[q]\) for any possible path
\[
    A = \sum_{c_n} e^{\frac{i}{\hbar}S(c_n)} \equiv \int \mathrm{D} q(t) \, e^{\frac{i}{\hbar}S[q(t)]}
\]

The final notation introduced here is that of the \textbf{path integral} or \textbf{functional integral}: $S[q]$ is a functional, i.e., a function of the functions $q(t)$, that indicate the possible ``paths'' of the system, and the symbol $\int \mathcal{D}q$ indicates the formal integration over the space of paths $\{q(t)\}$. Various mathematical subtleties on how to define exactly the path integration are still open. Nevertheless, path integrals have become one of the main tools to study quantum systems.
In this formulation, the classic limit is intuitive: macroscopic systems have large values of the action $S$ in units of $\hbar$, the quantum of action. Small variations of a path induce phase variations $i\frac{1}{\hbar} \delta S[q]$ much bigger with respect to $i\pi$ (recall that for such a phase $e^{i\pi} = -1$) and the amplitudes of nearby paths cancel by destructive interference. This happens except at the point in which the action has a minimum, $\delta S = 0$, which identifies the classic trajectory. Trajectories close to the classical one have amplitudes that add up coherently since the phase does not vary: the functional integral is dominated by the classic path!

\section{Action Principle}

Let us briefly review the action principle in mechanics and field theories, considering the case of a particle. The main purpose is to underline its relation to canonical quantization and to stress the relevance of symmetries. As anticipated, the action is essential for the path integral quantization.

Consider a non-relativistic particle of mass \(m\) that moves in a single dimension with coordinate \(q\) and subject to a conservative force \(F = -\frac{\partial}{\partial q}V\). Newton’s equations of motion reads
\[
    F = m \ddot{q},
\]
and can be derived from an action principle. The action is a functional of the trajectory of the particle \(q(t)\) (the dynamical variable of the system) and associates a real number to each function \(q(t)\). In general, physical systems are described by an action of the type
\[
    S[q(t)] = \int \d{t} \mathrm{L} (q(t),\,\dot{q}(t)), \quad \mathrm{L}(q(t),\,\dot{q}(t)) =
\]


[...]

\subsection{Hamiltonian formalism}

The basic idea of the hamiltonian formalism is to have equations of motion that are first order in time. To review it, we follow a simple example: a non-relativistic particle of coordinates \(q^i\) and configuration space lagrangian
\[
    pdf
\]
where indices are lowered with the metric \(\delta_{ij}\) (and are thus equivalent to upper indices in our model, the distinction of upper and lower indices is, however, useful in more general contexts). Transition to the hamiltonian formalism takes place as follows:
\begin{enumerate}
    \item The dynamical variables are doubled by introducing conjugate momentum \(p^i\) to each coordinate \(q_i\)
          \[
              pdf
          \]
          The set \((q_i,\,p^i)\) constitutes the coordinates of phase space.
    \item The hamiltonian \(H(q,\,p)\) is defined as the Legendre transform of the lagrangian \(\mathrm{L}\)
          \[
              pdf
          \]
          It is a function on phase space.
    \item The \textbf{Poisson brackets} are defined as follows. For any two functions A and B of phase space, their Poisson brackets are defined by
          \[
              pdf
          \]
          where we have used the summation convention for repeated indices. In particular,
          \[
              pdf
          \]
    \item The hamiltonian equations of motion can be written as
          \[
              pdf
          \]
          and are of the first order in time. In our example, they become
          \[
              pdf
          \]
          and are evidently equivalent to the lagrangian equations \([\dots]\) . The hamiltonian \(H\) is interpreted as the generator of time translations, and moves the initial conditions (a point in phase space) over time by an infinitesimal amount \(\mathrm{d} t\). The generator of these canonical transformations is given by \(H \mathrm{d} t\), and acts through the Poisson brackets (\([\dots ]\)).
\end{enumerate}

These equations can be obtained from an action. In phase space, the action takes the form
\[
    pdf
\]
and minimizing it, one finds
\[
    pdf
\]
from which one recognizes Hamilton’s equations of motion. Note that in this formulation one needs \(2n\) integration constants, which are given by specifying the coordinates \(q^i\) at initial and final times.

The hamiltonian structure is the starting point of canonical quantization:
\[
    pdf
\]
where the classical dynamical variables \(z^a\) are elevated to linear operators \(\hat{z}^a\) acting on a Hilbert space. The quantum commutation relations are fixed by the values of the classical Poisson bracket. The vectors of the Hilbert space describe the possible quantum states of the system, whose evolution is governed by the Schr¨odinger equation.

\section{Canonical Quantization}

Canonical quantization is constructed starting from the hamiltonian formulation of a classical system. It is obtained by lifting its phase space coordinates, the generalized coordinates \(x^i\) and conjugate momenta \(p_i\), to linear operators \(\hat{x}^i\) and \(\hat{p}_i\) that act on a linear space endowed with a positive definite norm, the Hilbert space of physical states \(\mathcal{H}\). The basic operators must satisfy commutation relations required to be equal \(i\hbar\) times the value of the corresponding classical Poisson brackets
\[
    pdf
\]
All classical observables \(A(x,p)\), which are functions on phase space, become linear operators \(\hat{A} (\hat{x} ,\hat{p})\) acting on the Hilbert space \(\mathcal{H}\). The most important example is given by the hamiltonian function \(H(x,p)\), which upon quantization becomes the hamiltonian operator \(\hat{H} (\hat{x} ,\hat{p})\). The latter generates the time evolution of any state \(\ket{\psi} \in \mathcal{H}\) through the Schrodinger equation
\[
    i\hbar \frac{\partial}{\partial t} \ket{\psi} = \hat{H} \ket{\psi}.
\]
The corresponding solution is a time-dependent state \(\ket{\psi}(t)\) that describes the evolution of the quantum system. This setup is known as the Schrodinger picture of quantum mechanics. It is a formal quantization procedure that becomes operative once one finds an irreducible unitary representation of the operator algebra in eq. (1). A mathematical result, known as the Stonevon Neumann theorem, states that in quantum mechanics all irreducible representations of (1) are unitarily equivalent, so that there is a unique procedure of quantizing a classical system.\footnote{Up to the problem of resolving ordering ambiguities, often present when one tries to relate the classical hamiltonian \(H(x,p)\) to its quantum counterpart \(\hat{H} (\hat{x} ,\hat{p})\).} Historically, this theorem made it clear that the Schr¨odinger formulation of quantum mechanics was equivalent to the one proposed by Heisenberg with its matrix mechanics (known as the Heisenberg picture). Let us consider, more specifically, the motion of a nonrelativistic particle in one dimension in the presence of an external potential \(V(x)\). The classical dynamics is fixed by the action

    [...] sofo [...] pdf [...]
So now we can study the transition amplitude
\[
    A = \bra{\psi_f} e^{-\frac{i}{\hbar} \hat{H} T} \ket{\psi_i} \implies i \hbar \frac{\partial}{\partial T} A = \bra{\psi_f} \hat{H} e^{-\frac{i}{\hbar} \hat{H} T} \ket{\psi_i}.
\]

\section{Path Integrals in Phase Space}

To derive a path integral expression for the transition amplitudes, we start by inserting twice the identity operator \(\mathbb{I}\), expressed using the eigenstates of the position operator
\[
    pdf
\]
then we can rewrite the transition amplitude derived previously by insterting two of this identityoperators
\[
    A = \bra{\psi_f} e^{-\frac{i}{\hbar} \hat{H} T} \ket{\psi_i} = \dots
\]
where \(\psi_i(x_i) = \bra{x}\ket{\psi_i}\) and \(\psi_f(x_f) = \bra{x}\ket{\psi_f}\) are the wave functions for the initial and final
states. This rewriting shows that it is enough to consider the matrix element of the evolution operator between position eigenstates
\[
    A(x_i,\,x_f,\,T) = \bra{x_f} e^{-\frac{i}{\hbar} \hat{H} T} \ket{x_i}
\]
where \(T = t_f - t_i\) is the total propagation time. It satisfies the Schrodinger equation
\[
    pdf
\]
with initial condition \([\dots ]\).

We are going to consider quantum hamiltonians of a paarticle interacting with a generc potential \(\hat{V}(\hat{x})\):
\[
    pdf
\]
The derivation of the path integral proceeds now as follows. One splits the transition amplitude \(A(x_i,\,x_f,\,T)\) as the product of \(N\) factors, and inserts the completeness relation \(N-1\) times in between the factors
\[
    \begin{aligned}
        A(x_i,\,x_f,\,T) & = \bra{x_f} e^{-\frac{i}{\hbar} \hat{H} T} \ket{x_i} = \bra{x_f} \left(e^{-\frac{i}{\hbar} \hat{H} \frac{T}{N}}\right)^N \ket{x_i} \\
                         & = \dots
    \end{aligned}
\]
where for convenience we have denoted \(x_0 = x_i\), \(x_n = x_f\), and \(\epsilon = \frac{T}{N}\). To evaluate this expression better, it is convenient to use the resolution of the identity N more times, now expressed in terms of the momentum eigenstates
\[
    \mathbb{I} = \int \frac{\d{p}}{2\pi \hbar} \ket{p} \bra{p}, \quad \text{with } \bra{p}\ket{p^{\prime}} = 2\pi\hbar \delta(p-p^{\prime}),
\]
to obtain
\[
    A = \int \left(\prod_{k=1}^{N-1} \d{x_k}\right) \dots
\]
This is an exact expression. Note that there is one more integration over momenta than integrations over coordinates, a consequence of choosing coordinate eigenstates as initial and final states in the transition amplitude. Now, one can manipulate this expression further by making approximations that are valid in the limit \(N \to \infty\) (i.e., \(\epsilon \to  0\)). The crucial point is the evaluation of the following matrix element
\[
    \begin{aligned}
        \bra{p} e^{-\frac{i\epsilon}{\hbar}\hat{H}(\hat{x},\,\hat{p})}\ket{x} & = \\
                                                                              & = \\
    \end{aligned}
\]
These approximations are all valid in the limit of small \(\epsilon\). The substitution \(\bra{p}\hat{H}(\hat{x},\,\hat{p})\ket{x} = \bra{p}\ket{x}H(x,p)\) follows from the simple structure of the considered hamiltonian (15), that allows one to act with the momentum operator on the left, and with the position operator on the right, to have the operators replaced by the corresponding eigenvalues. Notice that there is no need for commuting operators inside the hamiltonian, because of the simplicity of the hamiltonian
we have considered. The final result is that all operators are simply replaced by eigenvalues. This way the quantum hamiltonian \(\hat{H}(\hat{x} ,\, \hat{p})\) gets replaced by the classical function \(H(x,p) = \frac{p^2}{2m} + V(x)\). There exists a mathematically rigorous proof that these manipulations are correct for a wide class of physically interesting potentials \(V(x)\) (the “Trotter formula”). We do not review these subtleties, as the physically intuitive derivation given above is enough for our purposes.

Using now eq. (19), and remembering that the wave functions of the momentum eigenstates (the plane waves) are normalized as
\[
    \bra{x}\ket{p} = e^{\frac{i}{\hbar}px}, \quad \bra{p}\ket{x} = (\bra{x}\ket{p})^* = e^{-\frac{i}{\hbar}px}
\]
that follows from the normalization chosen in (11) and (17), one obtains
\[
    pdf
\]
up to terms that vanish for \(\epsilon \to 0\). This expression can now be inserted in (18). At this stage, the transition amplitude does not contain any more operators, bra and kets. It contains just integrations, though a big number of them, of ordinary functions
\[
    A = \lim_{N \to \infty} \int \left(\prod_{k=1}^{N-1} \d{x_k}\right) \left(\prod_{k=1}^{N} \frac{\mathrm{d} p_k}{2\pi \hbar}\right)e^{\frac{i\epsilon}{\hbar}\sum_{k=1}^N \left[p_k \frac{x_k - x_{k-1}}{\epsilon} - H(x_{k-1},\,p_k)\right] } = \int \mathrm{D} x \mathrm{D} p \, e^{\frac{i}{\hbar}S[x,\,p]}.
\]
This is the \textbf{path integral in phase space}. One recognizes in the exponent a discretization of the classical phase space action
\[
    S[x,\,p] = \int_{t_i}^{t_f} \d{t} \left(p\dot{x} - H(x,\,p)\right),\quad \implies \quad \sum_{k=1}^{N} \epsilon\left( p_k \frac{x_k - x_{k-1}}{\epsilon} - H(x_{k-1},\,p_k)\right),
\]
where \(t_f - t_i = T = N\epsilon\) is the total propagation time, with the paths in phase space discretized as
\[
    x(t),\,p(t) \longrightarrow x_k = x(t_i + k \epsilon),\, p_k = (t_i + k \epsilon).
\]
The last way of writing the amplitude in (22) is symbolic but suggestive: it indicates the sum over all paths in phase space weighted by the exponential of \(\frac{i}{\hbar}\) times the classical action. It depends implicitely on the boundary conditions assigned to the paths \(x(t)\).

\section{Path Integrals in Configuration Space}

The path integral in configurations space is easily derived by integrating over the momenta. The dependence on momenta in the exponent of (22) is at most quadratic and can be eliminated by gaussian integration
\[
    pdf
\]
which is valid for \(\alpha >0\), and extended analytically to include complex values of \(\alpha\) (see section 4.1 for details).\TODO{Put reference.} In particular, we consider gaussian integrals of the type
\[
    pdf
\]
obtained by square completion. Note that the final exponential is the original exponential inside the integral with argument evaluated at the minimum in \(p\) (saddle point).
Returning to the path integral, and considering the hamiltonian \(H(x, p) = \frac{p^2}{2m} + V(x)\), one completes the squares
\[
    footnote
\]
and performs the gaussian integrations over the momenta
\[
    A = \lim_{N \to \infty} \int \left(\prod_{k=1}^{N-1} \d{x_k}\right) \left(\frac{m}{2\pi i \hbar \epsilon}\right)^{\frac{N}{2}} e^{\frac{i\epsilon}{\hbar}\sum_{k=1}^N \left[\frac{m}{2}\frac{(x_k - x_{k-1})^2}{\epsilon^2} - V(x_{k-1})\right]} = \int \mathrm{D} x \, e^{\frac{i}{\hbar}S[x]}.
\]
Our exponent is complex, so the function keeps oscillating; by analytic continuation we can use this results also in the complex case.
This is the path integral in configuration space. It contains in the exponent the configuration space action suitably discretized
\[
    S[x] = \int_{t_i}^{t_f} \d{t} \left( \frac{m}{2} \dot{x}^2 - V(x) \right) \longrightarrow \sum_{k=1}^N \epsilon \left[\frac{m}{2}\frac{(x_k - x_{k-1})^2}{\epsilon^2} - V(x_{k-1})\right].
\]
Again, the last way of writing the path integral in (27) is symbolic and indicates the formal sum over paths in configuration space, weighted by the exponential of \(i\hbar\) times the classical action. The space of paths is given by the space of functions \(x(t)\) with boundary values \(x(t_i)=x_i\) and \(x(t_f) = x_f\). It is an infinite dimensional space. How to perform concretely the path integral over this functional space is defined precisely by the discretization, that approximates a function \(x(t)\) by its \(N+1\) values \(x_k = x(t_i + \epsilon k)\) at \(k = 0, 1, 2, ..., N\), as shown in fig.\TODO{Add figure.}

Thus, we have constructed the path integral that computes quantum mechanical amplitudes
\[
    A = \int \mathrm{D} x(t) \, e^{\frac{i}{\hbar}S[x(t)]},
\]
by the sum of all histories of paths weighted by a phase given by the action, with all paths contributing, as in fig. 2.

\subsection{Free Particle}

For a free particle (\(V(x)=0\)) one may use repeatedly gaussian integration and calculate from eq. (27) the exact transition amplitude\TODO{Add image.}
\[
    A(x_i,\,x_f,\,T) = \bra{x_f} e^{-\frac{i}{\hbar} \hat{H}T} \ket{x_i} = \sqrt{\frac{m}{2 \pi i \hbar T}} e^{\frac{i}{\hbar} \frac{m (x_f - x_i)^2}{2T}}.
\]
It satisfies the free Schrodinger equation
\[
    i\hbar \frac{\partial}{\partial T} A(x_i,\,x_f,\,T) = - \frac{\hbar^2}{2m}\frac{\partial^2}{\partial x_f^2} A(x_i,\,x_f,\,T)
\]
with initial condition \(A(x_i,\,x_i,\,0) = \delta(x_f - x_i)\).
This way, the path integral has produced a solution of the Schrodinger equation. The result is very suggestive: up to a prefactor, it is given by the exponential of \(i\hbar\) times the classical action evaluated on the classical path, i.e., the path that satisfies the classical equations of motion. This is typical for those cases in which the semiclassical approximation is exact. One may interpret the prefactor as due to quantum (“one-loop”) corrections to the classical (“tree-level”) result. The free particle case is also quite special: the exact final result is valid for any
\(N\), and there is no need to take the limit \(N \to  \infty\). The case \(N=1\), which carries no integration at all, is already exact.
A formal but useful way of calculating gaussian path integrals is achieved by working directly in the continuum limit. One does not need the precise definition of the path integral measure but uses only its formal properties, in particular, its translational invariance. The calculation is formal in the sense that one assumes properties of the path integral measure (that eventually must be proven by an explicit regularization and construction, as the one given earlier). The calculation goes as follows. The action is \(S[x]= \int_0^T \d{t} \frac{m}{2}\dot{x}^2\), and the classical equations of motion with the boundary conditions are solved by
\[
    x_{cl} (t) = x_i + (x_f - x_i)\frac{t}{T}.
\]
One can represent a generic path \(x(t)\) as the classical path \(x_{cl}(t)\) plus quantum fluctuations \(q(t)\)
\[
    x(t) = x_{cl}(t) + q(t),
\]
where the fluctuations q(t) must vanish at \(t=0\) and \(t=T\) to preserve the boundary conditions. One may interpret \(x_{cl}(t)\) as the origin in the space of functions. Then, one computes the path integral as follows
\[
    \begin{aligned}
        A(x_i,\,x_f,\,T) & = \int \mathrm{D} x \, e^{\frac{i}{\hbar}S[x]} = \int \mathrm{D} (x_{cl} + q) \, e^{\frac{i}{\hbar} S[x_{cl} + q]}                                                \\
                         & = e^{\frac{i}{\hbar} S[x_{cl}]} \int \mathrm{D} q \, e^{\frac{i}{\hbar} S[q]} = N e^{\frac{i}{\hbar} S[x_{cl}]} = N e^{\frac{i}{\hbar}\frac{m(x_{f-x_i})^2}{2T}},
    \end{aligned}
\]
where translational invariance of the path integral measure has been used in the form \(\mathrm{D}x = \mathrm{D}(x_{cl}+q) = \mathrm{D}q\). There is no linear term in \(q(t)\) in the action because the function \(x_{cl}(t)\) solves the classical equations of motion: for quadratic actions one has \(S[x_{cl} + q] = S[x_{cl}] + S[q]\). The normalization factor \(N = \int \mathrm{D} q\, e^ {\frac{i}{\hbar} S[q]}\) is undetermined by this method, but it is a constant that does not depend on \(x_i\) and \(x_f\). Very often, its precise value is not needed, but one can fix it by requiring that the final result satisfies the Schrodinger equation, finding \(N = \sqrt{\frac{m}{2\pi i \hbar T}}\).

\subsection{Euclidean Time and Statistical Mechanics}

Quantum mechanics can be related to statistical mechanics by an analytic continuation. We introduce this relation by considering the free particle.
Continuing analytically the time parameter to purely imaginary values by \(T \to  i \beta\) with real \(\beta\), and setting \(\hbar = 1\), the free Schrodinger equation (31) turns into the heat equation
\[
    \frac{\partial}{\partial \beta} A = \frac{1}{2m} \frac{\partial^2}{\partial x_f^2} A.
\]
Its fundamental solution, i.e. the solution with boundary condition \(A \xrightarrow{\beta \to 0} \delta(x_f - x_i)\) is
\[
    A = \sqrt{\frac{m}{2 \pi \beta}} e^{-\frac{m(x_f - x_i)^2}{2\beta}},
\]
and can be obtained from (30) by the same analytic continuation. This continuation is called “\textbf{Wick rotation}”, see fig. 3.\TODO{Add figure.}
The Wick rotation can be performed directly on the path integral to obtain euclidean path integrals. Analytically continuing the time variable \(t \to -i \tau\) , one finds that the action with “minkowskian” time (i.e. with a real time \(t\)) turns into an “euclidean” action \(S_E\) defined by
\[
    i S[x] = \dots
\]
where in the euclidean action one defines \(\dot{x} = \frac{\mathrm{d}}{\mathrm{d} \tau} x\), with \(\tau\) usually called “euclidean time” (or imaginary time). The euclidean action thus defined is positive definite. It appears in the path integral so that
\[
    \bra{x_f} e^{-\beta \hat{H}} \ket{x_i} = \int \mathrm{D} x \, e^{-S_E[x]}.
\]
The operator on the left-hand side is called the “heat kernel” and the path integrals computes its matrix elements. For a free theory, the path integral is truly gaussian, with exponential damping rather than with increasingly rapid phase oscillations. In this form, it coincides with the functional integral introduced by Wiener in the 1920's to study brownian motion and the heat equation (that explains why eq. (39) is called the heat kernel). Such euclidean path integrals are useful in statistical mechanics, where \(\beta\) is related to the inverse temperature \(\Theta\) by \(\beta = \frac{1}{k \Theta}\) , where \(k\) is the Boltzmann’s constant. To see this, let us consider the trace of the evolution operator \(e^{-\frac{i}{\hbar} \hat{H} T}\). It can be written using energy eigenstates (labeled by \(n\) if the spectrum is discrete) or equivalently using position eigenstates (which are continuous and labeled by \(x\))
\[
    Z \equiv \Tr(e^{-\frac{i}{\hbar}\hat{H}T}) = \sum_{n} e^{-\frac{i}{\hbar}E_n T} = \int \d{x} \bra{x} e^{-\frac{i}{\hbar} \hat{H} T} \ket{x}.
\]
It can be Wick rotated \(Z \to  Z_E\) (with \(T \to - i \beta\)) to obtain the statistical partition function \(Z_E\) of the quantum system with hamiltonian \(\hat{H}\). Setting \(\hbar=1\), it reads
\[
    Z_E = \Tr(e^{-\beta \hat{H}}) = \sum_{n} e^{- \beta E_n} = \int \d{x} \bra{x} e^{-\beta \hat{H}} \ket{x}.
\]
Again, it is immediate to find a path integral representation of the statistical partition function: one performs a Wick rotation of the path integral action, sets the initial state (at euclidean time \(\tau = 0\)) equal to the final state (at euclidean time \(\tau = \beta\)), and sums over all possible states, as indicated in (41). The paths become closed, as \(x(\beta)=x(0)\), and the partition function becomes
\[
    Z_E = \Tr(e^{-\beta \hat{H}}) = \int_{PBC} \mathrm{D} x\, e^{-S_E[x]},
\]
where \(PBC\) stands for “periodic boundary conditions”, indicating the sum over all paths that close onto themselves in an euclidean time \(\beta\).
Introduced here for the free theory, the Wick rotation is supposed to be of more general value, relating quantum mechanics to statistical mechanics in the interacting case as well. Even if one is interested in the theory with a real time, nowadays, one often works in the euclidean version of the theory, where factors of the imaginary unit \(i\) are absent, and path integral convergence is more easily kept under control. Only at the very end one performs the inverse Wick rotation to read off the result for the theory in real time.

The Wick rotation procedure is better appreciated by first considering the usual time as corresponding to the real line of a complex plane. Then, defining \(t_{\theta} = t e^{-i\theta} \), the usual real time appears at \(\theta=0\), while the euclidean time \(\tau\) appears at \(\theta = \frac{\pi}{2}\)  as \(t_{\frac{\pi}{2}} = -i \tau\). The analytical continuation of all physical quantities is achieved by continually increasing \(\theta\) form 0 to \(\frac{\pi}{2}\), a clockwise rotation of the real axis into the imaginary one. The generalized partition function
\[
    Z_{\theta} = \Tr(e^{-\frac{i}{\hbar} \hat{H} t_{\theta}})
\]
with a complex time \(t_{\theta} = t e^{-i\theta}\) with positive \(t\) has a damping factor for all \(0 < \theta < \frac{\pi}{2}\) and for all hamiltonians that are bounded from below (up to an inessential overall factor due to the value of the ground state energy, if that happens to be negative).

Similar considerations can be made for path integrals in minkowskian and euclidean times with other boundary conditions. Path integrals in euclidean times are mathematically better defined (one may develop a mathematically well-defined measure theory on the space of functions), at least for quadratic actions and perturbations thereof. Path integral with a minkowskian time are more delicate, and physicists usually use the argument of rapid phase oscillations to deduce that unwanted terms vanish. The Wick rotation suggests a way of defining the path integral in real time starting from the euclidean time one. These points of mathematical rigor are not needed for the applications that we are going to consider, and the derivation of path integrals described previously is enough for our purposes.