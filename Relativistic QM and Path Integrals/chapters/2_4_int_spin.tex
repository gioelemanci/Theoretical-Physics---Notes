\section{Integer Spin Particles}

At this point, it is relatively easy to describe relativistic wave equations for particles of spin \(s \geq 1\) with non-vanishing mass, modeling them on the Klein-Gordon equation (for bosonic fields of integer spin) and Dirac equation (for fermionic fields of semi-integer spin). They are known as Fierz-Pauli equations. The difficulties lie in the introduction of interactions, a non-trivial and subtle problem that we are not going to discuss in these notes.

\subsection{Fierz-Pauli equations}

In the massive case of integer spin \(s\) (i.e. \(s=0,\,1,\,2,\,\dots\) is an integer) the wave function is given by a completely symmetric tensor of rank \(s\), i.e. with \(s\) vector indices, \(\phi_{\mu_1,\,\dots,\,\mu_s}\). It satisfies the KG equation in addition to constraints that impose transversality and the condition of vanishing trace\footnote{The index notation is to indicate that the unnamed index (\(\mu_1\) for the second equation, \(\mu_1\) and \(\mu_2\) for the third one) can be chosen arbitrarily among \(\mu_1,\,\dots,\,\mu_s\), due to the symmetry of the tensor.}
\begin{equation}
    \begin{aligned}
        (\Box -m^2) \phi_{\mu_1,\,\dots,\,\mu_s} = 0,         \\
        \partial^{\mu}\phi_{\mu,\,\mu_2,\,\dots,\,\mu_s} = 0, \\
        \phi^{\mu}_{\ \mu,\,\mu_3,\,\dots,\,\mu_s} = 0,
    \end{aligned}
    \label{eq:fierz_pauli_equations_integer}
\end{equation}
to understand their meaning, it is useful to study plane wave solutions using the ansatz
\[
    \phi_{\mu_1,\,\dots,\,\mu_s}(x) = \epsilon_{\mu_1,\,\dots,\,\mu_s}(p) e^{i p_\mu x^{\mu}}.
\]
where the polarization tensor \(\epsilon_{\mu_1,\,\dots ,\,\mu_s}(p)\) describes covariantly the spin orientation. Then, eqs. \eqref{eq:fierz_pauli_equations_integer} reduce to
\[
    \begin{aligned}
        p^2 + m^2 = 0,                                         \\
        p^{\mu} \epsilon_{\mu,\,\mu_2,\,\dots,\,\mu_s}(p) = 0, \\
        \epsilon^{\mu}_{\ \mu,\,\mu_3,\,\dots,\,\mu_s}(p) = 0.
    \end{aligned}
\]
\begin{enumerate}
    \item The first equation imposes the correct relativistic relation between energy and momentum on the plane wave.
    \item The second equation (the transversality condition) eliminates in a covariant way non-physical degrees of freedom: choosing the frame of reference at rest with the particle, where the 4-momentum reduces to \(\overline{p}^{\mu} = (m,\,0,\,0,\,0)\), one recognizes that the independent components of the wave function must have only spatial indices:
          \[
              m \epsilon_{0,\,\mu_2,\,\dots,\,\mu_s}(\overline{p}) = 0 \implies \epsilon_{i_1,\,i_2,\,\dots,\,i_s}(\overline{p}) \neq 0.
          \]
          where the index \(\mu = (0,i)\) is split into time and space components. This is due to the fact that
          The remaining polarizations describe the possible orientations in space of the spin.
    \item The third equation (the vanishing trace condition) reduces the components of the polarization tensor to have only those components corresponding to the irreducible representation of spin s, which as known from quantum mechanics form a traceless, completely symmetric tensor of rank \(s\) of the rotation group \(\mathrm{SO}(3)\). It it has precisely \(2s+1\) independent components corresponding to the \(2s + 1\) possible spin projections along a quantization axis.
\end{enumerate}
Let us verify the number of polarizations by counting the number of independent components of a fully symmetric tensor with s indices taking only three values\footnote{The number of independent components of a completely symmetric tensor of rank \(s\) in \(d\) dimension is \(\frac{d(d+1)(d+2)\dots(d+s-1)}{s!}\).} (the spatial directions of the frame at rest with the particle) and then subtracting the components associated with the trace of the tensor (to be eliminated to get a vanishing trace)\footnote{A  trace can be taken on any two indices, but by symmetry, it is equivalent to taking the trace on the first two indices: the remaining tensor has \(s-2\) totally symmetric indices.}
\[
    \frac{3 \cdot 4 \cdot \cdot (3+s-1)}{s!}-\frac{3\cdot 4 \cdot \cdot (3+s-3)}{(s-2)!} = \frac{1}{2} (s+2)(s+1) - \frac{1}{2} s (s-1) = 2s + 1.
\]
This number is correct and supports the statement that the wave field \(\phi_{\mu_1,\,\dots,\,\mu_s}\) corresponds to massive quanta of spin \(s\). The calculation is valid for \(s\geq 2\) but easily extended to lower \(s\). In the case of a massive particle of half-integer spin \(s=n+\tfrac12\) (where \(n\) is an integer), the field is a spinor with in addition \(n\) symmetrical vector indices: \(\psi_{\mu_1,\,\dots,\,\mu_s}\). It satisfies a Dirac equation with additional constraints that impose transversality and gamma-tracelessness
\begin{equation}
    \begin{aligned}
        (\gamma^{\mu}\partial_\mu + m) \psi_{\mu_1,\,\dots,\,\mu_s} = 0, \\
        \partial^{\mu}\psi_{\mu,\,\mu_2,\,\dots,\,\mu_s} = 0,            \\
        \gamma^{\mu}\psi_{\mu,\,\mu_2,\,\dots,\,\mu_s} = 0,
    \end{aligned}
    \label{eq:fierz_pauli_equations_half_integer}
\end{equation}
We will not discuss these equations any further. In the limit of vanishing mass, the correct field equations must have only two physical polarizations\footnote{Recall that one cannot find a frame at rest with the particle, as the particle necessarily travels at the speed of light in all reference frames, and so the counting above must be modified.} (the two possible helicities \(h=\pm s\)). This is usually obtained by considering equations with gauge symmetries. They need a more detailed discussion, leading to the so-called \textit{Fronsdal equations}. They will not be presented in these notes except for the case \(s=1,\,2\). \textbf{Gauge symmetries} are responsible for reducing the number of degrees of freedom (i.e., the number of components of the wave function that satisfies the equations of motion) from \(2s+1\), corresponding to a massive particle of spin \(s\), to the 2 components required by massless particles. As said, we will briefly review the case of massless spin 1, which certainly admits non-trivial interactions with fields of spin 0, \(\tfrac12\), and 1, as used in the construction of the Standard Model, and briefly mention the case of massless spin 2 (the graviton, the quantum of the gravitational waves). This is done by reviewing first the massive case to better understand differences and similarities. Massless higher spin particles do not seem to admit non-trivial interactions, and they have not found phenomenological applications thus far.

\subsection{Proca Equations}

Massive particles of spin 1 are described by \eqref{eq:fierz_pauli_equations_integer} with \(s=1\). It is customary to denote the wave function \(\phi_\mu(x)\) by \(A_\mu(x)\), so that the equations read
\begin{equation}
    \begin{aligned}
        (\Box -m^2) A_{\mu} = 0, \\
        \partial^{\mu} A_{\mu} = 0.
    \end{aligned}
    \label{eq:proca_equations}
\end{equation}
For this specific case, they are known as \textbf{Proca equations}. They can be derived from an action
\begin{equation}
    S_P[A_\mu] = \int \mathrm{d}^4 x \left( -\frac{1}{4} F_{\mu \nu} F^{\mu \nu} - \frac{1}{2} m^2 A_\mu A^{\mu} \right),
    \label{eq:proca_action}
\end{equation}
where \(F_{\mu \nu} = \partial_{\mu} A_{\nu} - \partial_{\nu} A_{\mu}\) is the field strength tensor.
Integration by parts brings the action in an alternative form
\[
    \begin{aligned}
        S_P[A_\mu] & = \int \mathrm{d}^4 x -\frac{1}{4} \left( (\partial_\mu A_\nu - \partial_\nu A_\mu) (\partial^\mu A^\nu - \partial^\nu A^\mu) -\frac{1}{2} m^2 A_\mu A^{\mu} \right)                       \\
                   & = \int \mathrm{d}^4 x \left( -\frac{1}{2} (\partial_\mu A_\nu)(\partial^{\mu} A^{\nu}) + \frac{1}{2} (\partial_\mu A_\nu)(\partial^{\nu} A^{\mu}) - \frac{1}{2} m^2 A_\mu A^{\mu} \right),
    \end{aligned}
\]
since the we could integrate by parts the second term in the second step and neglect boundary terms
\[
    \frac{1}{2} (\partial_\mu A_\nu)(\partial^{\nu} A^{\mu}) = - \frac{1}{2} A_\nu \partial^{\nu} (\partial_\mu A^{\mu}) = \frac{1}{2} (\partial_\nu A^{\nu})(\partial_\mu A^{\mu}).
\]
The final result
\[
    S_P[A_\mu] = \int \mathrm{d}^4 x \left( -\frac{1}{2} (\partial_\nu A_\mu)(\partial^{\nu} A^{\mu}) - \frac{1}{2} m^2 A_\mu A^{\mu} + \frac{1}{2} (\partial_\mu A^{\mu})^2 \right)
\]
is similar to the action of four Klein-Gordon fields \(A_\mu\) (given by the first two terms), but with the crucial addition of the third term \((\partial_\mu A^{\mu})^2\) with a very precise coefficient. The latter is responsible for the emergence of a constraint that reduces the number of degrees of freedom from 4 to 3. Let us verify this statement. By varying the action with respect to \(A_\mu\) one finds
\[
    \begin{aligned}
        \delta S_P[A_\mu] & = \int \mathrm{d}^4 x \left( -\frac{1}{2} F_{\mu \nu}\delta F^{\mu \nu} - m^2 A_\nu \delta A^\nu \right)                                                                 \\
                          & = \int \mathrm{d}^4 x \left( -\frac{1}{2} F_{\mu \nu}\partial^{\mu}\delta A^\nu + \frac{1}{2} F_{\mu \nu}\partial^{\nu}\delta A^\mu - m^2 A_\nu \delta A^\nu \right)     \\
                          & = \int \mathrm{d}^4 x \left( + \frac{1}{2} (\partial^{\mu}F_{\mu \nu})\delta A^\nu - \frac{1}{2} \partial^{\nu}F_{\mu \nu} \delta A^\mu - m^2 A_\nu \delta A^\nu \right)
    \end{aligned}
\]
where in the last step we integrated by parts and neglected boundary terms. Renaming dummy indices, one finds the \textbf{Proca equations of motion}:
\[
    \frac{\delta S_P[A_\mu]}{\delta A^\nu(x)} = 0 = \partial^{\mu}F_{\mu \nu} - m^2 A_\nu.
\]
They are equivalent to the previous ones in \eqref{eq:proca_equations}. In fact, the identity \(\partial^{\mu} \partial^{\nu} F_{\mu \nu}\)  implies
\[
    \partial^{\nu} \partial^{\mu} F_{\mu \nu} = \partial^{\nu}(m^2 A_\nu) = 0,
\]
which is zero due to the antisymmetry of \(F_{\mu \nu}\) contracted with the symmetric operator \(\partial^{\mu} \partial^{\nu}\). Thus for \(m\neq 0\) one has a constraint
\[
    \partial^{\mu} A_\mu = 0.
\]
Using this relationship, one rewrites Proca equations found estremizing the lagrangian as four Klein-Gordon equations plus the constraint, as in eq. \eqref{eq:proca_equations}:
\[
    \begin{aligned}
        \partial^{\mu} F_{\mu \nu} & = \Box A_\nu - \partial_\nu (\partial^{\mu} A_\mu) = \Box A_\nu = m^2 A_\nu, \\
                                   & \implies \begin{dcases}
                                                  (\Box -m^2) A_{\mu} = 0, \\
                                                  \partial^{\mu} A_{\mu} = 0.
                                              \end{dcases}
    \end{aligned}
\]
The constraint tells that only three of the four components of \(A_\mu\) are independent, and the equations covariantly describe the three polarizations expected for a particle of spin 1. The invariance of the action and the equations of motion under Lorentz transformations is obvious, with \(A_\mu\) transforming in the vectorial representation as indicated by its index position
\[
    \begin{aligned}
        x^{\mu}    & \to x^{\prime \mu} = \Lambda^{\mu}_{\ \nu} x^{\nu},                  \\
        A_{\mu}(x) & \to A^{\prime}_{\mu}(x^{\prime}) = \Lambda_{\mu}^{\ \nu} A_{\nu}(x).
    \end{aligned}
\]

\paragraph{Plane wave solutions.}
Plane wave solutions of the Proca equation are obtained by inserting in \eqref{eq:proca_equations} the ansatz
\[
    A_{\mu}(x) = \epsilon_{\mu}(p) e^{i p_\mu x^{\mu}},
\]
to find that:
\begin{itemize}
    \item the momentum \(p_\mu\) must satisfy the “mass shell” condition \(p_\mu p^{\mu} = -m^2\) (first equation in \eqref{eq:proca_equations}),
    \item a linear combination of the four independent polarizations must vanish, \(p^\mu \epsilon_{\mu}(p) = 0\) (second equation in \eqref{eq:proca_equations}).
\end{itemize}
The three remaining polarizations describe the three degrees of freedom of a spin 1 particle in a manifestly covariant manner. In the rest frame, the polarization is given by a vector in three-dimensional space (spin 1). Real solutions can be obtained by combining with appropriate Fourier coefficients these physical plane waves. The associated quanta have mass m and spin 1, and antiparticles (corresponding to solutions with negative energies) coincide with the particles (no charge differentiates particles and antiparticles). If one considers a complex Proca field, particles and antiparticles are different: they have opposite charges under a \(\mathrm{U}(1)\)  symmetry, which may be interpreted as the electric charge, and used to describe the \(W^{\pm}\) particles of the Standard Model.

\subsubsection{Green Functions and Propagator}
It is useful to rewrite the action \eqref{eq:proca_action} using integrations by part to reach the form
\[
    S_P[A_\mu] = \int \mathrm{d}^4 x \left( -\frac{1}{2} A_\mu K^{\mu \nu}(\partial) A_\nu \right),
\]
obtained as
\[
    \begin{aligned}
        S_P[A_\mu] & = \int \mathrm{d}^4 x \left( -\frac{1}{4} F_{\mu \nu} F^{\mu \nu} - \frac{1}{2} m^2 A_\mu A^{\mu} \right)                                                                                 \\
                   & = \int \mathrm{d}^4 x \left( -\frac{1}{2} (\partial_\mu A_\nu)(\partial^{\mu} A^{\nu}) + \frac{1}{2} (\partial_\mu A_\nu)(\partial^{\nu} A^{\mu}) - \frac{1}{2} m^2 A_\mu A^{\mu} \right) \\
                   & = \int \mathrm{d}^4 x \left( \frac{1}{2} A_\nu \Box A^{\nu} - \frac{1}{2} A_\nu \partial^{\nu} (\partial_{\mu} A^{\mu}) - \frac{1}{2} m^2 A_\mu A^{\mu} \right)                           \\
                   & = \int \mathrm{d}^4 x \left( -\frac{1}{2} A_\mu \left( (-\Box + m^2) \eta^{\mu \nu} + \partial^{\mu}\partial^{\nu} \right) A_\nu \right),
    \end{aligned}
\]
The final expression identifies the differential operator
\[
    K^{\mu \nu}(\partial) = (-\Box + m^2)\eta^{\mu \nu} + \partial^{\mu}\partial^{\nu}.
\]
Using this notation, the Proca field equations read
\[
    K^{\mu \nu}(\partial) A_\nu(x) = 0,
\]
indeed
\[
    \begin{aligned}
        K^{\mu \nu}(\partial) A_\nu(x) & = \left( (-\Box + m^2)\eta^{\mu \nu} + \partial^{\mu}\partial^{\nu} \right) A_\nu(x) = -\Box A^{\mu} + m^2 A^{\mu} + \partial^{\mu} (\partial^{\nu} A_\nu) \\
                                       & = - \partial_\nu \partial^{\nu} A^{\mu} + m^2 A^{\mu} + \partial^{\mu} (\partial^{\nu} A_\nu) = \partial_{\nu} F^{\mu \nu} + m^2 A^{\mu}                   \\
                                       & = -\partial_\mu F^{\mu \nu} + m^2 A^{\nu} = 0, \implies \partial_\mu F^{\mu \nu} = m^2 A^{\nu}.
    \end{aligned}
\]
The relative Green function \(G_{\mu \nu}(x-y)\) by definition satisfies
\begin{equation}
    K^{\mu \rho}(\partial_x) G_{\rho \nu}(x-y) = \delta^{\mu}_{\ \nu} \delta^{(4)}(x-y).
    \label{eq:proca_green_function_definition}
\end{equation}
It is given in Fourier space by
\[
    G_{\mu \nu}(x-y) = \int \frac{\mathrm{d}^4 p}{(2\pi)^4} \tilde{G}_{\mu \nu}(p) e^{i p_\mu (x^\mu - y^\mu)},
\]
where, if we apply the operator \(K^{\mu \rho}(\partial_x)\) to the Fourier representation of the Green function, as required in \eqref{eq:proca_green_function_definition}, we find
\[
    K^{\mu \rho}(\partial_x) G_{\rho \nu}(x-y) = \int \frac{\mathrm{d}^4 p}{(2\pi)^4} \left( (p^2 + m^2) \eta^{\mu \rho} - p^{\mu} p^{\rho} \right) \tilde{G}_{\rho \nu}(p) e^{i p_\mu (x^\mu - y^\mu)},
\]
which should be equal to \(\delta^{\mu}_{\ \nu} \delta^{(4)}(x-y)\); this implies that the Fourier transform of the Green function must satisfy
\[
    \left( (p^2 + m^2) \eta^{\mu \rho} - p^{\mu} p^{\rho} \right) \tilde{G}_{\rho \nu}(p) = \delta^{\mu}_{\ \nu}.
\]
Indeed, by symmetry the \(\tilde{G}_{\mu \nu}(p)\) must have the form
\[
    \tilde{G}_{\mu \nu}(p) = A(p) \eta_{\mu \nu} + B(p) p_\mu p_\nu,
\]
and one finds
\[
    A(p) = \frac{1}{p^2 + m^2}, \quad B(p) = \frac{A(p)}{m^2}.
\]
Quantizing the Proca field with second quantized methods, one finds that the Green function is proportional to the propagator
\[
    \langle A_\mu(x) A_\nu(y) \rangle = -i G_{\mu \nu} (x-y) = -i \int \frac{\mathrm{d}^4 p}{(2\pi)^4} \frac{e^{i p_\mu (x^\mu - y^\mu)}}{p^2 + m^2 - i \epsilon} \left( \eta_{\mu \nu} + \frac{p_\mu p_\nu}{m^2} \right),
\]
where \(\epsilon \to  0^+\). It describes as usual the propagation of particles and antiparticles of spin 1. Note that the propagator is singular in the limit of vanishing mass, \(m \to 0\). Massless spin 1 particles require a separate treatment.

\subsection{Maxwell Equations}

For \(m \to 0\), the Proca action reduces to the Maxwell action
\begin{equation}
    S_M[A_\mu] = \int \mathrm{d}^4 x \left( -\frac{1}{4} F_{\mu \nu} F^{\mu \nu} \right),
    \label{eq:maxwell_action}
\end{equation}
that correctly describes the relativistic waves associated to massless particles of spin 1 (with helicities \(h=\pm 1\)): if we compute the variation of the action, we can find the homogeneous Maxwell as equations of motion
\[
    \delta S_M[A_\mu] = \int \mathrm{d}^4 x \left( -\frac{1}{2} F_{\mu \nu} \delta F^{\mu \nu} \right) = \int \mathrm{d}^4 x \left( -\frac{1}{2} F_{\mu \nu} (\partial^{\mu} \delta A^{\nu} - \partial^{\nu} \delta A^{\mu}) \right) [\dots ]
\]
so that integrating by parts and neglecting boundary terms one finds the equations of motion
\[
    \partial^{\mu} F_{\mu \nu} = 0.
\]
Using the definition of \(F_{\mu \nu} = \partial_{\mu} A_{\nu} - \partial_{\nu}A_{\mu}\), they can be written as
\begin{equation}
    \Box A_\nu - \partial_\nu (\partial^{\mu} A_\mu) = 0.
    \label{eq:maxwell_equations_homogeneous}
\end{equation}
The other half of Maxwell’s equations are automatically solved by having expressed \(F_{\mu \nu}\) in terms of the potential \(A_\mu\), and take the name of Bianchi identities
\[
    \partial_{\lambda} F_{\mu \nu} + \partial_{\mu} F_{\nu \lambda} + \partial_{\nu} F_{\lambda \mu} = 0.
\]
The novelty of this formulation of a relativistic wave equation for massless spin 1 particles is the presence of a gauge symmetry
\begin{equation}
    A_\mu(x) \to A'_\mu(x) = A_\mu(x) + \partial_\mu \alpha(x),
    \label{eq:gauge_transformation}
\end{equation}
which leaves the action \eqref{eq:maxwell_action} unchanged: \(F_{\mu \nu}\) is invariant
\[
    F'_{\mu \nu} = \partial_{\mu} A'_{\nu} - \partial_{\nu} A'_{\mu} = F_{\mu \nu}.
\]
and the full action remains invariant. As we shall see, this fact implies that the action describes only two degrees of freedom instead of three: they correspond to the maximum and minimum spin states when projected along the direction of motion (helicity \(h=\pm 1\)). The infinitesimal gauge transformation has the same form
\[
    \delta A_\mu(x) = \partial_\mu \alpha(x),
\]
with \(\alpha(x)\) taken now as an infinitesimal arbitrary function. We can think of this local symmetry as associated with the \(\mathrm{U}(1)\) group as one can write \eqref{eq:gauge_transformation} in the form
\[
    A'_\mu(x) = A_\mu(x) - i \left( e^{-i \alpha(x)} \partial_\mu e^{i \alpha(x)} \right),
\]
with \(e^{i \alpha(x)} \in \mathrm{U}(1)\) for any spacetime point.

\paragraph{Plane wave solutions.}
The equations of motion do not have a unique solution (even after fixing initial conditions) because of the gauge symmetry: there is a combination of the dynamical variables that does not have a unique evolution as its time evolution can be changed arbitrarily with a gauge transformation. Keeping this gauge redundancy is very useful to have Lorentz invariance manifest, which is instrumental for introducing interactions in a way consistent with relativistic invariance. The Standard Model is indeed a gauge theory with local symmetry group \(\mathrm{SU}(3) \times \mathrm{SU}(2) \times \mathrm{U}(1)\). Gauge invariance can be used to set auxiliary conditions (gauge-fixing conditions) that allow to find physical solutions by eliminating (sometimes only partially) equivalent configurations generated by the gauge symmetry. We choose to partially fix the gauge symmetry by imposing the covariant constraint (Lorenz gauge)
\begin{equation}
    \partial^{\mu} A_\mu(x) = 0.
    \label{eq:lorenz_gauge}
\end{equation}
One may verify that this constraint can always be imposed. This condition does not fix the gauge symmetry completely, but residual gauge transformations are left over, namely those with local parameter \(\alpha(x)\) that satisfies \(\Box \alpha(x)=0\). In the Lorenz gauge, the equations of motion are simplified to
\[
    \Box A_\mu(x) = 0.
\]
and the plane wave solutions are
\[
    A_{\mu}(x) = \epsilon_{\mu}(p) e^{i p_\mu x^{\mu}} \implies \begin{dcases}
        \Box A_\mu(x) = 0 \iff p_\mu p^{\mu} = 0, \\
        \partial^{\mu} A_\mu(x) = 0 \iff p_\mu \epsilon^{\mu}(p) = 0.
    \end{dcases}
\]
which contain 3 independent polarizations \(\epsilon^{\mu}(p)\), as one is removed by the Lorenz gauge constraint \(p_{\mu}\epsilon^{\mu}(p) = 0\). Of these three remaining polarizations, the longitudinal one, defined by \(\epsilon_\mu(p) = p_\mu\) , does not carry energy and momentum and carries vanishing electromagnetic fields \(\mathbf{E}\) and \(\mathbf{B}\). It is gauge equivalent to \(A_{\mu}=0\) and can be eliminated by a residual gauge transformation, i.e. a gauge transformation that preserves the Lorenz condition \eqref{eq:lorenz_gauge}. The residual gauge transformations have the form \(\delta A_\mu(x) = \partial_\mu \alpha(x)\) with \(\alpha(x)\) such that
\[
    \Box \alpha(x) = 0.
\]
so that the Lorenz gauge \eqref{eq:lorenz_gauge} is not modified. A plane wave \(\alpha(x) = -ie^{i p_\mu x^{\mu}}\) with \(p_\mu p^{\mu} = 0\) for the gauge function identifies a \textbf{non-physical solution} of the form
\[
    A_{\mu}(x) = \partial_\mu \alpha(x) = p_\mu e^{i p_\mu x^{\mu}},
\]
where the polarization is proportional to \(p_\mu\), and it is thus removable by a gauge transformation with the opposite parameter. We conclude that only two independent physical polarizations remain. They can be shown to correspond to the two possible helicities of the photon.

\subsection{Spin 2 Particles}

The general treatment of spin \(s\) can be specialized to the case s=2. The dynamical variables are grouped into a symmetric tensor of rank two, \(\phi_{\mu \nu}(x)\), which satisfies eq.
\[
    \begin{aligned}
        (\Box -m^2) \phi_{\mu \nu} = 0,    \\
        \partial^{\mu} \phi_{\mu \nu} = 0, \\
        \phi^{\mu}_{\ \mu} = 0,
    \end{aligned}
\]
and plane wave solutions carry 5 independent polarizations, corresponding precisely to those of a particle of spin 2 (\(-2,\,-1,\,0,\,1,\,2\)).

\subsubsection{Massless Particles}

The previous equations are not sufficient to describe the massless case, as only 2 physical polarizations are expected. They correspond to the maximum and minimum possible helicities of the particle, \(h=\pm 2\). Gauge symmetries must be present in a Lorentz covariant description, and they are used to eliminate the non-physical polarizations, just as for spin 1. We indicate the spin 2 field with the symmetric tensor \(h_{\mu \nu}(x)\), that in Einstein’s theory of gravitation corresponds to the deformation of the Minkowski metric \(\eta_{\mu \nu}\) to a curved metric
\begin{equation}
    g_{\mu \nu}(x)= \eta_{\mu \nu} + h_{\mu \nu}(x)
    \label{eq:metric_deformation}
\end{equation}
Using the notation \(h = h^{\mu}_{\ \mu}\) the gauge invariant equations (generalizing those in eq. \eqref{eq:maxwell_equations_homogeneous}) are given by
\begin{equation}
    \Box h_{\mu \nu} - \partial_\mu \partial^{\rho} h_{\rho \nu} - \partial_\nu \partial^{\rho} h_{\rho \mu} + \partial_\mu \partial_\nu h = 0.
    \label{eq:spin2_maxwell_equations}
\end{equation}
They are invariant under the local symmetries
\begin{equation}
    \delta h_{\mu \nu}(x) = \partial_\mu \xi_{\nu}(x) + \partial_\nu \xi_{\mu}(x),
    \label{eq:spin2_gauge_transformation}
\end{equation}
where \(\xi_{\mu}(x)\) are four arbitrary spacetime functions (they form a vector field). Gauge symmetry is verified by a direct calculation: varying eq. \eqref{eq:spin2_maxwell_equations} under \eqref{eq:spin2_gauge_transformation} produces a vanishing result
\[
    \begin{aligned}
        \Box \delta h_{\mu \nu} - \partial_\mu \partial^{\rho} \delta h_{\rho \nu} - \partial_\nu \partial^{\rho} \delta h_{\rho \mu} + \partial_\mu \partial_\nu \delta h                                                                                                                            \\
        = \Box (\partial_\mu \xi_{\nu} + \partial_\nu \xi_{\mu}) - \partial_\mu \partial^{\rho} (\partial_{\rho} \xi_{\nu} + \partial_\nu \xi_{\rho}) - \partial_\nu \partial^{\rho} (\partial_{\rho} \xi_{\mu} + \partial_\mu \xi_{\rho}) + \partial_\mu \partial_\nu (2 \partial^{\rho} \xi_{\rho}) \\
        = \Box (\partial_\mu \xi_{\nu} + \partial_\nu \xi_{\mu}) - \partial_\mu (\Box \xi_{\nu} + \partial_\nu \partial^{\rho} \xi_{\rho}) - \partial_\nu (\Box \xi_{\mu} + \partial_\mu \partial^{\rho} \xi_{\rho}) + 2 \partial_\mu \partial_\nu (\partial^{\rho} \xi_{\rho}) = 0.
    \end{aligned}
\]
Let us now study the plane wave solutions to check that there are indeed only two inequivalent polarizations. We use the four gauge symmetries to impose four gauge conditions, known as the \textbf{de Donder gauge}
\begin{equation}
    \partial^{\mu} h_{\mu \nu} = \frac{1}{2} \partial_\nu h,
    \label{eq:de_donder_gauge}
\end{equation}
so that the equations \eqref{eq:spin2_maxwell_equations} simplify to
\[
    \Box h_{\mu \nu} = 0.
\]
In analogy with the massless spin 1 case, we have residual gauge transformations with local parameters \(\xi_\mu(x)\) satisfying
\[
    \Box \xi_\mu = 0,
\]
indeed we have
\[
    \Box h^{\prime}_{\mu \nu} = \Box h_{\mu \nu} + \partial_\mu \Box \xi_\nu + \partial_\nu \Box \xi_\mu = 0,
\]
so that the truly physical solutions, which cannot be eliminated using gauge transformations, are 2. We calculate 10 (independent components of \(h_{\mu \nu}\)) \(-4\) (number of constraints in de Donder gauge) \(-4\) (number of solutions that can be eliminated with residual gauge transformations) \(=2\). A more refined analysis shows that these two independent polarizations correspond to the two physical helicities of the gravitational waves \(h=\pm 2\). Finally, let us mention that these equations emerge from the linearization of the Einstein equations in vacuum
\[
    R_{\mu \nu}(g) = 0,
\]
where \(R_{\mu \nu}(g)\) is the Ricci tensor built from the metric \(g_{\mu \nu}(x)= \eta_{\mu \nu} + h_{\mu \nu}(x)\). In the linearization, one keeps only terms linear in \(h_{\mu \nu}(x)\). Gauge symmetry is related to the invariance under an arbitrary change of coordinates, suitably linearized.