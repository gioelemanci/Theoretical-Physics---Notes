\section{Theoretical insights} \label{app:insights}
\subsection*{Noether’s theorem and Action formalism}\label{app:Noether}

\paragraph{Action Principle.}
Let us briefly review the action principle in mechanics and field theories, considering the case of a particle. The main purpose is to underline its relation to canonical quantization and to stress the relevance of symmetries. As anticipated, the action is essential for the path integral quantization.

Consider a non-relativistic particle of mass \(m\) that moves in a single dimension with coordinate \(q\) and subject to a conservative force \(F = -\frac{\partial}{\partial q}V\). Newton’s equations of motion reads
\[
    F = m \ddot{q},
\]
and can be derived from an action principle. The action is a functional of the trajectory of the particle \(q(t)\) (the dynamical variable of the system) and associates a real number to each function \(q(t)\). In general, physical systems are described by an action of the type
\[
    S[q(t)] = \int \d{t} \mathrm{L} (q(t),\,\dot{q}(t)), \quad \mathrm{L}(q(t),\,\dot{q}(t)) =
\]


[...]

\paragraph{Hamiltonian formalism.}
The basic idea of the hamiltonian formalism is to have equations of motion that are first order in time. To review it, we follow a simple example: a non-relativistic particle of coordinates \(q^i\) and configuration space lagrangian
\[
    pdf
\]
where indices are lowered with the metric \(\delta_{ij}\) (and are thus equivalent to upper indices in our model, the distinction of upper and lower indices is, however, useful in more general contexts). Transition to the hamiltonian formalism takes place as follows:
\begin{enumerate}
    \item The dynamical variables are doubled by introducing conjugate momentum \(p^i\) to each coordinate \(q_i\)
          \[
              pdf
          \]
          The set \((q_i,\,p^i)\) constitutes the coordinates of phase space.
    \item The hamiltonian \(H(q,\,p)\) is defined as the Legendre transform of the lagrangian \(\mathrm{L}\)
          \[
              pdf
          \]
          It is a function on phase space.
    \item The \textbf{Poisson brackets} are defined as follows. For any two functions A and B of phase space, their Poisson brackets are defined by
          \[
              pdf
          \]
          where we have used the summation convention for repeated indices. In particular,
          \[
              pdf
          \]
    \item The hamiltonian equations of motion can be written as
          \[
              pdf
          \]
          and are of the first order in time. In our example, they become
          \[
              pdf
          \]
          and are evidently equivalent to the lagrangian equations \([\dots]\) . The hamiltonian \(H\) is interpreted as the generator of time translations, and moves the initial conditions (a point in phase space) over time by an infinitesimal amount \(\mathrm{d} t\). The generator of these canonical transformations is given by \(H \mathrm{d} t\), and acts through the Poisson brackets (\([\dots ]\)).
\end{enumerate}

These equations can be obtained from an action. In phase space, the action takes the form
\[
    pdf
\]
and minimizing it, one finds
\[
    pdf
\]
from which one recognizes Hamilton’s equations of motion. Note that in this formulation one needs \(2n\) integration constants, which are given by specifying the coordinates \(q^i\) at initial and final times.

The hamiltonian structure is the starting point of canonical quantization:
\[
    pdf
\]
where the classical dynamical variables \(z^a\) are elevated to linear operators \(\hat{z}^a\) acting on a Hilbert space. The quantum commutation relations are fixed by the values of the classical Poisson bracket. The vectors of the Hilbert space describe the possible quantum states of the system, whose evolution is governed by the Schr¨odinger equation.