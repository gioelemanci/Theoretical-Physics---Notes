\section{Notation and Conventions} \label{app:notation}
\subsection*{Upper and Lower Indices, Dotted Indices}\label{app:indices}

In the previous examples, we defined Lie groups using matrices that directly identify a representation, the so-called defining (or fundamental) representation. We denote the dimension of the defining representation by \(N\). As mentioned earlier, we can think of the \(N \times N\) matrices of this representation as operators acting on a vector space \(V\) of dimension \(N\). We denote the vectors in \(V\) by their components \(v^a\), where the index \(a = 1, 2, \ldots, N\). The vectors \(v^a \in V\) are transformed by the matrices \([R(g)]^a_{\ b}\) of the representation. By definition, a generic vector \(v^a\) transforms under the action of the group \(G\) as follows:
\[
    v^a \quad \xrightarrow{g \in G} \quad v'^a = [R(g)]^a_{\ b} v^b.
\]
Note that the convention is used where repeated indices are automatically summed over all their possible values. Vectors that transform in the manner described above are defined to have upper indices. Vectors whose components have upper indices belong to vector spaces equivalent to \(V\) and transform the same way under the action of \(G\), as described by the equation above.
