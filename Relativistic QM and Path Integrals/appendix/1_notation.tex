\section{Notation and Conventions} \label{app:notation}
\subsection*{Upper and Lower Indices, Dotted Indices}\label{app:indices}
In section \ref{sec:representations}, we defined Lie groups using matrices that directly identify a representation, the so-called defining (or fundamental) representation. We denote the dimension of the defining representation by \(N\). As mentioned earlier, we can think of the \(N \times N\) matrices of this representation as operators acting on a vector space \(V\) of dimension \(N\). We denote the vectors in \(V\) by their components \(v^a\), where the index \(a = 1, 2, \ldots, N\). The vectors \(v^a \in V\) are transformed by the matrices \([R(g)]^a_{\ b}\) of the representation. By definition, a generic vector \(v^a\) transforms under the action of the group \(G\) as follows:
\[
    v^a \quad \xrightarrow{g \in G} \quad v'^a = [R(g)]^a_{\ b} v^b.
\]
Note that the convention is used where repeated indices are automatically summed over all their possible values. Vectors that transform in the manner described above are defined to have upper indices. Vectors whose components have upper indices belong to vector spaces equivalent to \(V\) and transform the same way under the action of \(G\), as described by the equation above.

\subsection*{Relativistic Framework Notation} \label{app:relativistic_notation}

In the relativistic framework, we use Greek letters \(\mu, \nu, \rho, \sigma, \ldots\) to denote spacetime indices that run from \(0\) to \(3\). The time component is indicated by the index \(0\), while the spatial components are indicated by Latin indices \(i, j, k, \ldots\) that run from \(1\) to \(3\). The metric tensor of Minkowski spacetime is denoted by \(\eta_{\mu \nu}\) and has the \textit{mostly plus} signature \(\eta^{\mu \nu} = \mathrm{diag}(-1, +1, +1, +1)\). The Einstein summation convention is used, where repeated upper and lower indices are summed over all their possible values.

    [...]