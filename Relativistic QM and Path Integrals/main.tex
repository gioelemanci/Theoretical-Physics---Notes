\documentclass[a4paper,11pt]{book}

% === Packages ===
\usepackage[T1]{fontenc}
\usepackage[utf8]{inputenc}
\usepackage[english]{babel}
\usepackage{amsmath, amssymb, amsthm, mathtools}
\usepackage{physics}          % Dirac notation, derivatives, etc.
\usepackage{bm}               % Bold math symbols
\usepackage{fancyhdr}
\usepackage{graphicx}
\usepackage{float}
\usepackage{geometry}
\usepackage{enumitem}
\usepackage{xcolor}
\usepackage{titlesec}
\usepackage{lipsum}
\usepackage{booktabs}
\usepackage[toc]{appendix}

% === Font and style ===
\usepackage{lmodern}

% === Custom settings ===
% ===== Page Layout and Style =====
\pagestyle{fancy}
\fancyhf{} % Clear all header and footer fields
\fancyhead[LE,RO]{\thepage} % Page numbers on outer side (Left Even, Right Odd)
\fancyhead[RE]{\leftmark}   % Chapter name on right side of even pages
\fancyhead[LO]{\rightmark}  % Section name on left side of odd pages

% ===== Chapter title formattation =====
\titleformat{\chapter}[hang]
  {\normalfont\bfseries\Huge} % text format
  {\thechapter\hspace{1em}\rule{1pt}{30pt}\hspace{1em}} % number + line + space
  {0pt}
  {} % title next to number
% Sections always in cleared page
\newcommand{\sectionbreak}{\clearpage}

% Document geometry settings
\geometry{
  a4paper,
  left=3cm,    % Left margin
  right=3cm,   % Right margin
  top=3cm,     % Top margin
  bottom=3cm,  % Bottom margin
}

% Line spacing and paragraph settings
\linespread{1.2}            % 1.2 line spacing
\setlength{\parskip}{0.5em} % Vertical space between paragraphs
\setlength{\parindent}{0pt} % No paragraph indentation
\setlength{\headheight}{14pt}

% ===== Equation Numbering =====
\numberwithin{equation}{section}

% ===== Theorem Environments =====
\theoremstyle{plain}
\newtheorem{theorem}{Theorem}[chapter]
\newtheorem{lemma}[theorem]{Lemma}
\newtheorem{proposition}[theorem]{Proposition}
\newtheorem{corollary}[theorem]{Corollary}

% ===== Definition Environment =====
\theoremstyle{definition}
\newtheorem{definition}{Definition}[chapter]

% ===== Custom Environments =====
\newenvironment{application}[1][]
  {\begin{adjustwidth}{1.5em}{0pt}
   \begingroup
   \setlength{\parindent}{0pt}
   \par\vspace{0.5em}\noindent\textbf{Application} #1.\par\nobreak\ignorespaces}
  {\par\vspace{0.5em}\endgroup\end{adjustwidth}}
\newenvironment{remark}[1][]
  {\begin{adjustwidth}{1.5em}{0pt}
   \begingroup
   \setlength{\parindent}{0pt}
   \par\vspace{0.5em}\noindent\textbf{Remark.} #1\ \itshape}
  {\par\vspace{0.5em}\endgroup\end{adjustwidth}}
\newenvironment{example}[1][]
  {\begin{adjustwidth}{1.5em}{0pt}
   \begingroup
   \setlength{\parindent}{0pt}
   \par\vspace{0.5em}\noindent\textbf{Example} #1.\par\nobreak\ignorespaces}
  {\par\vspace{0.5em}\endgroup\end{adjustwidth}}
\newenvironment{notation}[1][]
  {\begin{adjustwidth}{1.5em}{0pt}
   \begingroup
   \setlength{\parindent}{0pt}
   \par\vspace{0.5em}\noindent\textbf{Notation} #1.\ \ignorespaces}
  {\par\vspace{0.5em}\endgroup\end{adjustwidth}}
\input{macro.tex}

% === Hyperref and cleveref ===
\usepackage{hyperref}
\usepackage{cleveref}

% === Meta ===
\title{\textbf{Relativistic Quantum Mechanics and\\ Path Integrals} \\

\large{Notes from the course held by Prof. Fiorenzo Bastianelli\\ at the University of Bologna}}
\author{Gioele Mancino}
\date{\today}

\begin{document}

\maketitle

\frontmatter

\chapter*{TODO}

\begin{itemize}
    \item Check all the math notations and symbols used throughout the notes for consistency.
    \item Check if the introduction is coherent with the rest of the notes.
    \item Change spin statistics relations in section 1.2 (equations for commutation and anticommutation relations)
\end{itemize}

\tableofcontents

\mainmatter

\chapter*{Introduction}

\lipsum[1]\TODO{Scrivere introduzione}

\input{chapters/1_1_linear_algebra.tex}
\section{Definition of a Group}

Let us define a group \(G = \{g\}\) as a set of elements \(g\) that satisfy the following properties:

\begin{enumerate}
    \item \textbf{Composition law}: Given \(g_1, g_2 \in G\), then \(g_1 \cdot g_2 = g_3 \in G\)
    \item \textbf{Identity element}: There exists \(e \in G\) such that \(g \cdot e = e \cdot g = g\) for all \(g \in G\)
    \item \textbf{Inverse element}: For each \(g \in G\), there exists \(g^{-1} \in G\) such that \(g \cdot g^{-1} = g^{-1} \cdot g = e\)
    \item \textbf{Associativity}: \((g_1 \cdot g_2) \cdot g_3 = g_1 \cdot (g_2 \cdot g_3)\) for all \(g_1, g_2, g_3 \in G\)
\end{enumerate}

\subsection*{Types of Groups}

\begin{itemize}
    \item \textbf{Discrete groups}: Contain a finite number of elements. For example, the group \(Z_2 \equiv \{1, -1\}\) with the usual multiplication law defines a group with two elements.
    
    \item \textbf{Lie groups}: Groups with an infinite number of elements, where the elements depend continuously on certain parameters. For example, rotations around the \(z\)-axis of our three-dimensional space form a Lie group whose elements are parameterized by an angle \(\theta \in [0, 2\pi]\). For Lie groups, one can consider infinitesimal transformations leading to the concept of Lie algebras.
    
    \item \textbf{Abelian groups}: Groups whose elements commute under the composition law: \(g_1 \cdot g_2 = g_2 \cdot g_1\) for every \(g_1, g_2 \in G\). If this does not happen, the group is said to be \textbf{non-abelian}.
\end{itemize}

\subsection*{Examples of Discrete Groups}

\begin{itemize}
    \item The \textbf{cyclic group} \(Z_n\): The finite group generated by the powers of an element \(a\), \(Z_n = \{e, a, a^2, \ldots, a^{n-1}\}\), with the condition that \(a^n = a^0 = e\). It is isomorphic to the \(n\)-th roots of unity \(e^{\frac{2\pi i}{n}k}\), with \(k = 0, 1, \ldots, n-1\). It is an abelian group for any \(n\).
    
    \item The \textbf{symmetric group} \(S_n\): The group of permutations of \(n\) objects, containing \(n!\) elements. One can check that \(S_2 = Z_2\), while \(S_3\) contains six elements and is the simplest example of a non-abelian group.
\end{itemize}

\subsection*{Examples of Lie Groups}

\begin{itemize}
    \item \(GL(N,\mathbb{R})\): The group of real \(N \times N\) matrices with determinant \(\neq 0\).
    \item \(SL(N,\mathbb{R})\): The group of real \(N \times N\) matrices with determinant \(= 1\).
    \item \(O(N)\): The group of real orthogonal \(N \times N\) matrices. It describes the invariances of the scalar product \(x^T x\) with \(x \in \mathbb{R}^N\).
    \item \(SO(N)\): The group of real orthogonal \(N \times N\) matrices with determinant \(= 1\).
    \item \(GL(N,\mathbb{C})\): The group of complex \(N \times N\) matrices with determinant \(\neq 0\).
    \item \(SL(N,\mathbb{C})\): The group of complex \(N \times N\) matrices with determinant \(= 1\).
    \item \(U(1) = \{z \in \mathbb{C} \mid |z| = 1\} = \{e^{i\theta} \mid \theta \in [0, 2\pi]\}\): The group of phases. It describes the invariances of the product \(z^* z\) with \(z \in \mathbb{C}\).
    \item \(U(N)\): The group of unitary \(N \times N\) matrices. It describes the invariances of the scalar product \(z^\dagger z\) with \(z \in \mathbb{C}^N\).
    \item \(SU(N)\): The group of unitary \(N \times N\) matrices with determinant \(= 1\).
\end{itemize}
There are important relationships between these groups, for example:
\[
\begin{aligned}
U(1) &\cong SO(2), \\
O(N) &= Z_2 \otimes SO(N), \\
U(N) &= U(1) \otimes SU(N).
\end{aligned}
\]
These isomorphisms and decompositions reveal the underlying structure of these symmetry groups and are fundamental in many physical applications.

\section{Representations}

We now introduce the concept of group representation. A \textit{representation} of an abstract group \(G\) is a "realization" of the multiplicative relations of the group \(G\) in a corresponding group of square matrices, where the product is given by the usual matrix multiplication. These matrices must be thought of as \textit{linear operators} that act on a \textit{vector space} \(V\), whose dimension is called the \textit{dimension} of the representation.

Explicitly, a representation is given by a mapping:
\[
R: G \longmapsto \text{Square Matrices}, \quad g \longmapsto R(g),
\]
such that:
\begin{enumerate}
    \item \(R(g_1) R(g_2) = R(g_1 \cdot g_2)\).
    \item \(R(e) = \mathbb{I}\), where \(\mathbb{I}\) is the identity matrix.
\end{enumerate}
From these properties, it also follows that \(R(g^{-1}) R(g) = R(e) = \mathbb{I}\), hence \(R(g^{-1}) = R^{-1}(g)\) exists. Associativity is automatic because matrix multiplication is associative. Thus, all the properties of the group are explicitly realized by the matrices of a representation.

By thinking of the matrices of a representation as operators that act on a vector space \(V\) of dimension \(N\), the matrices are \(N \times N\) matrices, and that is why the representation is said to be of dimension \(N\).

\textbf{Simple Example: Representation of \(Z_2\)}

As a very simple example of a representation, consider the cyclic group \(Z_2 = \{e, a\}\) with the relation that \(a^2 = e\). Then, a simple two-dimensional representation is given by the following \(2 \times 2\) matrices:
\[
R(e) = \begin{pmatrix} 1 & 0 \\ 0 & 1 \end{pmatrix}, \quad R(a) = \begin{pmatrix} -1 & 0 \\ 0 & -1 \end{pmatrix}.
\]
It is easily checked that the matrices of this representation satisfy all the properties of the abstract group \(Z_2 = \{e, a\}\). As we shall understand soon, this representation is reducible, as it contains two copies of the more simple representation defined in terms of \(1 \times 1\) matrices, i.e., numbers:
\[
R(e) = 1, \quad R(a) = -1.
\]

\subsection*{Classification of Representations}

At this point, the problem arises of studying how many and what kinds of representations of a given group possibly exist. In particular, it is useful to know which are their dimensions. This problem is of great importance for physical applications because the "vectors" of a representation (generically called "tensors") are conveniently used to describe physical quantities associated with models where \(G\) acts as a symmetry group.

\textbf{Defining Representation}

In the list of Lie groups introduced in the previous examples, we have used matrices to define the groups. Thus, these matrices give rise immediately to a particular representation: the \textit{defining representation} (also called the \textit{fundamental representation}). 

The elements of the group in the defining representation naturally perform transformations on vectors belonging to a vector space \(V\), the space on which the matrices act as linear operators. Let \(v^a\) denote the components of a vector in \(V\). The matrix \(R(g)\), which represents the element \(g\) of the abstract group \(G\), transforms this vector as follows:
\[
v^a \quad \xrightarrow{g \in G} \quad v'^a = [R(g)]^a_{\ b} v^b,
\]
where, as usual, \([R(g)]^a_{\ b}\) describes, as the indices \(a\) and \(b\) vary, the elements of the matrix \(R(g)\). The row index \(a\) is the first index and is conventionally placed in the upper position, and the column index \(b\) is the second index and is conventionally placed in the lower position.

In this way, the vectors in the vector space \(V\) are transformed by operations associated with the group \(G\). Repeated indices are summed over all their possible values, and the convention is used that in the sum, one index is in the upper position and the other one in the lower position.

\textbf{Equivalent Representations}

In general, \textit{equivalent} representations are defined as those that are related by similarity transformations: \(R(g)\) and \(\tilde{R}(g)\) are equivalent representations if:
\[
\tilde{R}(g) = A R(g) A^{-1} \quad \forall g \in G,
\]
where \(A\) is a matrix independent of \(g\). This equivalence relation allows us to consider equivalent representations as essentially the same representation. Indeed, the similarity transformation simply represents a change of basis in the vector space \(V\): the matrices of the different equivalent representations identify the same linear operator expressed in different bases.

\textbf{Reducible and Irreducible Representations}

A \textit{reducible} representation is a representation equivalent to a representation whose matrices are block diagonal. For example, \(R(g)\) is reducible if:

\[
\tilde{R}(g) = A R(g) A^{-1} = \left(\begin{array}{c|c|c}
R_1(g) & 0 & 0 \\
\hline
0 & R_2(g) & 0 \\
\hline
0 & 0 & R_3(g)
\end{array}\right) \quad \forall g \in G,
\]
for an appropriate matrix \(A\). It is said that \(R(g)\) is reducible to the three representations \(R_1(g)\), \(R_2(g)\), \(R_3(g)\). In this example, the vector space \(V\) on which the reducible representation \(R(g)\) acts is naturally decomposed as a direct sum of the three vector spaces on which the representations \(R_1(g)\), \(R_2(g)\), \(R_3(g)\) act, i.e., \(V = V_1 \oplus V_2 \oplus V_3\). This reducibility is thus written as:
\[
R(g) = R_1(g) \oplus R_2(g) \oplus R_3(g).
\]
An \textit{irreducible} representation is a representation that cannot be decomposed as above\footnote{For some groups, there can exist reducible representations of a particular type, formed by upper triangular matrices, but we overlook this subtlety in a first exposition to group theory, as the simplest groups we are interested in do not show such a phenomenon.}.

In the classification of the possible representations of a group \(G\), it is useful to consider only inequivalent irreducible representations, as all other representations follow from them. Given a fixed integer \(N\), it is not guaranteed that an irreducible representation of dimension \(N\) exists. In general, only for certain values of \(N\) will there be representations of a fixed group \(G\) (sometimes even more than one with the same dimension).

\textbf{Unitary Representations}

A \textit{unitary} representation is a representation in terms of unitary matrices (operators). Unitary representations are very useful in applications of quantum mechanics, where the symmetries of a quantum system are described by unitary operators acting in the Hilbert space (an infinite-dimensional vector space endowed with a positive-definite norm).

\textbf{Derived Representations}

Given the defining representation \(R(g)\) that acts on the vector space \(V\), which corresponds to transforming vectors with upper indices, we can immediately construct three other representations:
\begin{itemize}
    \item \(R(g)^*\), the complex conjugate representation acting on \(V^*\).
    \item \(R(g)^{-1T}\), the inverse transposed representation acting on the dual space \(\tilde{V}\).
    \item \(R(g)^{-1\dagger}\), the inverse Hermitian conjugate\footnote{Given a matrix \(R\), its Hermitian conjugate (or adjoint) \(R^\dagger\) is defined as the complex conjugate of the transpose, \(R^\dagger = R^{T*}\).} representation acting on \(\tilde{V}^*\).
\end{itemize}
The vectors they act on have the following index structures by convention\footnote{More information about dotted indices can be found in Appendix \ref{app:indices}.}, respectively:
\begin{itemize}
    \item Vectors with "dotted upper indices" \(v^{\dot{a}}\) (vectors in the complex conjugate space \(V^*\)).
    \item Vectors with "lower indices" \(v_a\) (vectors in the dual space \(\tilde{V}\)).
    \item Vectors with "dotted lower indices" \(v_{\dot{a}}\) (vectors in the complex conjugate dual space \(\tilde{V}^*\)).
\end{itemize}
In formulae:
\[
\begin{aligned}
v^{\dot{a}} &\xrightarrow{g \in G} v'^{\dot{a}} = [R(g)^*]^{\dot{a}}_{\ \dot{b}} v^{\dot{b}}, \\
v_a &\xrightarrow{g \in G} v'_a = [R(g)^{-1T}]_a^{\ b} v_b, \\
v_{\dot{a}} &\xrightarrow{g \in G} v'_{\dot{a}} = [R(g)^{-1\dagger}]_{\dot{a}}^{\ \dot{b}} v_{\dot{b}}.
\end{aligned}
\]
It is immediate to verify that these are representations of the group \(G\) if \(R(g)\) is one. The different index structure associated with these matrices reflects the fact that they are operators acting on different vector spaces.

\subsection*{Invariant Quantities and Index Contractions}

Invariant quantities under the action of the group \(G\) can be obtained by taking the scalar product between vectors with upper indices (sometimes called contravariant) and those with lower indices (sometimes called covariant), whether dotted or undotted. One can verify the following identities:
\[
\begin{aligned}
v_a w^a &\xrightarrow{g \in G} v'_a w'^a = v'^T w' = (R(g)^{-1T} v)^T R(g) w = v^T R(g)^{-1} R(g) w = v^T w = v_a w^a, \\
x_{\dot{a}} y^{\dot{a}} &\xrightarrow{g \in G} x'_{\dot{a}} y'^{\dot{a}} = x'^T y' = (R(g)^{-1\dagger} x)^T R(g)^* y = x^T R(g)^{-1*} R(g)^* y = x^T y = x_{\dot{a}} y^{\dot{a}}.
\end{aligned}
\]
\textit{Exercise}: Rederive these equations using only the index notation.

In general, it makes no group-theoretic sense to contract indices of the vectors described above in any other way ("contracting" refers to the operation of equating two indices and summing over all possible values that these indices can assume).

\section{Tensors and Tensor Representations}

Other representations can be obtained from the tensor product of the previously described representations. By definition, these representations act on "tensors," which are elements of vector spaces obtained from the tensor product of copies of \(V\), \(V^*\), \(\tilde{V}\), and \(\tilde{V}^*\). Therefore, tensors, by definition, have a certain number of upper and lower indices, with transformation properties defined by the nature associated with those indices.

For example, a tensor \(F^{ab \phantom{c} \dot{d}}_{\phantom{ab} c \phantom{d} \dot{e}}\) is, by definition, an object with \(N^5\) components that transform exactly like the product of the components of the previously defined vectors (tensor product):
\[
F^{ab \phantom{c} \dot{d}}_{\phantom{ab} c \phantom{d} \dot{e}} \sim v^a u^b w_c x^{\dot{d}} y_{\dot{e}}.
\]
Thus, the tensor \(F^{ab \phantom{c} \dot{d}}_{\phantom{ab} c \phantom{d} \dot{e}}\) represents (the components of) an element of a vector space of dimension \(N^5\) (because each index can take \(N\) values; it corresponds to an element of the vector space \(V \otimes V \otimes \tilde{V} \otimes V^* \otimes \tilde{V}^*\) and we can write \(F^{ab \phantom{c} \dot{d}}_{\phantom{ab} c \phantom{d} \dot{e}} \in V \otimes V \otimes \tilde{V} \otimes V^* \otimes \tilde{V}^*\)).

\subsection*{Tensor Transformation Law}

Under the action of the group \(G\), the tensor transforms as follows:
\[
F'^{ab \phantom{c} \dot{d}}_{\phantom{ab} c \phantom{d} \dot{e}} = [R(g)]^a_{\ f} [R(g)]^b_{\ g} [R(g)^{-1T}]_c^{\ h} [R(g)^*]^{\dot{d}}_{\ \dot{m}} [R(g)^{-1\dagger}]_{\dot{e}}^{\ \dot{n}} F^{fg \phantom{h} \dot{m}}_{\phantom{fg} h \phantom{m} \dot{n}}.
\]
This linear transformation law identifies a representation of dimension \(N^5\) (the \(N^5\) components are mixed among themselves by an \(N^5 \times N^5\) matrix, implicitly defined by the above formula, thus providing a representation of the group).

\subsection*{Reducibility of Tensor Representations}

Typically, tensors correspond to reducible representations, i.e., are transformed by reducible representations. The problem of decomposing representations into irreducible ones now arises. One way to decompose a representation is to study the tensors on which they act. A first decomposition operation is to separate the tensors by considering their symmetry properties under permutations of indices of the same nature (it is, therefore, useful to know the properties of the permutation group of \(n\) objects, i.e., the symmetric group \(S_n\)).

For example, the tensor \(T^{ab}\) can be separated into its symmetric part (\(S^{ab} = S^{ba}\)) and its antisymmetric part (\(A^{ab} = -A^{ba}\)) as follows:
\[
T^{ab} = \underbrace{\frac{1}{2}(T^{ab} + T^{ba})}_{S^{ab}} + \underbrace{\frac{1}{2}(T^{ab} - T^{ba})}_{A^{ab}}.
\]
It is easy to convince oneself that the symmetric and antisymmetric parts with distinct symmetries do not mix under group transformations. Indeed, one can calculate the transformed symmetric part under an arbitrary group transformation and verify that it remains symmetric:
\[
\begin{aligned}
S^{ab} \xrightarrow{g \in G} S'^{ab} &= [R(g)]^a_{\ c}[R(g)]^b_{\ d} S^{cd} \\
&= [R(g)]^a_{\ c}[R(g)]^b_{\ d} S^{dc} \\
&= [R(g)]^b_{\ d}[R(g)]^a_{\ c} S^{dc} = S'^{ba}.
\end{aligned}
\]

Similarly, one can verify that the antisymmetric part remains antisymmetric:
\[
\begin{aligned}
A^{ab} \xrightarrow{g \in G} A'^{ab} &= [R(g)]^a_{\ c}[R(g)]^b_{\ d} A^{cd} \\
&= [R(g)]^a_{\ c}[R(g)]^b_{\ d} (-A^{dc}) \\
&= -[R(g)]^b_{\ d}[R(g)]^a_{\ c} A^{dc} = -A'^{ba}.
\end{aligned}
\]

Thus, symmetric parts and antisymmetric parts are never mixed by group transformations, so the tensor representation identified by the tensor \(T^{ab}\) is reducible. In a compact notation, we can denote the representation that transforms the tensor \(T^{ab} \sim T\) as \(R_T(g)\) so that:
\[
T' = R_T(g) T
\]
This representation is reducible:
\[
\begin{pmatrix} S' \\ A' \end{pmatrix} = \underbrace{\begin{pmatrix} R_S(g) & 0 \\ 0 & R_A(g) \end{pmatrix}}_{R_T(g)} \begin{pmatrix} S \\ A \end{pmatrix},
\]
where \(T \sim \begin{pmatrix} S \\ A \end{pmatrix}\) indicates the decomposition into symmetric and antisymmetric parts.

These parts may be further reduced if there are other invariant operations (such as the possibility of taking scalar products as seen previously). For the simpler representations, it is easy to study any further reducibility on a case-by-case basis.

\subsection*{Invariant Tensors}

Note that the Kronecker delta tensors \(\delta^a_{\ b}\) and \(\delta^{\dot{a}}_{\ \dot{b}}\), which are the matrix elements of the identity operators, remain invariant under group transformations if their indices are transformed according to their nature. For example:
\[
\begin{aligned}
\delta^a_{\ b} \xrightarrow{g \in G} (\delta')^a_{\ b} &= [R(g)]^a_{\ c}[R(g)^{-1T}]_b^{\ d} \delta^c_{\ d} \\
&= [R(g)]^a_{\ c}[R(g)^{-1T}]_b^{\ c} \\
&= [R(g)]^a_{\ c}[R(g)^{-1}]^c_{\ b} \\
&= [R(g)R(g)^{-1}]^a_{\ b} = \delta^a_{\ b}.
\end{aligned}
\]
These are called \textit{invariant tensors}. In contrast, \(\delta_{ab}\) does not identify any invariant tensor (unless there are special relations between the various types of indices): if we define a tensor that coincides with \(\delta_{ab}\) in a "reference frame," under a group transformation (a "change of reference frame") the components of the tensor change value.

The existence and number of invariant tensors depend on the group \(G\) under consideration. For example:
\begin{itemize}
    \item The group \(SO(N)\) admits an invariant tensor defined by the completely antisymmetric symbol \(\epsilon^{a_1 \ldots a_N}\), where the indices are those of the fundamental representation. This follows from the fact that the matrices of \(SO(N)\) have determinant 1.
    \item Similarly, the group \(SU(N)\) admits the invariant tensors given by the completely antisymmetric symbols \(\epsilon^{a_1 \ldots a_N}\) and \(\epsilon_{a_1 \ldots a_N}\).
\end{itemize}

\subsection{Representations of \texorpdfstring{$SO(N)$}{SO(N)}}

We describe here the simplest representations of \(SO(N)\), the \textbf{special orthogonal group} of real \(N \times N\) matrices:
\[
    SO(N) = \left\{ \text{real } N \times N \text{ matrices } R \mid R^T R = \mathbb{I}, \, \det R = 1 \right\}.
\]
This is the group that leaves invariant the scalar product of vectors \(\vec{v}, \vec{w} \in \mathbb{R}^N\), defined by \(\vec{v} \cdot \vec{w} = \delta_{ab} v^a w^b\), where the metric \(\delta_{ab}\) is recognized to be an invariant tensor (indices up and down are equivalent for \(SO(N)\), so that \(\delta_{ab} = \delta^a_{\ b}\), and we already know that \(\delta^a_{\ b}\) is an invariant tensor).

Remember that the group \(SO(N)\) describes rotations in an \(N\)-dimensional real vector space. The condition \(\det R = 1\) ensures that we are considering only proper rotations (excluding reflections).

Since this is a real and orthogonal group, upper and lower, dotted and undotted indices are equivalent, thus we have four equivalent ways of representing a vector:
\[
    v^a \sim v_a \sim v^{\dot{a}} \sim v_{\dot{a}}.
\]

\paragraph{Invariance of the scalar product.}
More directly, using matrix notation, we compute:
\[
    \begin{aligned}
        v'                    & = R v, \quad w' = R w                                                        \\
        \vec{v} \cdot \vec{w} & = v^T w \quad \rightarrow \quad v'^T w' = (R v)^T R w = v^T R^T R w = v^T w.
    \end{aligned}
\]
Equivalently, using components:
\[
    \begin{aligned}
        v'_a                  & = R_{ab} v_b, \quad w'_a = R_{ab} w_b                                                                                                                                 \\
        \vec{v} \cdot \vec{w} & = v_a w_a \quad \rightarrow \quad v'_a w'_a = R_{ab} v_b R_{ac} w_c = v_b R_{ab} R_{ac} w_c = v_b \underbrace{R^T_{ba} R_{ac}}_{\delta_{bc}} w_c = v_b w_b = v_a w_a,
    \end{aligned}
\]
where we used again the fact that upper and lower indices are equivalent.

\paragraph{Defining representation.}
Thus, the defining representation (also called vector representation) acts on the vectors \(v^a\). As already described, the four basic representations are all equivalent as \(v^a \sim v_a \sim v^{\dot{a}} \sim v_{\dot{a}}\). We denote this representation by \(N\), i.e., by its dimension.

The tensor product \(N \otimes N\) identifies the tensor representation that acts on tensors with two indices \(T^{ab}\) and thus corresponds to a representation of dimension \(N^2\). It is a reducible representation. To extract the irreducible representations that it contains, we proceed as follows.

\paragraph{Decomposing the tensor representation.}
We have already seen that the tensor \(T^{ab}\) can be separated into the symmetric part \(S^{ab}\) (of dimension \(\frac{N(N+1)}{2}\)) and the antisymmetric part \(A^{ab}\) (of dimension \(\frac{N(N-1)}{2}\)).

The symmetric part is still reducible because one can construct a scalar (an invariant under the group transformations) by taking its trace:
\[
    S \equiv \delta_{ab} S^{ab} = S^a_{\ a}.
\]
It is easily seen that this is a scalar, as we already know that the contraction of an upper index with a lower index produces a scalar:
\[
    S \quad \xrightarrow{g \in SO(N)} \quad S' = S
\]
It identifies a trivial one-dimensional representation: \(R_{\text{scal}}(g) = 1\). We can separate the trace from the symmetric tensor \(S^{ab}\) in the following way:
\[
    S^{ab} = \underbrace{S^{ab} - \frac{1}{N} \delta^{ab} S}_{\hat{S}^{ab}} + \frac{1}{N} \delta^{ab} S
\]
where we have defined the traceless symmetric tensor \(\hat{S}^{ab}\) (which satisfies \(\hat{S}^a_{\ a} = 0\)).\footnote{We have divided by \(N\) since the trace involves summing over \(N\) components, and for the delta in particular, \(\delta^a_{\ a} = N\).} Thus, collecting all pieces, we have succeeded in separating the tensor \(T^{ab}\) into its irreducible parts:
\[
    T^{ab} = \frac{\delta^{ab}}{N} S + A^{ab} + \hat{S}^{ab}
\]
They transform independently without ever mixing. Indicating the irreducible representations with their respective dimensions, the above translates into the following expression:
\[
    N \otimes N = 1 \oplus \frac{N(N-1)}{2} \oplus \left( \frac{N(N+1)}{2} - 1 \right)
\]
It can be shown that there are no further reductions. The representation acting on antisymmetric tensors with two indices \(A^{ab}\), the \(\frac{N(N-1)}{2}\), is also called the \textbf{adjoint representation}: its dimension corresponds to the number of independent parameters of the group, given by the angles describing the rotations in the \(a\)-\(b\) planes (with \(a \neq b\)).

In summary, for \(\mathrm{SO}(N)\), we understand that there exist the following irreducible representations, indicated by their dimension:
\[
    1, \quad N, \quad \frac{N(N-1)}{2}, \quad \left( \frac{N(N+1)}{2} - 1 \right), \quad \ldots
\]
where 1 is the trivial representation (the scalar), \(N\) is the vector representation (also called defining or fundamental), the \(\frac{N(N-1)}{2}\) is the adjoint representation, the \(\frac{N(N+1)}{2} - 1\) is the traceless symmetric representation, etc.

\begin{example}[\(\mathrm{SO}(3)\) case and quantum mechanics]
    In the specific case of \(\mathrm{SO}(3)\), the irreducible tensor decomposition becomes:
    \[
        3 \otimes 3 = 1 \oplus 3 \oplus 5.
    \]

    We see that in this special case, the adjoint representation coincides with the vector representation (the defining representation): the dimensions are the same, and a full proof is simple to produce. Translated into the language of quantum mechanics, this formula tells us that combining spin 1 (the vector representation "3") with another spin 1 yields spin 0 (the "1" representation, the scalar), spin 1 (again the "3" representation), and spin 2 (the "5" representation).

    Equivalently, defining the quantum numbers \(l\) by setting \(n = 2l + 1\) for \(n = 1, 3, 5\), this relation can be written as:
    \[
        (l = 1) \otimes (l = 1) = (l = 0) \oplus (l = 1) \oplus (l = 2)
    \]
    which is the formula for adding quantum angular momenta. In quantum mechanics, orbital angular momentum is quantized and is fixed by an integer quantum number \(l = 0, 1, 2, 3, \ldots\), indicating that the projection of the angular momentum along a fixed axis can only take \(2l + 1\) values. The \((2l + 1)\)-representation is the one acting on the traceless, symmetric tensor with \(l\) indices, \(\hat{S}^{a_1 a_2 \ldots a_l}\).

    The electron orbiting the nucleus can have angular momentum with \(l = 0\) (S orbital), angular momentum with \(l = 1\) (P orbital), angular momentum with \(l = 2\) (D orbital), etc. Continuing with the study of angular momentum in quantum mechanics, one discovers that intrinsic angular momenta (spins) are characterized by integer and half-integer values of the quantum number, i.e., \(s = 0, \frac{1}{2}, 1, \frac{3}{2}, 2, \ldots\). The rules for composing angular momentum in quantum mechanics correspond precisely to the decomposition of a tensor product into irreducible representations mentioned above.

    Strictly speaking, the representations with half-integer spin (spinors) are not truly representations of the \(\mathrm{SO}(N)\) group, as they are double valued (a rotation of \(2\pi\) is not the identity but minus the identity). They are representations of the covering group as well as of the \(\mathrm{SO}(N)\) Lie algebra, a concept that we shall introduce shortly.
\end{example}

\begin{example}[\(\mathrm{SO}(4)\) and Lorentz group \(\mathrm{SO}(3,1)\)]

    In the case of \(\mathrm{SO}(4)\), or the Lorentz group \(\mathrm{SO}(3,1)\), the irreducible tensor decomposition becomes:
    \[
        4 \otimes 4 = 1 \oplus 6 \oplus 9 .
    \]
    The 6-dimensional representation is the adjoint representation. It is the one that acts on the electromagnetic field, which indeed has six independent components that are mixed under Lorentz transformations. The electromagnetic field is described by an antisymmetric tensor with two indices \(F^{\mu\nu}\).

    In the case of the Lorentz group, upper and lower indices are equivalent, and the Minkowski metric is used to pass from one to the other (the metric describes the similarity transformation that connects the two representations).
\end{example}

\subsection{Representations of \texorpdfstring{\(\mathrm{SU}(N)\)}{SU(N)}}

Consider now \(\mathrm{SU}(N)\), the special unitary group of \(N\times N\) matrices:
\[
    \mathrm{SU}(N) = \left\{ \text{complex } N \times N \text{ matrices } U \mid U^\dagger U = \mathbb{I}, \det U = 1 \right\}.
\]

This is the group that preserves the inner product of vectors \(\vec{v}, \vec{w} \in \mathbb{C}^N\) defined by\footnote{Remember that, since the representation is unitary we have \(R(g)^{-1T} \cong R(g)^*\), thus \(v^{\dot{a}} \sim v_a\) (the adjoint is equivalent to the inverse).} \(\vec{v}^* \cdot \vec{w} = v^*_a w^a = \delta^a_{\ b} v^*_a w^b\), where \(^*\) denotes the complex conjugate. The metric \(\delta^a_{\ b}\) is an invariant tensor, as we have seen in eq. \eqref{eq:delta_invariant_tensor}.

Starting from the fundamental representation, \(N\) (corresponding to the vectors \(v^a\)), we immediately obtain another representation, the \textbf{antifundamental representation} (or complex conjugate of the fundamental, transforming the vectors \(v^{\dot{a}} \sim v_a\)), which is denoted by \(\overline{N}\). Note that, unlike the case of \(\mathrm{SO}(N)\), here not all indices are equivalent: the representation is unitary (\(U^{\dagger} U = \mathbb{I}\)), thus we have \(v^{a} \sim v_{\dot{a}}\) transforming in \(N\), while \(v^{\dot{a}} \sim v_a\) transforming in \(\overline{N}\).

\paragraph{Invariance of the scalar product.}
Using matrix notation, we compute:
\[
    \begin{aligned}
        v'                      & = U v, \quad w' = U w                                                                                            \\
        \vec{v}^* \cdot \vec{w} & = v^\dagger w \quad \rightarrow \quad v'^\dagger w' = (U v)^\dagger U w = v^\dagger U^\dagger U w = v^\dagger w.
    \end{aligned}
\]
Equivalently, using components:
\[
    \begin{aligned}
        v'^a                    & = U^a_{\ b} v^b, \quad w'^a = U^a_{\ b} w^b                                                                                                                                          \\
        \vec{v}^* \cdot \vec{w} & = v_a w^a \quad \rightarrow \quad v'_a w'^a = (U^a_{\ b} v^b)^* U^a_{\ c} w^c = v_b U^{\ b}_{a} U^a_{\ c} w^c = v_b \underbrace{U^b_{\ a} U^a_{\ c}}_{\delta^b_{\ c}} w^c = v_a w^a,
    \end{aligned}
\]
where we used \(U^b_{\ a} = (U^a_{\ b})^\dagger\), thus we have seen that this representation preserves the scalar product among daggered and undaggered vectors.

\paragraph{Decomposing the tensor representation.}
Now, let's find other irreducible representations by considering the tensor product:
\[
    N \otimes N = \frac{N(N+1)}{2} \oplus \frac{N(N-1)}{2},
\]
which corresponds to the decomposition of the tensor \(T^{ab}\) into its symmetric and antisymmetric parts, \(T^{ab} = S^{ab} + A^{ab}\), now an exhaustive decomposition.\footnote{Note that it is not possible to take traces to form scalars on these tensors because \(\delta_{ab}\) is not an invariant tensor for \(\mathrm{SU}(N)\): to see this, simply transform the tensor \(\delta_{ab}\) as dictated by the structure of its indices and see that it is not invariant. The invariant tensors would be \(\delta^a_{\ b}\) and \(\delta_a^{\ b}\): \(\delta_a^{\prime \, b} = U^{\dagger}U \delta_a^{\ b} = \delta_a^{\ b}\).} Hence, we have discovered the existence of two new representations and know their dimensions.

Consider now:
\[
    N \otimes \overline{N} = 1 \oplus (N^2 - 1),
\]
which corresponds to the decomposition of the tensor \(T^a_{\ b}\) into its trace part (the scalar) and its traceless part. This is possible because we know that contracting a raised index with a lowered index produces a scalar. In formulas, this separation is written as:
\[
    T^a_{\ b} = \frac{\delta^a_{\ b}}{N} T + \hat{T}^a_{\ b},
\]
where \(T \equiv T^a_{\ a}\) and \(\hat{T}^a_{\ b} \equiv T^a_{\ b} - \frac{1}{N} \delta^a_{\ b} T\). Note that the tensor \(\delta^a_{\ b}\) is an invariant tensor (it corresponds to the metric of the complex vector space \(\mathbb{C}^N\)). Thus, we have discovered the existence of the representation of dimension \(N^2 - 1\), the so-called \textbf{adjoint representation}.

One can consider other invariant tensors of \(\mathrm{SU}(N)\), such as the completely antisymmetric tensors with \(N\) indices, \(\epsilon_{a_1 a_2 \ldots a_N}\) and \(\epsilon^{a_1 a_2 \ldots a_N}\) (this can be demonstrated using the fact that the group matrices have determinants equal to one), which can be used to study the reduction (or equivalence) of other tensorial representations.

In summary, for \(SU(N)\), we have seen that there exist the following irreducible representations:
\[
    1, \quad N, \quad \bar{N}, \quad N^2 - 1, \quad \frac{N(N-1)}{2}, \quad \frac{N(N+1)}{2}, \quad \frac{\overline{N(N-1)}}{2}, \quad \frac{\overline{N(N+1)}}{2},
\]
where \(1\) is the trivial representation (the scalar), \(N\) is the fundamental representation (or defining), \(\bar{N}\) is the antifundamental (complex conjugate of the fundamental), and \(N^2 - 1\) is the adjoint representation (which is real) etc.

\begin{example}[\(\mathrm{SU}(2)\) case and quantum mechanics]
    Let's make this explicit for the case of \(\mathrm{SU}(2)\). We have:
    \begin{equation}
        2 \otimes 2 = 1 \oplus 3, \qquad 2 \otimes \bar{2} = 1 \oplus 3,
        \label{eq:SU2_tensor_decompositions}
    \end{equation}
    where \(2\) is the fundamental representation (acting on vectors \(v^a\)), \(\bar{2}\) is the antifundamental representation (acting on vectors \(v^{\dot{a}} \sim v_a\)), \(1\) is the scalar representation, and \(3\) is the adjoint representation (acting on traceless tensors \(T^a_{\ b}\)).

    These formulas suggest that perhaps \(2\) and \(\bar{2}\) are \textit{equivalent representations}, i.e. \(\bar{2} \sim 2\). This is indeed the case: using the invariant tensor \(\epsilon_{ab}\) we may relate the two representations by setting \(w_a = \epsilon_{ab} v^b\), then under a group transformation we see that:
    \[
        w'_a = \epsilon'_{ab} v'^b = \epsilon_{ab} v'^b
    \]
    which indicates that, up to a change of basis given by the \(\epsilon_{ab}\) tensor, the vectors \(v^a\) and \(w_a\) transform in the same way. Here, we used the fact that \(\epsilon_{ab}\) is an invariant tensor. The explicit proof is as follows: if \(R \in SU(2)\) then:
    \[
        \epsilon'^{ab} = R^a_{\ c} R^b_{\ d} \epsilon^{cd} = k \epsilon^{ab}, \quad \epsilon^{ab} = \begin{pmatrix} 0 & 1 \\ -1 & 0 \end{pmatrix},
    \]
    for some coefficient \(k\). This follows from the fact that an antisymmetric tensor transforms into another antisymmetric one, and from antisymmetric \(2 \times 2\) matrix having only one independent component (otherwise the coefficient \(k\) would have been a vector or a matrix). To determine \(k\), we calculate:
    \[
        \epsilon'^{12} = U^1_{\ c} U^2_{\ d} \epsilon^{cd} = U^1_{\ 1} U^2_{\ 2} - U^1_{\ 2} U^2_{\ 1} = \det U = 1, \quad U \in SU(2).
    \]
    So \(k = 1\) and \(\epsilon'^{ab} = \epsilon^{ab}\), and thus \(w_a\) transforms in the same way as \(v^a\) (up to a change of basis), confirming that \(\bar{2} \sim 2\).

    Translating \eqref{eq:SU2_tensor_decompositions} into the language of quantum mechanics, it means that combining spin \(\frac{1}{2}\) (the representation "\(2\)") with itself yields spin 0 (the representation "\(1\)", the scalar) and spin 1 (the representation "\(3\)"). Indeed, defining \(j = 2s + 1\) for \(s = 0, \frac{1}{2}, 1\), this relation can be equivalently written as:
    \[
        (j = \tfrac{1}{2}) \otimes (j = \tfrac{1}{2}) = (j = 0) \oplus (j = 1).
    \]

    The group \(\mathrm{SU}(2)\) describes space rotations, including the possibility of having half-integer spins associated with fermionic particles. In mathematical terms, one says that the group \(\mathrm{SU}(2)\) is the universal cover of the group \(\mathrm{SO}(3)\).
\end{example}

\begin{example}[\(\mathrm{SU}(3)\) case and quark model]
    Now, let us make explicit also the case of \(\mathrm{SU}(3)\). It has physical applications both as the \textbf{flavor symmetry group} \(\mathrm{SU}(3)_{\text{flavor}}\) which mixes the three "flavors" of quarks (up, down, strange), and as \textbf{color symmetry group} \(\mathrm{SU}(3)_{\text{color}}\) which mixes the three colors of each quark (conventionally red, green, blue). We have:
    \begin{equation}
        3 \otimes \bar{3} = 1 \oplus 8
        \label{eq:SU3_tensor_decomposition}
    \end{equation}
    where \(3\) is the fundamental representation (acting on vectors \(v^a\)), \(\bar{3}\) is the antifundamental representation (acting on vectors \(v^{\dot{a}} \sim v_a\)), \(1\) is the scalar representation, and \(8\) is the adjoint representation (acting on traceless tensors \(T^a_{\ b}\)).

    In \(\mathrm{SU}(3)_{\text{flavor}}\), 3 and \(\bar{3}\) correspond to the up, down, and strange quarks and their antiquarks:
    \[
        q^a = \begin{pmatrix} u \\ d \\ s \end{pmatrix} \sim 3, \qquad \bar{q}_a = \begin{pmatrix} \bar{u} \\ \bar{d} \\ \bar{s} \end{pmatrix} \sim \bar{3}.
    \]

    Flavor symmetry means that we can redefine flavors through \(\mathrm{SU}(3)\) group transformations without changing anything in the description of physical phenomena: \(\mathrm{SU}(3)_{\text{flavor}}\). In the static quark model of mesons, which are hadrons composed of bound states of quark-antiquark (\(q\bar{q}\)), the symmetry implies that only singlets or octets of flavor can emerge. The \textit{mesonic octet} containing the pions is the main example: there are eight mesons with identical properties, and one could not distinguish them from each other if the symmetry were exact (same masses, same spin, etc.). In reality, the symmetry is only approximate, so there are some small differences (e.g., they have slightly different mass, also they have different charges and electromagnetism violates this symmetry).

    Another application concerns the color of quarks and is associated with another \(\mathrm{SU}(3)\) group, called \(\mathrm{SU}(3)_{\text{color}}\). Each quark flavor has three colors (red, green, blue); for example, for the up quark, we can group them in a vector:
    \[
        u^a = \begin{pmatrix} u^{\text{red}} \\ u^{\text{green}} \\ u^{\text{blue}} \end{pmatrix} \sim 3, \quad \bar{u}_a = \begin{pmatrix} \bar{u}^{\text{red}} \\ \bar{u}^{\text{green}} \\ \bar{u}^{\text{blue}} \end{pmatrix} \sim \bar{3},
    \]
    where the colors associated with the antiquarks are called anticolors (antired, antigreen, antiblue).

    Color symmetry means that we can redefine colors through \(\mathrm{SU}(3)\) group transformations without changing anything (color symmetry is exact). The information contained in the relation \eqref{eq:SU3_tensor_decomposition} is that it is possible to combine the colors of a quark with the colors of an antiquark (the anticolors) to form a colorless state (the scalar) or states with eight possible different color combinations: indeed, quark/antiquark of the same flavor can combine into a photon (the scalar, or singlet, of color) or into a gluon (there are eight different possibilities, so that one says that the gluons form an octet of color).

    Moreover:
    \[
        3 \otimes 3 = 6 \oplus \bar{3}.
    \]

    The possible ambiguity in understanding whether the tensor \(A^{ab}\), which has three components, corresponds to 3 or \(\bar{3}\) is resolved in favor of the latter option considering that \(A^{ab} \sim A^{ab} \epsilon_{abc} \sim V_c\) (since \(\epsilon_{abc}\) is an invariant tensor for \(\mathrm{SU}(3)\)). This relation in \(\mathrm{SU}(3)_{\text{color}}\) tells us that combining the colors of two quarks is not possible to obtain a colorless state (the scalar).

    With a bit more effort, one can also deduce (considering the symmetries of the tensor \(T^{abc}\)) that:
    \begin{equation}
        3 \otimes 3 \otimes 3 = 1 \oplus 8 \oplus 8 \oplus 10
        \label{eq:SU3_three_tensor_decompositions}
    \end{equation}
    where the 1 corresponds to the completely antisymmetric part of \(T^{abc}\), the 10 to the completely symmetric part of \(T^{abc}\), and the two 8s to parts of the tensor with mixed symmetry. In applications in the static quark model of \textbf{baryons}, hadrons composed of bound states of three quarks (\(qqq\)), the symmetry \(\mathrm{SU}(3)_{\text{flavor}}\) predicts that families of similar particles can only exist with 1, 8, or 10 components (not all need to exist: some combinations might not appear for other reasons). There are several octets (the 8), like the eight baryons, which have similar properties concerning the strong interactions (a particular octet contains the proton and the neutron). Their antiparticles also form octets. There is also a famous decuplet of baryons, whose wave functions are symmetric in the flavors of the three constituent quarks. These wave functions transform into the 10 of \(\mathrm{SU}(3)\) under flavor symmetry transformations (the corresponding anti-baryons group into \(\overline{10}\)). Applying the relation \eqref{eq:SU3_three_tensor_decompositions} to color, the fact that the 1 appears on the right side is interpreted as the possibility of combining the colors of three quarks to form a colorless state (e.g., the proton is made of three quarks; in general mesons and baryons must be color scalars due to a dynamical process called \textit{color confinement}).
\end{example}

\subsection{Representations of \texorpdfstring{$\mathrm{U}(1)$}{U(1)}}

Let us also consider the case of representations of the group \(\mathrm{U}(1)\), which also plays a significant role in physics. The group \(\mathrm{U}(1)\) is the group of phases:
\[
    \mathrm{U}(1) = \{e^{i\theta} \mid \theta \in [0, 2\pi]\}.
\]
It can be shown that all its irreducible unitary representations are one-dimensional complex representations which are characterized by an integer number, positive or negative, called the "charge". The defining representation represents an element of the group \(\mathrm{U}(1)\) with the phase \(e^{i\theta}\) which "rotates" naturally a complex one-dimensional vector \(v\) (\(v \in \mathbb{C}\), where \(\mathbb{C}\) denotes the field of complex numbers, which we interpret here as a one-dimensional complex vector space):
\[
    v \quad \xrightarrow{g \in \mathrm{U}(1)} \quad v' = e^{i\theta} v, \qquad v \in \mathbb{C}.
\]
Thus, the vector space of the defining representation is one-dimensional and complex, and the matrices of the representation are complex \(1 \times 1\) matrices (i.e., complex numbers).

Objects that transform as tensor products of the defining representation:
\[
    v_{(q)} \sim \underbrace{v v \cdots v}_{q \text{ times}} = v^q
\]
with \(q\) an integer give rise to the \textit{representation of charge} \(q\):
\[
    v_{(q)} \quad \xrightarrow{g \in U(1)} \quad v'_{(q)} = e^{iq\theta} v_{(q)}.
\]
The number \(q\) can also be negative, as seen by tensoring the antifundamental representation acting on \(\bar{v}\), but it remains an integer. Thus, all irreducible representations of \(\mathrm{U}(1)\) are one-dimensional and characterized by an integer \(q\), called the \textbf{charge of the representation}.\footnote{Not to be confused with the dimension of the representation, which is always 1.}

The tensor product of a representation with charge \(q_1\) with a representation with charge \(q_2\) yields the representation with charge \(q_1 + q_2\). The symmetry group \(\mathrm{U}(1)\) is used in physics when there are \textit{quantized additive quantum numbers}. Since all its representations are one-dimensional, to distinguish the various inequivalent representations, one indicates the charge \(q\) of the representation rather than its dimension.

What has been analyzed so far also allows us to interpret the possible charges (\textbf{generalized charges}, such as electric charge, color charge, etc.) of particles and associate them with a representation of the corresponding symmetry group. For example, the Standard Model of elementary particles contains the symmetry group \(\mathrm{SU}(3) \times \mathrm{SU}(2) \times \mathrm{U}(1)\), called the \textbf{gauge symmetry group}. The fermions of the Standard Model have generalized charges under these groups.

We can indicate these charges using a notation of the form \((\mathrm{SU}(3), \mathrm{SU}(2))_{\mathrm{U}(1)}\), where for the non-abelian groups (\(\mathrm{SU}(N)\)) we specify the representation by its corresponding dimension, while for the abelian part we use the \(\mathrm{U}(1)\) charge, called \textbf{hypercharge} \(Y\). Anticipating that fermions can be decomposed into right-handed (\(R\)) and left-handed (\(L\)) fermions, with possibly different charges, one has so far discovered in nature elementary fermions with the following charges:

\begin{table}[H]
    \centering
    \renewcommand{\arraystretch}{1.2}

    \begin{minipage}{0.47\textwidth}
        \centering
        \begin{tabular}{ccc}
            \toprule
            \(\begin{pmatrix}
                  \nu_{e}^L \\ e^L
              \end{pmatrix}\)          & \(\nu_{e}^R\)    & \(e^R\)          \\
            \(\begin{pmatrix}
                  \nu_{\mu}^L \\ \mu^L
              \end{pmatrix}\)       & \(\nu_{\mu}^R\)  & \(\mu^R\)           \\
            \(\begin{pmatrix}
                  \nu_{\tau}^L \\ \tau^L
              \end{pmatrix}\)     & \(\nu_{\tau}^R\) & \(\tau^R\)            \\
            \midrule
            \((1,\,2)_{-\frac{1}{2}}\) & \((1,\,1)_{0}\)  & \((1,\,1)_{-1}\) \\
            \bottomrule
        \end{tabular}
    \end{minipage}
    \hfill
    \begin{minipage}{0.47\textwidth}
        \centering
        \begin{tabular}{ccc}
            \toprule
            \(\begin{pmatrix}
                  u^L \\ d^L
              \end{pmatrix}\)         & \(u^R\)                   & \(d^R\)                    \\
            \(\begin{pmatrix}
                  c^L \\ s^L
              \end{pmatrix}\)         & \(c^R\)                   & \(s^R\)                    \\
            \(\begin{pmatrix}
                  t^L \\ b^L
              \end{pmatrix}\)         & \(t^R\)                   & \(b^R\)                    \\
            \midrule
            \((3,\,2)_{\frac{1}{6}}\) & \((3,\,1)_{\frac{2}{3}}\) & \((3,\,1)_{-\frac{1}{3}}\) \\
            \bottomrule
        \end{tabular}
    \end{minipage}
    \caption{From this classification we can deduce the properties of the elementary fermions of the Standard Model under the gauge symmetry group \(\mathrm{SU}(3) \times \mathrm{SU}(2) \times \mathrm{U}(1)\). The left table contains the leptons (electron, muon, tau, and their corresponding neutrinos), while the right table contains the quarks (up, down, charm, strange, top, bottom).
        The left-handed fermions are grouped into doublets of \(\mathrm{SU}(2) = 2\), since the can interact weakly, while the right-handed ones are singlets, scalars of the weak interaction (\(\mathrm{SU}(2) = 1\)).
        Leptons do not feel the strong interaction, so they are singlets of \(\mathrm{SU}(3) = 1\) (and thus colorless), while quarks transform in the fundamental representation of \(\mathrm{SU}(3) = 3\), since they feel the strong interaction.
        The hypercharge \(Y\) under \(\mathrm{U}(1)\) is indicated as a subscript, and it is related to the electric charge as explained in the text.}
\end{table}

The group \(\mathrm{SU}(3)\) is called the \textbf{color group}, and the quarks transform in the fundamental representation (3), and thus have three "colors", while the corresponding antiparticles, the antiquarks, transform in the complex conjugate representation (\(\bar{3}\)), and thus have three "anticolors". Leptons do not feel the strong force and, therefore, are singlets under the color group.

The group \(\mathrm{SU}(2)\) is called the \textbf{weak isospin group}, and the \(\mathrm{SU}(2)\) doublets have been written above in the form of column vectors: they transform in the two-dimensional representation (2), and thus have weak isospin \(I = \frac{1}{2}\), with the third component \(I_3 = \frac{1}{2}\) for the upper element of the vector and \(I_3 = -\frac{1}{2}\) for the lower one. Note that the 2 is equivalent to the \(\bar{2}\), both identifying the same representation with weak isospin equal to \(\frac{1}{2}\).

\(\mathrm{U}(1)\) is the \textbf{hypercharge group}. If we denote by \(Y\) the hypercharge of a particle, the corresponding electric charge \(Q\) is given by
\[
    Q = I_3 + Y
\]
where \(I_3\) denotes the third component of the weak isospin. From the above table, one may extract which are the electric charges of these elementary particles.
\section{Lie Groups and Lie Algebras}

A Lie group is, by definition, a group whose elements depend continuously on some parameters. By studying the infinitesimal group transformations, i.e., those transformations that differ slightly from the identity, one obtains the so-called Lie algebra of the group, an algebra that summarizes essential information about the group. In particular, the Lie algebra captures the non-abelian structure of the group. To introduce these topics, we first study some of the simplest yet most commonly used groups in physics and then list general properties and definitions.

\subsection{SO(2)}

Consider the familiar group of rotations in two-dimensional Euclidean space, the group \(\mathrm{SO}(2)\) of real orthogonal $2 \times 2$ matrices with determinant equal to 1. These matrices generate the transformations of a vector
\[
    \vec{x} \rightarrow \vec{x}' = R \vec{x}
\]
or in tensor notation $x'^{i} = R^{i}_{\ j} x^{j}$ with $i,j = 1,2$. This is the defining (or vector) representation. The rotations that mix the two components of the vector $\vec{x} = (x, y) = (x^1, x^2)$ depend on an angle $\theta$ and can be written as
\begin{equation}
    R(\theta) =
    \begin{pmatrix}
        \cos(\theta)  & \sin(\theta) \\
        -\sin(\theta) & \cos(\theta)
    \end{pmatrix}
    \xrightarrow{\theta \to 0}
    \begin{pmatrix}
        1       & \theta \\
        -\theta & 1
    \end{pmatrix} = 1 + i \theta T
    \label{eq:SO2_infinitesimal_rotation}
\end{equation}
where the matrix $T$ is the operator that "generates" the infinitesimal part of the transformation
\begin{equation}
    T = \begin{pmatrix}
        0  & i \\
        -i & 0
    \end{pmatrix}.
    \label{eq:SO2_generator}
\end{equation}
The imaginary unit $i$ in \eqref{eq:SO2_infinitesimal_rotation} is conventional but allows us to present the generator $T$ as a Hermitian matrix (whose eigenvalues are real).

The group is abelian: its elements commute
\[
    R(\theta_1) R(\theta_2) = R(\theta_2) R(\theta_1),
\]
and thus
\begin{equation}
    [T, T] = 0,
    \label{eq:SO2_Lie_algebra}
\end{equation}
where $[.,.]$ denotes the commutator ($[A,B] = AB - BA$). This is called the Lie algebra of \(\mathrm{SO}(2)\). In general, the Lie algebra of a group is generated by the commutators of its infinitesimal generators, and it can be proven that if the commutator of the generators is non zero then the group is non-abelian.

\paragraph{Exponential map.}
Finite transformations can be obtained by iterating infinitesimal transformations (if the parameter $\theta$ is not infinitesimal, consider $\theta$ with $n$ large enough to make it infinitesimal). Then, one can write\footnote{In the end, if one expands the exponential map as a Taylor series, it can be recognized the final expression, since the generator \(T\) is idempotent: \(T^2 = \mathbb{I}\). Thus it appears only on the odd powers of the expansion, while the even powers give rise to the identity matrix.}
\begin{equation}
    [R(\theta)]^n \approx \left[R\left(\frac{\theta}{n}\right)\right]^n \approx (1 + i\frac{\theta}{n}T)^n \rightarrow e^{i \theta T} = \mathbb{I} \cos(\theta) + i T \sin(\theta),
    \label{eq:SO2_exponential_map}
\end{equation}
which reproduces the finite transformation in \eqref{eq:SO2_infinitesimal_rotation}. The notation $e^{i \theta T}$, which contains the infinitesimal generator $T$ and the continuous Lie parameter $\theta$ of the group, is the \textbf{exponential representation} of the elements of the group \(\mathrm{SO}(2)\). It generalizes to arbitrary Lie groups.

Here, we have obtained the Lie algebra of the group \(\mathrm{SO}(2)\) \eqref{eq:SO2_Lie_algebra} by considering the defining representation of \(\mathrm{SO}(2)\), which is enough to recognize its abstract Lie algebra. Then, one can study the various representations of the \(\mathrm{SO}(2)\) Lie algebra in terms of other matrices and classify inequivalent representations.

Note that by defining the complex number $z = x + i y$, the \(\mathrm{SO}(2)\) transformation of $(x, y)$ takes the form of \(\mathrm{U}(1)\) a phase transformation:
\[
    \begin{aligned}
        z' & = x' + i y' = (x \cos(\theta) + y \sin(\theta)) + i (-x \sin(\theta) + y \cos(\theta)) \\
           & = (\cos(\theta) - i \sin(\theta))(x + i y) = e^{-i \theta} z.
    \end{aligned}
\]
The groups \(\mathrm{SO}(2)\) and \(\mathrm{U}(1)\) are equivalent, $\mathrm{SO}(2) \cong \mathrm{U}(1)$.

\subsection{SO(3)}

Consider now the group of rotations in three-dimensional space, the group $\mathrm{SO}(3)$ of real orthogonal $3 \times 3$ matrices with determinant equal to 1. These matrices generate transformations of a three-dimensional vector $\vec{x} \rightarrow \vec{x}' = R \vec{x}$.
Consider the rotations around the three Cartesian axes with coordinates $(x, y, z) = (x^1, x^2, x^3)$
\[
    R_x(\theta_x) =
    \begin{pmatrix}
        1 & 0               & 0              \\
        0 & \cos(\theta_x)  & \sin(\theta_x) \\
        0 & -\sin(\theta_x) & \cos(\theta_x)
    \end{pmatrix}
    = 1 + \theta_x
    \begin{pmatrix}
        0 & 0  & 0 \\
        0 & 0  & 1 \\
        0 & -1 & 0
    \end{pmatrix}
    + \dots
\]
\[
    R_y(\theta_y) =
    \begin{pmatrix}
        \cos(\theta_y) & 0 & -\sin(\theta_y) \\
        0              & 1 & 0               \\
        \sin(\theta_y) & 0 & \cos(\theta_y)
    \end{pmatrix}
    = 1 + \theta_y
    \begin{pmatrix}
        0 & 0 & -1 \\
        0 & 0 & 0  \\
        1 & 0 & 0
    \end{pmatrix}
    + \dots
\]
\[
    R_z(\theta_z) =
    \begin{pmatrix}
        \cos(\theta_z)  & \sin(\theta_z) & 0 \\
        -\sin(\theta_z) & \cos(\theta_z) & 0 \\
        0               & 0              & 1
    \end{pmatrix}
    = 1 + \theta_z
    \begin{pmatrix}
        0  & 1 & 0 \\
        -1 & 0 & 0 \\
        0  & 0 & 0
    \end{pmatrix}
    + \dots
\]
so that the generators $T^i$ of the infinitesimal transformations are given by
\begin{equation}
    T^1 =
    \begin{pmatrix}
        0 & 0  & 0 \\
        0 & 0  & i \\
        0 & -i & 0
    \end{pmatrix}, \quad
    T^2 =
    \begin{pmatrix}
        0 & 0 & -i \\
        0 & 0 & 0  \\
        i & 0 & 0
    \end{pmatrix}, \quad
    T^3 =
    \begin{pmatrix}
        0  & i & 0 \\
        -i & 0 & 0 \\
        0  & 0 & 0
    \end{pmatrix}.
    \label{eq:SO3_generators}
\end{equation}
The corresponding Lie algebra is easily computed by calculating the commutators of the matrices just identified
\begin{equation}
    [T^i, T^j] = i \epsilon^{ijk} T^k,
    \label{eq:SO3_Lie_algebra}
\end{equation}
where $\epsilon^{ijk}$ is the Levi-Civita antisymmetric tensor defined by $\epsilon^{123} = 1$ and antisymmetric under the exchange of any pair of indices.

The right-hand side is not zero, indicating that the group is non-abelian (the group elements do not commute). The constants $\epsilon^{ijk}$ are called the \textbf{structure constants} of the $\mathrm{SO}(3)$ group because they encode the non-abelian structure of the group. A finite element of the group can be parameterized in exponential form as
\begin{equation}
    R(\vec{\theta}) = e^{i \vec{\theta} \cdot \vec{T}} = e^{i \theta_i T^i},
    \label{eq:SO3_exponential_map}
\end{equation}
where $\theta_i$ are the independent parameters of the group (a rotation of angle $\theta = \sqrt{\vec{\theta} \cdot \vec{\theta}}$ around the axis of the unit vector $\hat{n} = \vec{\theta}/\theta$).

To understand the role of the Lie algebra, let us study the product
\[
    R(\vec{\alpha}) R(\vec{\beta}) R^{-1}(\vec{\alpha}) R^{-1}(\vec{\beta}),
\]
that would be the identity of an abelian group (which have commuting operators, thus we could swap the positions of the elements and simplify them with their inverse). For infinitesimal parameters and working at the linear order in both $\vec{\alpha}$ and $\vec{\beta}$:
\[
    \begin{dcases}
        R(\vec{\alpha}) = \mathbb{I} + A = \mathbb{I} + i \alpha_i T^i, \\
        R(\vec{\beta}) = \mathbb{I} + B = \mathbb{I} + i \beta_i T^i,
    \end{dcases}
\]
one finds
\[
    \begin{aligned}
        R(\vec{\alpha}) R(\vec{\beta}) R^{-1}(\vec{\alpha}) R^{-1}(\vec{\beta}) & = (\mathbb{I} + A)(\mathbb{I} + B)(\mathbb{I} - A)(\mathbb{I} - B) \\
                                                                                & = \mathbb{I} + A^2 + B^2 - A^2 - B^2 + AB - BA + \dots             \\
                                                                                & = \mathbb{I} + [A, B],                                             \\
    \end{aligned}
\]
and since \(A = i \alpha_i T^i\), \(B = i \beta_i T^i\), we have (up to second order corrections):
\[
    \begin{aligned}
        R(\vec{\alpha}) R(\vec{\beta}) R^{-1}(\vec{\alpha}) R^{-1}(\vec{\beta}) & = \mathbb{I} + \left[i \alpha_i T^i,\, i \beta_j T^j \right]              \\
                                                                                & = \mathbb{I} - \alpha_i \beta_j [T^i, T^j] = \mathbb{I} + i \gamma_k T^k,
    \end{aligned}
\]
which is nonvanishing for the non-abelian group $\mathrm{SO}(3)$:\footnote{We have shown how this products can be rewritten as a rotation of an angle \(\vec{\gamma}\): \(\mathbb{I} + i \gamma_k T^k \sim R(\vec{\gamma})\).} the Lie algebra captures the non-commutative structure of the Lie group. In addition, one understands that the result must correspond to an infinitesimal group transformation, just like the left-hand side, so that the commutator $[T^i, T^j]$ must be proportional to a generator, as indeed verified in \eqref{eq:SO3_Lie_algebra}:
\[
    \gamma_k = - \alpha_i \beta_j f^{ij}_{\ \ k},
\]
where the constants \(f^{ij}_{\ \ k}\) are identified with the structure constants of the group, which in this case are recognized to be \(f^{ij}_{\ \ k} \sim \epsilon^{ijk}\).

We have obtained the Lie algebra using the defining representation, and now we can consider it as the abstract Lie algebra of the group $\mathrm{SO}(3)$ and study its different irreducible representations, as done for the representations of the group. From the representations of the group studied previously, one obtains the corresponding representations of the associated Lie algebra. Conversely, exponentiating the matrices of a representation of the Lie algebra yields finite transformations that provide a representation of the group.\footnote{Except for possible topological obstructions that might prevent the representation from being truly single-valued. This situation is exemplified by the spinor representations of $\mathrm{SO}(3)$, which, as we will see later, are true representations (i.e., single-valued representations) of the $\mathrm{SU}(2)$ group only.}

\paragraph{SO(3) and quantum mechanics.}
Let us comment on the $\mathrm{SO}(3)$ Lie algebra and relate it to known topics studied in quantum mechanics. In equation \eqref{eq:SO3_Lie_algebra}, we recognize the algebra of the quantum angular momentum operator. Renaming $T^i \rightarrow L^i$, we recognize the familiar algebra of the angular momentum (in units of $\hbar = 1$)
\[
    [L^i, L^j] = i \epsilon^{ijk} L^k.
\]
The study of its irreducible unitary representations is solved explicitly using the methods of quantum mechanics: the known result is that these irreducible representations are those given by the spherical harmonics $|l, m\rangle \sim Y_{lm}$, which for fixed $l$ form a basis of the spin-$l$ representation. It is $(2l + 1)$-dimensional, as for fixed $l$ the possible values of $m$ are $2l + 1$:
\[
    Y^{\prime}_{lm} = [R_{(l)}(\theta)]_{m}^{\ l} Y_{ln}, \quad l \text{ fixed}, \quad m, n \in [-l, -l+1, \dots, 0, \dots, l-1, l].
\]
In the case of spinorial representations (i.e., with half-integer spin, i.e., with $l \rightarrow j$ and $j$ half-integer), a rotation by $2\pi$ (which for $\mathrm{SO}(3)$ coincides with the identity) is represented by the matrix $-\mathbb{I}$, and thus we speak of a 2-valued representation (one needs to rotate by another $2\pi$ to get back to the identity). As we will see, these spinorial representations are true representations of the $\mathrm{SU}(2)$ group, which has the same Lie algebra as $\mathrm{SO}(3)$ and therefore has the same local structure but different global properties.

\paragraph{Hints on generalization.}
To appreciate future developments (such as the Lie algebras of \(\mathrm{SO}(N)\) and \(\mathrm{SO}(N,M)\)), let’s rewrite the matrices identifying the generators in the vector representation \eqref{eq:SO3_generators} and the corresponding Lie algebra in \eqref{eq:SO3_Lie_algebra} in an alternative way. We can rename the generator \(T^1\) as \(T^{23}\), as it generates a rotation in the 2-3 plane, and so on: \(T^2 \equiv T^{31}\), \(T^3 \equiv T^{12}\). The matrix elements in \eqref{eq:SO3_generators} can be written as
\[
    (T^1)^i_{\ j} \equiv (T^{23})^i_{\ j} = -i(\delta^{2i}\delta^3_{\ j} - \delta^{3i}\delta^2_{\ j}),
\]
and similarly for \(T^{31}\) and \(T^{12}\). Thus, the general expression obtained is
\[
    (T^{kl})^i_{\ j} = -i(\delta^{ki}\delta^l_{\ j} - \delta^{li}\delta^k_{\ j}),
\]
which can be used to recalculate the Lie algebra of \(\mathrm{SO}(3)\). Rewritten on this basis, the Lie algebra \eqref{eq:SO3_Lie_algebra} becomes
\[
    [T^{kl}, T^{mn}] = -i\delta^{lm}T^{kn} + i\delta^{km}T^{ln} + i\delta^{ln}T^{km} - i\delta^{kn}T^{lm}.
\]
Note the presence of the Euclidean (inverse) metric \(\delta^{ij}\) in this relation. Written in this form, the Lie algebra is valid for the generic group \(\mathrm{SO}(N)\), provided that the indices range from 1 to N. There are thus \(\tfrac12 N (N-1)\) indipendent generators. Moreover, by replacing the metric \(\delta^{ij}\) with a Minkowski metric \(\eta^{ij}\), appropriate for a spacetime with N spatial and M temporal dimensions, one obtains the Lie algebra of \(\mathrm{SO}(N, M)\).

\subsection{U(1)}

Consider the group $U(1) = \{ e^{i\theta} \mid \theta \in [0, 2\pi] \}$, the group of phases defined via its defining representation. For infinitesimal transformations (up to second order corrections in $\theta$) we have:
\[
    e^{i\theta} = 1 + i\theta,
\]
where the infinitesimal generator is given by $T = 1$ (we can think of it as a $1 \times 1$ matrix), which produces the Abelian Lie algebra of the $U(1)$ group given by the commutator
\begin{equation}
    [T, T] = 0.
    \label{eq:U1_Lie_algebra}
\end{equation}

In the charge $q$ representation, where the element $e^{i\theta}$ is represented by $e^{iq\theta}$, the infinitesimal generator is represented by $T = q$ and satisfies the same Lie algebra \eqref{eq:U1_Lie_algebra}. Therefore, we can think of the Lie algebra $[T, T] = 0$ as the abstract Lie algebra corresponding to the $U(1)$ group, which is represented by different matrices in different representations. Since the irreducible representations of the $U(1)$ group are all one-dimensional, all these matrices are $1 \times 1$ matrices and thus are simply numbers. In the charge $q$ representation, the generator of $U(1)$ is represented by $T = q$. It is also common to use the notation $Q$ (which often denotes a charge) instead of $T$ for the generator of the $U(1)$ group. The groups $U(1)$ and $SO(2)$ identify the same Abelian Lie group, as already described.

\subsection{SU(2)}

Let's now analyze the group $SU(2)$, the group of $2 \times 2$ unitary matrices with unit determinant:
\[
    SU(2) = \{ U \text{ complex matrices } 2 \times 2 \mid U^\dagger = U^{-1}, \ \det U = 1 \}.
\]
We can write the matrices that differ infinitesimally from the identity matrix as
\[
    U = 1 + i T, \quad T^i_{\ j} \ll 1.
\]

Now, the requirement of unitarity \(U^{\dagger} = U^{-1}\), alogn with the infinitesimal form above of \(U\)
\[
    \begin{dcases}
        U^\dagger & = 1 - i T^\dagger, \\
        U^{-1}    & = 1 - i T,
    \end{dcases} \text{ with } U^\dagger = U^{-1},
\]
implies that the matrices $T$ must be Hermitian:
\[
    T^\dagger = T,
\]
while the requirement for unit determinant, $\det U = 1 + i \Tr T = 1$,\footnote{Intuitively, when computing the determinant of \(\mathbb{I} + \epsilon\), where \(\epsilon\) is a small matrix, the determinant is approximately \(1 + \Tr \epsilon\) (since all the components out of the diagonal are infinitesimal).} implies that these matrices must be traceless:
\[
    \Tr T = 0.
\]
A basis of Hermitian traceless $2 \times 2$ matrices is given by the Pauli matrices:
\begin{equation}
    \sigma_1 =
    \begin{pmatrix}
        0 & 1 \\
        1 & 0
    \end{pmatrix}, \quad
    \sigma_2 =
    \begin{pmatrix}
        0 & -i \\
        i & 0
    \end{pmatrix}, \quad
    \sigma_3 =
    \begin{pmatrix}
        1 & 0  \\
        0 & -1
    \end{pmatrix},
    \label{eq:Pauli_matrices}
\end{equation}
so we can express an arbitrary matrix $T$ as a linear combination of the $\sigma^a$:
\begin{equation}
    T = \theta_a \frac{\sigma^a}{2} = \theta_a T^a, \quad T^a = \frac{1}{2} \sigma^a, \quad a = 1,2,3,
    \label{eq:SU2_generators}
\end{equation}
The normalization has been chosen to satisfy
\begin{equation}
    \Tr (T^a T^b) = \frac{1}{2} \delta^{ab},
    \label{eq:SU2_generator_normalization}
\end{equation}
With this normalization, the infinitesimal generators $T^a = \frac{1}{2} \sigma^a$ give rise to the following $\mathrm{SU}(2)$ Lie algebra:
\begin{equation}
    [T^a, T^b] = i \epsilon^{abc} T^c,
    \label{eq:SU2_Lie_algebra}
\end{equation}
which is recognized to coincide with the Lie algebra of $\mathrm{SO}(3)$ in \eqref{eq:SO3_Lie_algebra}. This shows that locally they are similar (they have the same structure constants), although globally there are differences: using the language of differential geometry, we can say that the group $\mathrm{SU}(2)$ is a double cover of the group $\mathrm{SO}(3)$. This difference is seen explicitly in the defining representation of $\mathrm{SU}(2)$ (the spin-$\tfrac{1}{2}$ or $2$ representation).

A finite rotation is obtained by exponentiating infinitesimal transformations to make them finite:
\begin{equation}
    U(\vec{\theta}) = \exp(i \theta_a T^a).
    \label{eq:SU2_exponential_map}
\end{equation}
In particular, a finite rotation around the $z$-axis is obtained by choosing $\theta^3 = \theta$ and $\theta^1 = \theta^2 = 0$, to find a matrix $U_3(\theta)$ given by
\[
    \begin{aligned}
        U_3(\theta) & = e^{i \theta T^3} = e^{i \theta \tfrac{\sigma^3}{2}} = \sum_{n=0}^\infty \frac{(i \tfrac{\theta}{2} \sigma^3)^n}{n!} = \mathbb{I} \cos\left( \theta / 2 \right) + i \sigma^3 \sin\left(\theta / 2 \right) \\
                    & = \begin{pmatrix}
                            \cos(\theta / 2) + i \sin(\theta / 2) & 0                                     \\
                            0                                     & \cos(\theta / 2) - i \sin(\theta / 2)
                        \end{pmatrix} = \begin{pmatrix}
                                            e^{i \theta/2} & 0               \\
                                            0              & e^{-i \theta/2}
                                        \end{pmatrix},
    \end{aligned}
\]
where again we have used the idempotent property of the Pauli matrices, \((\sigma^3)^2 = \mathbb{I}\) to group the even and odd powers of the expansion separately.

Setting $\theta = 2\pi$ gives the transformation
\[
    U_3(\theta = 2\pi) = -\mathbb{I},
\]
which does not coincide with the identity in $SU(2)$. The identity transformation is obtained only for $\theta = 4\pi$. As known from quantum mechanics, all irreducible unitary representations of $SU(2)$ are characterized by a quantum number $j$ that can be either an integer or a half-integer. They are of dimension $2j + 1$.

\textbf{Historical Note:} Pauli introduced the matrices in \eqref{eq:Pauli_matrices} to describe the electron's spin, defining the spin operator $\vec{S} = \frac{1}{2} \vec{\sigma}$, which acts on a two-component wave function (spinor).

\subsection{SU(3)}

The same analysis performed to extract the infinitesimal generators of $SU(2)$ applies also to the general $SU(N)$ group, whose generators are then seen to be traceless, Hermitian, $N \times N$ matrices. There are $N^2 - 1$ of such matrices, so that there are $N^2 - 1$ independent Lie parameters for the group $SU(N)$. In particular, the eight infinitesimal generators of $SU(3)$ in the fundamental representation are given by the Gell-Mann matrices $\lambda^a$, which form a basis of Hermitian $3 \times 3$ traceless matrices (generalizing the Pauli matrices $\sigma^a$ for $SU(2)$):
\begin{equation}
    T^a = \frac{1}{2} \lambda^a, \quad a = 1, \dots, 8,
    \label{eq:SU3_generators}
\end{equation}
where the Gell-Mann matrices are
\begin{equation}
    \begin{aligned}
        \lambda^1 & = \begin{pmatrix}
                          0 & 1 & 0 \\
                          1 & 0 & 0 \\
                          0 & 0 & 0
                      \end{pmatrix}, \quad \lambda^2 = \begin{pmatrix}
                                                           0 & -i & 0 \\
                                                           i & 0  & 0 \\
                                                           0 & 0  & 0
                                                       \end{pmatrix}, \quad \lambda^3 = \begin{pmatrix}
                                                                                            1 & 0  & 0 \\
                                                                                            0 & -1 & 0 \\
                                                                                            0 & 0  & 0
                                                                                        \end{pmatrix}, \\
        \lambda^4 & = \begin{pmatrix}
                          0 & 0 & 1 \\
                          0 & 0 & 0 \\
                          1 & 0 & 0
                      \end{pmatrix}, \quad \lambda^5 = \begin{pmatrix}
                                                           0 & 0 & -i \\
                                                           0 & 0 & 0  \\
                                                           i & 0 & 0
                                                       \end{pmatrix}, \quad \lambda^6 = \begin{pmatrix}
                                                                                            0 & 0 & 0 \\
                                                                                            0 & 0 & 1 \\
                                                                                            0 & 1 & 0
                                                                                        \end{pmatrix}, \\
        \lambda^7 & = \begin{pmatrix}
                          0 & 0 & 0  \\
                          0 & 0 & -i \\
                          0 & i & 0
                      \end{pmatrix}, \quad \lambda^8 = \frac{1}{\sqrt{3}} \begin{pmatrix}
                                                                              1 & 0 & 0  \\
                                                                              0 & 1 & 0  \\
                                                                              0 & 0 & -2
                                                                          \end{pmatrix},
    \end{aligned}
    \label{eq:Gell-Mann_matrices}
\end{equation}
where \(\lambda^3\) and \(\lambda^8\) are diagonal (and \(\lambda^8\) is not uniquely defined). The maximal number of simultaneously diagonalizable generators is called the rank of the group. The rank of \(\mathrm{SU}(3)\) is 2, while the rank of \(\mathrm{SU}(N)\) is \(N - 1\). The above matrices are normalized so that
\begin{equation}
    \Tr(T^a T^b) = \frac{\delta^{ab}}{2},
    \label{eq:SU3_generator_normalization}
\end{equation}
just as was done for $SU(2)$, see eq. \eqref{eq:SU2_generator_normalization}. An arbitrary element of the $SU(3)$ group in the fundamental representation is thus described by $3 \times 3$ matrices of the form
\begin{equation}
    U(\theta) = e^{i \theta_a T^a},
    \label{eq:SU3_exponential_map}
\end{equation}
where $\theta^a$ with $a = 1, \dots, 8$ are the eight parameters of the group. By calculating the Lie algebra, one finds the structure constants $f^{abc}$ that correspond to the $SU(3)$ group:
\begin{equation}
    [T^a, T^b] = i f^{abc} T^c.
    \label{eq:SU3_Lie_algebra}
\end{equation}
They are antisymmetric and given by:
\[
    f^{123} = 1, \quad
    f^{147} = -f^{156} = f^{246} = f^{257} = f^{345} = -f^{367} = \tfrac{1}{2}, \quad
    f^{458} = f^{678} = \tfrac{\sqrt{3}}{2}
\]
while all other $f^{abc}$ not related to these by permuting indices are zero. This group has important applications in the description of color associated with strong interactions and in the quark model that classifies the hadrons composed of the three lightest flavors of quarks (up, down, strange).

\subsection{General Case}

We summarize for arbitrary Lie groups what was illustrated above through examples. A Lie group is, by definition, a group of transformations that depend continuously on some parameters. By studying the infinitesimal transformations of the group, i.e., transformations that differ only slightly from the identity, we recognize the generators, operators that "generate" the infinitesimal transformations (and by repetition finite transformations). They identify the so-called Lie algebra of the group, which summarizes information about the group.

In general, an element $g(\theta)$ of a Lie group $G$ (or, more precisely, of the component connected to the identity) can be parametrized by an \textbf{exponential map} in the following way:
\begin{equation}
    g(\theta) = e^{i \theta_a T^a}, \quad a = 1, \dots, \dim G,
    \label{eq:general_Lie_group_exponential_map}
\end{equation}
where the parameters $\theta_a$ are real numbers that parametrize the various elements of the group. They are chosen so that for $\theta_a = 0$ one gets the identity $g = \mathbb{I}$. The operators $T^a$ are the \textbf{generators of the group}. Considering the group as a group of $N \times N$ matrices for some $N$ (for example, the defining representation), the generators are also $N \times N$ matrices. They generate infinitesimal transformations when $\theta_a \ll 1$. Simply expand the exponential function in a Taylor series and keep the lowest order terms:
\[
    g(\theta) = 1 + i \theta_a T^a + \dots
\]

By studying the relations that capture the composition properties of the group using infinitesimal transformations (which are generically non-commutative), one obtains the \textbf{Lie algebra} of the group $G$:
\begin{equation}
    [T^a, T^b] = i f^{ab}_{\ \ c} T^c.
    \label{eq:general_Lie_algebra}
\end{equation}
The constants $f^{ab}_{\ \ c}$ are called \textbf{structure constants} of the group and characterize it. Groups with the same Lie algebra may only differ in their topology but are locally similar. It is useful to mention the \textbf{Jacobi identities}:
\begin{equation}
    f^{ab}_{\ \ d} f^{dc}_{\ \ e} + f^{bc}_{\ \ d} f^{da}_{\ \ e} + f^{ca}_{\ \ d} f^{db}_{\ \ e} = 0,
    \label{eq:Jacobi_identities_structure_constants}
\end{equation}
which are quadratic relations satisfied by the structure constants and emerge as a consequence of the operatorial Jacobi identities (given by the associativity of the operator product):
\begin{equation}
    [[T^a, T^b], T^c] + [[T^b, T^c], T^a] + [[T^c, T^a], T^b] = 0.
    \label{eq:Jacobi_identities_operators}
\end{equation}
We have indeed:
\[
    \begin{aligned}
        [[T^a, T^b], T^c] & = i f^{ab}_{\ \ d} [T^d, T^c] = i f^{ab}_{\ \ d} (i f^{dc}_{\ \ e} T^e) = - f^{ab}_{\ \ d} f^{dc}_{\ \ e} T^e, \\
        [[T^b, T^c], T^a] & = - f^{bc}_{\ \ d} f^{da}_{\ \ e} T^e,                                                                         \\
        [[T^c, T^a], T^b] & = - f^{ca}_{\ \ d} f^{db}_{\ \ e} T^e,
    \end{aligned}
\]
so that summing these three expressions and imposing \eqref{eq:Jacobi_identities_operators} leads to \eqref{eq:Jacobi_identities_structure_constants}, and it is

The structure constants can be used to define the \textbf{adjoint representation} $T_{(A)}^a$ of the Lie algebra, given by the formula:
\begin{equation}
    (T_{(A)}^a)^b_{\ c} = -i f^{ab}_{\ \ c}.
    \label{eq:adjoint_representation_general_definition}
\end{equation}
It is verified to be a representation of the Lie algebra thanks to the Jacobi identities:
\[
    (T_{(A)}^a)^d_{\ g} (T_{(A)}^b)^g_{\ f} - (T_{(A)}^b)^d_{\ g} (T_{(A)}^a)^g_{\ f} = i f^{ab}_{\ \ c} (T_{(A)}^c)^d_{\ f},
\]
which, substituting the definition of the adjoint representation, is equivalent to \eqref{eq:Jacobi_identities_structure_constants}:
\[
    \begin{aligned}
        - f^{ad}_{\ \ g} f^{bg}_{\ \ f} + f^{bd}_{\ \ g} f^{ag}_{\ \ f} = f^{ab}_{\ \ c} f^{cd}_{\ \ f} \\
        \rightarrow f^{ab}_{\ \ g} f^{gd}_{\ \ f} + f^{bd}_{\ \ g} f^{ga}_{\ \ f} + f^{da}_{\ \ g} f^{gb}_{\ \ f} = 0,
    \end{aligned}
\]
after renaming indices appropriately and using the antisymmetry of the structure constants in the first two indices. It is a real representation because the structure constants are real numbers, and it is a representation of dimension equal to the dimension of the group, since the indices $a, b, c = 1, \dots, \dim G$.

Finally, it is useful to mention the \textbf{Baker-Campbell-Hausdorff formula} for the product of exponentials of two linear operators $A$ and $B$:
\begin{equation}
    e^A e^B = e^{A + B + \frac{1}{2}[A, B] + \frac{1}{12}[A, [A, B]] - \frac{1}{12}[B, [A, B]] + \dots}
    \label{eq:Baker-Campbell-Hausdorff_formula}
\end{equation}
where the dots indicate higher-order terms, always expressible in terms of commutators. This formula shows that the knowledge of the Lie algebra is sufficient to reconstruct the (generally non-commutative) product of the elements of the corresponding Lie group.

To summarize, let us list and review some of the main definitions and properties of Lie algebras:
\begin{itemize}
    \item[(i)] $g = e^{i \theta_a T^a} \in G, \quad a = 1, \dots, \dim G$.
    \item[(ii)] $[T^a, T^b] = i f^{ab}_{\ \ c} T^c$.
    \item[(iii)] $\Tr(T_{(F)}^a T_{(F)}^b) = \gamma^{ab}$ \quad (generators in the fundamental representation).
    \item[(iv)] $[[T^a, T^b], T^c] + [[T^b, T^c], T^a] + [[T^c, T^a], T^b] = 0 \Rightarrow f^{ab}_{\ \ d} f^{dc}_{\ \ e} + f^{bc}_{\ \ d} f^{da}_{\ \ e} + f^{ca}_{\ \ d} f^{db}_{\ \ e} = 0$.
    \item[(v)] $f^{abc} = f^{ab}_{\ \ d} \gamma^{dc}$ \quad (completely antisymmetric tensor).
\end{itemize}

Point \textbf{(i)} describes the \textit{exponential parametrization} of an arbitrary element of the group that is connected to the identity. The index $a$ takes as many values as the dimensions of the group. An element of the group is parametrized by the parameters $\theta_a$ with $a = 1, \dots, \dim G$.

Point \textbf{(ii)} corresponds to the \textit{Lie algebra} satisfied by the infinitesimal generators $T^a$. The structure constants $f^{ab}_{\ \ c}$ are antisymmetric on indices $a$ and $b$ and characterize the group $G$.

Point \textbf{(iii)} identifies (the inverse of) a metric $\gamma^{ab}$ called the "\textbf{Killing metric}". This metric is positive-definite only for compact and simple Lie groups, such as $SU(N)$ or $SO(N)$. Being positive, it is often normalized to the Kronecker delta: $\gamma^{ab} = \delta^{ab}$.

Point \textbf{(iv)} amounts to the so-called "\textit{Jacobi identities}" satisfied by the structure constants. They can be used to construct the adjoint representation of the Lie algebra. Denoting by $(T_{(A)}^a)^b_{\ c}$ the matrix elements of the generators of the adjoint representation $T_{(A)}$, we have $(T_{(A)}^a)^b_{\ c} = -i f^{ab}_{\ \ c}$. The Jacobi identities imply that this is a representation. It is real and of dimension equal to the dimension of the group since the indices $a, b, c = 1, \dots, \dim G$. By exponentiation, it gives rise to a representation of the group.

In point \textbf{(v)}, the Killing metric is used to raise an index of the structure constants. Then, $f^{abc}$ are completely antisymmetric in all indices: antisymmetry in the indices $a$ and $b$ is obvious from \textbf{(ii)}, while antisymmetry in the indices $b$ and $c$ is deduced by taking the trace of the Jacobi identities in \textbf{(iv)} and using \textbf{(ii)} and \textbf{(iii)}:
\[
    \begin{aligned}
        \Tr([[T^a, T^b], T^c]) = i f^{ab}_{\ \ d} \Tr(T^d T^c) = i f^{ab}_{\ \ d} \frac{\gamma^{dc}}{2} = \frac{i}{2} f^{abc} \\
        = \Tr(T^a T^b T^c - T^b T^a T^c) = \Tr(T^c T^a T^b - T^c T^a T^b)                                                     \\
        = -\Tr([[T^a, T^c], T^b]) = -\frac{i}{2} f^{acb},
    \end{aligned}
\]
where we have used the cyclicity property of the trace, thus proving antisimmetry in the indices $b$ and $c$. By combining the two antisymmetries, one finds that $f^{abc}$ is completely antisymmetric.

Finally, we conclude with the statement of a theorem which we shall not prove: \textit{the unitary irreducible representations of compact groups are finite-dimensional, while the unitary representations of non-compact groups must be infinite-dimensional}.

Thus, compact groups such as $\mathrm{SO}(N)$ and $\mathrm{SU}(N)$ have unitary finite-dimensional irreps. Non-compact groups, such as the Lorentz group $\mathrm{SO}(3,1)$ and the Poincaré group $\mathrm{ISO}(3,1)$, have unitary representations that must be infinite-dimensional. For applications in relativistic field theory, it is useful to have some knowledge of:

\begin{itemize}
    \item The finite-dimensional representations of the Lorentz group. They are not unitary and are used to label the quantum fields that define a given relativistic QFT.
    \item The unitary representations of the Poincaré group, which are infinite-dimensional and are realized in the Hilbert space of quantum field theories via unitary operators.
\end{itemize}

%\newpage
%\chapter{Exercises}
%\input{exercises/ex1.tex}

\newpage

\renewcommand\thesection{\Alph{section}}

\begin{appendices}
    \chapter*{Appendices}\label{sec:appendix}

    \section{Notation and Conventions} \label{app:notation}
    \lipsum[1]
\end{appendices}

\end{document}