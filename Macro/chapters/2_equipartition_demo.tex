\chapter{Equipartition Theorem — Demo}

This section illustrates both mathematical notation and a physics-oriented theorem.

\begin{theorem}[Generalized Equipartition]
  For any canonical variable $\xi_j$ such that
  \[
    \xi_j e^{-\beta \mathcal{H}}\Big|_{-\infty}^{+\infty} = 0,
  \]
  the following relation holds:
  \[
    k_B T = \left\langle \xi_j \frac{\partial \mathcal{H}}{\partial \xi_j} \right\rangle_c.
  \]
\end{theorem}

\begin{proof}
  Integrate by parts the average
  \[
    \langle \xi_j \partial_{\xi_j} \mathcal{H} \rangle_c
    = \frac{\int \xi_j (\partial_{\xi_j} \mathcal{H})e^{-\beta \mathcal{H}}\dd\xi_j}{\int e^{-\beta \mathcal{H}}\dd\xi_j}
  \]
  and use the boundary condition to eliminate the surface term.
\end{proof}

\begin{corollary}
  For $\mathcal{H} = A\xi_j^2$, we find
  \[
    \langle A\xi_j^2 \rangle = \frac{1}{2}k_BT.
  \]
\end{corollary}

\begin{example}[Free Particle]
  For a gas of $N$ free particles in $d$ dimensions,
  \[
    E = \frac{dN}{2} k_B T.
  \]
\end{example}

\begin{remark}
  This verifies that each quadratic degree of freedom contributes $\tfrac{1}{2}k_B T$ to the mean energy.
\end{remark}
