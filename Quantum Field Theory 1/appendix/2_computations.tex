\section{Computations and further developments} \label{app:computations}
\subsection*{Sign of complex radical}
\lipsum[2]

\subsection*{Unitary operators are generated by Hermitian operators}

Let \( U \) be a unitary operator, that is,
\[
    U^\dagger U = U U^\dagger = I.
\]
We want to show that every unitary operator can be written as the exponential of an Hermitian operator.

Consider an operator \( A \) such that
\[
    U = e^{iA}.
\]
We can check the unitarity of \( U \):
\[
    U^\dagger = (e^{iA})^\dagger = e^{-iA^\dagger}.
\]
Then
\[
    U^\dagger U = e^{-iA^\dagger} e^{iA} = e^{i(A - A^\dagger)}.
\]
For \( U \) to be unitary we must have \( U^\dagger U = I \), hence
\[
    A^\dagger = A.
\]
Therefore, \( A \) must be Hermitian. Conversely, given any Hermitian operator \( A \), the exponential \( e^{iA} \) is unitary because:
\[
    (e^{iA})^\dagger = e^{-iA} \quad \Rightarrow \quad (e^{iA})^\dagger e^{iA} = e^{-iA} e^{iA} = I.
\]

In summary, unitary operators are generated by Hermitian operators via the exponential map:
\[
    U = e^{iA}, \qquad A = A^\dagger.
\]
In quantum mechanics, this result implies that any continuous unitary transformation can be written as the exponential of an Hermitian generator, which corresponds physically to an observable. For instance, the time evolution operator
\[
    U(t) = e^{-\frac{i}{\hbar} H t}
\]
is generated by the Hamiltonian \( H \), an Hermitian operator representing the energy of the system.

\subsection*{Free EM field Lagrangian, field strenght and energy-momentum tensors}

As anticipated in section \ref{sec:free_em_field_lagrangian}, the lagrangian of a free EM field in absence of external sources reads:
\[
    \mathcal{L} = -\frac{1}{2}(\partial_{\mu} A_{\nu})(\partial^{\mu}A^{\nu}) + \frac{1}{2}(\partial_\mu A^{\mu})^2.
\]
We want to show how it is possible to rewrite it in the following form:
\[
    \mathcal{L} = -\frac{1}{4}F_{\mu \nu}F^{\mu \nu}.
\]
Let us develop further this last form, then we will find the same expression from the original lagrangian; then:
\[
    \begin{aligned}
        -\frac{1}{4}F_{\mu \nu}F^{\mu \nu} & = -\frac{1}{4} \left(\partial_\mu A_\nu - \partial_\nu A_\mu \right)\left(\partial^\mu A^\nu - \partial^\nu A^\mu \right)                                                              \\
                                           & = -\frac{1}{4} \left(\partial_\mu A_\nu\partial^\mu A^\nu - \partial_\mu A_\nu\partial^\nu A^\mu - \partial_\nu A_\mu\partial^\mu A^\nu + \partial_\nu A_\mu\partial^\nu A^\mu\right),
    \end{aligned}
\]
but since the last two terms are identical to the first two if we rename the indices \(\mu \leftrightarrow \nu\) (on the latter couple), then
\[
    -\frac{1}{4}F_{\mu \nu}F^{\mu \nu} = -\frac{1}{2} \left(\partial_\mu A_\nu\partial^\mu A^\nu - \partial_\mu A_\nu\partial^\nu A^\mu\right) = \frac{1}{2} (\partial_\mu A_\nu)(\partial^{\nu} A^{\mu}-\partial^{\mu}A^{\nu}).
\]
Now, developing from the first presented expression for the lagrangian, we implicitly perform some integrations by part with the idea to simplify the boundary terms: we integrate because the dynamics is governed by the action (through its estremization) and we neglect the boundary terms because we assume that \textit{no interesting physics is taking place on} \(\partial \mathbb{R}^4\) (at infinite distances in an infinitely further future). Instead of explicit integrals we substitute terms with the inverse product rule for derivatives and neglect the term associated to the total derivative:
\[
    \begin{aligned}
        \mathcal{L} & = -\frac{1}{2}(\partial_{\mu} A_{\nu})(\partial^{\mu}A^{\nu}) + \frac{1}{2}(\partial_\mu A^{\mu})(\partial_\nu A^{\nu})                                                                                                                  \\
                    & = -\frac{1}{2} (\partial_{\mu} A_{\nu})(\partial^{\mu}A^{\nu}) + \frac{1}{2}\left[\underbrace{\partial _\mu (A^{\mu}\partial_\nu A^{\nu})}_{\text{on }\partial \mathbb{R}^4 \to 0} - A^{\mu}\partial _\mu(\partial_\nu A^{\nu})\right]   \\
                    & = -\frac{1}{2} (\partial_{\mu} A_{\nu})(\partial^{\mu}A^{\nu}) - \frac{1}{2}\left[\underbrace{\partial_\nu(A^{\mu}\partial _\mu A^{\nu})}_{\text{on }\partial \mathbb{R}^4 \to 0} - (\partial_\nu A^{\mu})(\partial _\mu A^{\nu})\right] \\
                    & = \frac{1}{2} \left[\eta^{\nu \sigma} \eta_{\nu \rho}(\partial^{\rho}A^{\mu})(\partial_\mu A_{\sigma}) - (\partial_\mu A_\nu)(\partial^{\mu}A^{\nu})\right],
    \end{aligned}
\]
and finally, after the two "integrations by part" and since \(\eta^{\nu \sigma} \eta_{\nu \rho} = \delta^{\sigma}_{\rho}\), we rename the indices from the first term as \(\sigma,\,\rho \to \nu\) and we obtain
\[
    \mathcal{L} = \frac{1}{2} (\partial_\mu A_\nu)(\partial^{\nu} A^{\mu}-\partial^{\mu}A^{\nu}),
\]
obtaining in the end the same expression from the previous development and so the desired result: the Lagrangian in terms of \textbf{field strenght tensor}
\[
    \mathcal{L} = -\frac{1}{4}F_{\mu \nu}F^{\mu \nu}.
\]

---------

From this last form of the Lagrangian it's easy to derive a practical expression for the \textbf{energy-momentum tensor}: from the expression in equation \eqref{eq:energy_momentum_tensor} (using \(A^{\mu}\) as the dynamical field) we have:
\[
    \begin{aligned}
        T^{\mu \nu} & = \frac{\partial \mathcal{L}}{\partial (\partial_{\mu}A_\rho)}\,\partial^{\nu}A_{\rho} - \eta^{\mu\nu}\mathcal{L}                                                                                                    \\
                    & = \frac{\partial}{\partial (\partial_{\mu}A_\rho)}\left(-\frac{1}{4}F_{\sigma \lambda}F^{\sigma \lambda}\right) \partial^{\nu}A_{\rho} - \eta^{\mu\nu}\left(-\frac{1}{4}F_{\sigma \lambda}F^{\sigma \lambda}\right).
    \end{aligned}
\]
Let us ignore the last term, we develop the former:
\[
    \begin{aligned}
        \frac{\partial}{\partial (\partial_{\mu}A_\rho)}\left(-\frac{1}{4}F_{\sigma \lambda}F^{\sigma \lambda}\right) \partial^{\nu}A_{\rho} & = -\frac{1}{4}\frac{\partial}{\partial (\partial_{\mu}A_\rho)}\left(\eta^{\sigma \alpha}\eta^{\lambda \beta}F_{\sigma \lambda}F_{\alpha \beta}\right) \partial^{\nu}A_{\rho}                                                                                                                                                                                  \\
                                                                                                                                             & = -\frac{1}{4}\eta^{\sigma \alpha}\eta^{\lambda \beta} \left[F_{\alpha \beta} \left(\delta^{\mu}_{\ \sigma}\delta^{\rho}_{\ \lambda} - \delta^{\mu}_{\ \lambda}\delta^{\rho}_{\ \sigma}\right) + F_{\sigma \lambda}\left(\delta^{\mu}_{\ \alpha}\delta^{\rho}_{\ \beta} - \delta^{\mu}_{\ \beta}\delta^{\rho}_{\ \alpha}\right) \right]\partial^{\nu}A_{\rho} \\
                                                                                                                                             & = -\frac{1}{4} \left[F_{\alpha \beta}\left(\eta^{\mu \alpha}\eta^{\rho \beta} - \eta^{\rho \alpha}\eta^{\mu \beta}\right) + F_{\sigma \lambda}\left(\eta^{\sigma \mu}\eta^{\lambda \rho} - \eta^{\sigma \rho}\eta^{\lambda \mu}\right) \right]\partial^{\nu}A_{\rho}                                                                                          \\
                                                                                                                                             & = -\frac{1}{4} (F^{\mu \rho} - F^{\rho \mu} + F^{\mu \rho} - F^{\rho \mu})\partial^{\nu}A_{\rho} = -\frac{1}{2} (F^{\mu \rho}-F^{\rho \mu})\partial^{\nu}A_{\rho}                                                                                                                                                                                             \\
                                                                                                                                             & = -F^{\mu \rho} \partial^{\nu}A_{\rho}.
    \end{aligned}
\]
So, restoring the latter term we obtain the wnated expression for the energy-momentum tensor:
\[
    T^{\mu \nu} = -F^{\mu \rho} \partial^{\nu}A_{\rho} +\frac{1}{4} \eta^{\mu\nu}\left(F_{\sigma \lambda}F^{\sigma \lambda}\right).
\]

\subsection*{Hamiltonian operator for KG as harmonic oscillators}
Following the approach initiated in section \ref{sec:KG_Hamiltonian_ladder}, we substitute the mode expansions for the field and conjugate momenta into the Hamiltonian expression:
\[
    \hat{H} = \frac{1}{2} \int \mathrm{d}^3 \mathbf{x} \left( \hat{\pi}^2 + \vert \nabla \hat{\psi} \vert^2 + m^2 \hat{\psi}^2 \right).
\]
Recall the Fourier transform expressions in terms of creation and annihilation operators:
\[
    \begin{aligned}
        \hat{\pi}^2                     & = - \frac{\sqrt{\omega_{\mathbf{p}} \omega_{\mathbf{q}}}}{2}\left[ \hat{a}_{\mathbf{p}}e^{i \mathbf{p} \cdot \mathbf{x}} - \hat{a}^{\dagger}_{\mathbf{p}} e^{-i \mathbf{p} \cdot \mathbf{x}} \right]\left[ \hat{a}_{\mathbf{q}}e^{i \mathbf{q} \cdot \mathbf{x}} - \hat{a}^{\dagger}_{\mathbf{q}} e^{-i \mathbf{q} \cdot \mathbf{x}} \right],                         \\
        \vert \nabla \hat{\psi} \vert^2 & = \frac{1}{2 \sqrt{\omega_{\mathbf{p}} \omega_{\mathbf{q}}}} i \mathbf{p}\left[ \hat{a}_{\mathbf{p}}e^{i \mathbf{p} \cdot \mathbf{x}} - \hat{a}^{\dagger}_{\mathbf{p}} e^{-i \mathbf{p} \cdot \mathbf{x}} \right]i\mathbf{q}\left[ \hat{a}_{\mathbf{q}}e^{i \mathbf{q} \cdot \mathbf{x}} - \hat{a}^{\dagger}_{\mathbf{q}} e^{-i \mathbf{q} \cdot \mathbf{x}} \right], \\
        m^2 \hat{\psi}^2                & = \frac{m^2}{2 \sqrt{\omega_{\mathbf{p}} \omega_{\mathbf{q}}}}\left[ \hat{a}_{\mathbf{p}}e^{i \mathbf{p} \cdot \mathbf{x}} + \hat{a}^{\dagger}_{\mathbf{p}} e^{-i \mathbf{p} \cdot \mathbf{x}} \right]\left[ \hat{a}_{\mathbf{q}}e^{i \mathbf{q} \cdot \mathbf{x}} + \hat{a}^{\dagger}_{\mathbf{q}} e^{-i \mathbf{q} \cdot \mathbf{x}} \right] .
    \end{aligned}
\]
By substituting these terms into the Hamiltonian and regrouping, we obtain:
\[
    \begin{aligned}
        \hat{H} = \frac{1}{2} \int \mathrm{d}^3 \mathbf{x}\frac{\mathrm{d}^3 \mathbf{p}\,\mathrm{d}^3 \mathbf{q}}{(2\pi)^6} \Bigg[
         & e^{i(\mathbf{p} +\mathbf{q})\cdot \mathbf{x}} \left( -\frac{\sqrt{\omega_{\mathbf{p}} \omega_{\mathbf{q}}}}{2}    -\frac{\mathbf{p} \cdot \mathbf{q}}{2\sqrt{\omega_{\mathbf{p}}\omega_{\mathbf{q}}}} + \frac{m^2}{2\sqrt{\omega_{\mathbf{p}} \omega_{\mathbf{q}}}} \right) \hat{a}_{\mathbf{p}} \hat{a}_{\mathbf{q}}                       \\
         & +e^{-i(\mathbf{p} +\mathbf{q})\cdot \mathbf{x}} \left( -\frac{\sqrt{\omega_{\mathbf{p}} \omega_{\mathbf{q}}}}{2}    -\frac{\mathbf{p} \cdot \mathbf{q}}{2\sqrt{\omega_{\mathbf{p}}\omega_{\mathbf{q}}}} + \frac{m^2}{2\sqrt{\omega_{\mathbf{p}} \omega_{\mathbf{q}}}} \right) \hat{a}^{\dagger}_{\mathbf{p}} \hat{a}^{\dagger}_{\mathbf{q}} \\
         & +e^{i(\mathbf{p} -\mathbf{q})\cdot \mathbf{x}} \left(+\frac{\sqrt{\omega_{\mathbf{p}} \omega_{\mathbf{q}}}}{2}    +\frac{\mathbf{p} \cdot \mathbf{q}}{2\sqrt{\omega_{\mathbf{p}}\omega_{\mathbf{q}}}} + \frac{m^2}{2\sqrt{\omega_{\mathbf{p}} \omega_{\mathbf{q}}}}\right) \hat{a}_{\mathbf{p}} \hat{a}^{\dagger}_{\mathbf{q}}              \\
         & +e^{-i(\mathbf{p} -\mathbf{q})\cdot \mathbf{x}} \left(+\frac{\sqrt{\omega_{\mathbf{p}} \omega_{\mathbf{q}}}}{2}    +\frac{\mathbf{p} \cdot \mathbf{q}}{2\sqrt{\omega_{\mathbf{p}}\omega_{\mathbf{q}}}} + \frac{m^2}{2\sqrt{\omega_{\mathbf{p}} \omega_{\mathbf{q}}}}\right) \hat{a}^{\dagger}_{\mathbf{p}} \hat{a}_{\mathbf{q}}
            \Bigg].
    \end{aligned}
\]
We proceed by integrating over \(\mathrm{d}^3 \mathbf{x}\). Using the integral representation of the Dirac delta, \(\int \tfrac{\mathrm{d}^3 \mathbf{x}}{(2\pi)^3} e^{\pm i (\mathbf{p} \pm \mathbf{q})\cdot \mathbf{x}}=\delta(\mathbf{p} \pm \mathbf{q})\), we isolate the terms proportional to \(\delta(\mathbf{p} + \mathbf{q})\) and \(\delta(\mathbf{p} -\mathbf{q})\):
\[
    \begin{aligned}
        \hat{H} & = \frac{1}{2} \int \frac{\mathrm{d}^3 \mathbf{p}\,\mathrm{d}^3 \mathbf{q}}{(2\pi)^3} \, \delta(\mathbf{p} +\mathbf{q}) \left( \hat{a}_{\mathbf{p}} \hat{a}_{\mathbf{q}} + \hat{a}^{\dagger}_{\mathbf{p}} \hat{a}^{\dagger}_{\mathbf{q}} \right)\left( -\frac{\sqrt{\omega_{\mathbf{p}} \omega_{\mathbf{q}}}}{2}    -\frac{\mathbf{p} \cdot \mathbf{q}}{2\sqrt{\omega_{\mathbf{p}}\omega_{\mathbf{q}}}} + \frac{m^2}{2\sqrt{\omega_{\mathbf{p}} \omega_{\mathbf{q}}}} \right)  \\
                & + \frac{1}{2} \int \frac{\mathrm{d}^3 \mathbf{p}\,\mathrm{d}^3 \mathbf{q}}{(2\pi)^3} \, \delta(\mathbf{p} -\mathbf{q}) \left( \hat{a}_{\mathbf{p}} \hat{a}^{\dagger}_{\mathbf{q}} + \hat{a}^{\dagger}_{\mathbf{p}} \hat{a}_{\mathbf{q}} \right)\left( +\frac{\sqrt{\omega_{\mathbf{p}} \omega_{\mathbf{q}}}}{2}    +\frac{\mathbf{p} \cdot \mathbf{q}}{2\sqrt{\omega_{\mathbf{p}}\omega_{\mathbf{q}}}} + \frac{m^2}{2\sqrt{\omega_{\mathbf{p}} \omega_{\mathbf{q}}}} \right).
    \end{aligned}
\]
Evaluating the delta functions, which means integrating in \(\mathrm{d}^3 \mathbf{q}\), and noting the symmetry \(\omega_{\mathbf{p}} = \sqrt{\vert \mathbf{p} \vert^2 + m^2} = \omega_{-\mathbf{p}}\), the expression simplifies to:
\[
    \begin{aligned}
        \hat{H} & = \frac{1}{2} \int \frac{\mathrm{d}^3 \mathbf{p}}{(2\pi)^3} \, \frac{1}{2 \omega_{\mathbf{p}}}\left( \hat{a}_{\mathbf{p}} \hat{a}_{(-\mathbf{p})} + \hat{a}^{\dagger}_{\mathbf{p}} \hat{a}^{\dagger}_{(-\mathbf{p})} \right)\left( -\omega_{\mathbf{p}}^2 + \vert \mathbf{p} \vert^2 + m^2 \right) \\
                & + \frac{1}{2} \int \frac{\mathrm{d}^3 \mathbf{p}}{(2\pi)^3} \, \frac{1}{2 \omega_{\mathbf{p}}}\left( \hat{a}_{\mathbf{p}} \hat{a}^{\dagger}_{\mathbf{p}} + \hat{a}^{\dagger}_{\mathbf{p}} \hat{a}_{\mathbf{p}} \right)\left( \omega_{\mathbf{p}}^2 + \vert \mathbf{p} \vert^2 + m^2 \right),
    \end{aligned}
\]
The first term vanishes identically because the factor \(-\omega_{\mathbf{p}}^2 + \vert \mathbf{p} \vert^2 + m^2\) is zero by definition of the dispersion relation. We are thus left with the final Hamiltonian in terms of ladder operators:
\[
    \hat{H} = \frac{1}{2} \int \frac{\mathrm{d}^3 \mathbf{p}}{(2\pi)^3} \, \omega_{\mathbf{p}}\left( \hat{a}_{\mathbf{p}} \hat{a}^{\dagger}_{\mathbf{p}} + \hat{a}^{\dagger}_{\mathbf{p}} \hat{a}_{\mathbf{p}} \right).
\]