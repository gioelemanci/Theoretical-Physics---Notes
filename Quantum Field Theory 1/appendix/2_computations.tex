\section{Computations and further developments} \label{app:computations}
\subsection*{Sign of complex radical}
\lipsum[2]

\subsection*{Unitary operators are generated by Hermitian operators}

Let \( U \) be a unitary operator, that is,
\[
    U^\dagger U = U U^\dagger = I.
\]
We want to show that every unitary operator can be written as the exponential of an Hermitian operator.

Consider an operator \( A \) such that
\[
    U = e^{iA}.
\]
We can check the unitarity of \( U \):
\[
    U^\dagger = (e^{iA})^\dagger = e^{-iA^\dagger}.
\]
Then
\[
    U^\dagger U = e^{-iA^\dagger} e^{iA} = e^{i(A - A^\dagger)}.
\]
For \( U \) to be unitary we must have \( U^\dagger U = I \), hence
\[
    A^\dagger = A.
\]
Therefore, \( A \) must be Hermitian. Conversely, given any Hermitian operator \( A \), the exponential \( e^{iA} \) is unitary because:
\[
    (e^{iA})^\dagger = e^{-iA} \quad \Rightarrow \quad (e^{iA})^\dagger e^{iA} = e^{-iA} e^{iA} = I.
\]

In summary, unitary operators are generated by Hermitian operators via the exponential map:
\[
    U = e^{iA}, \qquad A = A^\dagger.
\]
In quantum mechanics, this result implies that any continuous unitary transformation can be written as the exponential of an Hermitian generator, which corresponds physically to an observable. For instance, the time evolution operator
\[
    U(t) = e^{-\frac{i}{\hbar} H t}
\]
is generated by the Hamiltonian \( H \), an Hermitian operator representing the energy of the system.
