\subsection{Finite dimensional representations}

Finite-dimensional representations of the Lorentz group play a central role in quantum field theory, as they determine how physical fields transform under Lorentz transformations. These representations describe the intrinsic angular momentum (or \textit{spin}) content of fields. The Lorentz group itself is non-compact, and therefore it admits only non-unitary finite-dimensional representations. Such representations are classified according to the group isomorphism:
\[
    \mathrm{SO}^+(1,3) \simeq \mathrm{SL}(2,\mathbb{C})/\mathbb{Z}_2,
\]
and labeled by two half-integer indices \((j_L, j_R)\), corresponding respectively to the two \(\mathrm{SU}(2)\) factors in the decomposition
\[
    \mathrm{SL}(2,\mathbb{C}) \sim \mathrm{SU}(2)_L \times \mathrm{SU}(2)_R.
\]
We now review the most important finite-dimensional representations, relevant for relativistic field theory.

\subsubsection{Trivial representation}

The simplest representation is the \textbf{trivial representation}, in which all Lorentz generators vanish:
\[
    M^{\mu \nu} = 0 \implies \Lambda = e^{-\frac{i}{2}\omega_{\mu \nu}M^{\mu \nu}} = \mathbb{I}.
\]
It is a one-dimensional (\(1D\)) representation acting on scalar fields \(\phi\):
\[
    \phi \to \phi' = \Lambda \phi = \phi.
\]
Such a field is invariant under Lorentz transformations and is thus denoted by the label \((0,0)\). It represents a \textbf{spinless particle}, such as the Higgs boson or the pion field, when regarded as a function defined on spacetime. This representation captures fields that do not transform under spatial rotations or boosts.

\subsubsection{Vectorial representation}

The next non-trivial case is the \textbf{vectorial representation}, in which the Lorentz generators act on four-vectors \(V^\mu\) according to:
\[
    (M^{\rho \sigma})^{\mu}_{\ \nu} = -i \left( \eta^{\mu \sigma} \delta^{\rho}_{\ \nu} - \eta^{\rho \mu} \delta^{\sigma}_{\ \nu} \right).
\]
These are \(4 \times 4\) matrices generating transformations of the form:
\[
    V^{\mu} \to V'^{\mu} = \Lambda^{\mu}_{\ \nu} V^{\nu} = e^{-\frac{i}{2} \omega_{\rho \sigma} (M^{\rho \sigma})^{\mu}_{\ \nu}} V^{\nu}.
\]
This is the \((\tfrac{1}{2}, \tfrac{1}{2})\) representation, often referred to as the \textit{fundamental vector representation}. It describes spin–1 particles, such as gauge bosons (photon, gluon, \(W^\pm\), and \(Z\) bosons), whose classical fields transform as four-vectors under Lorentz transformations.

\subsubsection{Spinorial representation}

The \textbf{spinorial representations} of the Lorentz group are more subtle. The proper orthochronous Lorentz group \(\mathrm{SO}^+(1,3)\) is not simply connected, and its double cover is the group \(\mathrm{SL}(2,\mathbb{C})\):
\[
    Spin^+(1,3) \equiv \mathrm{SL}(2,\mathbb{C}), \qquad \mathrm{SO}^+(1,3) \simeq \mathrm{SL}(2,\mathbb{C})/\mathbb{Z}_2.
\]
This means that for every Lorentz transformation \(\Lambda \in \mathrm{SO}^+(1,3)\) there exist two elements \(\pm A \in \mathrm{SL}(2,\mathbb{C})\) corresponding to it, establishing a two-to-one homomorphism:
\[
    A, B \in \mathrm{SL}(2,\mathbb{C}) \quad \longrightarrow \quad \Lambda \in \mathrm{SO}^+(1,3), \qquad \Lambda(A)\Lambda(B) = \Lambda(AB).
\]
The spinorial representations of \(\mathrm{SL}(2,\mathbb{C})\) correspond to the half-integer spin representations of the Lorentz group.

We introduce the Hermitian matrices
\[
    \sigma^{\mu} = (\mathbb{I}, \boldsymbol{\sigma}),
\]
where \(\boldsymbol{\sigma} = (\sigma^1, \sigma^2, \sigma^3)\) are the \textbf{Pauli matrices}:
\[
    \sigma^1 = \begin{pmatrix} 0 & 1 \\ 1 & 0 \end{pmatrix}, \quad
    \sigma^2 = \begin{pmatrix} 0 & -i \\ i & 0 \end{pmatrix}, \quad
    \sigma^3 = \begin{pmatrix} 1 & 0 \\ 0 & -1 \end{pmatrix}.
\]
A spacetime point \(x^{\mu}\) can then be represented by a \(2\times2\) Hermitian matrix:
\[
    X = x_{\mu}\sigma^{\mu} =
    \begin{pmatrix}
        x_0 + x_3   & x_1 - i x_2 \\
        x_1 + i x_2 & x_0 - x_3
    \end{pmatrix}.
\]
Under a Lorentz transformation, this object transforms as
\[
    X \longrightarrow X' = N X N^{\dagger}, \qquad N \in \mathrm{SL}(2,\mathbb{C}),
\]
which preserves the Minkowski norm, since
\[
    \det X' = \det N \det X \det N^{\dagger} = \det X = x_{\mu} x^{\mu} = x_0^2 - |\mathbf{x}|^2.
\]
Thus, \(N\) and \(-N\) generate the same Lorentz transformation, explaining the double-cover structure:
\[
    N = \pm \mathbb{I}_2 \quad \longrightarrow \quad \Lambda = \mathbb{I}_4.
\]

\paragraph{Representations of \(\mathrm{SL}(2,\mathbb{C})\).}

We can now define two inequivalent fundamental spinor representations:

\begin{enumerate}
    \item \textbf{Left-handed Weyl spinors} (fundamental representation \((\tfrac{1}{2},0)\)):
          \[
              \psi_{\alpha} =
              \begin{pmatrix}
                  \psi_1 \\ \psi_2
              \end{pmatrix}
              \to
              \psi'_{\alpha} = N_{\alpha}^{\ \beta} \psi_{\beta}, \quad \alpha,\beta = 1,2.
          \]
          These two-component spinors transform under the left-handed subgroup \(\mathrm{SU}(2)_L\).

    \item \textbf{Right-handed Weyl spinors} (complex conjugate representation \((0,\tfrac{1}{2})\)):
          \[
              \bar{\chi}_{\dot{\alpha}} =
              \begin{pmatrix}
                  \bar{\chi}_{\dot{1}} \\ \bar{\chi}_{\dot{2}}
              \end{pmatrix}
              \to
              \bar{\chi}'_{\dot{\alpha}} = N_{\dot{\alpha}}^{* \dot{\beta}} \bar{\chi}_{\dot{\beta}}, \quad \dot{\alpha},\dot{\beta} = 1,2.
          \]
          These transform under the right-handed subgroup \(\mathrm{SU}(2)_R\).
\end{enumerate}

\begin{remark}
    In the Lorentz group \(\mathrm{SO}^+(1,3)\), indices are raised and lowered using the Minkowski metric \(\eta_{\mu\nu}\). In contrast, within \(\mathrm{SL}(2,\mathbb{C})\) spinor indices are manipulated using the antisymmetric invariant tensors \(\epsilon^{\alpha\beta}\) and \(\epsilon^{\dot{\alpha}\dot{\beta}}\):
    \[
        \epsilon^{\alpha \beta} = \epsilon^{\dot{\alpha}\dot{\beta}} =
        \begin{pmatrix}
            0  & 1 \\
            -1 & 0
        \end{pmatrix}.
    \]
    These are invariant under \(\mathrm{SL}(2,\mathbb{C})\) transformations:
    \[
        \epsilon^{\alpha \beta} = \epsilon^{\gamma \delta} N^{\ \alpha}_{\gamma} N^{\ \beta}_{\delta} = \epsilon^{\alpha \beta} \det N = \epsilon^{\alpha \beta},
    \]
    allowing the relations
    \[
        \psi^{\alpha} = \epsilon^{\alpha \beta} \psi_{\beta}, \qquad
        \bar{\chi}^{\dot{\alpha}} = \epsilon^{\dot{\alpha}\dot{\beta}} \bar{\chi}_{\dot{\beta}}.
    \]
\end{remark}

\paragraph{Generators.}
The infinitesimal generators of \(\mathrm{SL}(2,\mathbb{C})\) are given by:
\[
    \left\{
    \begin{aligned}
        (\sigma^{\mu \nu})_{\alpha}^{\ \beta}                   & = \frac{i}{4} (\sigma^{\mu} \bar{\sigma}^{\nu} - \sigma^{\nu} \bar{\sigma}^{\mu})^{\ \beta}_{\alpha},             \\
        (\bar{\sigma}^{\mu \nu})^{\dot{\alpha}}_{\ \dot{\beta}} & = \frac{i}{4} (\bar{\sigma}^{\mu} \sigma^{\nu} - \bar{\sigma}^{\nu} \sigma^{\mu})^{\dot{\alpha}}_{\ \dot{\beta}}.
    \end{aligned}
    \right.
\]
Thus,
\[
    \begin{aligned}
        \psi_{\alpha}             & \to e^{-\frac{i}{2}\omega_{\mu\nu}(\sigma^{\mu\nu})_{\alpha}^{\ \beta}}\psi_{\beta}, \quad                               & \text{(LH Weyl spinor)} \\
        \bar{\chi}^{\dot{\alpha}} & \to e^{-\frac{i}{2}\omega_{\mu\nu}(\bar{\sigma}^{\mu\nu})^{\dot{\alpha}}_{\ \dot{\beta}}}\bar{\chi}^{\dot{\beta}}, \quad & \text{(RH Weyl spinor)}
    \end{aligned}
\]

\paragraph{Dirac spinors.}
A \textbf{Dirac spinor} is a four-component object obtained as the direct sum of a left-handed and a right-handed Weyl spinor:
\[
    \Psi =
    \begin{pmatrix}
        \psi_{\alpha} \\
        \bar{\chi}^{\dot{\alpha}}
    \end{pmatrix}.
\]
This reducible representation contains four complex (or eight real) components and transforms as:
\[
    \Psi \to \Psi' = e^{-\frac{i}{2} \omega_{\mu \nu} \Sigma^{\mu \nu}} \Psi,
\]
where the generators in the \textit{Weyl basis} are
\[
    \Sigma^{\mu \nu} = \frac{i}{4} [\gamma^{\mu}, \gamma^{\nu}] =
    \begin{pmatrix}
        \sigma^{\mu \nu} & 0                      \\
        0                & \bar{\sigma}^{\mu \nu}
    \end{pmatrix},
\]
and the gamma matrices are defined as
\[
    \gamma^{\mu} =
    \begin{pmatrix}
        0                  & \sigma^{\mu} \\
        \bar{\sigma}^{\mu} & 0
    \end{pmatrix}.
\]

\paragraph{Chirality operator.}
The \textbf{chirality operator} is defined as
\[
    \gamma^5 = i \gamma^0 \gamma^1 \gamma^2 \gamma^3 =
    \begin{pmatrix}
        -\mathbb{I}_2 & 0            \\
        0             & \mathbb{I}_2
    \end{pmatrix}, \quad (\gamma^5)^2 = \mathbb{I}.
\]
It distinguishes the two Weyl components through its eigenvalues:
\[
    \gamma^5 \Psi =
    \begin{pmatrix}
        -\psi_{\alpha} \\
        \bar{\chi}^{\dot{\alpha}}
    \end{pmatrix},
\]
so that left-handed spinors have eigenvalue \(-1\), and right-handed ones \(+1\). For massless fermions, chirality coincides with helicity, i.e., the projection of spin along the direction of motion.

\paragraph{Chiral projectors.}
The corresponding projectors are:
\[
    P_L = \frac{\mathbb{I}_4 - \gamma^5}{2}, \qquad
    P_R = \frac{\mathbb{I}_4 + \gamma^5}{2},
\]
which isolate the left- and right-handed components:
\[
    P_L \Psi =
    \begin{pmatrix}
        \psi_{\alpha} \\ 0
    \end{pmatrix}, \qquad
    P_R \Psi =
    \begin{pmatrix}
        0 \\ \bar{\chi}^{\dot{\alpha}}
    \end{pmatrix}.
\]

\paragraph{Dirac conjugate.}
The Dirac conjugate spinor is defined by
\[
    \bar{\Psi} = \Psi^{\dagger} \gamma^0 =
    \begin{pmatrix}
        \chi^{\alpha} & \bar{\psi}^*_{\dot{\alpha}}
    \end{pmatrix},
\]
so that the complex conjugate of a left-handed spinor transforms as a right-handed one and vice versa.

\paragraph{Charge conjugation.}
Charge conjugation exchanges particles with their antiparticles. Using the charge conjugation matrix \(C\),
\[
    \Psi^c = C \bar{\Psi}^T =
    \begin{pmatrix}
        \chi^{\alpha} \\ \bar{\psi}^*_{\dot{\alpha}}
    \end{pmatrix}, \quad
    C = \begin{pmatrix}
        \epsilon_{\alpha \beta} & 0                                  \\
        0                       & \epsilon^{\dot{\alpha}\dot{\beta}}
    \end{pmatrix}.
\]
Under \(C\), a left-handed particle becomes a right-handed antiparticle, and vice versa.

\paragraph{Majorana spinors.}
A \textbf{Majorana spinor} is defined by the condition of being its own charge conjugate:
\[
    \Psi_M^c = \Psi_M \iff \psi_{\alpha} = \chi_{\alpha}.
\]
In this case,
\[
    \Psi_M =
    \begin{pmatrix}
        \psi_{\alpha} \\
        \bar{\psi}^{\dot{\alpha}}
    \end{pmatrix} =
    \begin{pmatrix}
        \psi_1 \\ \psi_2 \\ \psi^{*1} \\ \psi^{*2}
    \end{pmatrix}.
\]
A generic Dirac spinor can be expressed as a combination of two real Majorana spinors:
\[
    \Psi = \Psi_{M_1} + i\Psi_{M_2}, \quad \Psi^c = \Psi_{M_1} - i\Psi_{M_2}.
\]
No experimental evidence for fundamental Majorana fermions has been found so far, although neutrinos, being electrically neutral, are prime candidates for Majorana particles.

\subsection{Infinite dimensional representations}
\lipsum[1]

\subsubsection{Field representations}
\lipsum[1]

\subsubsection{One particle Hilbert space representation}
\lipsum[1]
