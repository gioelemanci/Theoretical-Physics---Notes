\section{Energy--Momentum Tensor}

In classical mechanics, invariance under spatial translations leads to the conservation of linear momentum, while invariance under time translations yields the conservation of energy.
We now explore how this idea generalizes in the framework of quantum field theory.

\subsection{Infinitesimal spacetime translations}

Let us consider an \textit{infinitesimal spacetime translation}.
Starting from a general Poincaré transformation,
\[
    x^{\nu} \to x^{\prime\,\nu} = \Lambda^{\nu}_{\ \mu} x^{\mu} + a^{\nu},
\]
we restrict to the case in which there is no Lorentz transformation (\(\Lambda^{\nu}_{\ \mu} = \delta^{\nu}_{\ \mu}\)),
and \(a^{\nu} = \epsilon^{\nu}\) is an infinitesimal constant displacement.
In this case, the transformation acts on the field as a \textbf{scalar transformation}:
\begin{equation}
    \psi_i(x) \to \psi_i^{\prime}(x)
    = \psi_i(x^{\nu} + \epsilon^{\nu})
    = \psi_i(x) + \epsilon^{\nu}\partial_{\nu}\psi_i(x),
    \label{eq:infinitesimal_spacetime_translation}
\end{equation}
where we expanded to first order in \(\epsilon^{\nu}\).

\subsubsection*{Active and passive viewpoints}

It is important to distinguish between two equivalent but conceptually different interpretations of a transformation: the \textit{active} and the \textit{passive} viewpoints. Although they lead to the same mathematical result, they differ in what is considered to be ``moved'' — the field or the coordinate system — and this can be counterintuitive at first.

\begin{itemize}
    \item \textbf{Active transformation:} the field configuration itself is displaced in spacetime, while the coordinates remain fixed. In this picture,
          \[
              \psi_i^{\prime}(x^{\prime}) = \psi_i(x),
          \]
          meaning that what changes is the \textit{physical field}, which is ``pushed forward'' along the translation vector \(\epsilon^{\nu}\).
          One can imagine taking the same field profile and sliding it through spacetime without altering its shape.

    \item \textbf{Passive transformation:} here, we view the change as a relabeling of the coordinate system rather than a motion of the field itself:
          \[
              \psi_i^{\prime}(x) = \psi_i(x^{\prime}).
          \]
          The field configuration is left untouched, but the coordinate grid used to describe it has been shifted by \(\epsilon^{\nu}\).
          Consequently, the field \textit{appears} to change its functional form when expressed in the new coordinates.
\end{itemize}

Although the two interpretations seem opposite — one moves the field, the other moves the coordinates — they are mathematically equivalent.
The difference lies only in our point of view: in one case we deform the object within a fixed frame, in the other we deform the frame itself.
This equivalence is crucial in field theory, where symmetries are often expressed in one language or the other depending on convenience.

\subsection{Variation of the Lagrangian}

Since the Lagrangian density \(\mathcal{L}\) is a scalar function of the fields and their derivatives,
\(\mathcal{L} = \mathcal{L}(\psi_i(x), \partial_{\mu}\psi_i(x))\),
its transformation under \eqref{eq:infinitesimal_spacetime_translation} follows as
\[
    \mathcal{L}(x) \to \mathcal{L}^{\prime}(x)
    = \mathcal{L}(x) + \epsilon^{\nu}\partial_{\nu}\mathcal{L}(x).
\]
Therefore, its infinitesimal variation reads
\[
    \delta \mathcal{L} = \mathcal{L}^{\prime} - \mathcal{L}
    = \epsilon^{\nu}\partial_{\nu}\mathcal{L}(x).
\]
It is often convenient to rewrite this expression as a total divergence:
\[
    \delta \mathcal{L}
    = \epsilon^{\nu}\partial_{\mu}(\delta^{\mu}_{\ \nu}\mathcal{L})
    = \partial_{\mu}(\epsilon^{\nu}\delta^{\mu}_{\ \nu}\mathcal{L}).
\]
The introduction of the Kronecker symbol \(\delta^{\mu}_{\ \nu}\) has no numerical effect, since
\(\partial_{\mu}(\delta^{\mu}_{\ \nu}\mathcal{L}) = \partial_{\nu}\mathcal{L}\);
its purpose is purely formal, allowing us to display explicitly the divergence index \(\mu\).
This step makes the expression ready for application of Noether’s theorem,
since a variation that can be written as a total divergence directly corresponds to a conserved current.

The four independent spacetime translations (one temporal and three spatial)
therefore yield four conserved quantities, which we will identify shortly as the components of the \textbf{four-momentum}.

\subsection{Noether current and charges}

Using the general expression for the Noether current derived in eq.~\eqref{eq:Noether_current},
we have
\[
    (J^{\mu})_{\nu}
    = \frac{\partial \mathcal{L}}{\partial (\partial_{\mu}\psi_i)}\,\delta\psi_i
    - K^{\mu}(\psi_i)
    = \frac{\partial \mathcal{L}}{\partial (\partial_{\mu}\psi_i)}\, \epsilon^{\nu}\partial_{\nu}\psi_i
    - \epsilon^{\nu}\delta^{\mu}_{\ \nu}\mathcal{L} = \epsilon^{\nu} T^{\mu}_{\ \nu}.
\]
Factoring out the infinitesimal \(\epsilon^{\nu}\),
we recognize the expression of the \textbf{energy--momentum tensor} (also called the \textit{stress--energy tensor}):
\begin{equation}
    T^{\mu}_{\ \nu}
    = \frac{\partial \mathcal{L}}{\partial (\partial_{\mu}\psi_i)}\,\partial_{\nu}\psi_i
    - \delta^{\mu}_{\ \nu}\mathcal{L}.
    \label{eq:energy_momentum_tensor}
\end{equation}
The tensor \(T^{\mu}_{\ \nu}\) has the same physical dimensions as the Lagrangian density,
namely that of an energy density, \([\mathcal{L}] = [T] = 4\).
Each component of \(T^{\mu}_{\ \nu}\) corresponds to a conserved current associated with one of the four spacetime translations.

The conserved charges follow from Eq.~\eqref{eq:Noether_charge} as
\[
    Q_{\nu} = \int_{\mathbb{R}^3} \mathrm{d}^3\mathbf{x}\, T^{0}_{\ \nu}.
\]
Raising the index with the Minkowski metric gives the \textbf{four-momentum}:
\[
    P^{\nu} = \int_{\mathbb{R}^3} \mathrm{d}^3\mathbf{x}
    \left(
    \frac{\partial \mathcal{L}}{\partial (\partial_{0}\psi_i)}\,\partial^{\nu}\psi_i
    - \eta^{0\nu}\mathcal{L}
    \right).
\]
Let us compute its components explicitly:
\[
    P^{0}
    = \int_{\mathbb{R}^3} \mathrm{d}^3\mathbf{x}
    \left(
    \frac{\partial \mathcal{L}}{\partial \dot{\psi}_i}\,\dot{\psi}_i
    - \mathcal{L}
    \right)
    = \int_{\mathbb{R}^3} \mathrm{d}^3\mathbf{x}
    \left(
    \pi^i\dot{\psi}_i - \mathcal{L}
    \right)
    = \int_{\mathbb{R}^3} \mathrm{d}^3\mathbf{x}\,\mathcal{H}
    = H,
\]
where \(\pi^i = \partial\mathcal{L}/\partial\dot{\psi}_i\) are the canonical momenta,
and \(\mathcal{H}\) is the Hamiltonian density.
Hence, the temporal component of the four-momentum is the system’s total energy.

For the spatial components we obtain:
\[
    P^{j}
    = \int_{\mathbb{R}^3} \mathrm{d}^3\mathbf{x}
    \left(
    \frac{\partial \mathcal{L}}{\partial \dot{\psi}_i}\,\partial^{j}\psi_i
    - \eta^{0j}\mathcal{L}
    \right)
    = \int_{\mathbb{R}^3} \mathrm{d}^3\mathbf{x}\, \pi^i \partial^{j}\psi_i
    = -\int_{\mathbb{R}^3} \mathrm{d}^3\mathbf{x}\, \pi^i \partial_{j}\psi_i.
\]
Thus, the spatial components correspond to the total linear momentum of the field configuration. For a single particle, the linear momentum is defined as \(p = \frac{\partial L}{\partial \dot{q}}\), and the total momentum of a continuous system is obtained by integrating this density over space.
In field theory, the quantity
\[
    \pi^i(x) = \frac{\partial \mathcal{L}}{\partial \dot{\psi}_i(x)}
\]
plays the role of the conjugate momentum to the field value \(\psi_i(x)\). The integrand $\pi^i \nabla \psi_i$ acts as a local momentum density: it quantifies how the field's temporal variation (via $\pi^i$) couples to its spatial variation (via $\nabla \psi_i$). The minus sign reflects that a positive spatial derivative corresponds to momentum flowing in the opposite direction.

\begin{remark}
    The energy--momentum tensor \(T^{\mu}_{\ \nu}\) encodes how energy and momentum flow through spacetime.
    Its conservation law,
    \[
        \partial_{\mu}T^{\mu}_{\ \nu} = 0,
    \]
    expresses the local conservation of energy (\(\nu = 0\)) and momentum (\(\nu = 1,2,3\)).
    This tensor will later play a central role in both field theory and general relativity,
    where it acts as the source of spacetime curvature.
\end{remark}

\section{Electrodynamics}

Electromagnetism admits a very compact and powerful description in terms of a four-potential and a Lagrangian density. In this section we introduce the basic electromagnetic fields and charges, express the fields in terms of the four-potential \(A^{\mu}\), and present the standard field-theory Lagrangian whose Euler–Lagrange equations reproduce Maxwell's equations (inhomogeneous ones). We work in natural units \(c=\hbar=1\) and use the metric signature \(\eta_{\mu\nu}=\mathrm{diag}(+,-,-,-)\).

The electric and magnetic fields \(\mathbf{E}(t,\mathbf{x})\) and \(\mathbf{B}(t,\mathbf{x})\) and the sources (charge and current densities) \(\rho(t,\mathbf{x})\) and \(\mathbf{j}(t,\mathbf{x})\) satisfy Maxwell's equations (in vacuum, in differential form):
\begin{equation}
    \begin{aligned}
        \nabla\cdot\mathbf{B}                                          & = 0,          \\
        \nabla\times\mathbf{E} + \frac{\partial\mathbf{B}}{\partial t} & = 0,          \\
        \nabla\cdot\mathbf{E}                                          & = \rho,       \\
        \nabla\times\mathbf{B} - \frac{\partial\mathbf{E}}{\partial t} & = \mathbf{j}.
    \end{aligned}
    \label{eq:Maxwell_equations}
\end{equation}
The first two are the \textit{homogeneous} equations (no sources) and are kinematic identities once the fields are written in terms of a potential; the last two are the \textit{inhomogeneous} Maxwell equations (sources on the right-hand side).

\subsection{Four-potential and homogeneous Maxwell equations}

We introduce the electromagnetic four-potential as
\begin{equation}
    A^{\mu}(x) = (\phi(t,\mathbf{x}),\, \mathbf{A}(t,\mathbf{x})),
    \label{eq:EM_four_potential}
\end{equation}
where \(\phi\) denotes the scalar potential and \(\mathbf{A}\) the vector potential.
The electric and magnetic fields can be expressed in terms of \(A^{\mu}\) as
\[
    \mathbf{E} = -\nabla\phi - \frac{\partial\mathbf{A}}{\partial t},
    \qquad
    \mathbf{B} = \nabla\times\mathbf{A}.
\]
This representation makes it clear that both fields originate from derivatives of the same underlying potential, a fact that has deep geometrical meaning and automatically enforces part of Maxwell’s equations.

In particular, the homogeneous Maxwell equations \eqref{eq:Maxwell_equations}\(_{1,2}\) are identically satisfied. Indeed, the magnetic field is divergenceless:
\[
    \nabla \cdot \mathbf{B}
    = \nabla \cdot (\nabla \times \mathbf{A})
    = \partial_i \epsilon_{ijk} \partial_j A_k
    = \epsilon_{ijk} \partial_i \partial_j A_k
    = 0,
\]
because the contraction between the totally antisymmetric Levi-Civita tensor \(\epsilon_{ijk}\) and the symmetric derivatives \(\partial_i \partial_j\) vanishes identically.
Similarly, the second homogeneous equation follows:
\[
    \begin{aligned}
        \nabla\times\mathbf{E} + \frac{\partial\mathbf{B}}{\partial t}
         & = \nabla \times \left(-\nabla\phi - \frac{\partial\mathbf{A}}{\partial t}\right)
        + \frac{\partial}{\partial t}(\nabla \times \mathbf{A})                             \\
         & = \epsilon_{ijk} \partial_j (- \partial_k \phi - \partial_0 A_k)
        + \partial_0 \epsilon_{ijk} \partial_j A_k
        = - \epsilon_{ijk} \partial_j \partial_k \phi
        = 0.
    \end{aligned}
\]
This result shows that the potential formulation automatically encodes the structure of the electromagnetic field so that the homogeneous Maxwell equations
\[
    \nabla \cdot \mathbf{B} = 0,
    \qquad
    \nabla \times \mathbf{E} + \frac{\partial \mathbf{B}}{\partial t} = 0
\]
are identically fulfilled, independently of the specific field configuration.

Following this formalism, it is also convenient to collect the charge and current densities into a single four-vector known as the \textit{four-current}:
\[
    J^{\mu}(x) = (\rho,\, \mathbf{j}),
\]
which satisfies the local charge conservation law, or \textit{continuity equation},
\[
    \partial_{\mu} J^{\mu} = 0.
\]
This compact relation expresses the conservation of electric charge in covariant form, showing that the temporal and spatial components of \(J^{\mu}\) are intrinsically linked as parts of a single relativistic entity that sources the electromagnetic field.

\subsection{Lagrangian and inhomogeneous Maxwell equations} \label{sec:free_em_field_lagrangian}

The standard gauge-invariant Lagrangian density for the free electromagnetic field coupled to an external four-current \(J^\mu\) reads
\begin{equation}
    \mathcal{L} = -\frac{1}{2}(\partial_{\mu} A_{\nu})(\partial^{\mu}A^{\nu}) + \frac{1}{2}(\partial_\mu A^{\mu})^2 - A_{\mu}J^{\mu}.
    \label{eq:EM_free_lagrangian}
\end{equation}
The quadratic terms contain the kinetic terms for the gauge field while the second term is the minimal coupling to sources. This Lagrangian remain invariant under the gauge transformation
\[
    A_{\mu}(x) \mapsto A_{\mu}(x) + \partial_{\mu}\Lambda(x),
\]
provided the scalar function \(\Lambda\) vanishes suitably at the boundary, ensuring that no surface terms arise in the variation.

Let us do some observations:
\begin{itemize}
    \item The mass dimension of the Lagrangian density is \([\mathcal{L}]=4\), while that of the derivative operator is \([\partial_\mu]=1\). From these we infer the dimensions of the potential and the current:
          \[
              [A^{\mu}] = 1, \quad [J^{\mu}] = 3.
          \]
    \item The temporal component \(A_0\) of the four potential is not a dynamical variable, as the Lagrangian contains no quadratic term in its time derivative \(\dot{A}_0\). In fact, the term \(\tfrac12(\partial_0 A_0)(\partial^0 A^0)\) in the first kinetic contribution cancels exactly with \(\tfrac12(\partial_0 A^0)^2\) from the second one.
    \item Since \(A_i = -A^i\), the overall minus sign in the first kinetic term guarantees the correct positive sign for the kinetic energy of the vector potential:
          \[
              -\tfrac12 (\partial_0 A_i)(\partial^0 A^i) = \tfrac12 (\dot{A}_i)^2 = \tfrac12 (\dot{A}^{i})^2.
          \]
\end{itemize}
Hence, if \(A^0\) is non-dynamical, the field initially possesses three independent components, but \textit{gauge symmetry} will allow us to eliminate one more degree of freedom, leaving only two physical ones — corresponding to the \textbf{two transverse polarizations} of electromagnetic waves.

Treating \(A_\sigma\) as the dynamical field, we apply the Euler–Lagrange equations:
\[
    \partial_{\rho}\!\left(\frac{\partial\mathcal{L}}{\partial(\partial_\rho A_\sigma)}\right)
    - \frac{\partial\mathcal{L}}{\partial A_\sigma} = 0.
\]
one easily computes
\[
    \frac{\partial\mathcal{L}}{\partial A_\sigma} = -J^\sigma.
\]
For the other term we should write the Lagrangian explicitly with respect to \(\partial_\mu A_\nu\): we use the metric tensor in order to lower all the indices of similar terms
\[
    \mathcal{L} = -\frac{1}{2}\underbrace{(\partial_{\mu} A_{\nu})\eta^{\mu \alpha}\eta^{\nu \beta}(\partial_{\alpha}A_{\beta})}_{(\partial_{\mu} A_{\nu})(\partial^{\mu}A^{\nu})} + \frac{1}{2}\underbrace{\eta^{\mu \alpha}(\partial_\mu A_{\alpha})\eta^{\nu \beta} (\partial_{\nu}A_{\beta})}_{(\partial_\mu A^{\mu})(\partial_\nu A^{\nu})}- A_{\mu}J^{\mu},
\]
so that now it's easier to compute the derivative with respect to \(\partial_\mu A_\nu\) (knowing that \(\frac{\partial x^{\nu}}{\partial x^{\mu}}=\delta^{\mu}_{\nu}\)), so that
\[
    \begin{aligned}
        \frac{\partial\mathcal{L}}{\partial(\partial_\rho A_\sigma)} & = -\frac{1}{2}\eta^{\mu \alpha}\eta^{\nu \beta}\left[\delta^{\rho}_{\ \mu}\delta^{\sigma}_{\ \nu} (\partial_{\alpha}A_{\beta})+(\partial_{\mu} A_{\nu})\delta^{\rho}_{\ \alpha}\delta^{\sigma}_{\ \beta}\right]      \\
                                                                     & \quad +\frac{1}{2}\eta^{\mu \alpha}\eta^{\nu \beta}\left[\delta^{\rho}_{\ \mu}\delta^{\sigma}_{\ \alpha} (\partial_{\nu}A_{\beta})+ (\partial_{\mu}A_{\alpha})\delta^{\rho}_{\ \nu}\delta^{\sigma}_{\ \beta} \right] \\
                                                                     & = -\frac{1}{2} \left[ \eta^{\rho \alpha}\eta^{\sigma \beta}(\partial_{\alpha}A_{\beta}) + \eta^{\mu \rho}\eta^{\nu \sigma}(\partial_{\mu}A_{\nu}) \right]                                                            \\
                                                                     & \quad + \frac{1}{2} \left[ \eta^{\rho \sigma}\eta^{\nu \beta}(\partial_{\nu}A_{\beta}) + \eta^{\mu \alpha}\eta^{\rho \sigma}(\partial_{\mu}A_{\alpha}) \right]                                                       \\
                                                                     & = - (\partial^{\rho}A^{\sigma}) + \eta^{\rho \sigma}\left(\partial_{\alpha}A^{\alpha} \right),                                                                                                                       \\
    \end{aligned}
\]
and applying now the derivative \(\partial_{\rho}\) from EL equations, we get to:
\[
    \begin{aligned}
        \partial_{\rho} \left(\frac{\partial\mathcal{L}}{\partial(\partial_\rho A_\sigma)}\right) & = \partial_\rho \left[- (\partial^{\rho}A^{\sigma}) + \eta^{\rho \sigma}\left(\partial_{\alpha}A^{\alpha} \right)\right]                                                                    \\
                                                                                                  & = - \partial_{\rho} \partial^{\rho} A^{\sigma} + \partial^{\sigma} \partial_{\alpha} A^{\alpha} = - \partial_{\rho} \partial^{\rho} A^{\sigma} + \partial^{\sigma} \partial_{\rho} A^{\rho} \\
                                                                                                  & = - \partial_{\rho} (\partial^{\rho} A^{\sigma}-\partial^{\sigma} A^{\rho}).
    \end{aligned}
\]
Thus the equation of motion reads:
\[
    J^{\sigma} = \partial_{\rho} (\partial^{\rho} A^{\sigma}-\partial^{\sigma} A^{\rho}).
\]
This compact relation, as we will now see, is equivalent to the inhomogeneous Maxwell equations.

\paragraph{Field strength tensor.}
Introducing the \emph{field strength tensor}
\[
    F_{\mu\nu} \equiv \partial_{\mu}A_{\nu} - \partial_{\nu}A_{\mu},
\]
the equation of motion simplifies and acquires a direct physical interpretation.
Its components naturally contain the electric and magnetic fields:
\[
    F^{i0} = - F^{0i}
    = \partial^i A^0 - \partial^0 A^i
    = (-\nabla \phi - \partial_t \mathbf{A})^i = E^i,
\]
and
\[
    F^{ij} = -F^{ji} = \partial^i A^j - \partial^j A^i = - \epsilon^{ijk} B^k,
\]
with diagonal components \(F^{00} = F^{ii} = 0\).\footnote{When computing the magnetic terms, note that \(-\partial_i A^j + \partial_j A^i = -\epsilon^{ijk} \partial_i A^j = -(\nabla \times \mathbf{A})^k = -B^k\), thus the Levi-Civita tensor is put in to explicit the relation \(F^{ji} = - F^{ij}\) and the antisymmetry of the field strength tensor (the minus sign comes from the lowered indices for nabla definition).}

Using \(F_{\mu\nu}\), the Lagrangian \eqref{eq:EM_free_lagrangian} can be rewritten in the elegant and manifestly Lorentz-invariant form (see Appendix~\ref{app:computations} for the detailed derivation):
\[
    \mathcal{L} = -\frac{1}{4} F_{\mu \nu}F^{\mu \nu} - J_\mu A^{\mu},
\]
and the equations of motion become
\[
    \partial_\mu F^{\mu\nu} = J^{\nu}.
\]
This equation compactly represents the \textit{inhomogeneous} Maxwell equations. Let us show this explicitly:
\begin{itemize}
    \item \((\nu = 0)\) \textbf{Gauss’s law}
          \[
              \begin{aligned}
                  \partial_\mu F^{\mu 0} = \rho, \\
                  \partial_0 F^{00} + \partial_i F^{i0} = \rho,
              \end{aligned}
          \]
          but since \(\partial_0 F^{00}=0\) and \(F^{i0}=E^i\), we recovered equation \eqref{eq:Maxwell_equations}\(_3\):
          \[
              \partial_i E^i = \nabla \cdot \mathbf{E} = \rho.
          \]
    \item \((\nu = i)\) \textbf{Ampère–Maxwell’s law}
          \[
              \begin{aligned}
                  \partial_\mu F^{\mu i} = J^i, \\
                  \partial_0 F^{0i} + \partial_j F^{ji} = J^i,
              \end{aligned}
          \]
          but since \(F^{ji} = \epsilon^{ijk}B^k\) and \(\partial_0 F^{0i} = - \frac{\partial E^i}{\partial t}\), we can write
          \[
              - \frac{\partial E^i}{\partial t} + \partial_j \epsilon^{ijk}B^k = \left(- \frac{\partial \mathbf{E}}{\partial t} + \nabla \times \mathbf{B}\right)^i = J^i,
          \]
          we have recovered equation \eqref{eq:Maxwell_equations}\(_4\):
          \[
              - \frac{\partial \mathbf{E}}{\partial t} + \nabla \times \mathbf{B} = \mathbf{j}.
          \]
\end{itemize}
These are exactly the \textit{inhomogeneous} Maxwell equations for a free field coupled with a source \(J^{\mu}\) in covariant form.

\paragraph{Bianchi identity.}
Together with the antisymmetric definition of \(F_{\mu\nu}\), the cyclic identity
\[
    \partial_{[\lambda}F_{\mu\nu]} \equiv
    \partial_\lambda F_{\mu\nu} +
    \partial_\mu F_{\nu\lambda} +
    \partial_\nu F_{\lambda\mu} = 0
\]
is automatically satisfied.
This is known as the \textit{Bianchi identity}, and when expanded in components it reproduces the homogeneous Maxwell equations in \eqref{eq:Maxwell_equations}\(_{1,2}\).
Hence, the Lagrangian formalism not only yields the inhomogeneous equations of motion but also encodes the homogeneous ones as geometric identities following from the antisymmetry of \(F_{\mu\nu}\).

\paragraph{Gauge fixing and wave equation.}
If one imposes the Lorenz gauge condition \(\partial_{\mu}A^{\mu}=0\), the inhomogeneous equations simplify because the term proportional to \(\partial^{\sigma}\partial_{\rho}A^{\rho}\) vanishes. In this gauge the equations of motion reduce to the manifestly relativistic, decoupled wave equations for the components of the potential:
\[
    \square A^{\nu} \equiv \partial_{\mu}\partial^{\mu}A^{\nu} = J^{\nu}.
\]
This form is convenient both for solving classical radiation problems and for quantization, since each component \(A^\nu\) satisfies a standard sourced wave equation. Gauge fixing removes some redundancies associated with gauge transformations among descriptions of equivalent systems, therefore eliminating non-physical degrees of freedom. We will see in section \ref{sec:gauge_fixing_em_field} that after imposing the Lorentz gauge condition on the EM field, considerations can be made on the system in order to add restriction on the residual gauge freedom, removing effectively further unphysical components to isolate the two transverse physical polarizations of the photon.

\subsection{Energy--momentum tensor}

For the free electromagnetic field (no external sources), the Lagrangian becomes
\[
    \mathcal{L} = -\frac{1}{4} F_{\mu \nu}F^{\mu \nu}.
\]
Using the canonical expression \eqref{eq:energy_momentum_tensor} with \(A^{\mu}\) as the dynamical variable, we compute the canonical energy--momentum tensor:
\[
    \begin{aligned}
        T^{\mu \nu}
         & = \frac{\partial \mathcal{L}}{\partial (\partial_{\mu}A_{\rho})}\,\partial^{\nu}A_\rho - \eta^{\mu \nu}\mathcal{L} \\
         & = - \partial^{\mu}A^{\rho}\,\partial^{\nu} A_{\rho}
        + \eta^{\mu \rho}(\partial_\sigma A^{\sigma})\,\partial^{\nu}A_{\rho}
        + \eta^{\mu \nu}\frac{1}{4} F_{\lambda \sigma}F^{\lambda \sigma}                                                      \\
         & = - \partial^{\mu}A^{\rho}\,\partial^{\nu} A_{\rho}
        + \underbrace{\partial^{\nu}A^{\mu}(\partial_\sigma A^{\sigma})}_{\text{not symmetric under } \mu\leftrightarrow\nu}
        + \frac{1}{4}\eta^{\mu \nu}F_{\lambda \sigma}F^{\lambda \sigma}.
    \end{aligned}
\]
So the energy-momentum tensor is not symmetric for exchange of \(\mu\) and \(\nu\), but is that a problem?

\paragraph{Einstein field equation.}
In general relativity the dynamics of spacetime itself are determined by the distribution of energy and momentum of matter and fields through the celebrated \textit{Einstein field equation}:
\begin{equation}
    R_{\mu \nu} - \frac{1}{2} R\, g_{\mu \nu} + \Lambda g_{\mu \nu}
    = \frac{8 \pi G}{c^4}\, T_{\mu \nu},
    \label{eq:einstein_field_equation}
\end{equation}
where:
\begin{itemize}
    \item \(R_{\mu \nu}\) is the \textit{Ricci tensor}, obtained by contracting the Riemann curvature tensor, and it is symmetric under the exchange \(\mu \leftrightarrow \nu\);
    \item \(R = g^{\rho \sigma} R_{\rho \sigma}\) is the \textit{Ricci scalar}, which measures the curvature of spacetime as a whole;
    \item \(g_{\mu \nu}\) is the metric tensor of the curved spacetime, encoding its geometric structure;
    \item \(\Lambda\) is the \textit{cosmological constant}, originally introduced by Einstein to allow for a static universe, now interpreted as the energy density of the vacuum;
    \item \(G\) is Newton’s gravitational constant and \(c\) the speed of light in vacuum.
\end{itemize}
The left-hand side represents the purely geometric content of spacetime curvature, while the right-hand side contains the \textit{energy--momentum tensor} \(T_{\mu\nu}\), describing the distribution of matter and energy that act as the source of gravity.
For consistency, since the Einstein tensor \(G_{\mu\nu} = R_{\mu\nu} - \tfrac{1}{2}R\,g_{\mu\nu}\) is symmetric, the energy--momentum tensor must also satisfy \(T_{\mu\nu} = T_{\nu\mu}\).

\paragraph{Belinfante-Rosenfeld tensor.} In order to make \(T^{\mu \nu}\) symmetric for exchange of indices, the idea is to redefine it as a sum of two contributions:
\[
    \tilde{T}^{\mu \nu} = T^{\mu \nu} + \partial_{\lambda}\Xi^{\lambda \mu \nu},
\]
where \(\Xi^{\lambda \mu \nu}\) has to be a functions of the fields antysimmetric in the first two indices
\[
    \Xi^{\lambda \mu \nu} = -\Xi^{\mu \lambda \nu},
\]
so that we can preserve the condition \(\partial_{\mu}T^{\mu \nu} = 0\) even for the new \(\tilde{T}^{\mu \nu}\):
\[
    \partial_{\mu} \tilde{T}^{\mu \nu} = \partial_\mu T^{\mu \nu} + \partial_\mu \partial_{\lambda} \Xi^{\lambda \mu \nu} = 0,
\]
since the first term is already zero, and the second is the product of a symmetric quantity for an antisymmetric one in \(\mu \leftrightarrow \nu\).
In the EM case, this function can be defined as
\[
    \Xi^{\lambda \mu \nu} = F^{\mu \lambda}A^{\nu},
\]
since \( F^{\mu \lambda} = - F^{\lambda \mu} \), thus
\[
    \partial_{\lambda} F^{\mu \lambda} A^{\nu} = (\partial_{\lambda} F^{\mu \lambda})A^{\nu} + F^{\mu \lambda} (\partial_{\lambda}A^{\nu}),
\]
but since \(J^{\mu} = 0 = \partial_\lambda F^{\mu \lambda}\), then only the second term survives.
We should now recast the expression of the energy-momentum tensor in the new form (full computation in appendix \ref{app:computations})
\[
    T^{\mu \nu} =-F^{\mu \lambda} (\partial^{\nu}A_{\lambda}) + \frac{1}{4} \eta^{\mu\nu}\left(F_{\rho \sigma}F^{\rho \sigma}\right),
\]
so that the new tensor take the following form:
\[
    \begin{aligned}
        \tilde{T}^{\mu \nu} & =-F^{\mu \lambda} (\partial^{\nu}A_{\lambda}) +\frac{1}{4} \eta^{\mu\nu}\left(F_{\rho \sigma}F^{\rho \sigma}\right) + F^{\mu \lambda} (\partial_{\lambda}A^{\nu}) \\
                            & = F^{\mu \lambda} \left(\partial_{\lambda}A^{\nu} - \partial^{\nu}A_{\lambda}\right) +\frac{1}{4} \eta^{\mu\nu}\left(F_{\rho \sigma}F^{\rho \sigma}\right)        \\
                            & = F^{\mu \lambda} F_{\lambda}^{\ \nu} +\frac{1}{4} \eta^{\mu\nu}\left(F_{\rho \sigma}F^{\rho \sigma}\right),
    \end{aligned}
\]
which is now symmetric under exchange \(\mu \leftrightarrow \nu\). We notice indeed how \(\eta^{\mu \nu}\) is symmetric, while
\[
    F^{\mu \lambda} F_{\lambda}^{\ \nu} = F^{\mu}_{\ \lambda} F^{\lambda \nu} = (- F^{\ \mu}_{\lambda}) (- F^{\nu \lambda}) = F^{\nu \lambda} F^{\ \mu}_{\lambda}.
\]
This is the true form of the energy-momentum tensor which appears in Einstein's field equation, and it is called the \textbf{Belinfante-Rosenfeld tensor}:
\begin{equation}
    \tilde{T}^{\mu \nu} = F^{\mu \lambda} F_{\lambda}^{\ \nu} +\frac{1}{4} \eta^{\mu\nu}\left(F_{\rho \sigma}F^{\rho \sigma}\right).
    \label{eq:Belinfante-Rosenfeld_tensor}
\end{equation}

\paragraph{Noether current and charges.}
Since we can identify our Belinfante-Rosenfeld tensor as the Noether current, we know it contains 4 conserved charges. We expect the \textit{energy density} \(\mathcal{H}\) to be the first consered quantity:
\[
    \mathcal{H} = \tilde{T}^{00}, \quad H = \int \mathrm{d}^3 \mathbf{x} \, \mathcal{H} = \int \mathrm{d}^3 \mathbf{x} \, \tilde{T}^{00};
\]
we can compute the anergy density by computing the right term of \eqref{eq:Belinfante-Rosenfeld_tensor}:
\[
    \begin{aligned}
        \mathcal{H} & = \tilde{T}^{00} = F^{0 i} F_{i}^{\ 0} +\frac{1}{4} \eta^{00}\left(F_{0i}F^{0i} + F_{i0}F^{i0} + F_{ij}F^{ij}\right)                   \\
                    & = F^{0 i} F_{i}^{\ 0} + \frac{1}{2} F_{0i}F^{0i} + \frac{1}{4}F_{ij}F^{ij}                                                             \\
                    & = \eta_{i j}F^{0 i}F^{j0} + \frac{1}{2}\eta_{ij}F^{0j}F^{0i} +\frac{1}{4} \eta_{ia} \eta_{jb} F^{ab}F^{ij}                             \\
                    & = \eta_{i j} \left(\frac{1}{2}F^{0j}F^{0i} - F^{0 i}F^{0j}\right) + \frac{1}{4} \eta_{ia} \eta_{jb} F^{ab}F^{ij}                       \\
                    & = \frac{1}{2} F^{0i}F^{0i} + \frac{1}{4} F^{ij}F^{ij} = \frac{1}{2}\left(\vert \mathbf{E} \vert^2 + \vert \mathbf{B} \vert^2  \right),
    \end{aligned}
\]
where we have used the fact that for raising or lowering time-like indices the sign does not change, while \(\eta^{ij} = -\delta^{ij}\) on the space-like indices. We have therefore found the energy density. For the \textit{momentum density} \(\mathcal{P}\) we have:
\[
    \mathcal{P}^i = \tilde{T}^{0i}, \quad P^i = \int \mathrm{d}^3 \mathbf{x} \, \mathcal{P}^i = \int \mathrm{d}^3 \mathbf{x} \, \tilde{T}^{0i};
\]
we can compute it again from \eqref{eq:Belinfante-Rosenfeld_tensor} as follows
\[
    \begin{aligned}
        \mathcal{P}^i & = \tilde{T}^{0i} = F^{0 j} F_{j}^{\ i} +\frac{1}{4} \eta^{0i}\left(F_{\mu \nu} F^{\mu \nu}\right) = F^{0 j} F_{j}^{\ i} \\
                      & = - F^{j 0} (- F^{ji}) = (-E^j)(- \epsilon^{ijk} B^k) = (\mathbf{E} \times \mathbf{B})^i,
    \end{aligned}
\]
since \(\eta^{0i}=0\) and \(F_j^{\ i} = - F^{ji}\). Therefore we found the momentum density.