\subsection{Solutions of the Dirac Equation}

Each component of the dirac spinor \(\psi(x)\) satisfies the KG equation, so we can look for plane wave solutions of the form
\[
    \psi_{\alpha}(x) = u_{\alpha}(\mathbf{p}) e^{-i p_\mu x^{\mu}} = u_{\alpha}(\mathbf{p}) e^{-i (E_{\mathbf{p}} t - \mathbf{p} \cdot \mathbf{x})}, \quad E_{\mathbf{p}} = \sqrt{\mathbf{p}^2 + m^2},
\]
where \(\alpha = 1,\,2,\,3,\,4\) and \(u_{\alpha}(\mathbf{p})\) are complex coefficients depending on the momentum \(\mathbf{p}\) (it's a 4-component vector). Plugging this ansatz into the Dirac equation we get
\[
    (i \gamma^{\mu} \partial_{\mu} - m) \psi(x) = 0, \quad \implies \quad (\gamma^{\mu} p_{\mu} - m) u(\mathbf{p}) = 0.
\]
which in spinorial representation reads
\[
    \left[ \begin{pmatrix}
            0 & 1 \\
            1 & 0
        \end{pmatrix} p_0 + \begin{pmatrix}
            0         & \sigma^i \\
            -\sigma^i & 0
        \end{pmatrix} p_i - m \begin{pmatrix}
            1 & 0 \\
            0 & 1
        \end{pmatrix}\right] u(\mathbf{p}) = \begin{pmatrix}
        -m                         & p^{\mu} \sigma_{\mu} \\
        p^{\mu} \bar{\sigma}_{\mu} & -m
    \end{pmatrix} u(\mathbf{p}) = 0,
\]
where we have defined \(\sigma^{\mu} = (\mathbb{I},\,\sigma^i)\) and \(\bar{\sigma}^{\mu} = (\mathbb{I},\,-\sigma^i)\). Writing the spinor \(u(\mathbf{p})\) in terms of its two-component chiral parts
\[
    u(\mathbf{p}) = \begin{pmatrix}
        u_L(\mathbf{p}) \\
        u_R(\mathbf{p})
    \end{pmatrix},
\]
we get the system of equations
\[
    \begin{dcases}
        (p^{\mu} \sigma_{\mu}) u_R(\mathbf{p}) = m u_L(\mathbf{p}), \\
        (p^{\mu} \bar{\sigma}_{\mu}) u_L(\mathbf{p}) = m u_R(\mathbf{p}) .
    \end{dcases}
\]
It's a system of two coupled equations for the two chiral components of the spinor. We can solve for one component in terms of the other; for example, solving for \(u_R(\mathbf{p})\) from the second equation and plugging it into the first, we get
\[
    (p^{\mu} \sigma_{\mu})(p^{\nu} \bar{\sigma}_{\nu}) = (p_0 + p_i \sigma^i)(p_0 - p_j \sigma^j) = p_0^2 - p_i p_j \sigma^i \sigma^j = p_0^2 - \mathbf{p}^2 = m^2,
\]
where we have used the algebra of the Pauli matrices \(\sigma^i \sigma^j = \delta^{ij} + i \epsilon^{ijk} \sigma^k\)\footnote{The term proportional to \(\epsilon^{ijk}\sigma^k\) goes to zero due to antisymmetry when contracted with symmetric indices.} and we have found the relativistic dispersion relation \(p_0^2 - \mathbf{p}^2 = m^2\). Thus we can write
\[
    \begin{aligned}
        u_L(\mathbf{p})                              & = A (p^{\mu} \sigma_{\mu}) \chi,                                                               \\
        (p^{\mu} \bar{\sigma}_{\mu}) u_L(\mathbf{p}) & = (p^{\mu} \bar{\sigma}_{\mu}) (p^{\mu} \sigma_{\mu}) A \chi = m^2 A \chi = m u_R(\mathbf{p}), \\
        \implies u_R(\mathbf{p})                     & = m A \chi,
    \end{aligned}
\]
and getting for \(u_L(\mathbf{p})\):
\[
    \begin{aligned}
        u_L(\mathbf{p}) = (p^\mu \sigma_\mu) u_R(\mathbf{p}) \frac{1}{m} & = (p^\mu \sigma_\mu) (m A \chi) \frac{1}{m} = A (p^\mu \sigma_\mu) \chi, \\
        \implies u_L(\mathbf{p})                                         & = A (p^\mu \sigma_\mu) \chi,
    \end{aligned}
\]
where \(\chi\) is an arbitrary two-component spinors and \(A\) a normalization constant (thus the first equation is automatically satisfied). Thus the general solution for \(u(\mathbf{p})\) can be written as
\[
    u(\mathbf{p}) = A \begin{pmatrix}
        (p^{\mu} \sigma_{\mu}) \chi \\
        m \chi
    \end{pmatrix},
\]
where now the idea is to symmetrize the solution by choosing \(A = \frac{1}{m}\) and \(\chi = \sqrt{p^{\mu} \bar{\sigma}_{\mu}} \xi\), where \(\xi\) is a constant two-component spinor respecting
\[
    \xi^{\dagger} \xi = 1,
\]
so that
\[
    A (p^{\mu} \sigma_{\mu}) \chi = \frac{1}{m} \sqrt{(p^{\mu} \sigma_{\mu})(p^{\nu} \sigma_{\nu})} \sqrt{p^{\mu} \bar{\sigma}_{\mu}} \xi = \sqrt{p^{\mu} \sigma_{\mu}} \xi,
\]
since we have already computed that \((p^{\mu} \sigma_{\mu})(p^{\nu} \bar{\sigma}_{\nu}) = m^2\). Thus we get the final expression for the \textbf{positive frequency solution} of the Dirac equation:
\[
    \begin{dcases}
        u_L(\mathbf{p}) = \sqrt{p^{\mu} \sigma_{\mu}} \xi, \\
        u_R(\mathbf{p}) = \sqrt{p^{\mu} \bar{\sigma}_{\mu}} \xi,
    \end{dcases} \implies u(\mathbf{p}) = \begin{pmatrix}
        \sqrt{p^{\mu} \sigma_{\mu}} \xi \\
        \sqrt{p^{\mu} \bar{\sigma}_{\mu}} \xi
    \end{pmatrix},
\]
with plane wave solution in the final form
\begin{equation}
    \psi(x) = u(\mathbf{p}) e^{-i p_{\mu} x^{\mu}} = \begin{pmatrix}
        \sqrt{p^{\mu} \sigma_{\mu}} \xi \\
        \sqrt{p^{\mu} \bar{\sigma}_{\mu}} \xi
    \end{pmatrix} e^{-i (E_{\mathbf{p}} t - \mathbf{p} \cdot \mathbf{x})}.
    \label{eq:positive_frequency_dirac_plane_wave}
\end{equation}

We can also find \textbf{negative frequency solutions} of the Dirac equation by considering
\[
    \psi(x) = v(\mathbf{p}) e^{i p_{\mu} x^{\mu}},
\]
where \(v(\mathbf{p})\) satisfies
\[
    (\gamma^{\mu} p_{\mu} + m) v(\mathbf{p}) = \begin{pmatrix}
        m                          & p^{\mu} \sigma_{\mu} \\
        p^{\mu} \bar{\sigma}_{\mu} & m
    \end{pmatrix} v(\mathbf{p}) = 0,
\]
and following the same steps as before we get
\[
    v(\mathbf{p}) = \begin{pmatrix}
        \sqrt{p^{\mu} \sigma_{\mu}} \eta \\
        -\sqrt{p^{\mu} \bar{\sigma}_{\mu}} \eta
    \end{pmatrix}, \quad \eta^{\dagger} \eta = 1,
\]
with plane wave solution
\begin{equation}
    \psi(x) = v(\mathbf{p}) e^{i p_{\mu} x^{\mu}} = \begin{pmatrix}
        \sqrt{p^{\mu} \sigma_{\mu}} \eta \\
        -\sqrt{p^{\mu} \bar{\sigma}_{\mu}} \eta
    \end{pmatrix} e^{i (E_{\mathbf{p}} t - \mathbf{p} \cdot \mathbf{x})}.
    \label{eq:negative_frequency_dirac_plane_wave}
\end{equation}

If we were to apply the Hamiltonian on these ansatzs we would get
\[
    \hat{H} \psi(x) = i \partial_t (u(\mathbf{p}) e^{-i p_{\mu} x^{\mu}}) = E_{\mathbf{p}} (u(\mathbf{p}) e^{-i p_{\mu} x^{\mu}}) = + E_{\mathbf{p}} \psi(x),
\]
\[
    \hat{H} \psi(x) = i \partial_t (v(\mathbf{p}) e^{i p_{\mu} x^{\mu}}) = - E_{\mathbf{p}} (v(\mathbf{p}) e^{i p_{\mu} x^{\mu}}) = - E_{\mathbf{p}} \psi(x).
\]
Thus \(u(\mathbf{p})\) are positive energy solutions while \(v(\mathbf{p})\) are negative energy solutions of the Dirac equation. This problem of negative energy solutions will be solved in QFT interpreting them as antiparticles, but for now \(\psi(x)\) is just a classical field; after quantization \(\hat{\psi}(x)\) will be an operator acting on the Fock space and we will see how to interpret these solutions. Both particles and antiparticles will be needed to build a consistent quantum theory, with positive definite energy.

\begin{example}[Particle rest frame and spinor transformations]
    In the rest frame of a massive particle
    \[
        \mathbf{p} = 0, \quad E_{\mathbf{p}} = p^0 = m.
    \]
    Thus the positive frequency solutions read
    \[
        u(\mathbf{0}) = \begin{pmatrix}
            \sqrt{p^{\mu} \sigma_\mu} \xi \\
            \sqrt{p^{\mu} \bar{\sigma}_\mu} \xi
        \end{pmatrix} = \begin{pmatrix}
            \begin{pmatrix}
                \sqrt{p^0} & 0          \\
                0          & \sqrt{p^0}
            \end{pmatrix}\begin{pmatrix}
                             1 \\
                             0
                         \end{pmatrix} \\
            \begin{pmatrix}
                \sqrt{p^0} & 0          \\
                0          & \sqrt{p^0}
            \end{pmatrix}\begin{pmatrix}
                             1 \\
                             0
                         \end{pmatrix}
        \end{pmatrix} = \sqrt{p^0} \begin{pmatrix}
            \xi \\
            \xi
        \end{pmatrix} = \sqrt{m} \begin{pmatrix}
            \xi \\
            \xi
        \end{pmatrix},
    \]
    so that the plane wave solution is
    \[
        \psi(x) = \sqrt{m} \begin{pmatrix}
            \xi \\
            \xi
        \end{pmatrix} e^{-i m t}.
    \]

    Now recalling the Lorentz transformation for spinors
    \[
        \psi^{\prime \, \alpha}(x^{\prime}) = S(\Lambda)^{\alpha}_{\ \beta} \psi^{\beta}(x), \quad S(\Lambda)^{\alpha}_{\ \beta} = (e^{-\frac{i}{2} \omega_{\mu \nu} \Sigma^{\mu \nu}})^{\alpha}_{\ \beta},
    \]
    where the algebra generators of the Lorentz transformation are
    \[
        S^{\mu \nu} = \frac{1}{4} [\gamma^{\mu},\, \gamma^{\nu}] = i \Sigma^{\mu \nu},
    \]
    with parameters \(\omega_{\mu \nu}\) depending on the boost/rotation we are performing. For a pure spatial rotation we have
    \[
        S^{ij} = -\frac{i}{2} \epsilon_{ijk} \begin{pmatrix}
            \sigma^k & 0        \\
            0        & \sigma^k
        \end{pmatrix}, \quad (i \neq j),
    \]
    after computing the commutator and exploiting the Pauli matrices algebra; here \(k\) is the axis of rotation, and the three parameters of the rotation
    \[
        \omega_{ij} = - \epsilon_{ijk} \theta^k, \quad (i \neq j)
    \]
    are related to the rotation angles around the three spatial axis. Thus the spinor transformation under a spatial rotation reads
    \[
        e^{\frac{1}{2} \omega_{ij} S^{ij}} = \begin{pmatrix}
            e^{i \frac{\theta^k \sigma^k}{2}} & 0                                 \\
            0                                 & e^{i \frac{\theta^k \sigma^k}{2}}
        \end{pmatrix},
    \]
    and if we apply this transformation to the rest-frame spinor we get
    \[
        \psi(x) = \sqrt{m} \begin{pmatrix}
            \xi \\
            \xi
        \end{pmatrix} e^{-i m t} \to  \psi^{\prime}(x^{\prime}) = \begin{pmatrix}
            e^{i \frac{\theta^k \sigma^k}{2}} & 0                                 \\
            0                                 & e^{i \frac{\theta^k \sigma^k}{2}}
        \end{pmatrix} \psi(x) = \sqrt{m} \begin{pmatrix}
            e^{i \frac{\theta^k \sigma^k}{2}} \xi \\
            e^{i \frac{\theta^k \sigma^k}{2}} \xi
        \end{pmatrix} e^{-i m t},
    \]
    so that both chiral components transform in the same way under spatial rotations, as expected since in the rest frame chirality and helicity are not defined
    \[
        \xi \to \xi^{\prime} = e^{i \frac{\theta^k \sigma^k}{2}} \xi.
    \]
    This is the representation of the standard \(\mathrm{SU}(2)\) transformation for spin \(\tfrac12\) objects
    \[
        \mathbf{S} = \frac{\hbar}{2} \bs{\sigma}.
    \]
    We are considering particles at rest with spin \(\tfrac12\), so we can choose the basis where the spin is aligned along the \(z\)-axis:
    \[
        \xi_+ = \begin{pmatrix}
            1 \\
            0
        \end{pmatrix}, \quad \xi_- = \begin{pmatrix}
            0 \\
            1
        \end{pmatrix},
    \]
    which are eigenstates of the spin operator \(S_z = \frac{\hbar}{2} \sigma^3\) with eigenvalues \(\pm \frac{\hbar}{2}\):
    \[
        \sigma_3 \xi_{+} = \begin{pmatrix}
            1 & 0  \\
            0 & -1
        \end{pmatrix} \begin{pmatrix}
            1 \\
            0
        \end{pmatrix} = +1 \begin{pmatrix}
            1 \\
            0
        \end{pmatrix}, \quad \sigma_3 \xi_{-} = -1 \begin{pmatrix}
            0 \\
            1
        \end{pmatrix}.
    \]
\end{example}

\begin{example}[Massless limit and chirality]
    Now consider a particle with spin up along the \(z\)-axis moving with momentum \(\mathbf{p} = (0,\,0,\,p_z)= (0,\,0,\,p)\) along the \(z\)-axis
    \[
        p^{\mu} = (E_{\mathbf{p}},\,0,\,0,\,p), \quad E_{\mathbf{p}} = \sqrt{p^2 + m^2} = E.
    \]
    We have the positive frequency solution as
    \[
        \psi(x) = \begin{pmatrix}
            \sqrt{p^{\mu} \sigma_{\mu}} \xi_+ \\
            \sqrt{p^{\mu} \bar{\sigma}_{\mu}} \xi_+
        \end{pmatrix}e^{-i p_{\mu} x^{\mu}},
    \]
    where computing the square roots we get\footnote{Pay attention that \(p = \vert \mathbf{p} \vert\), thus we have to explicit the minus sign in the spatial part of the four-momentum when contracting with \(\bar{\sigma}^{\mu}\) or \(\sigma^{\mu}\).}
    \[
        \psi(x) = \begin{pmatrix}
            \begin{pmatrix}
                \sqrt{E - p} & 0            \\
                0            & \sqrt{E + p}
            \end{pmatrix}\begin{pmatrix}
                             1 \\
                             0
                         \end{pmatrix} \\
            \begin{pmatrix}
                \sqrt{E + p} & 0            \\
                0            & \sqrt{E - p}
            \end{pmatrix}\begin{pmatrix}
                             1 \\
                             0
                         \end{pmatrix}
        \end{pmatrix} e^{-i (E t - p z)} = \begin{pmatrix}
            \sqrt{E - p} \\
            0            \\
            \sqrt{E + p} \\
            0
        \end{pmatrix} e^{-i (E t - p z)}.
    \]
    where we have a right handed helicity state since both momentum and spin are aligned along the \(z\)-axis, but both chiral components since for massive particles chirality and helicity do not coincide. Now we can consider the massless limit \(m \to 0\), so that \(E = p\) and we get that the left chiral component vanishes
    \[
        \psi(x) = \sqrt{2 p} \begin{pmatrix}
            0 \\
            0 \\
            1 \\
            0
        \end{pmatrix} e^{-i p (t - z)},
    \]
    which is a purely right-handed spinor, as expected since in the massless limit chirality and helicity coincide: we were considering a particle with moentum and spin aligned along the \(z\)-axis, so it must be right-handed both in helicity and chirality. Thus in the massless limit we have
    \[
        \begin{aligned}
            \psi^{(w)}_L (x) & = 0,                        \\
            \psi^{(w)}_R (x) & = \sqrt{2 p} \begin{pmatrix}
                                                1 \\
                                                0
                                            \end{pmatrix}.
        \end{aligned}
    \]
\end{example}

\subsection{Useful Formulae}

We want to introduce some useful formulae for manipulating Dirac spinors and gamma matrices, in order to comprehend better the structure of the quantum theory. We will focus on the positive and negative frequency solutions \(u(\mathbf{p})\) and \(v(\mathbf{p})\) of the Dirac equation defined in equations \eqref{eq:positive_frequency_dirac_plane_wave} and \eqref{eq:negative_frequency_dirac_plane_wave}, which can be decomposed in a chosen basis of two-component spinors \(\xi\) and \(\eta\):
\[
    \xi^{n \, \dagger} \xi^m = \delta^{nm}, \quad \eta^{n \, \dagger} \eta^m = \delta^{nm},
\]
with trivial example
\[
    \xi^1 = \begin{pmatrix}
        1 \\
        0
    \end{pmatrix}, \quad \xi^2 = \begin{pmatrix}
        0 \\
        1
    \end{pmatrix}.
\]
Thus we have defined a basis \(\xi^n\) and \(\eta^n\) for the two-component spinors to build the positive and negative frequency solutions of the Dirac equation (with \(n = 1,\,2\)).

\paragraph{Inner products.}
We can label solutions of the Dirac equation with this index, in order to distinguish the two different spin states; thus we can start computing inner products between these solutions. For example for the positive frequency solutions we have
\[
    u^{n\,\dagger}(\mathbf{p}) u^m(\mathbf{p}) = \begin{pmatrix}
        \xi^{n\, \dagger} \sqrt{p^{\mu} \sigma_\mu} & \xi^{n\, \dagger} \sqrt{p^{\mu} \bar{\sigma}_\mu}
    \end{pmatrix} \begin{pmatrix}
        \sqrt{p^{\mu} \sigma_\mu} \xi^m \\
        \sqrt{p^{\mu} \bar{\sigma}_\mu} \xi^m
    \end{pmatrix}
\]
which can be computed as
\[
    \begin{aligned}
        u^{n\,\dagger}(\mathbf{p}) u^m(\mathbf{p}) & = \xi^{n\, \dagger} (p^{\mu} \sigma_\mu) \xi^m + \xi^{n\, \dagger} (p^{\mu} \bar{\sigma}_\mu) \xi^m         \\
                                                   & = 2 p^0 \xi^{n\, \dagger} \xi^m + p^i \sigma_i \xi^{n\,\dagger} \xi^m - p^i \sigma_i \xi^{n\,\dagger} \xi^m \\
                                                   & = 2 p^0 \xi^{n\, \dagger} \xi^m = 2 E_{\mathbf{p}} \delta^{nm},
    \end{aligned}
\]
which is not a Lorentz invariant quantity since it involves the time component of the four-momentum \(p^0 = E_{\mathbf{p}}\), but it will prove useful in the quantization procedure.

Instead if we have
\[
    \overline{u}^{n}(\mathbf{p}) u^m(\mathbf{p}) = \begin{pmatrix}
        \xi^{n\, \dagger} \sqrt{p^{\mu} \sigma_\mu} & \xi^{n\, \dagger} \sqrt{p^{\mu} \bar{\sigma}_\mu}
    \end{pmatrix} \begin{pmatrix}
        0 & 1 \\
        1 & 0
    \end{pmatrix} \begin{pmatrix}
        \sqrt{p^{\mu} \sigma_\mu} \xi^m \\
        \sqrt{p^{\mu} \bar{\sigma}_\mu} \xi^m
    \end{pmatrix},
\]
where we know the definition for the Dirac conjugate \(\overline{u}^n = u^{n\,\dagger} \gamma^0\), we get
\[
    \begin{aligned}
        u^{n\,\dagger}(\mathbf{p}) \gamma^0 u^m(\mathbf{p}) & = \xi^{n\, \dagger} \sqrt{p^{\mu} \sigma_\mu}\sqrt{p^{\mu} \bar{\sigma}_\mu}  \xi^m + \xi^{n\, \dagger} \sqrt{p^{\mu} \bar{\sigma}_\mu}\sqrt{p^{\mu} \sigma_\mu} \xi^m \\
                                                            & = \xi^{n\, \dagger} \sqrt{m^2}  \xi^m + \xi^{n\, \dagger} \sqrt{m^2} \xi^m                                                                                             \\
                                                            & = 2 m \xi^{n\, \dagger} \xi^m = 2 m \delta^{nm},
    \end{aligned}
\]
which is instead a Lorentz invariant quantity since it depends only on the mass \(m\) of the particle. Now we can summarize these two results as
\begin{equation}
    \begin{dcases}
        u^{n\,\dagger}(\mathbf{p}) u^m(\mathbf{p}) = 2 E_{\mathbf{p}} \delta^{nm}, \\
        \overline{u}^n (\mathbf{p}) u^m(\mathbf{p}) = 2 m \delta^{nm}.
    \end{dcases}
    \label{eq:Dirac_inner_products_positive_frequency}
\end{equation}

Similarly for the negative frequency solutions we have
\begin{equation}
    \begin{dcases}
        v^{n\,\dagger}(\mathbf{p}) v^m(\mathbf{p}) = 2 E_{\mathbf{p}} \delta^{nm}, \\
        \overline{v}^n (\mathbf{p}) v^m(\mathbf{p}) = -2 m \delta^{nm}.
    \end{dcases}
    \label{eq:Dirac_inner_products_negative_frequency}
\end{equation}
Indeed we can compute explicitly
\[
    \begin{aligned}
        v^{n\,\dagger}(\mathbf{p}) v^m(\mathbf{p}) & = \eta^{n\, \dagger} (p^{\mu} \sigma_\mu) \eta^m + \eta^{n\, \dagger} (p^{\mu} \bar{\sigma}_\mu) \eta^m           \\
                                                   & = 2 p^0 \eta^{n\, \dagger} \eta^m + p^i \sigma_i \eta^{n\,\dagger} \eta^m - p^i \sigma_i \eta^{n\,\dagger} \eta^m \\
                                                   & = 2 p^0 \eta^{n\, \dagger} \eta^m = 2 E_{\mathbf{p}} \delta^{nm},
    \end{aligned}
\]
and
\[
    \begin{aligned}
        \overline{v}^{n}(\mathbf{p}) v^m(\mathbf{p}) & = \eta^{n\, \dagger} \sqrt{p^{\mu} \sigma_\mu} (-\sqrt{p^{\mu} \bar{\sigma}_\mu})  \eta^m + \eta^{n\, \dagger} (-\sqrt{p^{\mu} \bar{\sigma}_\mu})\sqrt{p^{\mu} \sigma_\mu} \eta^m \\
                                                     & = - \eta^{n\, \dagger} \sqrt{m^2}  \eta^m - \eta^{n\, \dagger} \sqrt{m^2} \eta^m                                                                                                  \\
                                                     & = - 2 m \eta^{n\, \dagger} \eta^m = - 2 m \delta^{nm}.
    \end{aligned}
\]

Now we can compute the mixed products
\[
    \begin{aligned}
        \overline{u}^n (\mathbf{p}) v^m(\mathbf{p}) = & \begin{pmatrix}
                                                            \xi^{n\, \dagger} \sqrt{p^{\mu} \sigma_\mu} & \xi^{n\, \dagger} \sqrt{p^{\mu} \bar{\sigma}_\mu}
                                                        \end{pmatrix} \begin{pmatrix}
                                                                          0 & 1 \\
                                                                          1 & 0
                                                                      \end{pmatrix} \begin{pmatrix}
                                                                                        \sqrt{p^{\mu} \sigma_\mu} \eta^m \\
                                                                                        -\sqrt{p^{\mu} \bar{\sigma}_\mu} \eta^m
                                                                                    \end{pmatrix} =                     \\
        =                                             & \xi^{n\, \dagger} \sqrt{p^{\mu} \sigma_\mu} (-\sqrt{p^{\mu} \bar{\sigma}_\mu}) \eta^m + \xi^{n\, \dagger} \sqrt{p^{\mu} \bar{\sigma}_\mu} \sqrt{p^{\mu} \sigma_\mu} \eta^m = \\
        =                                             & - \xi^{n\, \dagger} \sqrt{m^2}  \eta^m + \xi^{n\, \dagger} \sqrt{m^2} \eta^m = 0,
    \end{aligned}
\]
and for the other combination (with the same cancellations) we get to the same result, thus we can write
\begin{equation}
    \begin{dcases}
        \overline{u}^n (\mathbf{p}) v^m(\mathbf{p}) = 0, \\
        \overline{v}^n (\mathbf{p}) u^m(\mathbf{p}) = 0.
    \end{dcases}
    \label{eq:Dirac_ mixed_inner_products}
\end{equation}
Thus we derived the \textbf{orthogonality relations}, which ensure that positive and negative frequency solutions are orthogonal to each other in the Dirac inner product sense.

We can also compute the mixed inner products with opposite momentum but without the Dirac conjugate, which will be useful in the quantization procedure:
\[
    \begin{aligned}
        u^{n\,\dagger}(\mathbf{p}) v^m(-\mathbf{p}) & = \begin{pmatrix}
                                                            \xi^{n\, \dagger} \sqrt{p^{\mu} \sigma_\mu} & \xi^{n\, \dagger} \sqrt{p^{\mu} \bar{\sigma}_\mu}
                                                        \end{pmatrix} \begin{pmatrix}
                                                                          \sqrt{p^{\prime\mu} \sigma_\mu} \eta^m \\
                                                                          -\sqrt{p^{\prime\mu} \bar{\sigma}_\mu} \eta^m
                                                                      \end{pmatrix}            \\
                                                    & = \xi^{n\, \dagger} \left( \sqrt{p^{\mu}\sigma_{\mu} p^{\prime \nu} \sigma_{\nu}} - \sqrt{p^{\prime \mu}\bar{\sigma}_{\mu} p^{\nu} \bar{\sigma}_{\nu}} \right) \eta^m \\
                                                    & = \xi^{n\, \dagger} \left( \sqrt{p_0^2 - p_i p_j \sigma^i \sigma^j} - \sqrt{p_0^2 - p_i p_j \sigma^i \sigma^j} \right) \eta^m                                         \\
                                                    & = \xi^{n\, \dagger} \left( m - m \right) \eta^m = 0,
    \end{aligned},
\]
and similarly for the other combination, so that we have
\begin{equation}
    \begin{dcases}
        u^{n\,\dagger}(\mathbf{p}) v^m(-\mathbf{p}) = 0, \\
        v^{n\,\dagger}(\mathbf{p}) u^m(-\mathbf{p}) = 0.
    \end{dcases}
    \label{eq:Dirac_mixed_inner_products_opposite_momentum}
\end{equation}

\paragraph{Outer products.}
We can compute the outer products of the spinors, starting from the positive frequency solutions:
\[
    \sum_{n} u^n(\mathbf{p}) \overline{u}^n(\mathbf{p}) = \sum_n \begin{pmatrix}
        \sqrt{p^{\mu} \sigma_\mu} \xi^n \\
        \sqrt{p^{\mu} \bar{\sigma}_\mu} \xi^n
    \end{pmatrix} \begin{pmatrix}
        \xi^{n\, \dagger} \sqrt{p^{\mu} \sigma_\mu} & \xi^{n\, \dagger} \sqrt{p^{\mu} \bar{\sigma}_\mu}
    \end{pmatrix} \begin{pmatrix}
        0 & 1 \\
        1 & 0
    \end{pmatrix},
\]
where we have used the definition of \(\overline{u}^n\). Computing the sum over the basis we get
\[
    \sum_{n} u^n(\mathbf{p}) \overline{u}^n(\mathbf{p}) = \sum_{n} \begin{pmatrix}
        \sqrt{p^{\mu} \sigma_\mu} \xi^n \xi^{n\, \dagger} \sqrt{p^{\mu} \overline{\sigma}_\mu}            & \sqrt{p^{\mu} \sigma_\mu} \xi^n \xi^{n\, \dagger} \sqrt{p^{\mu} \sigma_\mu}            \\
        \sqrt{p^{\mu} \overline{\sigma}_\mu} \xi^n \xi^{n\, \dagger} \sqrt{p^{\mu} \overline{\sigma}_\mu} & \sqrt{p^{\mu} \overline{\sigma}_\mu} \xi^n \xi^{n\, \dagger} \sqrt{p^{\mu} \sigma_\mu}
    \end{pmatrix}.
\]
Now if we use the completeness relation for the basis
\[
    \sum_{n} \xi^n \xi^{n\, \dagger} = \begin{pmatrix}
        1 \\
        0
    \end{pmatrix} \begin{pmatrix}
        1 & 0
    \end{pmatrix} + \begin{pmatrix}
        0 \\
        1
    \end{pmatrix} \begin{pmatrix}
        0 & 1
    \end{pmatrix} = \begin{pmatrix}
        1 & 0 \\
        0 & 1
    \end{pmatrix} = \mathbb{I}_2,
\]
then we find
\[
    \sum_{n} u^n(\mathbf{p}) \overline{u}^n(\mathbf{p}) = \begin{pmatrix}
        \sqrt{(p \sigma)(p \overline{\sigma})} & (p \sigma)                             \\
        (p \overline{\sigma})                  & \sqrt{(p \overline{\sigma})(p \sigma)}
    \end{pmatrix} = \begin{pmatrix}
        m                          & p^{\mu} \sigma_{\mu} \\
        p^{\mu} \bar{\sigma}_{\mu} & m
    \end{pmatrix} = \gamma^{\mu} p_{\mu} + m \mathbb{I}_4.
\]
Similarly for the negative frequency solutions we have
\[
    \begin{aligned}
        \sum_{n} v^n(\mathbf{p}) \overline{v}^n(\mathbf{p}) & = \sum_n \begin{pmatrix}
                                                                           \sqrt{p^{\mu} \sigma_\mu} \eta^n \\
                                                                           -\sqrt{p^{\mu} \bar{\sigma}_\mu} \eta^n
                                                                       \end{pmatrix} \begin{pmatrix}
                                                                                         \eta^{n\, \dagger} \sqrt{p^{\mu} \sigma_\mu} & -\eta^{n\, \dagger} \sqrt{p^{\mu} \bar{\sigma}_\mu}
                                                                                     \end{pmatrix} \begin{pmatrix}
                                                                                                       0 & 1 \\
                                                                                                       1 & 0
                                                                                                   \end{pmatrix}                                                                                                                                        \\
                                                            & = \sum_{n} \begin{pmatrix}
                                                                             \sqrt{p^{\mu} \sigma_\mu} \eta^n \eta^{n\, \dagger} \sqrt{p^{\mu} \sigma_\mu}        & -\sqrt{p^{\mu} \sigma_\mu} \eta^n \eta^{n\, \dagger} \sqrt{p^{\mu} \bar{\sigma}_\mu}       \\
                                                                             -\sqrt{p^{\mu} \bar{\sigma}_\mu} \eta^n \eta^{n\, \dagger} \sqrt{p^{\mu} \sigma_\mu} & +\sqrt{p^{\mu} \bar{\sigma}_\mu} \eta^n \eta^{n\, \dagger} \sqrt{p^{\mu} \bar{\sigma}_\mu}
                                                                         \end{pmatrix} \begin{pmatrix}
                                                                                           0 & 1 \\
                                                                                           1 & 0
                                                                                       \end{pmatrix} \\
                                                            & = \begin{pmatrix}
                                                                    -\sqrt{(p \sigma)(p \overline{\sigma})} & (p \sigma)                              \\
                                                                    (p \overline{\sigma})                   & -\sqrt{(p \overline{\sigma})(p \sigma)}
                                                                \end{pmatrix} = \begin{pmatrix}
                                                                                    -m                         & p^{\mu} \sigma_{\mu} \\
                                                                                    p^{\mu} \bar{\sigma}_{\mu} & -m
                                                                                \end{pmatrix} = \gamma^{\mu} p_{\mu} - m \mathbb{I}_4.
    \end{aligned}
\]
Thus we can summarize these results as
\begin{equation}
    \begin{dcases}
        \sum_{n} u^n(\mathbf{p}) \overline{u}^n(\mathbf{p}) = \gamma^{\mu} p_{\mu} + m \mathbb{I}_4, \\
        \sum_{n} v^n(\mathbf{p}) \overline{v}^n(\mathbf{p}) = \gamma^{\mu} p_{\mu} - m \mathbb{I}_4,
    \end{dcases}
    \label{eq:Dirac_outer_products}
\end{equation}
remembering that the result is a \(4 \times 4\) matrix.