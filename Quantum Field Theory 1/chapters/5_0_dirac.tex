\chapter{Dirac theory}

The starting point is the lagrangian, then we show the most general solution to the equations of motion, then we proceed to quantize the system and promote the observables to operators on the Fock Space.
The lagrangian has to be Lorentz invariant.
Lorentz group admits a spinor representation, which for dirac translates In
\[
    \psi_D \to \dots
\]
and \(\Sigma_{\mu \nu} = \dots \) where
\[
    \gamma^{\mu \nu} \dots
\]
repects the clifford algebra \(\dots \) and are \(4\times 4\) complex matrices.
We adopt the Weyl (or chiral) representation of dirac matrices
\[
    \dots
\]
Sometimes the generators can be expressed As
\[
    S_{\mu \nu}\dots
\]
or by introducing \textbf{spinorial indices}:
\[
    \dots
\]
where we have Lorentz transformed the coordinates too.

\section{Action and Lagrangian}

We want to construct a lorentz invariant lagrangian dependent upon \(\psi\)
\[
    \psi^{\dagger} \dots
\]
The first attempt i to consider the spinor bilinear \(\psi^{\dagger} \psi\) as a building block; it's going to be a real number, but we have to check if it is Lorentz invariant.

\subsection{Lorentz scalar as a building block}

We have seen that
\[
    \psi \to  \psi^{\prime} = S \psi, \quad \psi^{\dagger} \to  \psi^{\dagger \prime} = S^{\dagger} \psi^{\dagger},
\]
and the product
\[
    \dots
\]
is not a lorentz scalar, it would require \(S^{\dagger} = S^{-1}\), which is not the case, since this representation is not unitary
\[
    \dots
\]
and therefore the spinor bilinear is not a good building block for our lagrangian. Let's check why \(S^{\dagger} \neq S^{-1}\):
\[
    \begin{aligned}
        S = \dots \quad \implies \begin{dcases}
                                     , ; \\
                                     .
                                 \end{dcases} \\
    \end{aligned}
\]
The generators \(S^{\mu \nu}\) are \textit{anti-hermitian}, while
\[
    \dots
\]
the generators \(\Sigma^{\mu \nu}\) are indeed hermitian
\[
    \dots
\]
and if \((\gamma^{\mu})^{\dagger} = \pm \gamma^{\mu}\) only if \(\iff\) all the gamma matrices are hermitian or anti-hermitian, so let's just check from the representations: we find out this is not possible since from the clifford algebra
\[
    \begin{aligned}
        \{\gamma^{\mu} ,\, \gamma^{\nu}\} = 2 \eta^{\mu \nu} \mathbb{I}_4 \implies \dots
        C^{-1} \gamma^0 C C^{-1} \gamma^0 C = \mathbb{I}_4 \implies (\gamma^0_{\text{diag}})^2 = \mathbb{I}_4
    \end{aligned}
\]
so \(\gamma^0\) is harmitian, with real eigenvalues \(\lambda_i^2 = 1\) while \(\gamma^i\) is antihermitian with imaginary eigenvalues \(\lambda_i^2 = -1\).

We then need to find another building block for our lagrangiuan.
Note that \((\gamma^{\mu})^{\dagger} = \gamma^0 \gamma^{\mu} \gamma^0\) \(\forall \mu\). Lets check it:
\[
    \begin{aligned}
        \mu = 0, \quad &   \\
        \mu = i, \quad & .
    \end{aligned}
\]
If we now look at \((S^{\mu \nu})^{\dagger}\)
\[
    \dots
\]
where we have expanded the commutator, recognized identities. So the right relation is \((S^{\mu \nu})^{\dagger} = \gamma^0 S^{\mu \nu} \gamma^0\). Remeber we are looking for the scalar building block of the lagrangian. Following this result we can write
\[
    S^{\dagger} = e^{\frac{1}{2}\omega_{\mu \nu}(S^{\mu \nu})^{\dagger}} = \dots
\]
Let's factor \(\gamma^0\) out again, manipulating the expression until
\[
    S^{\dagger} = \gamma^0 \left( \dots \right) \gamma^0 = \gamma^0 e^{\frac{1}{2}\omega_{\mu \nu}S^{\mu \nu}} \gamma^0 = \gamma^0 S^{\mu \nu} \gamma^0.
\]
This tells u that the right building block for our lagrangian is
\[
    \overline{\psi}(x)\psi(x),
\]
using the \textbf{adjoint Dirac Spinor} \(\overline{\psi}(x) = \psi^{\dagger} \gamma^0\). So in the end let's finally check the invarinace of this scalar:
\[
    \begin{aligned}
        \dots
    \end{aligned}
\]
it is indeed lorentz invariant. \(\overline{\psi} \psi\) is the lorentz scalar we will use to build our lagrangian.

But this is not enough. If we remember the KG Lagrangian
\[
    KG \mathcal{L}
\]
the analog of the last term would be \(m \overline{\psi}(x)\psi(x)\), while the first one, where the indices are contracted into a loremtz scalar, but that involves derivative. we have to look at anothe spinor bilinear for the kinetic term.

\subsection{Kinetic term}

Consider a different spinor bilinear
\[
    \overline{\psi}(x) \gamma^{\mu} \psi(x)
\]
which in the end is a vector of four components, so we can ask ourselves if that is a Lorentz vector or not, So that
\[
    \overline{\psi}^{\prime} \gamma^{\mu} \psi^{\prime} = \Lambda^{\mu}_{\nu} \overline{\psi}\gamma^{\nu} \psi
\]
If this is true, we can contract that \(\mu\) indices in order to obtain a lorentz scalar (invariant)
\[
    \overline{\psi} \gamma^{\mu}\partial_{\mu}\psi.
\]
So we found another building block, if true. Let's check it
\[
    \overline{\psi} \gamma^{\mu} \psi \to \overline{\psi}^{\prime} \gamma^{\mu} \psi^{\prime} = \overline{\psi} S^{-1}\gamma^{\mu} S \psi,
\]
So
\[
    \Lambda = e^{-\frac{i}{2} \dots } = \dots
\]
and in parallel
\[
    S = e^{\dots } = \dots
\]
Where we are analizyng \(4\times 4\) matrices, the first from the foundamental representation \(\mathrm{S=}(3,1)\), the second from the Dirac representation; also \(\Sigma^{\mu \nu}\) and \(M^{\mu \nu}\) are generators of lorentz transformation in foundamental representation. if we use the representation of \(M^{\mu \nu}\)
\[
    (\mathcal{M}^{\rho \sigma})^{\mu}_{\ \nu} \gamma^{\nu} = \dots
\]
Now we want some kind of relation among (??) since this last expression is the corresponding of \(\Lambda^{\mu}_{\ \nu} \overline{\psi} \gamma^{\nu} \psi\):
\[
    S^{\rho \sigma} = \frac{1}{4}\left[\gamma^{\rho},\, \gamma^{\sigma} \right] = \dots
\]
and
\[
    \left\{ \gamma^{\rho},\, \gamma^{\sigma} \right\} = 2 \eta^{\rho \sigma} \iff \dots
\]
Now we can compute the following commutator
\[
    [S^{\rho \sigma},\,\gamma^{\mu}] = \frac{1}{2} \dots
\]
So using the previous \(\gamma^{\sigma} \gamma^{\rho} = 2\eta^{\rho \sigma} - \gamma^{\rho} \gamma^{\sigma}\) we get to
\[
    [S^{\rho \sigma},\,\gamma^{\mu}] = \dots
\]
so we can take the previous computation and recognize
\[
    (\mathcal{M}^{\rho \sigma})^{\mu}_{\ \nu} \gamma^{\nu} = - [S^{\rho \sigma},\,\gamma^{\mu}].
\]
And now finally
\[
    \Lambda^{\mu}_{\ \nu} \gamma^{\nu} = \left[\delta^{\mu}_{\ \nu} + \frac{1}{2} \omega_{\rho \sigma} (\mathcal{M}^{\rho \sigma})^{\mu}_{\ \nu} + O(\omega_{\rho \sigma}^2)\right]\gamma^{\nu} = \gamma^{\mu} -\frac{1}{2} \omega_{\rho \sigma}[S^{\rho \sigma},\,\gamma^{\mu}] + O(\omega_{\rho \sigma}^2),
\]
but at an infinitesimal level it is identical to
\[
    S^{-1} \gamma^{\mu} S = \dots
\]
So we finally found
\[
    S^{-1} \gamma^{\mu} S = \Lambda^{\mu}_{\ \nu} \gamma^{\nu}.
\]
This implies that the Dirac bilinear \(\overline{\psi} \gamma^\mu \psi\) transforms as a Lorentz four vector since
\[
    \overline{\psi} \gamma^{\mu} \psi \to \overline{\psi}^{\prime} \gamma^{\mu} \psi^{\prime} = \overline{\psi} S^{-1} \gamma^{\mu} S \psi = \Lambda^{\mu}_{\ \nu} \overline{\psi} \gamma^{\mu} \psi.
\]

We can now obtain another Lorentz scalar by contracting this four vector with another one\dots
Examples:
\begin{itemize}
    \item \(\overline{\psi} \gamma^{\mu} \partial_\mu \psi\), good for kinetic terms;
    \item \(\overline{\psi} \gamma^{\mu}A_\mu \psi\), which is a contraction with a field in vectorial representation, this term represents interaction between spin \(\tfrac12\) and spin 1 gauge bosons (e.g. photons);
    \item \(\overline{\psi} \gamma^{\mu \nu} \psi\), where there is a Lorentz tensor, i.e. \(\overline{\psi}^{\prime} \gamma^{\mu \nu} \psi^{\prime} = \Lambda^{\mu}_{\ \rho} \Lambda^{\nu}_{\sigma} \overline{\psi} \gamma^{\rho \sigma} \psi\).
\end{itemize}
We can introduce the \textbf{slash notation} for indicating an object contracted with a gamma matrix.

We can now find an expression for the Lorentz invariant Dirac Lagrangian:
\[
    \mathcal{L} = \overline{\psi} i \gamma^{\mu} \partial_{\mu} \psi - m \overline{\psi} \psi = \overline{\psi} \left(i \gamma^{\mu} \partial_{\mu} - m\right) \psi.
\]
Comments
\begin{itemize}
    \item the \(i\) factor ensures the lagrangian to be real: we have to check the realness of
          \[
              (\overline{\psi} \psi)^{\dagger} = \dots = \overline{\psi} \psi.
          \]
          \[
              (\overline{\psi} \gamma^{\mu} \partial_{\mu} \psi)^{\dagger} = \dots = - \overline{\psi} \gamma^{\mu} \partial_{\mu} \psi.
          \]
          That's why we need the imaginary unit.\footnote{Since this lagrangian is integrated to obtain the Lagrangian, we can always use the trick of integrating by parts and then neglect the boundary term.}
    \item Some unit analysis in mass units:
          \[
              \begin{aligned}
                  [S] = 0, \quad [\mathrm{d}^4 x]=-4 \quad \implies [\mathcal{L}] = 4, \\
                  [\partial_\mu] = 1,\quad [m]= 1 \quad \implies [\psi]=[\overline{\psi}] = \frac{3}{2}.
              \end{aligned}
          \]
          So this field has a different mass dimention from KG scalar field, which had: \([\psi]=1\).
    \item KG Lagrangian contains two derivatives \(\partial_\mu \psi \partial^{\mu} \psi\), while Dirac's only one: The KG lagrangian was of the second order, while Dirac's of the first; it changes the order of the equations of motion.
    \item Upon quantization the Dirac theory will describe particles/antiparticles (for whose we needed a complex field for the charge dof, which is our case) with spin \(\tfrac12\) and mass \(m\); in principle we have four complex dof, which are LH/RH and spin up/down (doubled because 4 for particle and 4 for antiparticle real dof).
\end{itemize}




------



\section{Dirac Equation}

The Dirac Lagrangian is
\[
    \dots
\]
where \(\overline{\psi} = \psi^{\dagger} \gamma^0\)  is the Dirac conjugate. We can find the equations of motion using the Euler-Lagrange equations for fields:
\[
    \dots
\]
notice that \(\frac{\partial}{\partial \overline{\psi}} \mathcal{L} = 0\), in fact \(\dots \) and so \(\dots \) too. Meanwhile the derivatives with respect to \(\psi\) are
\[
    \dots
\]
Hence we get the Dirac equation
\[
    \dots
\]
where the arrow on the derivatives tells us it is to be applied to the left. This is a first order partial differential equation for the spinor field \(\psi(x)\).

We can build such an equation of motion just thatks to the presence of \(\gamma^{\mu}\) which grants lorentz invariance. Dirac equation si a first order differential equation, while for KG we could only get a second order one. Furthermore KG is a scalar equation, while Dirac is a spinor equation (vectorial in spinor space, with 4 components).

Notice that dirac equation mixes components of the spinor, since \(\gamma^{\mu}\) are \(4\times 4\) matrices:
\[
    \psi = \begin{pmatrix}
        \psi_1 \\
        \psi_2 \\
        \psi_3 \\
        \psi_4
    \end{pmatrix}, \quad \gamma^0 = \begin{pmatrix}
        0            & \mathbb{I}_2 \\
        \mathbb{I}_2 & 0
    \end{pmatrix}, \quad \gamma^i = \begin{pmatrix}
        0         & \sigma^i \\
        -\sigma^i & 0
    \end{pmatrix}.
\]
So in components in the spinor space, Dirac equations reads:
\[
    i \begin{pmatrix}
        \partial_t \psi_3 \\
        \partial_t \psi_4 \\
        \partial_t \psi_1 \\
        \partial_t \psi_2
    \end{pmatrix} + i \begin{pmatrix}
        \partial_x \psi_4  \\
        \partial_x \psi_3  \\
        -\partial_x \psi_2 \\
        -\partial_x \psi_1
    \end{pmatrix} \dots
\]
in general it becomes a system of four coupled first order differential equations, mixing the four components of the spinor.

However each component \(\psi_{\alpha}(x)\) satosfy the KG equation, since in effect each components describe a dof of a relativistic particle with mass \(m\): if we multiply Dirac equation by \((i \gamma^{\mu} \partial_{\mu} + m)\) from the left (it is still valid since zero multiplied by anything remains xero), we get
\[
    \begin{aligned}
        (i \gamma^{\mu} \partial_{\mu} + m)(i \gamma^{\nu} \partial_{\nu} - m) \psi(x) = 0                                \\
        \implies \left(-\gamma^{\mu} \gamma^{\nu} \partial_{\mu} \partial_{\nu} - m^2\right) \psi(x) = 0                  \\
        \implies \left(-\frac{1}{2}\{\gamma^{\mu},\,\gamma^{\nu}\} \partial_{\mu} \partial_{\nu} - m^2\right) \psi(x) = 0 \\
        \implies \left(-\eta^{\mu \nu} \partial_{\mu} \partial_{\nu} - m^2\right) \psi(x) = 0                             \\
        \implies (\Box + m^2) \psi(x) = 0.
    \end{aligned}
\]
In terms of matrices we have used the clifford algebra to simplify the product of gamma matrices. So each component of the dirac spinor satisfies the KG equation, thus \((\Box + m^2) \psi_{\alpha} (x) = 0\) \(\forall \alpha = 1,\,2,\,3,\,4\).

\subsection{Chiral Spinors}

Chirality means that Dirac representation \((\tfrac12,\,0) \oplus (0,\,\tfrac12)\) can be decomposed into two irreducible representations, \textbf{Weyl representation}, of the Lorentz group. Weyl or Chiral spinors are two-component objects (complex dof) with different transformation properties:
\begin{itemize}
    \item \textbf{Left-Handed} weyl spinors: \(\psi_L \sim (\tfrac12,\,0)\), transform under \(S_L\) only;
    \item \textbf{Right-Handed} weyl spinors: \(\psi_R \sim (0,\,\tfrac12)\), transform under \(S_R\) only.
\end{itemize}

We can write the Dirac spinor as a combination of two weyl spinors:
\[
    \psi_D = \begin{pmatrix}
        \psi_L^{(w)} \\
        \psi_R^{(w)}
    \end{pmatrix}, \quad \psi_L^{(w)} \xleftrightarrow{\text{Parity}} \psi_R^{(w)}.
\]

\paragraph{Chirality operator.} In order to project out the two chiral components from a Dirac spinor we can introduce the chirality operator
\[
    \gamma^5 = \dots
\]
using which we can define projectors able to select the desired chiral component from the Dirac spinor:
\[
    \begin{aligned}
        \psi_L = \\
        \psi_R =.
    \end{aligned}
\]
Thus the dirac spinor is the sum of the two chiral components \(\psi_D = \psi_L + \psi_R\) and the projectors satisfy the usual properties:
\[
    \begin{dcases}
        P_L^2 = P_L \\
        P_L^{\dagger} = P_L
    \end{dcases} \quad \dots
\]
Note that the eigenvalues of \(\gamma^5\) are \(\pm 1\), so the chirality operator measures the chirality of a spinor:
\[
    \dots
\]
We will see how chirality is related to helicity in the massless limit.

\paragraph{Lagrangian and Chirality.}To understand better, let's write the dirac lagrangian in terms of chiral components (which is very useful for statistical field theory and the standard model, also for understanding the massless limit):
\begin{enumerate}
    \item aa
          \[
              \begin{aligned}
                  \overline{\psi} \gamma^{\mu} \psi = \psi^{\dagger} \gamma^0 \gamma^{\mu} \left(\psi_L + \psi_R\right) = \dots
              \end{aligned}
          \]

          Now exploiting hermitianity of the gamma matrices we can write
          \[
              \overline{\psi} \gamma^{\mu} \psi = \overline{\psi}_R \gamma^{\mu} \psi_R + \overline{\psi}_L \gamma^{\mu} \psi_L.
          \]
          This found vector current does not mix chiral components, its parity invariant; this is not good for electroweak interactions and interpretation \(\mathrm{SU}(2)_L\).

    \item aa
          \[
              \overline{\psi} \gamma^{\mu} \gamma^{5} \psi = \dots
          \]
          same steps as before, but now we find another current, called \textbf{axial vector current}: it changes sign under parity and not good again for EW interpretation.

    \item aa
          \[
              \frac{1}{2} (V-A) = \overline{\psi} \dots
          \]
          V-A current violates parity, and only LH components enter the interarction: it is a good candidate for EW interactions.

    \item mass term
          \[
              \overline{\psi} \psi 0 = \psi^{\dagger} \gamma^0 (P_R^2 + P_L^2) \psi = \dots =
          \]
          with respect to before we dont have \(\gamma^{\mu}\), so we have to put a minus while writing in terms of chiral components: mass term mixes LH and RH components:
          \[
              \dots
          \]
\end{enumerate}

Now we have everything to write the dirac lagrangian in terms of chiral components:
\[
    \mathcal{L} = \overline{\psi}_L i \gamma^{\mu} \partial_{\mu} \psi_L + \overline{\psi}_R i \gamma^{\mu} \partial_{\mu} \psi_R - m \left(\overline{\psi}_L \psi_R + \overline{\psi}_R \psi_L\right).
\]
We can see that in the massless limit the two chiral components decouple, and we get two independent Weyl equations for each chiral component: this is important, since kinetic terms evolves independently the two chiral components, while the mass term allow us to perform a boost and mix the two chiralities: since the particle is massive there will be frames where the particle is RH and others where it is seen as LH.

If you take an electron whith both the chiral components, only the LH component will interact weakly, while the RH will not; but since the electron is massive you can always boost to a frame where the electron appears as RH, so both components are needed to describe a massive fermion. When we see only the RH component, the electron will not seem to interact weakly.

\paragraph{Dirac in Weyl components.}
If we write the Dirac equation in terms of chiral components we get:
\[
    \dots
\]
which is a system of two coupled first order coupled differential equations
\[
    \begin{dcases}
        \dots \\
        \dots \\
    \end{dcases}
\]
In the massless limit the two equations decouple and we get two Weyl equations:
\[
    \begin{dcases}
        \dots \\
        \dots \\
    \end{dcases}
\]
and we call this last set of equations the \textbf{Weyl equations} for massless fermions.
Now it is clearer the meaning of Left and Right-Handed Weyl spinors: in terms of operators we have
\[
    i \partial_t = \hat{H},\quad -i \nabla = \hat{\mathbf{p}}, \quad \bs{\sigma} = \hat{\mathbf{S}},
\]
so that we can see the last two equations as
\[
    \dots
\]
Now, dropping the hat notation for operators (so that we can use it to indicate versors), we find clearly that the weyl components are eigenstate of the \textbf{helicity}:
\[
    \begin{dcases}
        \dots \\
        \dots
    \end{dcases}
\]
where helicity operator is defined as \(\mathbf{S} \cdot \mathbf{p}\); since the eigenvalues are \(\pm 1\), we can affirm that in the massless limit chirality and helicity coincide: they have same eigenstates and eigenvalues.

If the neutrinos were massless, only the LH component would exist, since only that interacts weakly; but since neutrinos have a small mass, both chiral components exist, even if the RH component has never been observed (it interacts only gravitationally, so it is very difficult to detect it).

We now want to prove that under parity we can pass from one weyl component to the other:
\[
    \psi_L^{(w)} \xleftrightarrow{\text{Parity}} \psi_R^{(w)}.
\]
Let's start from the Dirac equation:
\[
    \begin{aligned}
        \dots
    \end{aligned}
\]
which, given the action of the parity:
\[
    \dots
\]
then transforms the Dirac equation as
\[
    \dots
\]
But how does \(\psi\) transform under parity?
\[
    \dots
\]
but gamma matrices satisfy
\[
    \dots
\]
so we gave
\[
    \dots
\]
[...] sofi [...]

and we find finally that
\[
    \psi^{\prime} = \begin{pmatrix}
        \dots
    \end{pmatrix} = \begin{pmatrix}
        0            & \mathbb{I}_2 \\
        \mathbb{I}_2 & 0
    \end{pmatrix} \begin{pmatrix}
        \psi_L^{(w)} \\
        \psi_R^{(w)}
    \end{pmatrix},
\]
so that
\[
    \begin{dcases}
        \psi_L^{\prime (w)} = \psi_R^{(w)}, \\
        \psi_R^{\prime (w)} = \psi_L^{(w)}.
    \end{dcases}
\]