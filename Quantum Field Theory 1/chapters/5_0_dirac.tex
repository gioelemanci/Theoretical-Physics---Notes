\chapter{Dirac theory}

From now on, we aim to describe particles with spin different from zero. The simplest case is spin \(\tfrac12\) fermions, which are described by the Dirac equation. The starting point will be the Lagrangian, then we will show the most general solution to the equations of motion and finally proceed to quantize the system and promote the observables to operators on the Fock Space.

The lagrangian has to be Lorentz invariant, so we recall that the Lorentz group admits a \(4D\) spinor representation, which for Dirac translates into
\[
    \psi_D \to \psi_D^{\prime} = e^{-\tfrac{i}{2}\omega_{\mu \nu} \Sigma^{\mu \nu}} \psi_D,
\]
the \textbf{Dirac spinor}, where the \(4\times 4\) antisymmetric generators of the Lorentz group in the Dirac representation\footnote{\(\Sigma^{\mu \nu}\) has only six independent components, corresponding to the six generators of the Lorentz group (as given by eq. \eqref{eq:Dirac_representation_generators}): it is antisymmetric in \(\mu\) and \(\nu\) and traceless, thus \((4 \times 4 - 4) / 2 = 6\).} are given by
\[
    \Sigma_{\mu \nu} = \frac{i}{4} \gamma^{\mu \nu}, \quad \gamma^{\mu \nu} = [\gamma^{\mu},\, \gamma^{\nu}] = \gamma^{\mu} \gamma^{\nu} - \gamma^{\nu} \gamma^{\mu},
\]
where \(\gamma^{\mu}\) are the \textbf{Dirac gamma matrices}, respecting the clifford algebra
\begin{equation}
    \{ \gamma^{\mu}, \gamma^{\nu} \} = 2 \eta^{\mu \nu} \mathbb{I}_4,
    \label{eq:clifford_algebra}
\end{equation}
of \(4\times 4\) complex matrices.

We adopt the \textbf{Weyl representation} (or chiral representation, eq. \eqref{eq:Dirac_gamma_matrices}) of the Dirac gamma matrices, which assumes the form
\[
    \gamma^0 = \begin{pmatrix}
        0            & \mathbb{I}_2 \\
        \mathbb{I}_2 & 0
    \end{pmatrix}, \quad \gamma^i = \begin{pmatrix}
        0         & \sigma^i \\
        -\sigma^i & 0
    \end{pmatrix},
\]
where \(\sigma^i\) are the Pauli matrices (as given in eq. \eqref{eq:Pauli_matrices}):
\[
    \sigma^1 = \begin{pmatrix}
        0 & 1 \\
        1 & 0
    \end{pmatrix}, \quad \sigma^2 = \begin{pmatrix}
        0 & -i \\
        i & 0
    \end{pmatrix}, \quad \sigma^3 = \begin{pmatrix}
        1 & 0  \\
        0 & -1
    \end{pmatrix}.
\]

The generators of Lorentz transformations in the Dirac representation can be written also as
\[
    S^{\mu \nu} = - i \Sigma^{\mu \nu} = \frac{1}{4} \gamma^{\mu \nu}, \quad \psi_D^{\prime} = e^{\frac{1}{2} \omega_{\mu \nu} S^{\mu \nu}} \psi_D.
\]
If we introduce now the \textbf{spinorial indices}, we can write the transformation law as:
\begin{equation}
    \psi_D^{\alpha}(x) \to \psi_D^{\prime \alpha}(x^{\prime}) = S^{\alpha}_{\ \beta}(\Lambda) \psi_D^{\beta}(x) = \left(e^{\frac{1}{2} \omega_{\mu \nu} S^{\mu \nu}}\right)^{\alpha}_{\ \beta} \psi_D^{\beta}(x),
    \label{eq:Dirac_spinor_transformation}
\end{equation}
where \(\Lambda\) is the Lorentz transformation, which also acted on the coordinates (they got transformed along with the Dirac spinor). From now on we will drop the subscript \(D\) for Dirac spinors, since we will only deal with them.

\section{Action and Lagrangian}

Now the idea is to derive a Lorentz invariant Lagrangian dependent upon the Dirac spinor \(\psi\), remembering that \(\psi\) has four complex components, so eight real degrees of freedom, and
\[
    \psi^{\dagger}(x) = (\psi^*)^T(x).
\]
The lagrangian must be a Lorentz scalar, so we need to find building blocks which are lorentz scalars, vectors, tensors (whih will be contracted into scalars since the legrangian has to be one) built from the spinor \(\psi\) and its derivatives. We will consider different \textbf{spinor bilinears}, i.e. quantities built from two spinors, since a single spinor cannot build a Lorentz scalar alone.

\subsection{Building Block for the Mass Term}

The simplest bilinear we can consider as a candidate for the mass term building block is
\begin{equation}
    \psi^{\dagger}(x) \psi(x),
    \label{eq:first_scalar_spinor_bilinear}
\end{equation}
which is going to be a real number, but we have to check its transformation properties under Lorentz transformations to prove wether it is a Lorentz scalar.\footnote{By parallelism with the KG lagrangian we know that the mass term is made of Lorentz scalars built with fields, while the kinetic term (as we will soon see) should be made of two Lorentz vectors contracted into a scalar.} We know that
\[
    \psi \to  \psi^{\prime} = S \psi, \quad \psi^{\dagger} \to  \psi^{\prime \dagger} = \psi^{\dagger} S^{\dagger},
\]
hence their product
\[
    \psi^{\dagger} \psi \to \psi^{\prime \dagger} \psi^{\prime} = \psi^{\dagger} S^{\dagger} S \psi,
\]
which could be Lorentz scalar if and only if \(S^{\dagger} = S^{-1}\), which is not the case, since the Dirac representation is not unitary:
\[
    S^{\dagger} S \neq \mathbb{I}_4 \implies \psi^{\prime \dagger} \psi^{\prime} \neq \psi^{\dagger} \psi,
\]
and therefore the spinor bilinear is not a good building block for our Lagrangian. Let's check explicitly why this representation \(S^{\dagger} \neq S^{-1}\) is not unitary:
\[
    S = e^{\tfrac{1}{2} \omega_{\mu \nu} S^{\mu \nu}} \quad \implies \begin{dcases}
        S^{-1}      & = e^{-\tfrac{1}{2} \omega_{\mu \nu} S^{\mu \nu}};           \\
        S^{\dagger} & = e^{\tfrac{1}{2}\omega_{\mu \nu} (S^{\mu \nu})^{\dagger}}.
    \end{dcases} \\
\]
Thus for the representation to be unitary, we are requiring the generators \(S^{\mu \nu}\) to be \textit{anti-hermitian}
\[
    (S^{\mu \nu})^{\dagger} = - S^{\mu \nu} \implies (-i \Sigma^{\mu \nu})^{\dagger} = - (-i \Sigma^{\mu \nu});
\]
in other words we are requiring the generators \(\Sigma^{\mu \nu}\) to be hermitian
\[
    (\Sigma^{\mu \nu})^{\dagger} = \Sigma^{\mu \nu}.
\]
But we can explicitly verify that this is never the case:
\[
    (S^{\mu \nu})^{\dagger} = \frac{1}{4} \left[\gamma^{\mu},\,\gamma^{\nu}\right]^\dagger = \frac{1}{4}(\gamma^{\nu})^{\dagger} (\gamma^{\mu})^{\dagger} - \frac{1}{4}(\gamma^{\mu})^{\dagger} (\gamma^{\nu})^{\dagger} = - \frac{1}{4} \left[(\gamma^{\mu})^{\dagger},\, (\gamma^{\nu})^{\dagger}\right],
\]
which would be equal to \(- S^{\mu \nu}\) only if
\[
    (S^{\mu \nu})^{\dagger} = - \frac{1}{4} \left[(\gamma^{\mu})^{\dagger},\, (\gamma^{\nu})^{\dagger}\right] = - \frac{1}{4} \left[\gamma^{\mu},\, \gamma^{\nu}\right] = - S^{\mu \nu}.
\]
Thus \((\gamma^{\mu})^{\dagger} = \pm \gamma^{\mu}\) \(\iff\) the gamma matrices are hermitian or anti-hermitian; we can check from the Clifford algebra \eqref{eq:clifford_algebra} that this is not possible for all \(\mu\):
\[
    \{\gamma^{\mu} ,\, \gamma^{\nu}\} = 2 \eta^{\mu \nu} \mathbb{I}_4 \implies (\gamma^0)^2 = \mathbb{I}_4 \quad (\gamma^i)^2 = -\mathbb{I}_4,
\]
where we have computed the anticommutator for \(\mu = \nu = 0\) and \(\mu = \nu = i\). If we apply those squared matrices to a generic complex four-vector \(v\), we get
\[
    (\gamma^0)^2 v = \lambda_0^2 v = v, \quad (\gamma^i)^2 v = \lambda_i^2 v = -v,
\]
thus \(\gamma^0\) has real eigenvalues \(\lambda_0 = \pm 1\), while \(\gamma^i\) have imaginary eigenvalues \(\lambda_i = \pm i\); hence \(\gamma^0\) is hermitian, while \(\gamma^i\) are anti-hermitian. Therefore the gamma matrices are not all hermitian or anti-hermitian, so the generators \(S^{\mu \nu}\) are not hermitian, and the Dirac representation is not unitary:
\[
    (\gamma^0)^\dagger = \gamma^0, \quad (\gamma^i)^{\dagger} = - \gamma^i \implies (S^{\mu \nu})^{\dagger} \neq - S^{\mu \nu} \implies S^{\dagger} \neq S^{-1}.
\]
We then need to find another building block for our Lagrangian.

------

We can start from the following gamma matrices property: \((\gamma^{\mu})^{\dagger} = \gamma^0 \gamma^{\mu} \gamma^0\) for all \(\mu\). Lets check it:
\[
    \begin{aligned}
        \mu = 0) \quad & (\gamma^0)^\dagger = \gamma^0 \gamma^0 \gamma^0 = \gamma^0 = (\gamma^0)^\dagger,   \\
        \mu = i) \quad & (\gamma^i)^\dagger = \gamma^0 \gamma^i \gamma^0 = - \gamma^i = (\gamma^i)^\dagger,
    \end{aligned}
\]
where in the first case we used the idempotency of \(\gamma^0\) and its hermitianity, while in the second case we used the anticommutation relations among gamma matrices.\footnote{Indeed, computing it explicitly from the Clifford algebra we get \(\{\gamma^i,\,\gamma^0\} = 2 \eta^{i0} =0 \implies \gamma^i \gamma^0 = - \gamma^0 \gamma^i\).} So this relation holds for all \(\mu\). Using this we can compute \((S^{\mu \nu})^{\dagger}\) as:
\[
    \begin{aligned}
        (S^{\mu \nu})^{\dagger} & = \frac{1}{4} \left[\gamma^{\mu},\, \gamma^{\nu}\right]^{\dagger} = \frac{1}{4} (\gamma^{\nu})^{\dagger} (\gamma^{\mu})^{\dagger} - \frac{1}{4} (\gamma^{\mu})^{\dagger} (\gamma^{\nu})^{\dagger} \\
                                & = \frac{1}{4} \gamma^0 \gamma^{\nu} \gamma^0 \gamma^0 \gamma^{\mu} \gamma^0 - \frac{1}{4} \gamma^0 \gamma^{\mu} \gamma^0 \gamma^0 \gamma^{\nu} \gamma^0                                           \\
                                & = \gamma^0 \left(\frac{1}{4} [\gamma^{\nu},\, \gamma^{\mu}] \right) \gamma^0 = - \gamma^0 \left(\frac{1}{4} [\gamma^{\mu},\, \gamma^{\nu}] \right) \gamma^0 = - \gamma^0 S^{\mu \nu} \gamma^0,
    \end{aligned}
\]
where we have expanded the commutator and simplified the \((\gamma^0)^2\). Since we are looking for the scalar building block for the lagrangian, we are lead to follow this result and compute \(S^{\dagger}\):
\[
    S^{\dagger} = e^{\frac{1}{2}\omega_{\mu \nu}(S^{\mu \nu})^{\dagger}} = e^{-\frac{1}{2}\omega_{\mu \nu} \gamma^0 S^{\mu \nu} \gamma^0}.
\]
Let's factor \(\gamma^0\) out again, Taylor expanding the exponential:
\[
    \begin{aligned}
        S^{\dagger} & \sim \mathbb{I}_{4} - \frac{1}{2}\omega_{\mu \nu} \gamma^0 S^{\mu \nu} \gamma^0  + \frac{1}{4}\omega_{\mu \nu}\omega_{\rho \sigma} \gamma^0 S^{\mu \nu} \gamma^0\gamma^0 S^{\rho \sigma} \gamma^0 - \frac{1}{8}\cdots \\
                    & = \gamma^0 \gamma^0 - \frac{1}{2}\omega_{\mu \nu} \gamma^0 S^{\mu \nu} \gamma^0  + \frac{1}{4}\omega_{\mu \nu}\omega_{\rho \sigma} \gamma^0 S^{\mu \nu} \mathbb{I}_4 S^{\rho \sigma} \gamma^0 - \frac{1}{8}\cdots     \\
                    & = \gamma^0 \left(\mathbb{I}_{4} - \frac{1}{2}\omega_{\mu \nu} S^{\mu \nu} + \frac{1}{4}\omega_{\mu \nu}\omega_{\rho \sigma} S^{\mu \nu} S^{\rho \sigma} - \frac{1}{8}\cdots \right) \gamma^0                          \\
                    & = \gamma^0 e^{-\frac{1}{2}\omega_{\mu \nu}S^{\mu \nu}} \gamma^0 = \gamma^0 S^{\mu \nu} \gamma^0.
    \end{aligned}
\]
Now we have definitive proof that the right expression for \(S^{\dagger}\) is
\[
    S^{\dagger} = \gamma^0 S^{-1} \gamma^0,
\]
since we already knew that \(S^{-1} = e^{-\frac{1}{2}\omega_{\mu \nu} S^{\mu \nu}}\).

Thus finally we have insight on how to build a lorentz scalar from spinors: instead of \(\psi^{\dagger} \psi\) we can consider the following bilinear
\begin{equation}
    \overline{\psi}(x)\psi(x),
    \label{eq:second_scalar_spinor_bilinear}
\end{equation}
built using the \textbf{adjoint Dirac Spinor} \(\overline{\psi}(x) = \psi^{\dagger} \gamma^0\) (more details in \eqref{eq:Dirac_conjugate}). Let's check its transformation properties:
\[
    \begin{aligned}
        \overline{\psi}(x) \psi(x) & \to \overline{\psi}^{\prime}(x^{\prime}) \psi^{\prime}(x^{\prime}) = \psi^{\prime \dagger}(x^{\prime}) \gamma^0 \psi^{\prime}(x^{\prime}) = \psi^{\dagger}(x) S^{\dagger} \gamma^0 S \psi(x) \\
                                   & = \psi^{\dagger}(x) \gamma^0 S^{-1} \gamma^0 \gamma^0 S \psi(x) = \psi^{\dagger}(x) \gamma^0 S^{-1} S \psi(x)                                                                                \\
                                   & = \psi^{\dagger}(x) \gamma^0 \psi(x) = \overline{\psi}(x) \psi(x).
    \end{aligned}
\]
which is lorentz invariant indeed. Thus we could write the transormation of the adjoint spinor as
\begin{equation}
    \overline{\psi} \to \overline{\psi}^{\prime} = \psi^{\dagger} S^{\dagger} \gamma^0 = \psi^\dagger \gamma^0 S^{-1}= \overline{\psi} S^{-1}.
    \label{eq:adjoint_spinor_transformation}
\end{equation}

Thus \(\overline{\psi} \psi\) is the lorentz scalar we will use to build our lagrangian, in particular for the mass term, where there is no derivative involved:
\[
    \mathcal{L}_{mass} = - m \overline{\psi}(x) \psi(x).
\]

Recalling the expression for the KG Lagrangian
\[
    \mathcal{L}_{KG} = \frac{1}{2} \left(\partial_{\mu} \phi \partial^{\mu} \phi - m^2 \phi^2\right),
\]
we have indeed found the analog of the last term: \(m \overline{\psi}(x)\psi(x)\); the first one instead, where the indices has to be contracted into a Lorentz scalar, involves derivatives. Thus we have to look for another spinor bilinear for the kinetic term, preserving Lorentz invariance with derivatives.

\subsection{Building Block for the Kinetic Term}

Considering the following spinor bilinear
\begin{equation}
    \overline{\psi}(x) \gamma^{\mu} \psi(x),
    \label{eq:vector_spinor_bilinear}
\end{equation}
which is a vector of four components, we have to understand if that is a Lorentz vector or not, So that
\[
    \overline{\psi}^{\prime} \gamma^{\mu} \psi^{\prime} = \Lambda^{\mu}_{\ \nu} \overline{\psi}\gamma^{\nu} \psi.
\]
If this is true, we can contract that \(\mu\) indices in order to obtain a lorentz scalar (invariant)
\[
    \overline{\psi} \gamma^{\mu}\partial_{\mu}\psi.
\]
So, summarizing, if we verify that \(\overline{\psi} \gamma^{\mu} \psi\) transforms as a lorentz vector, we can use it to build the kinetic term of the lagrangian. Let's check how it transforms:
\[
    \overline{\psi} \gamma^{\mu} \psi \to \overline{\psi}^{\prime} \gamma^{\mu} \psi^{\prime} = \overline{\psi} S^{-1}\gamma^{\mu} S \psi,
\]
where we have used the results of the previous section to write the expression for the adjoint Dirac spinor transformation of eq. \eqref{eq:adjoint_spinor_transformation}. Thus if we now use the infinitesimal form of the Lorentz transformation in the foundamental representation of the Lorentz group \(\mathrm{SO}(1,3)\)
\[
    \Lambda = e^{-\frac{i}{2} \omega_{\mu \nu} M^{\mu \nu}},
\]
and in parallel its spinorial representation
\[
    S = e^{-\frac{i}{2} \omega_{\mu \nu} \Sigma^{\mu \nu}},
\]
we are studying the \(4\times 4\) matrices realizing the Lorentz transformation, in two different representations; also \(\Sigma^{\mu \nu}\) and \(M^{\mu \nu}\) are generators of lorentz transformation in the Dirac and foundamental representation.

We can even make the parallel substitution of \(S^{\mu \nu} = i \Sigma^{\mu \nu}\) in the foundamental representation, so that
\[
    \mathcal{M}^{\mu \nu} = - i M^{\mu \nu},
\]
so that now it is possible to write both infinitesimal transformations in the same form:
\[
    \begin{aligned}
        \Lambda & = e^{\frac{1}{2} \omega_{\mu \nu}\mathcal{M}^{\mu \nu}} \sim 1 + \frac{1}{2} \omega_{\mu \nu} \mathcal{M}^{\mu \nu} + O(\omega_{\mu \nu}^2), \\
        S       & = e^{\frac{1}{2} \omega_{\mu \nu} S^{\mu \nu}} \sim 1 + \frac{1}{2} \omega_{\mu \nu} S^{\mu \nu} + O(\omega_{\mu \nu}^2).
    \end{aligned}
\]

Now if we stick to the foundamental representation (vectorial representation of Lorentz group, which generators are given by \eqref{eq:vectorial_representation_generators}), we have that
\[
    (\mathcal{M}^{\rho \sigma})^{\mu}_{\ \nu} \gamma^{  \nu} = (\eta^{\rho \mu} \delta^{\sigma}_{\ \nu} - \eta^{\sigma \mu} \delta^{\rho}_{\ \nu}) \gamma^{\nu} = \eta^{\rho \mu} \gamma^{\sigma} - \eta^{\sigma \mu} \gamma^{\rho}.
\]
Now we want some kind of relation among the two representations, since this last expression is practically the corresponding of \(\Lambda^{\mu}_{\ \nu} \overline{\psi} \gamma^{\nu} \psi\). The idea is to expand the expression for \(S^{\rho \sigma}\) in order to find a way to relate the previous expression to a commutator with gamma matrices. We have
\[
    \begin{aligned}
        S^{\rho \sigma} & = \frac{1}{4} [\gamma^{\rho},\, \gamma^{\sigma}] = \frac{1}{4} (\gamma^{\rho} \gamma^{\sigma} - \gamma^{\sigma} \gamma^{\rho})                              \\
                        & = \frac{1}{4} ( 2 \gamma^{\rho} \gamma^{\sigma} - \{\gamma^{\rho},\, \gamma^{\sigma}\}) = \frac{1}{2} (\gamma^{\rho} \gamma^{\sigma} - \eta^{\rho \sigma}),
    \end{aligned}
\]
where we have used the clifford algebra to rewrite the commutator in terms of the anticommutator. Now if we compute the following commutator among gamma matrices and the generators in the Dirac representation
\[
    \begin{aligned}
        \left[S^{\rho \sigma},\,\gamma^{\mu}\right] & = \frac{1}{2} \left[ \gamma^{\rho} \gamma^{\sigma},\, \gamma^{\mu} \right] -\frac{1}{2} \eta^{\rho \sigma}\left[\mathbb{I},\, \gamma^{\mu} \right] \\
                                                    & = \frac{1}{2} (\gamma^{\rho} \gamma^{\sigma} \gamma^{\mu} - \gamma^{\mu} \gamma^{\rho} \gamma^{\sigma}) - 0,
    \end{aligned}
\]
and using again the Clifford algebra to compute \(\gamma^{\sigma} \gamma^{\rho} = 2\eta^{\rho \sigma} - \gamma^{\rho} \gamma^{\sigma}\), we get to
\[
    \begin{aligned}
        [S^{\rho \sigma},\,\gamma^{\mu}] & = \frac{1}{2} \gamma^{\rho} \left(2 \eta^{\mu \sigma} - \gamma^{\mu}\gamma^{\sigma}\right) - \frac{1}{2} \left(2 \eta^{\rho \mu} - \gamma^{\rho} \gamma^{\mu} \right) \gamma^{\sigma} \\
                                         & = \eta^{\mu \sigma} \gamma^{\rho} - \eta^{\rho \mu} \gamma^{\sigma} = - (\eta^{\rho \mu} \gamma^{\sigma} - \eta^{\sigma \mu} \gamma^{\rho}).
    \end{aligned}
\]
Recalling in the end the previous computation, it is clear that\footnote{Since the Minkowski metric \(\eta^{\mu \sigma} = \eta^{\sigma \mu}\) is symmetric, we can exchange the indices in the last term.}
\[
    (\mathcal{M}^{\rho \sigma})^{\mu}_{\ \nu} \gamma^{\nu} = - [S^{\rho \sigma},\,\gamma^{\mu}].
\]
And now finally we can compare the two infinitesimal transformations:
\[
    \Lambda^{\mu}_{\ \nu} \gamma^{\nu} = \left[\delta^{\mu}_{\ \nu} + \frac{1}{2} \omega_{\rho \sigma} (\mathcal{M}^{\rho \sigma})^{\mu}_{\ \nu} + O(\omega_{\rho \sigma}^2)\right]\gamma^{\nu} = \gamma^{\mu} -\frac{1}{2} \omega_{\rho \sigma}[S^{\rho \sigma},\,\gamma^{\mu}] + O(\omega_{\rho \sigma}^2),
\]
which is identical to
\[
    \begin{aligned}
        S^{-1} \gamma^{\mu} S & = \left( 1 - \frac{1}{2} \omega_{\rho \sigma} S^{\rho \sigma} + O(\omega_{\rho \sigma}^2) \right) \gamma^{\nu} \left(1 + \frac{1}{2} \omega_{\rho \sigma} S^{\rho \sigma} + O(\omega_{\rho \sigma}^2) \right) \\
                              & = \gamma^{\mu} -\frac{1}{2}\omega_{\rho \sigma} S^{\rho \sigma}\gamma^{\mu} + \frac{1}{2}\omega_{\rho \sigma} \gamma^{\mu} S^{\rho \sigma} + O(\omega_{\rho \sigma}^2)                                        \\
                              & = \gamma^{\mu} - \frac{1}{2} \omega_{\rho \sigma} [S^{\rho \sigma},\, \gamma^{\mu}] + O(\omega_{\rho \sigma}^2).
    \end{aligned}
\]
So we finally proved the identity
\begin{equation}
    S^{-1} \gamma^{\mu} S = \Lambda^{\mu}_{\ \nu} \gamma^{\nu},
    \label{eq:gamma_matrix_transformation_spin-foundamental}
\end{equation}
which implies that the Dirac bilinear \(\overline{\psi} \gamma^\mu \psi\) transforms as a Lorentz four vector, since
\[
    \overline{\psi} \gamma^{\mu} \psi \to \overline{\psi}^{\prime} \gamma^{\mu} \psi^{\prime} = \overline{\psi} S^{-1} \gamma^{\mu} S \psi = \Lambda^{\mu}_{\ \nu} \overline{\psi} \gamma^{\nu} \psi.
\]

We can now obtain other Lorentz scalars, vectors or tensors by contracting this four vector with other objects transforming in the vectorial representation of the Lorentz group, for example \(\partial_{\mu}\) or a gauge field \(A_{\mu}\):
\begin{itemize}
    \item \(\overline{\psi} \gamma^{\mu} \partial_\mu \psi\), which is a contraction with the derivative operator, this term represents the kinetic term of the lagrangian and it is a Lorentz scalar;
    \item \(\overline{\psi} \gamma^{\mu}A_\mu \psi\), which is a contraction with a field in vectorial representation, this term represents interaction among spin \(\tfrac12\) particles and spin 1 gauge bosons (e.g. photons);
    \item \(\overline{\psi} \gamma^{\mu \nu} \psi\), where there is a Lorentz tensor, i.e. \(\overline{\psi}^{\prime} \gamma^{\mu \nu} \psi^{\prime} = \Lambda^{\mu}_{\ \rho} \Lambda^{\nu}_{\sigma} \overline{\psi} \gamma^{\rho \sigma} \psi\) (every index transforming accordingly with a Lorentz transformation).
\end{itemize}
We can introduce the \textbf{slash notation} for indicating an object contracted with a gamma matrix:
\begin{equation}
    \slashed{A} = \gamma^{\mu} A_{\mu}.
    \label{eq:slash_notation}
\end{equation}

\subsection{Dirac Lagrangian}

We can finally write an expression for the manifestly Lorentz invariant Dirac lagrangian:
\begin{equation}
    \mathcal{L} = \overline{\psi} i \gamma^{\mu} \partial_{\mu} \psi - m \overline{\psi} \psi = \overline{\psi} \left(i \gamma^{\mu} \partial_{\mu} - m\right) \psi.
    \label{eq:Lorentz_invariant_Dirac_Lagrangian}
\end{equation}

It is worth to notice some features of this lagrangian:
\begin{itemize}
    \item the \(i\) factor ensures the lagrangian to be real: we have to check the realness of the two terms separately. For the mass term we have
          \[
              (\overline{\psi} \psi)^{\dagger} = \psi^{\dagger} (\psi^{\dagger} \gamma^0)^{\dagger}= \psi^{\dagger} \gamma^0 \psi = \overline{\psi} \psi,
          \]
          for hermitianity of \(\gamma^0\). Instead for the kinetic term we have
          \[
              \begin{aligned}
                  (\overline{\psi} \gamma^{\mu} \partial_{\mu} \psi)^{\dagger} & = (\partial_{\mu} \psi)^{\dagger} (\gamma^{\mu})^{\dagger} \overline{\psi}^{\dagger} = (\partial_{\mu} \psi)^{\dagger} (\gamma^{\mu})^{\dagger}\gamma^0 \psi = (\partial_{\mu} \psi)^{\dagger} \gamma^0 \gamma^{\mu} \psi                 \\
                                                                               & = \partial_{\mu} (\psi^{\dagger} \gamma^0 \gamma^{\mu} \psi) - \psi^{\dagger} \gamma^0 \gamma^{\mu} \partial_{\mu} \psi = - \psi^{\dagger} \gamma^0 \gamma^{\mu} \partial_{\mu} \psi= - \overline{\psi} \gamma^{\mu} \partial_{\mu} \psi.
              \end{aligned}
          \]
          where we have used the usual trick of integrating by parts and then neglecting the boundary term computed at infinite times and distances, since this lagrangian is integrated to obtain the action. As we have shown, the kinetic term is anti-hermitian, so multiplying it by \(i\) we get a hermitian expression, ensuring the lagrangian to be real.
    \item Since the action is dimensionless, we can deduce the mass dimension of the field \(\psi\):
          \[
              \begin{aligned}
                  [S] = 0, \quad [\mathrm{d}^4 x]=-4 \quad \implies [\mathcal{L}] = 4, \\
                  [\partial_\mu] = 1,\quad [m]= 1 \quad \implies [\psi]=[\overline{\psi}] = \frac{3}{2}.
              \end{aligned}
          \]
          So this field has a different mass dimention from KG scalar field, which had: \([\psi]=1\).
    \item KG Lagrangian contains two derivatives \(\partial_\mu \psi \partial^{\mu} \psi\) in the kinetic term, while Dirac's only one \(\overline{\psi} i \gamma^{\mu} \partial_{\mu} \psi\): The KG lagrangian was of the second order, while Dirac's of the first; it changes the order of the equations of motion.
    \item Upon quantization the Dirac theory will describe particles/antiparticles (for whose description we needed a complex field for the charge dof, which is our case) with spin \(\tfrac12\) and mass \(m\); in principle we have four complex dof, which are LH/RH and spin up/down (doubled because we need 4 real dof for the particle description and 4 real dof for the antiparticle one).
\end{itemize}

\section{Dirac Equation}

The Dirac Lagrangian is
\[
    \mathcal{L} = \overline{\psi} \left(i \gamma^{\mu} \partial_{\mu} - m\right) \psi,
\]
where \(\overline{\psi} = \psi^{\dagger} \gamma^0\) is the Dirac conjugate. We can find the equations of motion using the Euler-Lagrange equations for fields \eqref{eq:euler_lagrange_fields}, treating \(\psi\) and \(\overline{\psi}\) as independent fields:
\[
    \frac{\partial \mathcal{L}}{\partial \psi} = \partial_{\mu} \left(\frac{\partial \mathcal{L}}{\partial (\partial_{\mu} \psi)}\right), \quad \frac{\partial \mathcal{L}}{\partial \overline{\psi}} = \partial_{\mu} \left(\frac{\partial \mathcal{L}}{\partial (\partial_{\mu} \overline{\psi})}\right);
\]
in the second equation we have to notice that \(\frac{\partial}{\partial \overline{\psi}} \mathcal{L} = 0\), since \( \frac{\partial}{\partial (\partial_\mu \overline{\psi})} \mathcal{L} = 0 \). This implies that if we differentiate the lagrangian with respect to \(\overline{\psi}\) we get
\[
    \frac{\partial \mathcal{L}}{\partial \overline{\psi}} = \left(i \gamma^{\mu} \partial_{\mu} - m\right) \psi = 0,
\]
which is the \textbf{Dirac equation}:
\begin{equation}
    \left(i \gamma^{\mu} \partial_{\mu} - m\right) \psi = 0,
    \label{eq:Dirac_equation}
\end{equation}
which is a first order partial differential equation for the spinor field \(\psi(x)\).

We can also derive the equation of motion by considering the derivatives with respect to \(\psi\), in order to get the conjugate equation:
\[
    \frac{\partial \mathcal{L}}{\partial \psi} = - m \overline{\psi}, \quad \frac{\partial \mathcal{L}}{\partial (\partial_{\mu} \psi)} = i \overline{\psi} \gamma^{\mu} \implies \partial_{\mu} \left(\frac{\partial \mathcal{L}}{\partial (\partial_{\mu} \psi)}\right) = i (\partial_{\mu} \overline{\psi}) \gamma^{\mu},
\]
so that the Euler-Lagrange equation gives
\[
    - m \overline{\psi} = i (\partial_{\mu} \overline{\psi}) \gamma^{\mu} \implies \overline{\psi} ( - i \overleftarrow{\partial}_{\mu} \gamma^{\mu} - m) = 0,
\]
which is the \textbf{conjugate Dirac equation}:
\begin{equation}
    \overline{\psi} ( - i \overleftarrow{\partial}_{\mu} \gamma^{\mu} - m) = 0,
    \label{eq:conjugate_Dirac_equation}
\end{equation}
where the arrow on the derivatives tells us it is to be applied to the left.

We can build such an equation of motion just thanks to the presence of \(\gamma^{\mu}\) which grants lorentz invariance. Dirac equation is a \textbf{first order differential equation}, while for KG we could only get a second order one. Furthermore KG is a scalar equation, while Dirac is a spinor equation (vectorial in spinor space, with 4 components).
The presence of gamma matrices in Dirac equation is crucial: they ensure lorentz invariance with their transformation properties, and they allow to have a first order equation, since they contract with the derivative index \(\mu\) in \(\partial_\mu \psi\).

\textbf{Lorentz invariance} determines the form of Dirac equation: if we want to describe a scalar particle, we need a second order equation (KG), while if we want to describe a spin \(\tfrac12\) particle (spinor) a first order equation (Dirac) is sufficient.

Notice that dirac equation mixes components of the spinor, since \(\gamma^{\mu}\) are \(4\times 4\) matrices:
\[
    \psi = \begin{pmatrix}
        \psi_1 \\
        \psi_2 \\
        \psi_3 \\
        \psi_4
    \end{pmatrix}, \quad \gamma^0 = \begin{pmatrix}
        0            & \mathbb{I}_2 \\
        \mathbb{I}_2 & 0
    \end{pmatrix}, \quad \gamma^i = \begin{pmatrix}
        0         & \sigma^i \\
        -\sigma^i & 0
    \end{pmatrix}.
\]
So in components in the spinor space, Dirac equations reads:
\[
    i \begin{pmatrix}
        \partial_t \psi_3 \\
        \partial_t \psi_4 \\
        \partial_t \psi_1 \\
        \partial_t \psi_2
    \end{pmatrix} + i \begin{pmatrix}
        \partial_x \psi_4  \\
        \partial_x \psi_3  \\
        -\partial_x \psi_2 \\
        -\partial_x \psi_1
    \end{pmatrix} + i \begin{pmatrix}
        -\partial_y \psi_4 \\
        \partial_y \psi_3  \\
        \partial_y \psi_2  \\
        -\partial_y \psi_1
    \end{pmatrix} + i \begin{pmatrix}
        \partial_z \psi_3  \\
        -\partial_z \psi_4 \\
        -\partial_z \psi_1 \\
        \partial_z \psi_2
    \end{pmatrix} - m \begin{pmatrix}
        \psi_1 \\
        \psi_2 \\
        \psi_3 \\
        \psi_4
    \end{pmatrix} = 0.
\]
In general it becomes a system of four coupled first order differential equations, mixing the four components of the spinor: for example the first equation is
\[
    i \partial_t \psi_3 + i \partial_x \psi_4 - i \partial_y \psi_4 + i \partial_z \psi_3 - m \psi_1 = 0.
\]

However each component \(\psi_{\alpha}(x)\) satosfy the KG equation, since in effect each components describe a degree of freedom of a relativistic scalar particle with mass \(m\): if we multiply Dirac equation by \((i \gamma^{\mu} \partial_{\mu} + m)\) from the left (it is still valid since zero multiplied by anything remains zero), we get
\[
    \begin{aligned}
        (i \gamma^{\mu} \partial_{\mu} + m)(i \gamma^{\nu} \partial_{\nu} - m) \psi(x) = 0,                       \\
        \left(-\gamma^{\mu} \gamma^{\nu} \partial_{\mu} \partial_{\nu} - m^2\right) \psi(x) = 0,                  \\
        \left(-\frac{1}{2}\{\gamma^{\mu},\,\gamma^{\nu}\} \partial_{\mu} \partial_{\nu} - m^2\right) \psi(x) = 0, \\
        \left(-\eta^{\mu \nu} \partial_{\mu} \partial_{\nu} - m^2\right) \psi(x) = 0,                             \\
        \implies (\Box + m^2) \psi(x) = 0.
    \end{aligned}
\]
In terms of matrices we have used the \textbf{clifford algebra} to simplify the product of gamma matrices, since we can insert the anticommutator for symmetry of \(\partial_\mu \partial_\nu\):
\[
    \gamma^{\mu} \gamma^{\nu} \partial_{\mu} \partial_{\nu} = \frac{1}{2} \gamma^{\mu} \gamma^{\nu} \left\{\partial_{\mu}, \partial_{\nu}\right\} = \frac{1}{2} \{\gamma^{\mu},\, \gamma^{\nu}\} \partial_{\mu} \partial_{\nu} = \eta^{\mu \nu} \partial_{\mu} \partial_{\nu},
\]
just renaming indces. Thus each component of the Dirac spinor satisfies the KG equation:
\[
    (\Box + m^2) \psi_{\alpha} (x) = 0 \quad \forall \alpha = 1,\,2,\,3,\,4.
\]

\subsection{Chiral Spinors}

Chirality means that Dirac representation \((\tfrac12,\,0) \oplus (0,\,\tfrac12)\) can be decomposed into two irreducible representations, \textbf{Weyl representation}, of the Lorentz group (for more information on the representations see section \ref{sec:finite_dimensional_representations}). Weyl or Chiral spinors are two-component objects (complex dof) with different transformation properties:
\begin{itemize}
    \item \textbf{Left-Handed} weyl spinors: \(\psi_L \sim (\tfrac12,\,0)\), transform under \(S_L\) only (as in eq. \eqref{eq:LH_Weyl_spinor_transformation});
    \item \textbf{Right-Handed} weyl spinors: \(\psi_R \sim (0,\,\tfrac12)\), transform under \(S_R\) only (as in eq. \eqref{eq:RH_Weyl_spinor_transformation}).
\end{itemize}

We can write the Dirac spinor as a combination of two weyl spinors:
\[
    \psi_D = \begin{pmatrix}
        \psi_L^{(w)} \\
        \psi_R^{(w)}
    \end{pmatrix}, \quad \psi_L^{(w)} \xleftrightarrow{\text{Parity}} \psi_R^{(w)}.
\]

\paragraph{Chirality operator.} In order to project out the two chiral components from a Dirac spinor we can introduce the chirality operator
\[
    \gamma^5 = i \gamma^0 \gamma^1 \gamma^2 \gamma^3 = \begin{pmatrix}
        -\mathbb{I}_2 & 0            \\
        0             & \mathbb{I}_2
    \end{pmatrix}, \quad (\gamma^5)^2 = \mathbb{I}_4,
\]
using which we can define projectors able to select the desired chiral component from the Dirac spinor (as we did in eq. \eqref{eq:chiral_projectors} for spinors):
\[
    \begin{aligned}
        \mathbb{P}_L \psi = \frac{1 - \gamma^5}{2} \psi = \psi_L = \begin{pmatrix}
                                                                       \psi_L^{(w)} \\
                                                                       0
                                                                   \end{pmatrix}, \\
        \mathbb{P}_R \psi = \frac{1 + \gamma^5}{2} \psi = \psi_R = \begin{pmatrix}
                                                                       0 \\
                                                                       \psi_R^{(w)}
                                                                   \end{pmatrix}.
    \end{aligned}
\]
Thus the dirac spinor is the sum of the two chiral components \(\psi_D = \psi_L + \psi_R\) and the projectors satisfy the usual properties:
\[
    \begin{dcases}
        \mathbb{P}_L^2 = \mathbb{P}_L, \\
        \mathbb{P}_L^{\dagger} = \mathbb{P}_L,
    \end{dcases} \quad \begin{dcases}
        \mathbb{P}_R^2 = \mathbb{P}_R, \\
        \mathbb{P}_R^{\dagger} = \mathbb{P}_R,
    \end{dcases} \quad \begin{dcases}
        \mathbb{P}_L \mathbb{P}_R = \mathbb{P}_R \mathbb{P}_L = 0, \\
        \mathbb{P}_L + \mathbb{P}_R = \mathbb{I}_4.
    \end{dcases}
\]
Note that the eigenvalues of \(\gamma^5\) are \(\pm 1\), so the chirality operator measures the chirality of a spinor:
\[
    \begin{dcases}
        \gamma^5 \psi_L = (-1) \psi_L, \\
        \gamma^5 \psi_R = (+1) \psi_R.
    \end{dcases}
\]
We will see how chirality is related to helicity in the massless limit.
This decomposition is very useful in the standard model, since weak interactions only involve LH particles and RH antiparticles; so we can use chiral projectors to select the interacting components.

\subsubsection{Lagrangian and Chirality}

To understand better, let's write the Dirac lagrangian
\[
    \mathcal{L} = \overline{\psi} \left(i \gamma^{\mu} \partial_{\mu} - m\right) \psi,
\]
in terms of chiral components (which is very useful for \textit{statistical field theory} and the \textit{standard model}, also for understanding the \textit{massless limit}): we will exploit the vector and axial vector currents, defined as \(\overline{\psi} \gamma^{\mu} \psi\) and \(\overline{\psi} \gamma^{\mu} \gamma^5 \psi\) respectively, but we will need only the first one to rewrite the kinetic term; then we will use the properties of the chirality operator to rewrite the mass term. Let's proceed step by step:
\begin{enumerate}
    \item Starting from the \textbf{vector current} which is contracted with the derivative in the kinetic term:
          \[
              \begin{aligned}
                  \overline{\psi} \gamma^{\mu} \psi & = \psi^{\dagger} \gamma^0 \gamma^{\mu} \left(\psi_L + \psi_R\right) = \psi^{\dagger} \gamma^0 \gamma^{\mu} \left(\mathbb{P}_{L}^2  + \mathbb{P}_R^2\right) \psi      \\
                                                    & = \psi^{\dagger} \gamma^0 \gamma^{\mu} \left(\frac{1 - \gamma^5}{2}\right)^2 \psi + \psi^{\dagger} \gamma^0 \gamma^{\mu} \left(\frac{1 + \gamma^5}{2}\right)^2 \psi, \\
              \end{aligned}
          \]
          where, since the chirality matrix \(\gamma^5\) anticommutes with all gamma matrices \(\{\gamma^5,\, \gamma^{\mu}\} = 0\), we can use the identity \(\gamma^0 \gamma^{\mu}  \gamma^5 = - \gamma^{0} \gamma^5 \gamma^{\mu} = \gamma^5 \gamma^0 \gamma^{\mu}\) to obtain:
          \[
              \overline{\psi} \gamma^{\mu} \psi = \psi^{\dagger} \left(\frac{1 - \gamma^5}{2}\right) \gamma^0 \gamma^{\mu} \left(\frac{1 - \gamma^5}{2}\right) \psi + \psi^{\dagger} \left(\frac{1 + \gamma^5}{2}\right) \gamma^0 \gamma^{\mu} \left(\frac{1 + \gamma^5}{2}\right) \psi.
          \]
          Now exploiting hermitianity of the gamma matrices we can rewrite the projectors acting on \(\psi^{\dagger}\) as \(\psi^{\dagger} \mathbb{P}_{R/L} = \left(\mathbb{P}_{R/L} \psi\right)^{\dagger}\) to finally get (after reapplying \(\gamma^0\) matrices to get \(\overline{\psi}\)):
          \[
              \overline{\psi} \gamma^{\mu} \psi = \overline{\psi}_L \gamma^{\mu} \psi_L + \overline{\psi}_R \gamma^{\mu} \psi_R.
          \]
          This found vector current does not mix chiral components, it is parity invariant and we will use this to rewrite the lagrangian. However, this current is not good for electroweak interactions and interpretations of \(\mathrm{SU}(2)_L\) interactions: electroweak currents violate parity, this vector current is invariant under its action.

    \item Moving on, we can try to insert a \(\gamma^5\) in the vector current: in this way maybe we can get a parity violation and obtain a better candidate for weak interactions. So we consider
          \[
              \overline{\psi} \gamma^{\mu} \gamma^{5} \psi = \psi^{\dagger} \gamma^0 \gamma^{\mu} \gamma^5 \left(\psi_L + \psi_R\right) = \psi^{\dagger} \gamma^0 \gamma^{\mu} \gamma^5 \left(\mathbb{P}_L^2 + \mathbb{P}_R^2\right) \psi.
          \]
          We can easily compute that \(\gamma^5 \mathbb{P}_{L/R} = \mp \mathbb{P}_{L/R}\) (the first sign is for the left projector, the second for the right), so that we can write
          \[
              \overline{\psi} \gamma^{\mu} \gamma^5 \psi = \psi^{\dagger} \gamma^0 \gamma^{\mu} \left(- \frac{1 - \gamma^5}{2}\right) \psi_L + \psi^{\dagger} \gamma^0 \gamma^{\mu} \left(\frac{1 + \gamma^5}{2}\right) \psi_R = \overline{\psi}_R \gamma^{\mu} \psi_R - \overline{\psi}_L \gamma^{\mu} \psi_L,
          \]
          which now, following the same steps as before, bring us to the expression for another current, called \textbf{axial vector current}: it changes sign under parity, but experiments show that weak interactions \textit{violate parity maximally}, so this is not enough.

    \item Now we can combine the two previous currents to get the \textbf{V-A current}:
          \[
              \frac{1}{2} (V-A) = \overline{\psi} \gamma^{\mu} \frac{1 - \gamma^5}{2} \psi = \overline{\psi}_L \gamma^{\mu} \psi_L.
          \]
          V-A current violates parity maximally, since it involves only left-handed components, and it makes explicit the fact that only LH components enter the interaction: it is a good candidate for EW interactions.

    \item Expanding the \textbf{mass term} instead:
          \[
              \begin{aligned}
                  \overline{\psi} \psi & = \psi^{\dagger} \gamma^0 (P_R^2 + P_L^2) \psi = \psi^{\dagger} \gamma^0 \left(\frac{1 + \gamma^5}{2}\right)^2 \psi_L + \psi^{\dagger} \gamma^0 \left( \frac{1 - \gamma^5}{2} \right)^2 \psi_R                \\
                                       & = \psi^{\dagger}\left(\frac{1 - \gamma^5}{2}\right) \gamma^0 \left(\frac{1 + \gamma^5}{2}\right) \psi + \psi^{\dagger}\left(\frac{1 + \gamma^5}{2}\right) \gamma^0  \left(\frac{1 - \gamma^5}{2}\right) \psi,
              \end{aligned}
          \]
          where we have used the identity \(\gamma^0 \mathbb{P}_{L/R} = \mathbb{P}_{R/L} \gamma^0\) to find that the mass term mixes chiral components:
          \[
              \overline{\psi} \psi = \overline{\psi}^{\dagger}_L \gamma^0 \psi_R + \overline{\psi}^{\dagger}_R \gamma^0 \psi_L = \overline{\psi}_L \psi_R + \overline{\psi}_R \psi_L.
          \]
\end{enumerate}
Now we have everything to write the Dirac lagrangian in terms of chiral components:
\begin{equation}
    \mathcal{L} = \overline{\psi}_L i \gamma^{\mu} \partial_{\mu} \psi_L + \overline{\psi}_R i \gamma^{\mu} \partial_{\mu} \psi_R - m \left(\overline{\psi}_L \psi_R + \overline{\psi}_R \psi_L\right).
    \label{eq:dirac_lagrangian_chiral_components}
\end{equation}
We can see that \textit{in the massless limit the two chiral components decouple}, and we get two independent Weyl equations for each chiral component: this is important, since kinetic terms evolves independently the two chiral components, while the mass term allow us to perform a boost and mix the two chiralities: since the particle is massive there will be frames where the particle is RH and others where it is seen as LH.

If you take an electron whith both the chiral components, only the LH component will interact weakly, while the RH will not; but since the electron is massive you can always boost to a frame where the electron appears as RH, so both components are needed to describe a massive fermion. When we see only the RH component, the electron will not seem to interact weakly.

\subsubsection{Dirac Equation in Weyl Components}

If we write the Dirac equation in terms of chiral components we get:
\[
    i \left( \gamma^0 \partial_0 + \gamma^i \partial_i - m \right) \psi = 0, \quad \gamma^0 = \begin{pmatrix}
        0            & \mathbb{I}_2 \\
        \mathbb{I}_2 & 0
    \end{pmatrix}, \quad \gamma^i = \begin{pmatrix}
        0         & \sigma^i \\
        -\sigma^i & 0
    \end{pmatrix},
\]
which reads
\[
    i \begin{pmatrix}
        0                                     & \partial_t + \bs{\sigma} \cdot \nabla \\
        \partial_t - \bs{\sigma} \cdot \nabla & 0                                     \\
    \end{pmatrix} \begin{pmatrix}
        \psi_L^{(w)} \\
        \psi_R^{(w)}
    \end{pmatrix} - m \begin{pmatrix}
        \psi_L^{(w)} \\
        \psi_R^{(w)}
    \end{pmatrix} = 0,
\]
which is a system of two coupled first order coupled differential equations
\[
    \begin{dcases}
        i (\partial_t + \bs{\sigma} \cdot \nabla) \psi_R^{(w)} - m \psi_L^{(w)} = 0, \\
        i (\partial_t - \bs{\sigma} \cdot \nabla) \psi_L^{(w)} - m \psi_R^{(w)} = 0. \\
    \end{dcases}
\]
In the massless limit the two equations decouple and we get two Weyl equations:
\begin{equation}
    \begin{dcases}
        i (\partial_t + \bs{\sigma} \cdot \nabla) \psi_R^{(w)} = 0, \\
        i (\partial_t - \bs{\sigma} \cdot \nabla) \psi_L^{(w)} = 0, \\
    \end{dcases}
    \label{eq:weyl_equations}
\end{equation}
and we call this last set of equations the \textbf{Weyl equations} for massless fermions.
Now it is clearer the meaning of Left and Right-Handed Weyl spinors: in terms of operators (in natural units) we have
\[
    i \partial_t = \hat{H},\quad -i \nabla = \hat{\mathbf{p}}, \quad \bs{\sigma} = \hat{\mathbf{S}},
\]
so that we can see the last two equations as
\[
    \begin{dcases}
        (\hat{H} - \hat{\mathbf{S}} \cdot \hat{\mathbf{p}}) \psi_R^{(w)} = 0, \\
        (\hat{H} + \hat{\mathbf{S}} \cdot \hat{\mathbf{p}}) \psi_L^{(w)} = 0. \\
    \end{dcases}
\]
Now, dropping momentaneously the hat notation for operators (so that we can use it to indicate versors), we find clearly that the weyl components are eigenstate of the \textbf{helicity}: for massless particles indeed \(E = \vert \mathbf{p} \vert\) so that
\[
    \begin{dcases}
        \vert \mathbf{p} \vert \psi_R^{(w)} = \mathbf{S} \cdot \mathbf{p} \, \psi_R^{(w)}, \\
        \vert \mathbf{p} \vert \psi_L^{(w)} = - \mathbf{S} \cdot \mathbf{p} \, \psi_L^{(w)},
    \end{dcases} \iff \begin{dcases}
        \mathbf{S} \cdot \hat{\mathbf{p}} \, \psi_R^{(w)} = (+1) \psi_R^{(w)}, \\
        \mathbf{S} \cdot \hat{\mathbf{p}} \, \psi_L^{(w)} = (-1) \psi_L^{(w)},
    \end{dcases}
\]
where the helicity operator is defined as
\begin{equation}
    h = \frac{\mathbf{S} \cdot \mathbf{p}}{\vert \mathbf{p} \vert} = \mathbf{S} \cdot \hat{\mathbf{p}}.
    \label{eq:helicity_operator}
\end{equation}
Since the eigenvalues are \(\pm 1\), we can affirm that in the massless limit chirality and helicity coincide: they have same eigenstates and eigenvalues.

If the neutrinos were massless, only the LH component would exist, since only that interacts weakly; but since neutrinos have a small mass, both chiral components exist, even if the RH component has never been observed (it interacts only gravitationally, so it is very difficult to detect it).

We now want to prove that under parity we can pass from one weyl component to the other:
\[
    \psi_L^{(w)} \xleftrightarrow{\text{Parity}} \psi_R^{(w)}.
\]
Let's start from the Dirac equation:
\[
    \begin{aligned}
        \left(i \gamma^{\mu}\partial_{\mu} - m\right) \psi(x) = 0, \\
        \left(i \gamma^{0} \partial_{0} + i \gamma^{i} \partial_{i} - m\right) \psi(t,\,\mathbf{x}) = 0,
    \end{aligned}
\]
which, given the action of the parity:
\[
    \partial_t \xrightarrow{\text{Parity}} \partial_t, \quad \nabla \xrightarrow{\text{Parity}} - \nabla,
\]
then transforms the Dirac equation as
\[
    \left(i \gamma^{0} \partial_{0} - i \gamma^{i} \partial_{i} - m\right) \psi^{\prime}(t,\,\mathbf{x}^{\prime}) = 0, \quad \mathbf{x}^{\prime} = - \mathbf{x}.
\]
But how does \(\psi\) transform under parity? We want to write the transformed spinor \(\psi^{\prime}(t,\,\mathbf{x}^{\prime})\) in terms of the original one \(\psi(t,\,\mathbf{x})\). Since gamma matrices satisfy
\[
    (\gamma^0)^2 = \mathbb{I}_4, \quad \{\gamma^0,\,\gamma^i\} = 0,
\]
we can insert \((\gamma^0)^2\) in the transformed Dirac equation to get
\[
    \left(i \gamma^{0} \partial_{0} - i \gamma^{0} \gamma^{0} \gamma^{i} \partial_{i} - \gamma^{0} \gamma^{0}m\right) \psi^{\prime}(t,\,\mathbf{x}^{\prime})= 0,
\]
which we can rewrite as
\[
    \gamma^{0} \left(i \partial_{0} + i \gamma^{i} \gamma^{0} \partial_{i} - \gamma^{0} m\right) \psi^{\prime}(t,\,\mathbf{x}^{\prime}) = 0.
\]
where we have grouped \(\gamma^0\) on the left and changed the sign of the second term swapping \(\gamma^0\) and \(\gamma^i\). Now, inserting \((\gamma^0)^2\) one last time in the first term, we can group another \(\gamma^0\) on the right:
\[
    \gamma^{0} \left(i \gamma^{0} \partial_{0} + i \gamma^{i} \partial_{i} - m\right) \gamma^{0} \psi^{\prime}(t,\,\mathbf{x}^{\prime}) = 0,
\]
which, simplifying \(\gamma^0\) on the left, and comparing with the original Dirac equation, tells us that the transformed spinor must respect
\[
    \psi^{\prime}(t,\,\mathbf{x}^{\prime}) = \gamma^{0} \psi(t,\,\mathbf{x}).
\]
Thus under parity the Dirac spinor transforms as
\[
    \psi =\begin{pmatrix}
        \psi_L^{(w)} \\
        \psi_R^{(w)}
    \end{pmatrix} \xrightarrow{\text{Parity}}
    \psi^{\prime} = \begin{pmatrix}
        \psi_L^{\prime (w)} \\
        \psi_R^{\prime (w)}
    \end{pmatrix} = \begin{pmatrix}
        0            & \mathbb{I}_2 \\
        \mathbb{I}_2 & 0
    \end{pmatrix} \begin{pmatrix}
        \psi_L^{(w)} \\
        \psi_R^{(w)}
    \end{pmatrix},
\]
so that
\[
    \begin{dcases}
        \psi_L^{\prime (w)} = \psi_R^{(w)}, \\
        \psi_R^{\prime (w)} = \psi_L^{(w)}.
    \end{dcases}
\]

As we wanted to prove, under parity the two weyl components are indeed exchanged:
\[
    \psi_L^{(w)} \xleftrightarrow{\text{Parity}} \psi_R^{(w)}.
\]
