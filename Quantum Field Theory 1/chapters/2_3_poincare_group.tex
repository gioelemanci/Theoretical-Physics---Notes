\section{The Poincaré Group}

The \textbf{Poincaré group} is the fundamental symmetry group of relativistic spacetime.
It extends the Lorentz group by including spacetime translations, and thus encodes the full structure of special relativity:
\[
    x^{\mu} \to x'^{\mu} = \Lambda^{\mu}_{\ \nu} x^{\nu} + a^{\mu},
\]
where \(\Lambda\) is a Lorentz transformation and \(a^\mu\) a spacetime translation vector.
Hence, it is also called the \textit{inhomogeneous proper orthochronous Lorentz group}, often denoted as \(ISO^+(1,3)\).
In quantum field theory, the Poincaré group plays a central role: every isolated physical system must admit a unitary representation of this group on its Hilbert space, ensuring the invariance of the theory under rotations, boosts, and translations.

From an algebraic point of view, the Poincaré group is a \textbf{ten-parameter Lie group} generated by the four-momentum operators \(P_\mu\) (associated with translations) and the six Lorentz generators \(M_{\mu\nu}\) (associated with rotations and boosts), which satisfy the \textit{Poincaré algebra}:
\[
    [P^\mu, P^\nu] = 0, \qquad
    [M^{\mu\nu}, P^\sigma] = i \left(P^\mu \eta^{\nu\sigma} - P^\nu \eta^{\mu\sigma}\right),
\]
together with the usual Lorentz commutation relations for \(M^{\mu\nu}\).
The fact that the momentum operators commute among themselves shows that the translation subgroup is \textbf{abelian}. Physically, this reflects that momentum is additive and that free particles can be simultaneously diagonalized with respect to their energy and momentum. This also connects to gauge theories: interactions mediated by abelian gauge bosons (like the photon) do not self-interact, while non-abelian gauge bosons (like \(W\) and \(Z\) bosons) have self-interactions due to the non-commuting structure of their symmetry generators.

\paragraph{Field representation.} In the \textit{field representation}, the Poincaré generators act on fields as differential operators. In particular, spacetime translations are represented by
\[
    P^\mu = i \frac{\partial}{\partial x_\mu} = i \partial^\mu,
\]
so that a translation of a field \(\Phi(x)\) is realized as
\[
    \Phi(x) \to \Phi'(x) = e^{-i a_\mu P^\mu} \Phi(x) = \Phi(x + a).
\]
This makes explicit how continuous spacetime symmetries are represented by \textit{infinitesimal differential operators} acting on classical or quantum fields.

\subsection{Representations: one-particle Hilbert space}

The representation theory of the Poincaré group on the \textit{one-particle Hilbert space} provides the foundation for the relativistic description of quantum particles.
Each irreducible unitary representation corresponds to a distinct type of elementary particle, characterized by the eigenvalues of the Casimir operators that we are about to introduce.
This framework, originally developed by Wigner, unifies the treatment of spin and momentum in relativistic quantum mechanics and serves as the bridge between the abstract group-theoretic structure and the physical notion of particles.

\subsubsection{Casimir operators}

Recall that operators \(A\) that commute with a generator \(T\) of a symmetry represent conserved quantities under that transformation:
\[
    [A, T] = 0 \implies AT = TA.
\]
Explicitly, if \(\ket{\psi}\) is an eigenstate of \(A\):
\[
    A \ket{\psi} = a \ket{\psi},
\]
then under the transformation generated by \(T\):
\[
    \ket{\psi} \to \ket{\psi'} = e^{i \alpha T} \ket{\psi} = \left(\mathbb{I} + i \alpha T + \mathcal{O}(\alpha^2) \right) \ket{\psi},
\]
we have
\[
    \begin{aligned}
        A \ket{\psi'} & = A \left(\mathbb{I} + i \alpha T \right) \ket{\psi} = A \ket{\psi} + i \alpha A T \ket{\psi}      \\
                      & = a \ket{\psi} + i \alpha T A \ket{\psi} = a (\mathbb{I} + i \alpha T) \ket{\psi} = a \ket{\psi'}.
    \end{aligned}
\]
This shows that the eigenvalue \(a\) is invariant under the transformation: the conserved quantity remains the same.

An operator that commutes with all generators of a group is called a \textbf{Casimir operator}. Casimir operators are important because their eigenvalues are invariants that can be used to label irreducible representations, i.e., multiplets of states sharing the same conserved quantities.

For the Poincaré group, there are two independent Casimir operators:
\begin{equation}
    C_1 = P^\mu P_\mu, \qquad C_2 = W^\mu W_\mu,
    \label{eq:casimir_operators_poincare}
\end{equation}
where \(W_\mu\) is the \textbf{Pauli-Lubanski vector}:
\begin{equation}
    W_\mu = +\frac{1}{2} \epsilon_{\mu\nu\rho\sigma} P^{\nu} M^{\rho\sigma}.
    \label{eq:Pauli_Lubanski_vector}
\end{equation}

The physical interpretation is:

\begin{itemize}
    \item \(C_1 = P^\mu P_\mu\) corresponds to the squared mass of the particle. Since \(P_\mu\) commutes with itself and with all Lorentz generators, all states in an irreducible representation have the same mass.
    \item \(C_2 = W^\mu W_\mu\) is related to the intrinsic spin (for massive particles) or helicity (for massless particles). It provides a Lorentz-invariant characterization of the particle's rotational degrees of freedom.
\end{itemize}

Hence, different irreducible representations of the Poincaré group are classified by the eigenvalues of \(C_1\) and \(C_2\), giving a systematic way to organize particle types in relativistic quantum mechanics.

\subsubsection{Massive representation}

For \textbf{massive particles} (\(m > 0\)), the first Casimir operator of the Poincaré group satisfies
\[
    C_1 = P^{\mu} P_{\mu} = P^2, \quad \text{with eigenvalue } p^2 = E_{\mathbf{p}}^2 - |\mathbf{p}|^2 = m^2 \neq 0.
\]
This defines a family of states all having the same mass \(m\), obtained by applying the full set of Poincaré transformations to a chosen reference four-momentum \(p^{\mu}\). The resulting set of states can be denoted by \(\{p^\mu\}\), and a generic state can be initially labeled as \(\ket{m,\,p^\mu}\).

To classify these states further, we consider the second Casimir operator
\[
    C_2 = W^\mu W_\mu,
\]
where \(W_\mu\) is the Pauli-Lubanski vector defined in eq. \eqref{eq:Pauli_Lubanski_vector}.
Its eigenvalues are associated with intrinsic spin degrees of freedom of the particle.

\paragraph{Little group.}
To understand the internal structure, we fix \(p^\mu\) and look for all Poincaré generators that commute with \(P^\mu\). This subset of generators forms a subgroup of the Poincaré group called the \textit{little group}: this subgroup leaves the four-momentum \(p^\mu\) invariant and its representations classify the internal degrees of freedom of the particle, such as spin.

It is defined as the subgroup of Lorentz transformations \(\Lambda\) that leaves this vector unchanged:
\[
    \Lambda^{\mu}_{\ \nu} p^{\nu} = p^{\mu}.
\]

Importantly, Wigner showed that the structure of the little group does not depend on the specific choice of \(p^\mu\) within its equivalence class \(\{p^\mu\}\), allowing us to select the simplest frame for calculations: the rest frame
\[
    p^\mu = (m, 0, 0, 0).
\]

In the rest frame, it turns out that only the spatial rotation generators
\[
    J_i = \frac{1}{2} \epsilon_{ijk} M^{jk}, \quad i=1,2,3,
\]
commute with \(P^\mu\), so the little group is isomorphic to \(SO(3)\), the group of spatial rotations. In more complicated situations, finding the little group may require evaluating nontrivial commutators, but for massive particles in the rest frame the identification is straightforward.

\paragraph{Pauli-Lubanski vector.}
The Pauli-Lubanski vector components in the rest frame are
\[
    \begin{aligned}
        W_0 & = \frac{1}{2} \epsilon_{0\nu\rho\sigma} P^\nu M^{\rho\sigma} = \frac{1}{2} \epsilon_{00\rho\sigma} m M^{\rho\sigma} = 0,                              \\
        W_i & = \frac{1}{2} \epsilon_{i\nu\rho\sigma} P^\nu M^{\rho\sigma} = \frac{1}{2} \epsilon_{i0jk} m M^{jk} = - \frac{m}{2} \epsilon_{0ijk} M^{jk} = - m J_i,
    \end{aligned}
\]
where the antisymmetry of the Levi-Civita tensor ensures that terms with repeated indices vanish. We thus recover the familiar rotation generators \(J_i\), so we can hypothesize that \(W_\mu\) generates transformations which leave \(p^\mu\) invariant:
\[
    \begin{aligned}
        [W_\mu,\, P_\nu] & = \frac{1}{2} \epsilon_{\mu\alpha\beta\gamma} [P^\alpha M^{\beta\gamma},\, P_\nu] = \frac{1}{2} \epsilon_{\mu\alpha\beta\gamma} \eta_{\nu \rho} [P^\alpha M^{\beta\gamma},\, P^\rho]                                                                         \\
                         & = \frac{1}{2} \epsilon_{\mu\alpha\beta\gamma} \eta_{\nu \rho} \left( P^\alpha [M^{\beta\gamma},\, P^\rho] + [P^\alpha,\, P^\rho] M^{\beta\gamma} \right)                                                                                                     \\
                         & = \frac{1}{2} \epsilon_{\mu\alpha\beta\gamma} \eta_{\nu \rho} \left( P^\alpha i (P^{\beta} \eta^{\gamma \rho} - P^{\gamma} \eta^{\beta \rho}) + 0 M^{\beta\gamma} \right)                                                                                    \\
                         & =\frac{i}{2} \epsilon_{\mu\alpha\beta\gamma} P^\alpha \left( P^{\beta} \eta_{\nu \rho} \eta^{\gamma \rho} - P^{\gamma} \eta_{\nu \rho} \eta^{\beta \rho} \right)                                                                                             \\
                         & = \frac{i}{2} \epsilon_{\mu\alpha\beta\gamma} P^\alpha (P^{\beta} \delta_{\nu}^{\ \gamma} - P^{\gamma} \delta_{\nu}^{\ \beta}) = \frac{i}{2} \epsilon_{\mu\alpha\beta\nu} P^\alpha P^{\beta} - \frac{i}{2} \epsilon_{\mu\alpha\nu\gamma} P^\alpha P^{\gamma} \\
                         & = \frac{i}{2} \epsilon_{\mu\alpha\beta\nu} P^\alpha P^{\beta} + \frac{i}{2} \epsilon_{\mu\alpha\gamma\nu} P^\alpha P^{\gamma} = i \epsilon_{\mu\alpha\beta\nu} P^\alpha P^{\beta} = 0,
    \end{aligned}
\]
since \(P^\alpha P^{\beta}\) is symmetric while \(\epsilon_{\mu\alpha\beta\nu}\) is antisymmetric for exchange of \(\alpha\) and \(\beta\). This results hold for both massive and massless particles.

Consequently, the second Casimir operator reduces to
\[
    C_2 = W^\mu W_\mu = - m^2 \mathbf{J}^2,
\]
with \(\mathbf{J}^2\) the squared angular momentum operator. Its eigenvalues are
\[
    j(j+1), \quad j = 0, \frac{1}{2}, 1, \frac{3}{2}, \dots,
\]
while the eigenvalues of \(J_3\) are
\[
    j_3 = -j, -j+1, \dots, j-1, j.
\]

\paragraph{Labeling the multiplet.}
Each state in the massive representation can thus be labeled as
\[
    \ket{m,\, j,\, j_3,\, p^\mu},
\]
where \(m\) is the mass, \(j\) the total spin, \(j_3\) the spin projection along a chosen axis, and \(p^\mu\) the four-momentum. The first three labels classify the multiplet, encoding internal spin degrees of freedom, while \(p^\mu\) specifies the individual state within the multiplet.

This construction shows explicitly how internal symmetries of a particle, such as spin, are directly derived from spacetime symmetries, illustrating how the Poincaré group unifies the treatment of momentum and spin in relativistic quantum mechanics.

\subsubsection{Massless representation}

For \textbf{massless particles} (\(m = 0\)), the first Casimir operator of the Poincaré group satisfies
\[
    C_1 = P^{\mu} P_{\mu} = P^2, \quad \text{with eigenvalue } p^2 = E_{\mathbf{p}}^2 - |\mathbf{p}|^2 = m^2 = 0.
\]
This condition identifies all possible states having vanishing invariant mass, i.e.\ those lying on the light cone of momentum space. Such states correspond to particles moving at the speed of light, for which energy and momentum are related by \(E_{\mathbf{p}} = |\mathbf{p}|\).

As in the massive case, we must now understand how the internal structure of these states --- their intrinsic degrees of freedom --- emerges from the symmetries of spacetime. To this purpose, we again study the subgroup of the Poincaré group that leaves a reference four-momentum \(p^{\mu}\) invariant: the so-called \textbf{little group}.

\paragraph{Little group.}
To identify the little group, we fix a representative momentum \(p^{\mu}\) within the orbit of all lightlike four-momenta under Lorentz transformations. A convenient and symmetric choice is
\[
    p^{\mu} = (E, 0, 0, E),
\]
which describes a massless particle propagating along the \(z\)-axis with energy \(E\) and momentum magnitude \(|\mathbf{p}| = E\).

Geometrically, these transformations conserve the direction of motion of the particle (commute with \(p^\mu\)), but can include both rotations around that direction and certain ``null'' boosts that act as translations in the plane transverse to it. Algebraically, this subgroup turns out to be isomorphic to the two-dimensional Euclidean group \(\mathrm{ISO}(2)\), consisting of rotations and translations in a plane.

To see how this arises, we can exploit the fact that the Pauli--Lubanski vector \(W_{\mu}\) commutes with the four-momentum operator \([W_{\mu},\, P_{\nu}] = 0\). Therefore, the components of \(W_{\mu}\) provide a convenient way to identify the generators of the little group that leave \(p^{\mu}\) invariant. Let us explicitly compute these components in the chosen frame:
\[
    \begin{aligned}
        W_0 & = \frac{1}{2} \epsilon_{0\nu\rho\sigma} P^\nu M^{\rho\sigma} = \frac{1}{2} \epsilon_{0 0 \rho \sigma} E M^{\rho\sigma} + \frac{1}{2} \epsilon_{0 3 \rho \sigma} E M^{\rho\sigma} \\
            & = \frac{1}{2} \left(\epsilon_{0 3 1 2} E M^{12} + \epsilon_{0 3 2 1} E M^{21}\right)                                                                                             \\
            & = E \frac{1}{2} \left( M^{12} - M^{12} \right) = E J_3.
    \end{aligned}
\]
and similarly
\[
    \begin{aligned}
        W_1 & = -E \left( J_1 + K_2 \right), \\
        W_2 & = -E \left( J_2 - K_1 \right), \\
        W_3 & = -E J_3 = - W_0.
    \end{aligned}
\]
Thus, three independent combinations of the Lorentz generators appear naturally: \(J_3\), \(J_1 + K_2\), and \(J_2 - K_1\). The first corresponds to rotations around the direction of propagation, while the other two mix boosts and rotations in such a way that they act as “translations” in the plane orthogonal to the momentum.

We can confirm that these generators indeed satisfy the commutation relations of the \(\mathrm{ISO}(2)\) algebra:
\[
    \begin{aligned}
        [W_1,\, W_2] & = 0,         \\
        [W_3,\, W_1] & = - i E W_2, \\
        [W_3,\, W_2] & = i E W_1.
    \end{aligned}
\]
This algebra corresponds to the group of isometries of the two-dimensional Euclidean plane: \(W_1\) and \(W_2\) generate translations, while \(W_3\) generates rotations about the axis perpendicular to that plane. From a group-theoretical point of view, this means that the internal symmetry space of a massless particle is not spherical (as in the massive case, where it was associated to \(\mathrm{SO}(3)\)), but planar. The plane is the space orthogonal to the direction of motion, and the physical states are classified according to how they transform under rotations and translations within it.

If the eigenvalues of \(W_1\) and \(W_2\) are non-zero, the resulting representations are infinite-dimensional, because the translations generate a continuous family of states labeled by continuous parameters -- a manifestation of the non-compactness of the subgroup generated by \(W_1\) and \(W_2\). In contrast, \(W_3\) generates a compact subgroup corresponding to \(\mathrm{SO}(2)\), whose eigenvalues are discrete.

However, in nature, only the latter type of representations are realized: physical massless particles correspond to the case in which the eigenvalues of \(W_1\) and \(W_2\) vanish. This restriction eliminates the continuous degrees of freedom associated with the non-compact part of the little group, leading to finite-dimensional representations that are fully characterized by the eigenvalues of \(W_3\).

Such representations describe the internal rotational properties of massless particles — properties that will later be associated with \textit{helicity}, the projection of the spin along the direction of motion.

\paragraph{Helicity.}
In the physically relevant case where the transverse components of the Pauli--Lubanski vector vanish, i.e.\ by setting \(W_1 = W_2 = 0\), the remaining non-zero components simplify drastically:
\[
    \begin{aligned}
        W_0   & = - W_3 = E J_3,                              \\
        W_0   & = W^0 = E J_3 = W^3 = - W_3,                  \\
        W^\mu & = J_3 \left(E,\,0,\,0,\,E\right) = J_3 P^\mu.
    \end{aligned}
\]
This shows that \(W^\mu\) becomes proportional to the four-momentum \(P^\mu\), with the proportionality factor given by the operator \(J_3\), the generator of rotations around the direction of propagation.
In this limit, the only remaining internal degree of freedom is the projection of the spin along the momentum direction — a quantity known as the \textbf{helicity}.

The helicity operator is thus defined as
\[
    h = \frac{W_\mu P^\mu}{P^\nu P_\nu}.
\]
However, since for massless particles \(P^\nu P_\nu = 0\), this expression must be interpreted carefully. In practice, one considers the proportionality relation \(W_\mu = h\,P_\mu\), valid on the physical subspace of states with definite helicity.
The operator \(h\) therefore measures how the intrinsic angular momentum (spin) is aligned or anti-aligned with the momentum of the particle. Its eigenvalues \(h\) are discrete and can take integer or half-integer values depending on the spin of the field under consideration.

Physically, \(h>0\) corresponds to a particle whose spin points in the same direction as its momentum (\textit{right-handed} or positive helicity state), while \(h<0\) corresponds to the opposite alignment (\textit{left-handed} or negative helicity state). These two possibilities represent the only independent polarization states available to a massless particle, since there is no rest frame in which to define a third (longitudinal) polarization component.

The second Casimir operator of the Poincaré group,
\[
    C_2 = W^\mu W_\mu,
\]
vanishes identically in this case, \(C_2 = 0\), because \(W^\mu\) is proportional to the null vector \(P^\mu\). Thus, unlike the massive case where \(C_2 = -m^2 \mathbf{J}^2\) provided a discrete spin multiplet, here the intrinsic structure of the representation is fully captured by the helicity eigenvalue.

\paragraph{Labeling the multiplet.}
Each state in the massless representation can therefore be labeled as
\[
    \ket{0,\,0,\,p^\mu,\,\pm h} \equiv \ket{p^\mu,\, \pm h},
\]
where \(h\) denotes the helicity and \(p^\mu\) the four-momentum of the state. The sign \(\pm\) distinguishes the two helicity states corresponding to opposite spin alignments relative to the direction of motion.

Under a \textit{parity} transformation, which reverses spatial orientation (\(\mathbf{p} \to -\mathbf{p}\)), the helicity changes sign:
\[
    h \xrightarrow{P} -h.
\]
Hence, parity inversion exchanges right-handed and left-handed states.
For this reason, a complete relativistic theory must generally include both helicities in order to be invariant under the combined \textit{CPT} transformation\footnote{CPT stands for Charge conjugation (C), Parity inversion (P), and Time reversal (T).}.

In the Standard Model, different particle species correspond to different allowed helicities:
\[
    \begin{aligned}
        h = 0                & \quad \text{Higgs boson (scalar)},             \\
        h = \pm \tfrac{1}{2} & \quad \text{leptons and quarks (fermions)},    \\
        h = \pm 1            & \quad \text{photon and gluons (gauge bosons)}, \\
        h = \pm 2            & \quad \text{graviton (tensor boson)}.
    \end{aligned}
\]
The two helicity states \(\pm h\) thus correspond to the two independent polarization states of a massless field. For instance, in the case of the photon, they represent the right-handed and left-handed circular polarizations of light, which are experimentally observable and fundamental to our understanding of electromagnetism in quantum field theory.

\begin{remark}
    It is conceptually illuminating to note that, at its most fundamental level, quantum field theory is intrinsically a \textbf{massless theory}.
    The full Poincaré symmetry---which combines Lorentz invariance and spacetime translations---is naturally realized on representations with \(m=0\).
    Introducing a nonzero mass ``by hand'' would in fact \textit{bend} this symmetry, since a massive term explicitly selects a preferred reference frame (the rest frame) and breaks conformal invariance.

    From this viewpoint, all fundamental fields are most naturally described as massless excitations of underlying quantum fields, each transforming under irreducible representations of the Poincaré group.
    Mass then arises dynamically through interaction with the scalar Higgs field: the Higgs mechanism endows particles with an effective rest mass while preserving the local gauge and Lorentz symmetries of the theory.
    This elegant construction reconciles the need for mass with the fundamental symmetry principles that govern relativistic quantum dynamics.
\end{remark}
