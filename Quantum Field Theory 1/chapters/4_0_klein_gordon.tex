\chapter{Klein Gordon Theory}

\section{KG Field as Harmonic Oscillators}


\[
    \hat{Q} = \dots
\]
Note that there is an ambiguity: \(\hat{Q}\) is an hermitian operator (good, physical quantity), but if it is conserved and I take \(c_1 \hat{Q} + c_2\) is also conserved:
\[
    \ddt c_1 \hat{Q} + c_2 = c_2 \ddt \hat{Q} = 0,
\]
so \(c_1\) controls \textit{units} in which \(Q\) is measured; \(c_2\) instead is just a constant which can be removed by normal ordering (similar to what we did for the Hamiltonian and the vacuum energy).

\[
    \begin{aligned}
        \hat{Q}\ket{0}         & = \int \frac{\mathrm{d}^3 \mathbf{p}}{(2\pi)^3} \left[ \dots \right] = 0,   \\
        \hat{\tilde{Q}}\ket{0} & = c_2, \iff \bra{0}\hat{\tilde{Q}} \ket{0} = c_2 \langle 0|0 \rangle = c_2, \\
        :\hat{\tilde{Q}}:      & = \hat{\tilde{Q}} - \bra{0}\hat{\tilde{Q}} \ket{0}.
    \end{aligned}
\]

Let us now determine the spectrum of the theory: let us define:
\[
    \begin{aligned}
        \hat{a}_{\pm,\,\mathbf{p}}           & = \\
        \hat{a}_{\pm,\,\mathbf{p}}^{\dagger} & = \\
    \end{aligned}
\]

\textbf{Exercise:} show the commutators:
\[
    fi l l l l l
\]

Let us now consider the state \(\ket{s}\) with charge \(q_s \longrightarrow \hat{\tilde{Q}}\ket{s} = q_s \ket{s}\), then:
\[
    \hat{\tilde{Q}}(\hat{a}_{\pm,\,\mathbf{p}}^{\dagger}\ket{s}) = \dots,
\]
thus we recognize \(\hat{a}_{\pm,\,\mathbf{p}}^{\dagger}\) to be a \textit{ladder operator}\footnote{Operators that let you go up and down in states referring to a particular eigenvalue.} for \(\hat{\tilde{Q}}\); they are ladder operators even for \(\hat{H}\) and \(\hat{P}\) (linear combinations of \(\hat{a}_{(1),\,\mathbf{p}}^{\dagger}\) and \(\hat{a}_{(2),\,\mathbf{p}}^{\dagger}\) are ladder operators for \(\hat{H}\) and \(\hat{P}\)).

Our goal is now to find \textit{common eigenstates} of \(\hat{H}\) and \(\hat{P}\) and \(\hat{\tilde{Q}}\):
\[
    \begin{aligned}
        \hat{\tilde{Q}} \ket{0} = 0, \\
        \ket{s_n^{\pm}} = \prod_{i=1}^n  \hat{a}_{\pm,\,\mathbf{p}}^{\dagger} \ket{0}
    \end{aligned}
\]
so we have \(n\) particle states with positive (\(\ket{s_n^{+}}\)) or negative charge (\(\ket{s_n^{-}}\)) \(\pm nq\):
\[
    \begin{aligned}
        \hat{H} \ket{s_n^{\pm}}         & =   \\
        \hat{P} \ket{s_n^{\pm}}         & =   \\
        \hat{N} \ket{s_n^{\pm}}         & =   \\
        \hat{\tilde{Q}} \ket{s_n^{\pm}} & = .
    \end{aligned}
\]
This symmetry exists only for \(m_1 = m_2 = m\), otherwise we have no internal symmetries and cannot find a Noether's charge.

\(\hat{\tilde{Q}}\) is the \textbf{electric charge} and it can describe \textbf{particle/antiparticle} with positive energy: \(\ket{s_n^{\pm}}\) describes the sign of the charge: \(\ket{s_n^{+}}\) for particles and \(\ket{s_n^{-}}\) for antiparticles.

Note that a real KG field can only describe a \(Q=0\) particle
\[
    Q = \int \dots
\]
if \(\psi_1=\psi_2=\psi\), we need at least 2 degrees of freedom if we wnat to have a more complete description of a picture with particles and antiparticles with nonzero electric charge: \textbf{complex KG field}.

The Lagrangian can be rewritten also as follows
\[
    \mathcal{L} =\partial_\mu \psi^* \partial^{\mu} \psi\dots
\]
with
\[
    \psi(x) = \frac{1}{\sqrt{2}}(\phi_1(x)+i \phi_2(x)).
\]
Thus we have a transformation of the form:
\[
    \psi(x) \xrightarrow[\mathrm{U}(1)]{\text{global}} \psi^{\prime}(x) = e^{i \theta} \psi(x)
\]
where it's important that \(\theta \neq \theta(x)\) so that
\[
    paper1
\]
hence \(\mathcal{L} \to \mathcal{L} ^{\prime} =\mathcal{L}\) and for infinitesimal transformations we have
\[
    \begin{aligned}
        paper2
    \end{aligned}
\]
and the total charge (which is conserved) can be computed as\TODO{compute it}
\[
    paper3.
\]