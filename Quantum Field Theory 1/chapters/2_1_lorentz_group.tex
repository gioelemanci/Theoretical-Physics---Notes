\section{The Lorentz Group}

The most fundamental symmetry group in relativistic field theory is the \textbf{Lorentz group}, denoted by \(\mathrm{O}(1,3)\).
It consists of all linear transformations \(\Lambda: \mathbb{R}^{1,3} \to \mathbb{R}^{1,3}\) that leave the Minkowski metric invariant:
\[
    \eta_{\mu\nu} = \mathrm{diag}(1,-1,-1,-1),
\]
i.e.
\[
    \eta_{\mu\nu} \, \Lambda^{\mu}{}_{\rho} \Lambda^{\nu}{}_{\sigma} = \eta_{\rho\sigma}.
\]

These transformations preserve the spacetime interval
\[
    s^2 = \eta_{\mu\nu} x^\mu x^\nu = (x^0)^2 - \vec{x}^{\,2},
\]
or, infinitesimally,
\[
    \mathrm{d}s^2 = \eta_{\mu\nu} \, \mathrm{d}x^\mu \mathrm{d}x^\nu.
\]

Indeed, under a Lorentz transformation \(x^\mu \mapsto x'^\mu = \Lambda^\mu{}_\rho x^\rho\), we have
\[
    \begin{aligned}
        \mathrm{d}s'^2 & = \eta_{\mu\nu} \, \mathrm{d}x'^\mu \mathrm{d}x'^\nu
        = \eta_{\mu\nu} \, \Lambda^\mu{}_\rho \mathrm{d}x^\rho \, \Lambda^\nu{}_\sigma \mathrm{d}x^\sigma                     \\
                       & = \Lambda^\mu{}_\rho \, \eta_{\mu\nu} \, \Lambda^\nu{}_\sigma \, \mathrm{d}x^\rho \mathrm{d}x^\sigma
        = \eta_{\rho\sigma} \, \mathrm{d}x^\rho \mathrm{d}x^\sigma
        = \mathrm{d}s^2,
    \end{aligned}
\]
confirming that Lorentz transformations leave the Minkowski interval invariant.


The determinant of a Lorentz transformation can be either \(+1\) or \(-1\), and \(\Lambda^0{}_0\) can be greater or smaller than \(1\).
The subgroup with \(\det \Lambda = +1\) and \(\Lambda^0{}_0 \ge 1\) is called the \textbf{proper orthochronous Lorentz group}, denoted by \(\mathrm{SO}^+(1,3)\).
It is this connected component that is continuously connected to the identity and relevant for most applications in physics.

\subsection{Infinitesimal Transformations and Generators}

An infinitesimal Lorentz transformation can be written as
\[
    \Lambda^\mu{}_\nu = \delta^\mu{}_\nu + \omega^\mu{}_\nu, \qquad |\omega^\mu{}_\nu| \ll 1,
\]
with the condition
\[
    \eta_{\mu\nu} \Lambda^\mu{}_\rho \Lambda^\nu{}_\sigma = \eta_{\rho\sigma}
    \quad \Rightarrow \quad
    \omega_{\mu\nu} = -\omega_{\nu\mu}.
\]
Hence, \(\omega_{\mu\nu}\) is an antisymmetric tensor containing six independent parameters, corresponding to the six generators of the Lorentz algebra.

We can introduce a set of generators \(M_{\mu\nu}\) (antisymmetric in their indices) such that a general infinitesimal transformation acts on a four-vector \(x^\rho\) as:
\[
    x'^\rho = \left( \delta^\rho{}_\sigma + \frac{i}{2}\,\omega^{\mu\nu} (M_{\mu\nu})^\rho{}_\sigma \right) x^\sigma,
\]
as taylor expansion of
\[
    \Lambda^{\rho}{}_{\sigma} =  e^{\frac{i}{2}\,\omega^{\mu\nu} (M_{\mu\nu})^\rho{}_\sigma},
\]
where the generators of the group are
\[
    (M_{\mu\nu})^\rho_{\ \sigma} = -i \left( \eta_{\mu\sigma} \, \delta^\rho_{\ \nu} - \eta_{\nu\sigma} \, \delta^\rho_{\ \mu} \right),
\]

The commutation relations among the generators define the \textbf{Lorentz algebra} \(\mathfrak{so}(1,3)\):
\[
    [M_{\mu\nu}, M_{\rho\sigma}] = i \left(
    \eta_{\nu\rho} M_{\mu\sigma}
    - \eta_{\mu\rho} M_{\nu\sigma}
    + \eta_{\mu\sigma} M_{\nu\rho}
    - \eta_{\nu\sigma} M_{\mu\rho}
    \right).
\]

\subsection{Generators: Rotations and Boosts}

The six generators can be naturally separated into:
\[
    \begin{aligned}
        J_i & \equiv \frac{1}{2} \epsilon_{ijk} M_{jk} \quad &  & \text{(spatial rotations)}, \\
        K_i & \equiv M_{0i} \quad                            &  & \text{(Lorentz boosts)}.
    \end{aligned}
\]
In terms of these generators, the Lorentz algebra decomposes as:
\[
    [J_i, J_j] = i \epsilon_{ijk} J_k, \qquad
    [J_i, K_j] = i \epsilon_{ijk} K_k, \qquad
    [K_i, K_j] = -i \epsilon_{ijk} J_k.
\]
This structure reveals the non-compact nature of the Lorentz group: unlike spatial rotations, boosts do not form a compact subgroup, as rapidities are unbounded.\footnote{When we later discuss the relation between \(\mathrm{SO}^+(1,3)\) and \(\mathrm{SU}(2) \times \mathrm{SU}(2)\), this distinction will be important: \(\mathrm{SU}(2) \times \mathrm{SU}(2)\) is compact, whereas the proper orthochronous Lorentz group is not.}


For finite transformations, one can write:
\[
    R(\bs{\theta}) = e^{-i \bs{\theta}\cdot \mathbf{J}},
    \qquad
    B(\bs{\beta}) = e^{-i \bs{\beta}\cdot \mathbf{K}},
\]
where \(\bs{\theta} = \left( \theta_1,\,\theta_2,\,\theta_3 \right)\) are the rotation angles and \(\bs{\beta} = \left( \beta_1,\,\beta_2,\,\beta_3\right) \) the boost parameters. We can ultimately reconstruct the expression for \(\omega_{\mu\nu}\):
\[
    \begin{aligned}
        \omega_{ij} & = \epsilon_{ijk}\bs{\theta}^k, \\
        \omega_{0i} & = \bs{\beta}_i.
    \end{aligned}
\]
Together, they parametrize any proper orthochronous Lorentz transformation as a combination of a boost and a rotation:
\[
    \Lambda = B(\bs{\beta}) \, R(\bs{\theta}).
\]

\subsection{Representations}

The Lorentz algebra \(\mathfrak{so}(1,3)\) is locally isomorphic to the direct sum \(\mathfrak{su}(2) \oplus \mathfrak{su}(2)\) (algebra of $SU(2) \times SU(2)$, equivalent up to different hermiticity relations arising because of compactess).
This correspondence allows the classification of all finite-dimensional representations of the Lorentz group in terms of pairs \((j_+, j_-)\) of \(\mathrm{SU}(2)\) spins.

One can then construct the combinations
\[
    \mathbf{A} = \frac{1}{2} (\mathbf{J} + i \mathbf{K}), \qquad
    \mathbf{B} = \frac{1}{2} (\mathbf{J} - i \mathbf{K}),
\]
which satisfy independent \(\mathfrak{su}(2)\) algebras:
\[
    [A_i, A_j] = i \epsilon_{ijk} A_k, \quad
    [B_i, B_j] = i \epsilon_{ijk} B_k, \quad
    [A_i, B_j] = 0.
\]

This shows explicitly the algebra decompiìosition into a direct sum of two commuting algebras.
A concrete realization of the Lorentz algebra generators can be constructed using \textbf{spinor spaces}.
Consider spin-$\tfrac{1}{2}$ representations, and let \(\sigma_i\) denote the Pauli matrices.
Define the combinations
\[
    \mathbf{A} = \frac{1}{2} \boldsymbol{\sigma} \otimes \mathbf{1}, \qquad
    \mathbf{B} = \frac{1}{2} \mathbf{1} \otimes \boldsymbol{\sigma},
\]
which act on the four-dimensional space \(\mathbb{C}^2 \otimes \mathbb{C}^2\).

In this language, the \textbf{left-handed} (LH) \textbf{spinors} transform under the \((\tfrac{1}{2},0)\) representation, corresponding to the action of \(\mathbf{A}\) on the first factor, while the \textbf{right-handed} (RH) \textbf{spinors} transform under the \((0,\tfrac{1}{2})\) representation, corresponding to the action of \(\mathbf{B}\) on the second factor.

The Dirac spinor arises as the direct sum \(\psi = \psi_L \oplus \psi_R \in \mathbb{C}^2 \oplus \mathbb{C}^2\), and higher-dimensional representations of the Lorentz group can be constructed by taking tensor products of these spinor spaces, yielding \((j_+, j_-)\) representations with arbitrary spins.

\paragraph{Trivial representation.}
In addition to the nontrivial representations, every Lie group admits a trivial representation, denoted by $(0,0)$ in the $\mathrm{SU}(2)\times\mathrm{SU}(2)$ classification. In this representation, the group acts identically on the vector space: for any group element $g \in G$ and any vector $v$ in the representation space,
\[
    g \cdot v = v.
\]
The trivial representation is one-dimensional and corresponds to \textit{scalars} under the symmetry: objects that remain invariant under all group transformations, such as the Higgs field under Lorentz transformations.

Although the underlying representation space $\mathbb{C}^2 \otimes \mathbb{C}^2$ is the same, different choices of the $(j_+,j_-)$ representation allow us to classify distinct particles and fields according to their transformation properties under the Lorentz group, as shown in the following table.

\begin{table}[H]
    \centering
    \begin{tabular}{clc}
        \toprule
        Representation $(j_+,j_-)$               & Field / Particle type   & Total spin $s$ \\
        \midrule
        $(0,0)$                                  & Scalar                  & $0$            \\
        $\left(\tfrac{1}{2},0\right)$            & LH Weyl spinor          & $\tfrac{1}{2}$ \\
        $\left(0,\tfrac{1}{2}\right)$            & RH Weyl spinor          & $\tfrac{1}{2}$ \\
        $\left(\tfrac{1}{2},0\right)\oplus\left(0,\tfrac{1}{2}\right)$
                                                 & Dirac spinor            & $\tfrac{1}{2}$ \\
        $\left(\tfrac{1}{2},\tfrac{1}{2}\right)$ & Vector                  & $1$            \\
        $\left(1,\tfrac{1}{2}\right)\oplus\left(\tfrac{1}{2},1\right)$
                                                 & Rarita--Schwinger field & $\tfrac{3}{2}$ \\
        $(1,1)$                                  & Graviton                & $2$            \\
        \bottomrule
    \end{tabular}
\end{table}
The representations listed above correspond to the basic field types used in relativistic quantum field theory. Scalars describe spinless particles such as the Higgs boson. Weyl spinors transform under chiral representations of the Lorentz group, and their direct sum forms the Dirac spinor, which accounts for massive spin-$\tfrac{1}{2}$ fermions like the electron. The vector representation corresponds to spin-1 gauge bosons, such as the photon or gluons. The Rarita--Schwinger field describes spin-$\tfrac{3}{2}$ particles, notably the gravitino, while the symmetric rank-2 tensor representation corresponds to the graviton, a hypothetical spin-2 quantum of the gravitational field.

This representation theory is the foundation for classifying relativistic fields and particles by their transformation properties under the Lorentz group.
