\chapter{Spin 1 Particles}

We are going to describe the Electromagnetic field as a quantum field theory of spin 1 particles, the photons. We will see that the photon is \textbf{massless} and has only \textbf{two polarization states}, which makes it different from the massive spin 1 particles we will study later, such as the W and Z bosons.

We have alredy gone through the quantization procedure for scalar fields (spin 0 particles) and spin \(\frac{1}{2}\) fields (fermions), so we will focus here on the peculiarities of spin 1 fields, which arise mainly due to the gauge invariance of the electromagnetic field.

\section{The Classical Electromagnetic Field}

The lagrangian density for the classical electromagnetic field is given by
\[
    \mathcal{L} = -\frac{1}{4} F_{\mu\nu} F^{\mu\nu},
\]
where the field strength tensor is defined as
\[
    F^{\mu \nu} = \partial^\mu A^\nu - \partial^\nu A^\mu, \quad A^\mu = ( \phi, \mathbf{A} ).
\]
We can study the equations of motion using the Euler-Lagrange equations:
\[
    \partial_\mu \left( \frac{\partial \mathcal{L}}{\partial (\partial_\mu A_\nu)} \right) = \frac{\partial \mathcal{L}}{\partial A_\nu} = 0, \implies \partial_\mu F^{\mu \nu} = 0.
\]
It can be shown that the field strength tensor \(F^{\mu \nu}\) respects the \textbf{Bianchi identity} (as we have already seen in section \ref{sec:free_em_field_lagrangian}):
\[
    \partial_{[\lambda}F_{\mu\nu]} \equiv
    \partial_\lambda F_{\mu\nu} +
    \partial_\mu F_{\nu\lambda} +
    \partial_\nu F_{\lambda\mu} = 0.
\]
The electric and magnetic fields can be expressed in terms of the potentials as
\[
    \mathbf{E} = -\nabla \phi - \frac{\partial \mathbf{A}}{\partial t}, \quad \mathbf{B} = \nabla \times \mathbf{A},
\]
as the spatial and temporal components of the field strength tensor:
\begin{equation}
    F_{\mu \nu} = \begin{pmatrix}
        0    & E_x  & E_y  & E_z  \\
        -E_x & 0    & -B_z & B_y  \\
        -E_y & B_z  & 0    & -B_x \\
        -E_z & -B_y & B_x  & 0
    \end{pmatrix}, \quad F^{\mu \nu} = \begin{pmatrix}
        0   & -E^x & -E^y & -E^z \\
        E^x & 0    & -B^z & B^y  \\
        E^y & B^z  & 0    & -B^x \\
        E^z & -B^y & B^x  & 0
    \end{pmatrix}.
    \label{eq:field_strength_tensor_components_explicit}
\end{equation}
Finally we can see how the Maxwell equations are encapsulated in this formalism, since they arise from the equations of motion and the Bianchi identity:
\[
    \begin{dcases}
        \nabla \cdot \mathbf{E} = 0, \\
        \nabla \times \mathbf{B} - \frac{\partial \mathbf{E}}{\partial t} = 0,
    \end{dcases} \quad
    \begin{dcases}
        \nabla \cdot \mathbf{B} = 0, \\
        \nabla \times \mathbf{E} + \frac{\partial \mathbf{B}}{\partial t} = 0,
    \end{dcases}
\]
where the first two equations come from \(\partial_\mu F^{\mu \nu} = 0\) and need further terms when sources are present, while the last two come from the Bianchi identity, after recognizing the definitions of the electric and magnetic fields, and are true even in the presence of sources.

If we now expand the lagrangian density in terms of the potentials, we can exploit the fact that not every component of the vector potential \(A^{\mu}\) is physical, since the EM field is known to have only two physical degrees of freedom (the two polarization states of the photon). Thus the lagrangian density can be rewritten as
\[
    \mathcal{L} = -\frac{1}{4} F_{\mu \nu} F^{\mu \nu} = -\frac{1}{2} F_{0i}F^{0i} - \frac{1}{4} F_{ij}F^{ij}.
\]
Thus we understand that the lagrangian density depends only on the spatial components of the vector potential \(A^i\) and their time derivatives, while the time component \(A^0\) does not have a kinetic term (no dependence on \(\frac{1}{2}(\partial_0 A^0)^2\)):
\[
    \mathcal{L} \supset \sum_{i} \frac{1}{2}(\dot{A}^{i})^2, \quad \text{ but } \quad \mathcal{L} \not\supset \frac{1}{2}(\dot{A}^{0})^2.
\]
This indicates that \(A^0\) is not a dynamical propagating degree of freedom, but rather acts as a Lagrange multiplier enforcing a constraint on the physical states of the theory.

Thus if we were to solve the equations of motion directly for \(A_\mu\), we would find that only \(A_i\) and \(\dot{A}_i\) are necessary as initial conditions \((A_i(t_0, \mathbf{x}),\,\dot{A}_i(t_0, \mathbf{x}))\) to determine the evolution of the field, while \(A_0\) is determined by the equation \(\nabla\cdot \mathbf{E}=0\) and does not represent an independent degree of freedom. We have indeed
\[
    - \nabla\cdot \mathbf{E} = \nabla\cdot\nabla A_0 + \nabla\cdot \dot{\mathbf{A}} = 0 \implies \nabla^2 A_0 = - \nabla\cdot \dot{\mathbf{A}},
\]
so that \(A_0\) is fixed by the spatial components \(A_i\) and their time derivatives. We could compute a solution for \(A_0\) using Green's functions of the Laplacian operator:
\[
    \begin{dcases}
        \nabla^2 A_0(t_0, \mathbf{x}) = - \nabla \cdot \dot{\mathbf{A}}(t_0, \mathbf{x}), \\
        \nabla^2 G(\mathbf{x}, \mathbf{y}) = \delta^{(3)}(\mathbf{x} - \mathbf{y}) \iff G(\mathbf{x}, \mathbf{y}) = \frac{-1}{4\pi \vert \mathbf{x} - \mathbf{y} \vert},
    \end{dcases}
\]
so that
\[
    A_0(t_0, \mathbf{x}) = \int \mathrm{d}^3 \mathbf{y} \, G(\mathbf{x}, \mathbf{y}) \left(- \nabla \cdot \dot{\mathbf{A}}(t_0, \mathbf{y})\right) = - \int \mathrm{d}^3 \mathbf{y} \frac{\nabla \cdot \dot{\mathbf{A}}(t_0, \mathbf{y})}{4\pi \vert \mathbf{x} - \mathbf{y} \vert}.
\]

In the end , we have only three independent degrees of freedom in the vector potential \(A^{\mu}\) (the three spatial components \(A^i\)), but we know that the photon has only two physical polarization states. This discrepancy arises because the electromagnetic field is invariant under \textbf{gauge transformations} of the form
\[
    A_\mu(x) \to A'_\mu(x) = A_\mu(x) + \partial_\mu \alpha(x),
\]
where \(\alpha(x)\) is an arbitrary scalar function of the spacetime whose derivative has to vanish at infinity. This gauge invariance implies that not all configurations of \(A_\mu\) correspond to physically distinct states: different choices of \(\alpha(x)\) can lead to the same physical electromagnetic fields \(\mathbf{E}\) and \(\mathbf{B}\). The field strength tensor \(F_{\mu\nu}\) is invariant under these gauge transformations:
\[
    F'_{\mu\nu} = \partial_\mu A'_\nu - \partial_\nu A'_\mu = \partial_\mu (A_\nu + \partial_\nu \alpha) - \partial_\nu (A_\mu + \partial_\mu \alpha) = F_{\mu\nu}.
\]
\(A_\mu\) and \(A'_\mu\) describe the same physical situation, they need to be identified, since this is a physical equivalence class. This is very different from global transformations: \(\psi(x) \to \psi^{\prime}(x) = e^{i\theta} \psi(x)\) is a transformation that changes the field configuration, but does not identify different configurations as physically equivalent, since the phase factor \(e^{i\theta}\) is constant and does not depend on spacetime. Here we have just a redundancy in the description of the electromagnetic field.

We can do another example of gauge transformation: consider a solution for the vector potential \(A_\mu(x)\). After performing a gauge transformation with a function \(\alpha(x)\):\TODO{insert picture of gauge orbit}
\[
    A_\mu(x) \to A'_\mu(x) = A_\mu(x) + \partial_\mu \alpha(x),
\]
we could repeat the gauge transformation with a different function \(\beta(x)\):
\[
    A^{\prime}_\mu(x) \to A^{\prime\prime}_\mu(x) = A'_\mu(x) + \partial_\mu \beta(x) = A_\mu(x) + \partial_\mu (\alpha(x) + \beta(x)).
\]
Each of these configurations describes the same physical electromagnetic field, since the field strength tensor remains unchanged. This shows that there is an infinite number of gauge equivalent configurations of \(A_\mu\) that correspond to the same physical situation.

\subsection{Popular Gauge Fixing Choices}

We need to \textit{choose a representative} for each equivalence class of gauge equivalent configurations in order to uniquely describe the physical electromagnetic field. This process is called \textbf{gauge fixing}: we fix some condition on the vector potential \(A_\mu\) to eliminate the redundancy and select a unique element on the \textbf{gauge orbit} for each physical configuration. Some physiscal properties, such as the number of physical degrees of freedom, may become clearer after gauge fixing.

There are several popular choices for gauge fixing in electromagnetism, each with its own advantages and disadvantages. Two of the most common gauges are:
\begin{enumerate}
    \item \textbf{Lorentz Gauge}: The Lorentz gauge condition is given by
          \[
              \partial_\mu A^\mu = 0 = \partial_0 A^0 + \nabla \cdot \mathbf{A}.
          \]
          This gauge is Lorentz invariant, making it convenient for relativistic calculations. In this gauge, the independent degrees of freedom are reduced down to two, corresponding to the two physical polarization states of the photon.
          But \(A^{\mu} \) does not necessarily satisfy the wave equation:
          \[
              \partial_\mu A^\mu(x) = f(x) \neq 0,
          \]
          but we can always perform a gauge transformation to ensure that it does: if there exists a function \(\alpha(x)\) such that \(\Box \alpha(x) = - f(x)\), then the gauge transformed field
          \[
              A'^{\mu}(x) = A^{\mu}(x) + \partial^{\mu} \alpha(x)
          \]
          will satisfy the Gauge condition; but note that this does not completely fix the gauge, since we can still perform gauge transformations and obtain new fields that satisfy the Lorentz gauge condition: if \(\Box \beta(x) = 0\), then
          \[
              A''^{\mu}(x) = A'^{\mu}(x) + \partial^{\mu} \beta(x)
          \]
          will also satisfy the Lorentz gauge condition \(\partial_\mu A''^\mu = 0\).

          But in this gauge, we have still some residual gauge freedom, since we can perform gauge transformations with functions \(\beta(x)\) (called harmonic functions) that satisfy the homogeneous wave equation \(\Box \beta(x) = 0\). Furthermore, this gauge fixing has the advantage of being \textbf{manifestly Lorentz invariant}, making it suitable for relativistic calculations.

    \item \textbf{Coulomb Gauge}: The Coulomb gauge condition is given by
          \[
              \nabla \cdot \mathbf{A} = 0,
          \]
          which focuses on the spatial components of the vector potential. In this gauge, the scalar potential \(\phi\) is determined by the charge distribution, while the vector potential \(\mathbf{A}\) describes the transverse electromagnetic waves. This gauge is particularly useful in non-relativistic quantum mechanics and in problems involving static charges, since it is not manifestly Lorentz invariant.

          One can use the residual gauge freedom in the Lorentz gauge to reach the Coulomb gauge by choosing a suitable gauge function \(\alpha(x)\) that satisfies the appropriate conditions:
          \[
              \nabla \cdot \mathbf{A} = \partial_i A^i = 0 \implies \partial_\mu A^{\mu} = \partial_0 A^0 + \nabla \cdot \mathbf{A} = \partial_0 A^0,
          \]
          so that \(A_0\) is constant in time, since \(\partial_\mu A^{\mu}=\partial_0 A^0 = 0\) from the Lorentz gauge. However, the temporal component \(A^0\)
          \[
              A_0(t_0,\,\mathbf{x}) = \int \mathrm{d}^3 \mathbf{y} \frac{\partial_t \left(\nabla \cdot \mathbf{A}(t_0, \mathbf{y})\right)}{4\pi \vert \mathbf{x} - \mathbf{y} \vert} = 0,
          \]
          since \(\nabla \cdot \mathbf{A} = 0\). Thus, in the Coulomb gauge, the scalar potential is fixed to zero
          \[
              A_0(\mathbf{x}) = 0,
          \]
          and the vector potential \(\mathbf{A}\) describes the two transverse polarization states of the photon: since we have imposed \(\nabla \cdot \mathbf{A} = 0\), only the components of \(\mathbf{A}\) perpendicular to the direction of propagation remain as physical degrees of freedom.
\end{enumerate}

Note that in both gauges, we have successfully reduced the number of independent degrees of freedom in the vector potential \(A_\mu\) from four (actually three, since we already know from Gauss' law that the temporal component is constrained) to two, corresponding to the two physical polarization states of the photon. The residual gauge freedom in the Lorentz gauge (which we used to go in the Coulomb gauge directly from Lorentz') was not a physical degree of freedom, but rather a redundancy in the description of the electromagnetic field: in Lorentz gauge \(A_0\) is a function of \(A_i\), while in Coulomb gauge \(A_0\) is fixed to zero.

\section{Gauge Fixing of the Electromagnetic Field}\label{sec:gauge_fixing_em_field}

Before starting the quantization procedure, we need to find the conjugate momenta associated with the fields \(A^{\mu}\): we will use it to find the expression for the Hamiltonian (via Legendre transform) and impose the canonical commutation relations.

From the lagrangian density
\[
    \mathcal{L} = -\frac{1}{2} \left( F_{0i}F^{0i} + \frac{1}{2}F_{ij}F^{ij} \right),
\]
the \textbf{conjugate momenta} are defined as
\[
    \pi^{\mu} = \frac{\partial \mathcal{L}}{\partial (\partial_0 A_{\mu})} = \begin{dcases}
        \pi^0 = \frac{\partial \mathcal{L}}{\partial (\partial_0 A_{0})} = 0, \text{ since } A_0 \text{ is not a dynamical field,} \\
        \pi^i = \frac{\partial \mathcal{L}}{\partial (\partial_0 A_{i})} = \frac{\partial}{\partial (\dot{A}_{i})} \left[-\frac{1}{2}\left(\dot{A}_j - \partial_j A_0\right)\left(\dot{A}^j - \partial^j A^0\right)\right] = E^i,
    \end{dcases}
\]
since, for the spatial component, we can compute
\[
    \frac{\partial}{\partial (\dot{A}_{i})} \left[-\frac{1}{2}\left(\dot{A}_j - \partial_j A_0\right)\left(\dot{A}^j - \partial^j A^0\right)\right] = -\frac{1}{2} F^{0i} -\frac{1}{2}F_{0j}\eta^{ji} = -F^{0i} = E^i.
\]
Thus we have \(\pi = (0, \mathbf{E})\) and the Hamiltonian density of the system can be computed via Legendre transform as
\[
    \mathcal{H} = \pi^{\mu} \partial_0 A_{\mu} - \mathcal{L} = \pi^{i} \partial_0 A_{i} - \mathcal{L},
\]
since \(\pi^0 = 0\). Recalling the expressions for \(\pi^i\) and \(\mathcal{L}\), we get
\[
    \begin{aligned}
        \pi^{i} \partial_0 A_{i} & = E^{i} \partial_0 A_{i} = E^{i} \left( F_{0i} + \partial_i A_0 \right) = E^{i} F_{0i} + E^{i} \partial_i A_0,                                                                                                                            \\
        \mathcal{L}              & = -\frac{1}{2} \left( F_{0i}F^{0i} + \frac{1}{2}F_{ij}F^{ij} \right) = -\frac{1}{2} \left( -\vert \mathbf{E} \vert^2 + \vert \mathbf{B} \vert^2 \right) = \frac{1}{2} \left( \vert \mathbf{E} \vert^2 - \vert \mathbf{B} \vert^2 \right),
    \end{aligned}
\]
we get to an hamiltonian density of
\[
    \mathcal{H} = E^{i} F_{0i} + E^{i} \partial_i A_0 - \frac{1}{2} \left( \vert \mathbf{E} \vert^2 - \vert \mathbf{B} \vert^2 \right) = \frac{1}{2} \left( \vert \mathbf{E} \vert^2 + \vert \mathbf{B} \vert^2 \right) + (\mathbf{E} \cdot \nabla) A_0.
\]
So that the Hamiltonian takes the form
\[
    H = \int \mathrm{d}^3 \mathbf{x} \left( \frac{1}{2} \left( \vert \mathbf{E} \vert^2 + \vert \mathbf{B} \vert^2 \right) - A_0 (\nabla \cdot \mathbf{E}) \right).
\]
In this expression, the term involving \(A_0\) (found after an integration by parts) does not act as a physical variable, but as a \textit{Lagrange multiplier} enforcing Gauss's law \(\nabla \cdot \mathbf{E} = 0\) in the absence of charges:
\[
    \frac{\partial \mathcal{H}}{\partial A_0} = -\nabla \cdot \mathbf{E} = 0 \implies \nabla \cdot \mathbf{E} = 0.
\]
We have still to fix a gauge in order to proceed with the quantization, so the field has still some redundant degrees of freedom, and this is a constraint on the physical states of the theory (on the elements of \(A^{\mu}\)).

\paragraph{Lorentz gauge fixing.}
Lastly, before starting the quantization procedure, we have to impose the Lorentz gauge condition \(\partial_\mu A^{\mu} = 0\), since its Lorentz invariance makes it suitable for relativistic quantum field theory. From this choice, we get the equations of motion
\[
    \partial_\mu F^{\mu \nu} = 0 = \partial_\mu \partial^\mu A^\nu - \partial^\nu (\partial_\mu A^\mu) = \partial_\mu \partial^\mu A^\nu = \Box A^{\nu}.
\]
Since \(\Box A^{\nu} = 0\), each component of the vector potential \(A^{\mu}\) satisfies the wave equation, particularly the massless KG scalar equation: at the quantum level, the field \(A^{\mu}\) will describe massless particles, the \textbf{photons}, with energy \(E_{\mathbf{p}} = \sqrt{\vert \mathbf{p} \vert^2} = \vert \mathbf{p} \vert \).

\paragraph{Feynman gauge fixing.}
Instead of imposing the gauge condition \(\partial_\mu A^{\mu} = 0\) by hand, we can modify the lagrangian density slightly by adding a gauge-fixing term:
\[
    \mathcal{L} = -\frac{1}{4} F_{\mu\nu} F^{\mu\nu} - \frac{1}{2\xi} (\partial_\mu A^{\mu})^2,
\]
where \(\xi\) let us choose different gauges: \(\xi = 1\) corresponds to the \textbf{Feynman gauge}, while \(\xi \to 0\) (computed after quantization) corresponds to the Landau gauge. Here we will choose \(\xi = 1\) for simplicity, so
\[
    \mathcal{L} = -\frac{1}{4} F_{\mu\nu} F^{\mu\nu} - \frac{1}{2} (\partial_\mu A^{\mu})^2.
\]
This modification does not change the equations of motion for physical fields (since we have redundancy in the description, we are just fixing a gauge implicitly from the Lagrangian), but it allows us to derive the Lorentz gauge condition from the equations of motion themselves. The new equations of motion become
\[
    \partial_\mu \left( \frac{\partial \mathcal{L}}{\partial (\partial_\mu A_\nu)} \right) = \frac{\partial \mathcal{L}}{\partial A_\nu} = 0 \implies \partial_\mu F^{\mu \nu} + \partial_\mu \eta^{\mu \nu} (\partial_\sigma A^{\sigma}) = 0,
\]
which simplifies to
\[
    \partial_\mu \partial^\mu A^\nu - \partial_\mu \partial^\nu A^\mu + \partial^\nu (\partial_\mu A^\mu) = \Box A^{\nu} = 0.
\]
We reached the same equations as before, but now the gauge condition \(\partial_\mu A^{\mu} = 0\) is automatically satisfied by the solutions of the equations of motion.

Proceeding to find the conjugate momenta, we get
\[
    \begin{dcases}
        \pi^0 = \frac{\partial \mathcal{L}}{\partial (\partial_0 A_{0})} = -\partial_\mu A^{\mu}; \\
        \pi^i = \frac{\partial \mathcal{L}}{\partial (\partial_0 A_{i})} = \partial^i A^{0} - \partial^0 A^{i} = F^{i0};
    \end{dcases} \quad \iff \quad \pi^{\mu} = F^{\mu 0} - \eta^{\mu 0} (\partial_\nu A^{\nu}),
\]
different from the previous case where \(\pi^0 = 0\), since now the Lagrangian has a dependence on \(\partial_0 A_0\) in the new gauge-fixing term.

\section{Quantization of the Electromagnetic Field}

Sticking to the Feynman gauge, we can finally proceed to the quantization for the electromagnetic field. The classical fields \(A_{\mu}(x)\) and \(\pi^{\mu}(x)\) are going to be promoted to operators acting on a Hilbert space. We impose the \textbf{canonical commutation relations} at equal times:\footnote{Since we know that integer spin particles are bosons and the correct spin-statistics relations are imposed via commutation.}
\begin{equation}
    \begin{aligned}
        \left[\hat{A}_{\mu}(\mathbf{x}, t), \hat{\pi}^{\nu}(\mathbf{y}, t)\right]   & = i \delta_{\mu}^{\ \nu} \delta^{(3)}(\mathbf{x} - \mathbf{y}), \\
        \left[\hat{A}_{\mu}(\mathbf{x}, t), \hat{A}_{\nu}(\mathbf{y}, t)\right]     & = 0,                                                            \\
        \left[\hat{\pi}^{\mu}(\mathbf{x}, t), \hat{\pi}^{\nu}(\mathbf{y}, t)\right] & = 0.
    \end{aligned}
    \label{eq:canonical_commutation_relations_em_field}
\end{equation}
This relations will ensure the correct quantization of the electromagnetic field, with symmetric multiparticle states corresponding to the bosonic nature of photons, while the use of anticommutation relations would lead to inconsistencies with the spin-statistics theorem.

The general solution to the equations of motion \(\Box A^{\mu} = 0\) can be expressed as a Fourier expansion in terms of plane waves whose coefficients should operate as creation and annihilation operators for photons. We can write:
\[
    \hat{A}_{\mu}(x) = \int \frac{\mathrm{d}^3 \mathbf{p}}{(2\pi)^3} \frac{1}{\sqrt{2 E_{\mathbf{p}}}} \left( \hat{\xi}_{\mu}(\mathbf{p}) e^{i \mathbf{p} \cdot \mathbf{x}} + \hat{\xi}_{\mu}^{\dagger}(\mathbf{p}) e^{-i \mathbf{p} \cdot \mathbf{x}} \right),
\]
where \(p^{\mu} = (E_{\mathbf{p}}, \mathbf{p})\) with \(E_{\mathbf{p}} = \vert \mathbf{p} \vert\) for massless particles.

The operators \(\hat{\xi}_{\mu}(\mathbf{p})\) and \(\hat{\xi}_{\mu}^{\dagger}(\mathbf{p})\) will be identified as annihilation and creation operators for photons with momentum \(\mathbf{p}\) and polarization indexed by \(\mu\). To explicitly see this, we can decompose these operators in terms of a \textbf{polarization basis}: we introduce polarization vectors \(\epsilon_{\mu}^{(\lambda)}(\mathbf{p})\) with \(\lambda = 0, 1, 2, 3\) labeling the four possible polarization states of the photon
\begin{equation}
    \epsilon_\mu^{(\lambda)}(\mathbf{p}), \quad \lambda = 0, 1, 2, 3, \quad \begin{dcases}
        \epsilon^{(\lambda)}_{\mu}(\mathbf{p}) \epsilon^{(\lambda') \mu}(\mathbf{p})                                   & = \eta^{\lambda \lambda'}, \\
        \epsilon_{\mu}^{(\lambda)}(\mathbf{p}) \epsilon_{\nu}^{(\lambda^{\prime})}(\mathbf{p}) \eta_{\lambda \lambda'} & = \eta_{\mu \nu}.
    \end{dcases}
    \label{eq:polarization_basis_and_ON_relations}
\end{equation}
which satisfy the orthonormality with respect to the Minkowski metric \(\eta_{\mu \nu}\). Lambda \(\lambda = 0\) corresponds to the timelike polarization, \(\lambda = 1, 2\) correspond to the two transverse polarizations, while \(\lambda = 3\) corresponds to the longitudinal polarization: it labels the only non zero component of the polarization vector.

We can then express the operators \(\hat{\xi}_{\mu}(\mathbf{p})\) and \(\hat{\xi}_{\mu}^{\dagger}(\mathbf{p})\) in terms of creation and annihilation operators \(\hat{a}^{(\lambda)}_{\mathbf{p}}\) and \(\hat{a}^{(\lambda)\,\dagger}_{\mathbf{p}}\) for photons with definite polarization:
\[
    \hat{\xi}_{\mu}(\mathbf{p}) = \sum_{\lambda=0}^{3} \epsilon_{\mu}^{(\lambda)}(\mathbf{p}) \hat{a}^{(\lambda)}_{\mathbf{p}}, \quad \hat{\xi}_{\mu}^{\dagger}(\mathbf{p}) = \sum_{\lambda=0}^{3} \epsilon_{\mu}^{(\lambda)}(\mathbf{p}) \hat{a}^{(\lambda)\,\dagger}_{\mathbf{p}}.
\]
We can now write the final expression for the quantized electromagnetic field and its conjugate momenta:\footnote{In the expression for the conjugate momenta, we have a term \((+i)\) changing sign with respect to the scalar field case, due to the different definition of \(\pi^{\mu}\) in terms of the fields \(A^{\mu}\): after using equation \eqref{eq:conjugate_momentum}, for KG we had \(\pi = \partial_0 \phi\), while here we have \(\pi^{\mu} = - \dot{A}^{\mu} + \dots\), so that in the end this sign changes.}
\begin{equation}
    \begin{dcases}
        \hat{A}_\mu(\mathbf{x})   & = \int \frac{\mathrm{d}^3 \mathbf{p}}{(2\pi)^3} \frac{1}{\sqrt{2 E_{\mathbf{p}}}} \sum_{\lambda=0}^{3} \epsilon_{\mu}^{(\lambda)}(\mathbf{p}) \left( \hat{a}^{(\lambda)}_{\mathbf{p}} e^{i \mathbf{p} \cdot \mathbf{x}} + \hat{a}^{(\lambda)\,\dagger}_{\mathbf{p}} e^{-i \mathbf{p} \cdot \mathbf{x}} \right),   \\
        \hat{\pi}^\mu(\mathbf{x}) & = \int \frac{\mathrm{d}^3 \mathbf{p}}{(2\pi)^3} (+i) \sqrt{\frac{E_{\mathbf{p}}}{2}} \sum_{\lambda=0}^{3} \epsilon^{\mu\,(\lambda)}(\mathbf{p}) \left( \hat{a}^{(\lambda)}_{\mathbf{p}} e^{i \mathbf{p} \cdot \mathbf{x}} - \hat{a}^{(\lambda)\,\dagger}_{\mathbf{p}} e^{-i \mathbf{p} \cdot \mathbf{x}} \right).
    \end{dcases}
    \label{eq:field_momenta_expansion_ladder_em}
\end{equation}
We have the four vector polarization depending on the four momentum \(p^{\mu} = \left(\vert \mathbf{p} \vert, \mathbf{p} \right)\), but we know that the photon has only two physical polarization states. This discrepancy arises because we have not yet fully fixed the gauge: the presence of unphysical polarization states (longitudinal and timelike) is a consequence of the gauge redundancy in the electromagnetic field. To make contact with the two physical polarization states of the photon, we choose \(\epsilon^{(1)}_{\mu}\) and \(\epsilon^{(2)}_{\mu}\) as the two transverse polarization vectors
\[
    \epsilon^{(1)}_{\mu} p^{\mu} = \epsilon^{(2)}_{\mu} p^{\mu} = 0,\quad \text{ Lorentz invariant condition,}
\]
thus they are perpendicular to the direction of the momentum, while \(\epsilon^{(0)}_{\mu}\) and \(\epsilon^{(3)}_{\mu}\) correspond to the unphysical timelike and longitudinal polarizations, respectively.

\begin{example}[Momentum along the z-axis]
    Consider a photon with momentum along the z-axis:
    \[
        p^{\mu} = (E, 0, 0, E).
    \]
    A possible choice for the polarization vectors is:
    \[
        \begin{aligned}
            \epsilon^{(0)}_{\mu} & = (1, 0, 0, 0) \quad \text{(timelike)},     \\
            \epsilon^{(1)}_{\mu} & = (0, 1, 0, 0) \quad \text{(transverse)},   \\
            \epsilon^{(2)}_{\mu} & = (0, 0, 1, 0) \quad \text{(transverse)},   \\
            \epsilon^{(3)}_{\mu} & = (0, 0, 0, 1) \quad \text{(longitudinal)}.
        \end{aligned}
    \]
    This choice satisfies the orthonormality conditions, since clearly
    \[
        \epsilon^{(1)}_{\mu} p^{\mu} = E \left(\epsilon^{(1)}_{0} + \epsilon^{(1)}_{3}\right) = 0, \quad \epsilon^{(2)}_{\mu} p^{\mu} = E \left(\epsilon^{(2)}_{0} + \epsilon^{(2)}_{3}\right) = 0,
    \]
    we have then
    \[
        \epsilon^{(1)}_0 = \epsilon^{(1)}_3 = 0, \quad \epsilon^{(2)}_0 = \epsilon^{(2)}_3 = 0,
    \]
    since the transverse polarization vectors have zero time and longitudinal components. We need to choose the transverse polarization vectors to be orthogonal to each other and normalized:
    \[
        \epsilon^{(1)}_{\mu} \epsilon^{(2)\,\mu} = 0 = \epsilon^{(1)}_1 \epsilon^{(2)\,1} + \epsilon^{(1)}_2 \epsilon^{(2)\,2} = 0,
    \]
    so now it is justified the initial choice of \(\epsilon^{(1)}_{\mu} = (0, 1, 0, 0)\) and \(\epsilon^{(2)}_{\mu} = (0, 0, 1, 0)\),\footnote{We could have chosen the opposite identification, it would have been the same.} which clearly satisfies the orthonormality conditions with respect to the Minkowski metric. To clarify, we can write
    \[
        \epsilon^{(1)\,\mu} = \begin{pmatrix}
            0 \\
            1 \\
            0 \\
            0
        \end{pmatrix}, \quad \epsilon^{(1)}_{\mu} = \begin{pmatrix}
            0  \\
            -1 \\
            0  \\
            0  \\
        \end{pmatrix}.
    \]
    since we had written only row vectors before. The same applies to \(\epsilon^{(2)}_{\mu}\).\footnote{Practically we are treating \(\epsilon^{\mu\,(\lambda)}\) as a four vector in the Minkowski space, while \(\epsilon_{\mu}^{(\lambda)}\) is its covariant counterpart, obtained by lowering the index with the Minkowski metric: thus the sign changes only in the spatial components.}
\end{example}

The creation and annihilation operators \(a^{(\lambda)}_{\mathbf{p}}\) and \(a^{(\lambda)\,\dagger}_{\mathbf{p}}\) satisfy the commutation relations\TODO{expliciit derivation of commutation relations.}
\[
    \begin{aligned}
        \left[\hat{a}^{(\lambda)}_{\mathbf{p}}, \hat{a}^{(\lambda')\,\dagger}_{\mathbf{q}}\right] & = - \eta^{\lambda \lambda'} (2\pi)^3 \delta^{(3)}(\mathbf{p} - \mathbf{q}),                               \\
        \left[\hat{a}^{(\lambda)}_{\mathbf{p}}, \hat{a}^{(\lambda')}_{\mathbf{q}}\right]          & = \left[\hat{a}^{(\lambda)\,\dagger}_{\mathbf{p}}, \hat{a}^{(\lambda')\,\dagger}_{\mathbf{q}}\right] = 0,
    \end{aligned}
\]
which can be derived from the canonical commutation relations for the fields \(\hat{A}_{\mu}\) and \(\hat{\pi}^{\mu}\). Note the minus sign in the first commutation relation, which arises from the Minkowski metric \(\eta^{\lambda \lambda'}\) and reflects the presence of unphysical polarization states.

If we indeed check the norm of the states created by the creation operators acting on the vacuum \(\ket{0}\), defined by \(\hat{a}^{(\lambda)}_{\mathbf{p}} \ket{0} = 0\) for all \(\lambda\) and \(\mathbf{p}\), we find that for spacelike polarizations \(\lambda = 1, 2, 3\) we have positive norm states:
\[
    \begin{aligned}
        \bra{\mathbf{p}, \lambda} \ket{\mathbf{q}, \lambda'} & = \bra{0} \hat{a}^{(\lambda)}_{\mathbf{p}} \hat{a}^{(\lambda')\,\dagger}_{\mathbf{q}} \ket{0} = -\eta^{\lambda \lambda'} (2\pi)^3 \delta^{(3)}(\mathbf{p} - \mathbf{q}), \\
                                                             & \implies \bra{\mathbf{p}, \lambda} \ket{\mathbf{p}, \lambda} = + (2\pi)^3 \delta^{(3)}(0) > 0, \quad \text{for } \lambda = 1, 2, 3,
    \end{aligned}
\]
while for the timelike polarization \(\lambda = 0\) we have negative norm states:
\[
    \bra{\mathbf{p}, 0} \ket{\mathbf{q}, 0} = \bra{0} \hat{a}^{(0)}_{\mathbf{p}} \hat{a}^{(0)\,\dagger}_{\mathbf{q}} \ket{0} = -\eta^{00} (2\pi)^3 \delta^{(3)}(\mathbf{p} - \mathbf{q}) = - (2\pi)^3 \delta^{(3)}(\mathbf{p} - \mathbf{q}) <0.
\]
This comes from the commutator relations
\[
    \begin{aligned}
        \left[\hat{a}^{(0)}_{\mathbf{p}}, \hat{a}^{(0)\,\dagger}_{\mathbf{q}}\right] & = - \eta^{00} (2\pi)^3 \delta^{(3)}(\mathbf{p} - \mathbf{q}) = - (2\pi)^3 \delta^{(3)}(\mathbf{p} - \mathbf{q}),                    \\
        \left[\hat{a}^{(i)}_{\mathbf{p}}, \hat{a}^{(i)\,\dagger}_{\mathbf{q}}\right] & = - \eta^{ii} (2\pi)^3 \delta^{(3)}(\mathbf{p} - \mathbf{q}) = + (2\pi)^3 \delta^{(3)}(\mathbf{p} - \mathbf{q}), \quad i = 1, 2, 3,
    \end{aligned}
\]
which in turn arise from the switching of the operators in the calculation of the norm.

This is the same problem we encountered when trying to quantize the Dirac field with commutator instead of anticommutators:\footnote{The only difference being that we cannot try to use \(\hat{a}^{\dagger}_{\mathbf{p}}\) as an annihilation operator, since timelike polarized states would still have a negative norm: this time the problem is of gauge nature, but instead of redundancies in the description of the system this time it may create bigger problems in the physical interpretation.} the presence of \textbf{negative norm states} indicates that our Hilbert space has an indefinite metric, which is unphysical since it usually leads to \textbf{negative probabilities} and \textbf{negative energies}. This states ended up being named \textbf{ghost states}.

\subsection{Imposing the Gauge Condition on the Fock Space}

One can trace back the origin of these negative norm states to the gauge-fixing term in the lagrangian:
\[
    \begin{aligned}
        \mathcal{L} & = -\frac{1}{4} F_{\mu\nu} F^{\mu\nu} -\frac{1}{2} (\partial_\mu A^{\mu})^2,                       \\
        \mathcal{L} & \supset \frac{1}{2} \left( (\dot{A}^1)^2 + (\dot{A}^2)^2 + (\dot{A}^3)^2 - (\dot{A}^0)^2 \right),
    \end{aligned}
\]
where the positive time derivatives are contained in the original lagrangian \(F_{\mu\nu}F^{\mu \nu}\) while the minus sign in front of the gauge-fixing term leads to negative time derivatives \(\dot{A}_0^2\), thus to negative norm states for the unphysical polarizations (timelike). This is a manifestation of the indefinite metric in the space of states, which is a common feature in gauge theories before imposing physical state conditions.
The solution is to \textit{impose the gauge condition at the quantum level}, directly on the Fock space of states.\footnote{We did not already impose the gauge conditions indeed. The term in the Lagrangian was added to impose implicitly the Lorentz gauge condition via the equations of motion, but when quantizing we have to take care of the unphysical states, which classically would be eliminated, but quantum mechanically they still appear in the Hilbert space.}\QUESTION{do not quite understand, ask}

If we want to make the spin 1 particle massive, we need an additional degree of freedom, since a massive spin 1 particle has three polarization states, while a massless spin 1 particle has only two physical polarization states. We have seen that from the representations of the little group, a massive particle has the little group \(SO(3)\) with three generators corresponding to the three spatial rotations, while a massless particle has the little group \(ISO(2)\) with only one generator corresponding to rotations around the direction of motion, leading to only two physical polarization states (one rotation which implies two transverse polarizations). We are dealing with a massless spin 1 particle, the photon, so in the end we aim to describe its behavior with only two physical polarization states.

Moving now to Heisemberg picture, we will need to understand how to impose the gauge condition at the quantum level, and what does that condition imply for the physical states of the theory. Thus we promote the fields to operators
\[
    \hat{A}^{\mu} = \hat{A}^{\mu}(x) = \hat{A}^{\mu}(\mathbf{x}, t),
\]
which has a time dependence in Heisemberg picture. We chose the gauge condition
\[
    \partial_\mu \hat{A}^{\mu} = 0,
\]
since it should lead us towards Lorentz invariant solution while removing the negative time derivative in the Lagrangian. This condition makes sense only as an operator equation acting on the Hilbert space of states. However, we have at least three ways of imposing this condition:
\begin{itemize}
    \item \(\partial_\mu \hat{A}^{\mu} = 0\) as a strong operator equation, imposing that the matrix has to vanish identically. This approach is too restrictive and leads to inconsistencies in our theory.
          If we go back to the expression of the conjugate momenta
          \[
              \hat{\pi}^\mu = \hat{F}^{\mu 0} - \eta^{\mu 0} (\partial_\nu \hat{A}^{\nu}),
          \]
          so that if we compute the time component
          \[
              \hat{\pi}^0 = \hat{F}^{0 0} - \eta^{0 0} (\partial_\nu \hat{A}^{\nu}) = - (\partial_\nu \hat{A}^{\nu}) = 0,
          \]
          which would imply that \(\hat{\pi}^0 = 0\) as an operator equation. But this is in contradiction with the canonical commutation relations imposed before:
          \[
              \left[\hat{A}_{\mu}(\mathbf{x}, t), \hat{\pi}_{\nu}(\mathbf{y}, t)\right] = i \delta^{(3)}(\mathbf{x} - \mathbf{y}) \eta_{\mu \nu}, \\
              \implies \left[\hat{A}_{0}(\mathbf{x}, t), \hat{\pi}_{0}(\mathbf{y}, t)\right] = i \delta^{(3)}(\mathbf{x} - \mathbf{y}) \eta_{00} \neq 0,
          \]
          so we have a contradiction requesting that all the entries of the operator \(\partial_\mu \hat{A}^{\mu}\) vanish identically.
    \item An alternative is to impose the gauge condition in a weak sense, meaning that we require that the expectation value of the operator \(\partial_\mu \hat{A}^{\mu}\) vanishes when applied on physical states \(\ket{\psi}\): We are \textit{applying the gauge condition directly on the Fock space}, with the idea to consider unphysical all the states which do not respect
          \[
              (\partial_\mu \hat{A}^{\mu}) \ket{\psi} = 0.
          \]
          This approach is less restrictive, but it still does not fully work in the recognition of physical and unphysical polarization states, since the vacuum state \(\ket{0}\) would be recognized as unphysical: in Heisemberg picture (as done in \eqref{eq:ladder_operators_heisenberg}), we have indeed
          \[
              \hat{A}_\mu(x) = \int \frac{\mathrm{d}^3 \mathbf{p}}{(2\pi)^3} \frac{1}{\sqrt{2 E_{\mathbf{p}}}} \sum_{\lambda=0}^{3} \epsilon_{\mu}^{(\lambda)}(\mathbf{p}) \left( a^{(\lambda)}_{\mathbf{p}} e^{i p_\mu x^{\mu}} + a^{(\lambda)\,\dagger}_{\mathbf{p}} e^{-i p_\mu x^{\mu}} \right) = \hat{A}_\mu^{+}(x) + \hat{A}_\mu^{-}(x),
          \]
          where we have separated the positive and negative frequency parts of the field operator. The gauge condition operator \(\partial_\mu \hat{A}^{\mu}\) then can be applied to the vacuum state:
          \[
              \partial_\mu \hat{A}^{\mu} \ket{0} = \partial_\mu \hat{A}^{\mu +} \ket{0} + \partial_\mu \hat{A}^{\mu -} \ket{0} = \partial_\mu \hat{A}^{\mu +} \ket{0} \neq 0,
          \]
          since when we act on the vacuum with the negative frequency part \(\hat{A}^{\mu -}\), we get zero, but the positive frequency part \(\hat{A}^{\mu +}\) acting on the vacuum does not vanish, leading to a non-zero result. Thus the vacuum state is not physical in this approach, it does not satisfy the gauge condition imposed on the Fock space \(\partial_\mu \hat{A}^{\mu} \ket{0} = 0\).
    \item The most common and effective approach is the \textbf{Gupta-Bleuler formalism}, where we impose the gauge condition only on the positive frequency part of the field operator: in this way the vacuum remains a physical state, and we can consistently define the subspace of physical states in the Hilbert space. The Gupta-Bleuler condition is given by
          \begin{equation}
              \partial_\mu \hat{A}^{\mu +} \ket{\psi} = 0, \iff \bra{\psi} \partial_\mu \hat{A}^{\mu -} = 0,
              \label{eq:gupta-bleuler-condition}
          \end{equation}
          for physical states \(\ket{\psi}\). This condition effectively eliminates the unphysical polarization states from the theory, while still allowing for a consistent quantization of the electromagnetic field. The physical states are then defined as those that satisfy this condition, leading to a well-defined physical subspace of the full Hilbert space.

          We could note that the negative frequency part \(\partial_\mu \hat{A}^{\mu -}\) functioning on physical states only when applied to the left, so that if we look at the matrix element of the operator \(\partial_\mu \hat{A}^{\mu}\) between two physical states \(\ket{\psi}\) and \(\ket{\phi}\), we have:
          \[
              \bra{\phi} \partial_\mu \hat{A}^{\mu} \ket{\psi} = \bra{\phi} \partial_\mu \hat{A}^{\mu +} \ket{\psi} + \bra{\phi} \partial_\mu \hat{A}^{\mu -} \ket{\psi} = 0 + 0 = 0.
          \]
\end{itemize}

\subsection{Physical Implications of the Gupta-Bleuler Condition}

Now that we found a consistent way to impose the gauge condition at the quantum level, we can analyze the implications of this condition on the physical states of the theory. The Gupta-Bleuler condition effectively removes the unphysical polarization states from the theory, leaving only the two transverse polarization states as physical degrees of freedom.

We can check that starting from
\[
    \partial^{\mu} \hat{A}_{\mu}^{+} \ket{\psi} = \int \frac{\mathrm{d}^3 \mathbf{p}}{(2\pi)^3} \frac{1}{\sqrt{2 E_{\mathbf{p}}}} \sum_{\lambda=0}^{3} \left( -i \epsilon_{\mu}^{(\lambda)}(\mathbf{p}) p^{\mu} \right) a^{(\lambda)}_{\mathbf{p}} e^{-i p_\mu x^{\mu}} \ket{\psi} = 0,
\]
since one can notice how the contraction \(\epsilon_{\mu}^{(\lambda)}(\mathbf{p}) p^{\mu}\) vanishes for the transverse polarizations \(\lambda = 1, 2\), orthogonal to the momentum \(p^{\mu}\): choosing the momentum along the z-axis as before \(p^{\mu} = \left(E,\,0,\,0,\,E\right)\), we have
\[
    \epsilon^{(1)}_{\mu} p^{\mu} = 0, \quad \epsilon^{(2)}_{\mu} p^{\mu} = 0,
\]
so that the only condition becomes
\[
    \partial^{\mu} \hat{A}_{\mu}^{+} \ket{\psi} = 0 \iff \left( \epsilon_{\mu}^{(0)}(\mathbf{p}) p^{\mu} \hat{a}^{(0)}_{\mathbf{p}} + \epsilon_{\mu}^{(3)}(\mathbf{p}) p^{\mu} \hat{a}^{(3)}_{\mathbf{p}} \right) \ket{\psi} = 0.
\]
This implies a relation between the creation and annihilation operators for the unphysical polarization states
\[
    E \left( \epsilon_{0}^{(0)}(\mathbf{p}) \hat{a}^{(0)}_{\mathbf{p}} + \epsilon_{3}^{(3)}(\mathbf{p}) \hat{a}^{(3)}_{\mathbf{p}} \right) \ket{\psi} = 0,
\]
which leads to a series of implications useful to understand the nature of physical states:\footnote{Remember that \(\epsilon^{(\lambda)}_{i}\), with \(i = 1,\,2,\,3\), are spacelike components, so they have a minus sign when the index is low (or equivalently we can have high index in the polarization and low index in the momentum, but the minus sign is always present).}
\begin{equation}
    \begin{dcases}
        \left( \hat{a}^{(0)}_{\mathbf{p}} - \hat{a}^{(3)}_{\mathbf{p}} \right) \ket{\psi} = 0,                               \\
        \hat{a}^{(0)}_{\mathbf{p}} \ket{\psi}                           = \hat{a}^{(3)}_{\mathbf{p}} \ket{\psi},             \\
        \bra{\psi} \hat{a}^{(0)\,\dagger}_{\mathbf{p}}                     = \bra{\psi} \hat{a}^{(3)\,\dagger}_{\mathbf{p}}, \\
        \bra{\psi} \hat{a}^{(0)\,\dagger}_{\mathbf{p}} \hat{a}^{(0)}_{\mathbf{p}} \ket{\psi} = \bra{\psi} \hat{a}^{(3)\,\dagger}_{\mathbf{p}} \hat{a}^{(3)}_{\mathbf{p}} \ket{\psi}.
    \end{dcases}
    \label{eq:gupta-bleuler-implications}
\end{equation}
This last relation shows that the number of timelike photons is equal to the number of longitudinal photons in physical states (since we are computing the expectation value of the number operators of the two unphysical polarizations photons on both sides), leading to a cancellation of their contributions to physical observables. Thus, only the two transverse polarization states remain as physical degrees of freedom, consistent with the known properties of photons (negative norm timelike photons are not physical).

To check this, we can consider \(\ket{\mathbf{p},\, \lambda=0} = \hat{a}^{(0)\,\dagger}_{\mathbf{p}} \ket{0}\) the state with timelike polarization, which if our condition is correct should be unphysical:
\[
    \implies \left( \hat{a}^{(0)}_{\mathbf{p}} -  \hat{a}^{(3)}_{\mathbf{p}} \right) \ket{\mathbf{p},\, \lambda=0} \neq 0,
\]
but let us compute that explicitly:
\[
    \begin{aligned}
        \left( \hat{a}^{(0)}_{\mathbf{p}} -  \hat{a}^{(3)}_{\mathbf{p}} \right) \ket{\mathbf{q},\, 0} & = \hat{a}^{(0)}_{\mathbf{p}} \hat{a}^{(0)\,\dagger}_{\mathbf{q}} \ket{0} - \hat{a}^{(3)}_{\mathbf{p}} \hat{a}^{(0)\,\dagger}_{\mathbf{p}} \ket{0} = \hat{a}^{(0)}_{\mathbf{p}} \hat{a}^{(0)\,\dagger}_{\mathbf{q}} \ket{0} \\
                                                                                                      & = \left[ \hat{a}^{(0)}_{\mathbf{p}}, \hat{a}^{(0)\,\dagger}_{\mathbf{q}} \right] \ket{0} + \hat{a}^{(0)\,\dagger}_{\mathbf{p}} \hat{a}^{(0)}_{\mathbf{q}} \ket{0}                                                          \\
                                                                                                      & = \left[ \hat{a}^{(0)}_{\mathbf{p}}, \hat{a}^{(0)\,\dagger}_{\mathbf{q}} \right] \ket{0}                                                                                                                                   \\
                                                                                                      & = - (2\pi)^3 \delta^{(3)}(0) \ket{0} \neq 0.
    \end{aligned}
\]
This is good news: a state of only a timelike photon (which has negative norm) is not physical; but the Hilbert space defined by \(\left( \hat{a}^{(0)}_{\mathbf{p}} -  \hat{a}^{(3)}_{\mathbf{p}} \right) \ket{\psi} = 0\) contains also \textbf{zero norm states}. This comes as good news, since zero norm states do not contribute to physical observables, so they can be present in the physical Hilbert space without causing any issues: we need only two polarization states to describe the photon, so the presence of \(\ket{S}-\ket{L}\) polarization (scalar minus longitudinal, the states created by \(\hat{a}^{(0)\,\dagger}_{\mathbf{p}} - \hat{a}^{(3)\,\dagger}_{\mathbf{p}}\)) could be a problem, but if they are zero norm states they do not contribute to physical observables, so they can be safely included in the physical Hilbert space.

To see this clearly, we have to perform a change of basis on the Fock space:
\[
    \hat{a}^{(\lambda)}_{\mathbf{p}} \quad \longrightarrow \quad \hat{a}^{(1)}_{\mathbf{p}}, \hat{a}^{(2)}_{\mathbf{p}}, \hat{b}^{(\pm)}_{\mathbf{p}} = \frac{1}{\sqrt{2}} \left( \hat{a}^{(0)}_{\mathbf{p}} \pm \hat{a}^{(3)}_{\mathbf{p}} \right).
\]
In this new basis, we should be able to create states with definite numbers of transverse photons and definite combinations of timelike and longitudinal photons:
\[
    \begin{aligned}
        \hat{a}^{(1)\,\dagger}_{\mathbf{p}} \ket{0} & = \ket{\mathbf{p},\, 1},                                                           \\
        \hat{a}^{(2)\,\dagger}_{\mathbf{p}} \ket{0} & = \ket{\mathbf{p},\, 2},                                                           \\
        \hat{b}^{(+)\,\dagger}_{\mathbf{p}} \ket{0} & = \frac{1}{\sqrt{2}} \left( \ket{\mathbf{p},\, 0} + \ket{\mathbf{p},\, 3} \right), \\
        \hat{b}^{(-)\,\dagger}_{\mathbf{p}} \ket{0} & = \frac{1}{\sqrt{2}} \left( \ket{\mathbf{p},\, 0} - \ket{\mathbf{p},\, 3} \right).
    \end{aligned}
\]
so that we can create states with only transverse photons, or states with definite combinations of timelike and longitudinal photons (and combinations of these). Thus the Gupta-Bleuler condition \(\left( \hat{a}^{(0)}_{\mathbf{p}} -  \hat{a}^{(3)}_{\mathbf{p}} \right) \ket{\psi} = 0\) can be rewritten as
\[
    \hat{b}^{(-)}_{\mathbf{p}} \ket{\psi} = 0,
\]
which means that physical states \(\ket{\psi}\) cannot contain any \(\hat{b}^{(-)\,\dagger}_{\mathbf{p}}\) excitations.

We can now check explicitly that states with only transverse photons are physical states applying the Gupta-Bleuler condition through the operator \(\hat{b}^{(-)}_{\mathbf{p}}\). To do so we will need the commutation relations for the new operators \(\hat{b}^{(\pm)}_{\mathbf{p}}\):
\[
    \begin{aligned}
        \left[\hat{b}^{(\pm)}_{\mathbf{p}},\, \hat{b}^{(\pm)\,\dagger}_{\mathbf{q}}\right] & = \left[ \hat{a}^{(0)}_{\mathbf{p}},\, \hat{a}^{(0)\,\dagger}_{\mathbf{q}} \right] + \left[ \hat{a}^{(3)}_{\mathbf{p}},\, \hat{a}^{(3)\,\dagger}_{\mathbf{q}} \right] \\
                                                                                           & = - (2\pi)^3 \delta^{(3)}(\mathbf{p} - \mathbf{q}) + (2\pi)^3 \delta^{(3)}(\mathbf{p} - \mathbf{q}) = 0,                                                              \\
        \left[\hat{b}^{(\pm)}_{\mathbf{p}},\, \hat{b}^{(\mp)\,\dagger}_{\mathbf{q}}\right] & = \left[ \hat{a}^{(0)}_{\mathbf{p}},\, \hat{a}^{(0)\,\dagger}_{\mathbf{q}} \right] - \left[ \hat{a}^{(3)}_{\mathbf{p}},\, \hat{a}^{(3)\,\dagger}_{\mathbf{q}} \right] \\
                                                                                           & = - 2 (2\pi)^3 \delta^{(3)}(\mathbf{p} - \mathbf{q}) \neq 0,
    \end{aligned}
\]
with all other commutators null. Now we can compute the action of \(\hat{b}^{(-)}_{\mathbf{p}}\) on states with:
\begin{enumerate}
    \item \textbf{only transverse photons}, created by \(\hat{a}^{(1/2)\,\dagger}_{\mathbf{q}}\):
          \[
              \hat{b}^{(-)}_{\mathbf{p}} \left( \hat{a}^{(1/2)\,\dagger}_{\mathbf{q}} \ket{0} \right) = 0,
          \]
          since the commutators \(\left[\hat{b}^{(-)}_{\mathbf{p}}, \hat{a}^{(1/2)\,\dagger}_{\mathbf{q}}\right]\) are proportional to the null commutators among \(\hat{a}^{(1/2)}_{\mathbf{q}}\) and \(\hat{a}^{(0/3)}_{\mathbf{p}}\). So the states with only transverse photons \textbf{are physical}.
    \item \textbf{only timelike photons}, created by \(\hat{a}^{(0)\,\dagger}_{\mathbf{q}} = \frac{1}{2}\left(\hat{b}^{(+)\,\dagger}_{\mathbf{q}} + \hat{b}^{(-)\,\dagger}_{\mathbf{q}}\right)\):
          \[
              \begin{aligned}
                  \hat{b}^{(-)\,\dagger}_{\mathbf{p}} \left( \hat{a}^{(0)\,\dagger}_{\mathbf{q}} \ket{0} \right) & = \frac{1}{2}\hat{b}^{(-)\,\dagger}_{\mathbf{p}} \left(\hat{b}^{(+)\,\dagger}_{\mathbf{q}} \ket{0} +  \hat{b}^{(-)\,\dagger}_{\mathbf{q}} \ket{0}\right)                                                         \\
                                                                                                                 & = \frac{1}{2}\left[ \hat{b}^{(-)\,\dagger}_{\mathbf{p}} ,\, \hat{b}^{(+)\,\dagger}_{\mathbf{p}}  \right] +\frac{1}{2} \left[ \hat{b}^{(-)\,\dagger}_{\mathbf{p}} ,\, \hat{b}^{(-)\,\dagger}_{\mathbf{p}} \right] \\
                                                                                                                 & = \frac{1}{2}\left[ \hat{b}^{(-)\,\dagger}_{\mathbf{p}} ,\, \hat{b}^{(+)\,\dagger}_{\mathbf{p}}  \right] \neq 0.
              \end{aligned}
          \]
          This shows that timelike photons alone \textbf{form unphysical states}, since they do not satisfy the Gupta-Bleuler condition. This is expected, since timelike photons have negative norm.
    \item \textbf{only longitudinal photons}, created by \(\hat{a}^{(3)\,\dagger}_{\mathbf{q}} = \frac{1}{2}\left(\hat{b}^{(+)\,\dagger}_{\mathbf{q}} - \hat{b}^{(-)\,\dagger}_{\mathbf{q}}\right)\):
          \[
              \begin{aligned}
                  \hat{b}^{(-)\,\dagger}_{\mathbf{p}} \left( \hat{a}^{(0)\,\dagger}_{\mathbf{q}} \ket{0} \right) & = \frac{1}{2}\hat{b}^{(-)\,\dagger}_{\mathbf{p}} \left(\hat{b}^{(+)\,\dagger}_{\mathbf{q}} \ket{0} -  \hat{b}^{(-)\,\dagger}_{\mathbf{q}} \ket{0}\right)                                                         \\
                                                                                                                 & = \frac{1}{2}\left[ \hat{b}^{(-)\,\dagger}_{\mathbf{p}} ,\, \hat{b}^{(+)\,\dagger}_{\mathbf{p}}  \right] -\frac{1}{2} \left[ \hat{b}^{(-)\,\dagger}_{\mathbf{p}} ,\, \hat{b}^{(-)\,\dagger}_{\mathbf{p}} \right] \\
                                                                                                                 & = \frac{1}{2}\left[ \hat{b}^{(-)\,\dagger}_{\mathbf{p}} ,\, \hat{b}^{(+)\,\dagger}_{\mathbf{p}}  \right] \neq 0.
              \end{aligned}
          \]
          This shows that longitudinal photons alone \textbf{form unphysical states}, since they do not satisfy the Gupta-Bleuler condition. This is expected, since longitudinal photons have negative norm.
    \item \textbf{combination of timelike and longitudinal photons}, created by \(\hat{b}^{(+)\,\dagger}_{\mathbf{q}}\) and \(\hat{b}^{(-)\,\dagger}_{\mathbf{q}}\):
          \begin{itemize}
              \item For the combination created by \(\hat{b}^{(+)\,\dagger}_{\mathbf{q}}\):
                    \[
                        \hat{b}^{(-)}_{\mathbf{p}} \left( \hat{b}^{(+)\,\dagger}_{\mathbf{q}} \ket{0} \right) = \left[ \hat{b}^{(-)}_{\mathbf{p}}, \hat{b}^{(+)\,\dagger}_{\mathbf{q}} \right] \ket{0} \neq 0,
                    \]
                    so that states with one timelike and one longitudinal photon in the combination \(\ket{S}+\ket{L}\)\footnote{We call it \(\ket{S} + \ket{L}\) since for timelike states we would have \(\ket{T}\), but then it creates confusion with the transversal polarizations: transversal polarizations remain \(\ket{T}\), while timelike ones will be addressed as \textit{scalar}, or \(\ket{S}\).} \textbf{are unphysical}, since they do not satisfy the Gupta-Bleuler condition (even if they have zero norm).
              \item For the combination created by \(\hat{b}^{(-)\,\dagger}_{\mathbf{q}}\):
                    \[
                        \hat{b}^{(-)}_{\mathbf{p}} \left( \hat{b}^{(-)\,\dagger}_{\mathbf{q}} \ket{0} \right) = \left[ \hat{b}^{(-)}_{\mathbf{p}}, \hat{b}^{(-)\,\dagger}_{\mathbf{q}} \right] \ket{0} = 0,
                    \]
                    so that states with one timelike and one longitudinal photon in the combination \(\ket{S}-\ket{L}\) \textbf{are physical}, since they satisfy the Gupta-Bleuler condition.
          \end{itemize}
\end{enumerate}

It seems that we have found a physical state which is a combination of timelike and longitudinal photons, which is unexpected since both these polarization states have negative norm. However, if we compute the norm of the state created by \(\hat{b}^{(-)\,\dagger}_{\mathbf{p}}\):
\[
    \begin{aligned}
        \bra{0} \hat{b}^{(-)}_{\mathbf{p}} \hat{b}^{(-)\,\dagger}_{\mathbf{p}} \ket{0} & = \bra{0} \left[ \hat{b}^{(-)}_{\mathbf{p}}, \hat{b}^{(-)\,\dagger}_{\mathbf{p}} \right] \ket{0} + \bra{0} \hat{b}^{(-)\,\dagger}_{\mathbf{p}} \hat{b}^{(-)}_{\mathbf{p}} \ket{0} \\
                                                                                       & = \bra{0} \left[ \hat{b}^{(-)}_{\mathbf{p}}, \hat{b}^{(-)\,\dagger}_{\mathbf{p}} \right] \ket{0} + 0 = 0,
    \end{aligned}
\]
we find that this linear combination (and its multiples) of timelike and longitudinal photons is the only physical one, but its norm is zero. This property indicates that it is a null state in the physical subspace, hence it does not contribute to physical observables: this linear combination is physical but cannot be measured (observed).

Computing the norm of the transverse polarization states, we find that they have positive norm:
\[
    \begin{aligned}
        \bra{\mathbf{p}, \lambda=(1/2)} \ket{\mathbf{p}, \lambda=(1/2) } & = \bra{0} \hat{a}^{(1/2)}_{\mathbf{p}} \hat{a}^{(1/2)\,\dagger}_{\mathbf{p}} \ket{0}                                                                                                      \\
                                                                         & = \bra{0} \left[ \hat{a}^{(1/2)}_{\mathbf{p}}, \hat{a}^{(1/2)\,\dagger}_{\mathbf{p}} \right] \ket{0} + \bra{0} \hat{a}^{(1/2)\,\dagger}_{\mathbf{p}} \hat{a}^{(1/2)}_{\mathbf{p}} \ket{0} \\
                                                                         & = (2\pi)^3 \delta^{(3)}(\mathbf{0}) > 0.
    \end{aligned}
\]
Thus, the Gupta-Bleuler condition effectively removes the unphysical polarization states from the theory, leaving only the two transverse polarization states as physical degrees of freedom, while the timelike and longitudinal combinations which survived still lead to zero norm states that do not contribute to physical observables.

Now if we consider a state with two photons, one with transverse polarization \(\ket{T}\) and one with \(\ket{S}-\ket{L}\) polarization (scalar minus longitudinal, the zero norm states):
\[
    \hat{b}^{(-)\,\dagger}_{\mathbf{q}} \ket{\mathbf{p} ,\,T}                                                          = \ket{\mathbf{p},\,T;\; \mathbf{q},\,S-L},
\]
and if we compute its norm:
\[
    \begin{aligned}
        \bra{\mathbf{p^{\prime} },\,T;\; \mathbf{q^{\prime} },\,S-L}\ket{\mathbf{p},\,T;\; \mathbf{q},\,S-L} & = \bra{0} \hat{a}^{(T)}_{\mathbf{p^{\prime}}} \hat{b}^{(-)}_{\mathbf{q^{\prime}}} \hat{b}^{(-)\,\dagger}_{\mathbf{q}} \hat{a}^{(T)\,\dagger}_{\mathbf{p}} \ket{0}                      \\
                                                                                                             & = \bra{0} \hat{b}^{(-)}_{\mathbf{q^{\prime}}}\hat{b}^{(-)\,\dagger}_{\mathbf{q}}\left[ \hat{a}^{(T)}_{\mathbf{p^{\prime}}} ,\, \hat{a}^{(T)\,\dagger}_{\mathbf{p}} \right] \ket{0} + 0 \\
                                                                                                             & = (2\pi)^3 \delta^{(3)}(\mathbf{p} - \mathbf{p^{\prime}}) \bra{0} \left[ \hat{b}^{(-)}_{\mathbf{q^{\prime}}} ,\, \hat{b}^{(-)\,\dagger}_{\mathbf{q}} \right] \ket{0} + 0 = 0.          \\
    \end{aligned}
\]
So in the end we can generalize this result to any state with any number of transverse photons and zero norm states (scalar minus longitudinal), leads to a total zero norm for the entire state (one photon in \(\ket{S}-\ket{L}\) state is sufficient to zero out the norm of the state). Thus, the physical subspace of the Hilbert space is spanned only by states with transverse photons, while any state containing timelike or longitudinal photons leads to zero norm states that do not contribute to physical observables.

We are interested in th system's observables indeed, which are represented by operators acting on the Hilbert space. Physical observables do not mix physical and unphysical states, since due to the zero norm, any matrix element involving unphysical states vanishes. Thus, physical observables are effectively restricted to the physical subspace spanned by transverse photon states. Furthermore, if we were to compute the energy (or any other observable) of a state containing any number of transverse photons and any number of zero norm states, we would find that the contribution from the zero norm states vanishes always, leaving only the contribution from the transverse photons. This ensures that physical observables are well-defined and consistent with the known properties of photons.

\paragraph{Intuitive Picture}\TODO{insert figure}
We can look at the total Fock space \(\mathcal{F}\), which will contain all the possible states created by the creation operators on the vacuum: there will be states with transverse photons, timelike photons, longitudinal photons, and combinations thereof (it contains physical and unphysical states). Inside this total Fock space, we can identify the physical subspace \(\mathcal{F}_{\text{phys}}\), which is spanned only by states with transverse photons (the physical degrees of freedom, respecting Gupta-Bleuler condition). The unphysical states (timelike and longitudinal photons) form a subspace \(\mathcal{F}_{\text{unphys}}\) that is orthogonal to the physical subspace, and any state in this unphysical subspace has zero norm when projected onto the physical subspace.

Inside \(\mathcal{F}_{\text{phys}}\) we will have both positive norm states (transverse photons) and zero norm states (combinations of timelike and longitudinal photons satisfying Gupta-Bleuler condition): transverse states and mixtures of the same number of trannsverse photons plus scalar-longitudinal combinations are gauge equivalent, since they differ by zero norm states. This gauge orbits are equivalence classes of states (states with the same physical observables) in \(\mathcal{F}_{\text{phys}}\) that differ only by zero norm states, representing the same physical configuration.

Thus, the physical content of the theory is captured by the equivalence classes of states in \(\mathcal{F}_{\text{phys}}\), while the unphysical states in \(\mathcal{F}_{\text{unphys}}\) do not contribute to physical observables.

\subsection{Energy States}

Finally, we can check that the energy operator (hamiltonian) acting on physical states gives positive energy eigenvalues, ensuring the stability of the vacuum and the consistency of the theory. The hamiltonian for the quantized electromagnetic field can be expressed in terms of the creation and annihilation operators as
\begin{equation}
    \hat{H} = \int \frac{\mathrm{d}^3 \mathbf{p}}{(2\pi)^3} E_{\mathbf{p}} \left( \sum_{\lambda = 1}^3  \hat{a}^{(\lambda)\,\dagger}_{\mathbf{p}}  \hat{a}^{(\lambda)}_{\mathbf{p}} - \hat{a}^{(0)\,\dagger}_{\mathbf{p}}\hat{a}^{(0)}_{\mathbf{p}} \right),
    \label{eq:hamiltonian_em_field}
\end{equation}
after normal ordering, where \(E_{\mathbf{p}} = |\mathbf{p}|\) for massless photons. In order to derive this expression, we have used the Legendre transformation of the Lagrangian density for the electromagnetic field, so we need to compute the conjugate momenta and the lagrangian density. Starting from the latter:
\[
    \mathcal{L} = - \frac{1}{4} F_{\mu \nu} F^{\mu \nu} - \frac{1}{2} (\partial_\mu A^{\mu})^2 = - \frac{1}{2} \partial_{\mu} A_{\nu} \partial^{\mu} A^{\nu} = - \frac{1}{2} \left( \partial_0 A_{\mu} \partial^0 A^{\mu} - \partial_i A_{\mu} \partial^i A^{\mu} \right),
\]
since the mixed terms in the first part cancel out with the gauge fixing term. We can then compute the conjugate momenta as:
\[
    \pi^{\mu} = \frac{\partial \mathcal{L}}{\partial (\partial_0 A_{\mu})} = -\frac{1}{2} \frac{\partial}{\partial \dot{A}_{\mu}} \eta^{\mu \nu} \dot{A}_{\mu} \dot{A}_{\nu} = - \partial^0 A^{\mu} = - \dot{A}^{\mu},
\]
leading to the hamiltonian:
\[
    H = \int \mathrm{d}^3 \mathbf{x} \left( \pi^{\mu} \partial_0 A_{\mu} - \mathcal{L} \right) = \frac{1}{2} \int \mathrm{d}^3 \mathbf{x} \left( \partial_i A_{\mu} \partial^i A^{\mu} - \pi_\mu \pi^{\mu} \right),
\]
where we used \(\pi^{\mu}\dot{A}_\mu = \pi_{\mu}\dot{A}^\mu = - \pi_{\mu}\pi^{\mu}\). Let us compute terms separately: starting from the spatial derivative of the fields and substituting the field expansion of \eqref{eq:field_momenta_expansion_ladder_em}, we have
\[
    \begin{aligned}
        \partial_i \hat{A}_{\mu}(\mathbf{x}) & = \int \frac{\mathrm{d}^3 \mathbf{p}}{(2\pi)^3} \frac{-i p_i}{\sqrt{2 \vert \mathbf{p} \vert}} \sum_{\lambda=0}^{3} \epsilon_{\mu}^{(\lambda)}(\mathbf{p}) \left( \hat{a}^{(\lambda)}_{\mathbf{p}} e^{-i\omega t} e^{i p^j x^j} - \hat{a}^{(\lambda)\,\dagger}_{\mathbf{p}} e^{i\omega t} e^{-i p^j x^j} \right),                                    \\
        \partial^i \hat{A}^{\mu}(\mathbf{x}) & = \int \frac{\mathrm{d}^3 \mathbf{q}}{(2\pi)^3} \frac{-i q^i}{\sqrt{2 \vert \mathbf{q} \vert}} \sum_{\lambda^{\prime}=0}^{3} \epsilon^{\mu\,(\lambda^{\prime})}(\mathbf{q}) \left( \hat{a}^{(\lambda^{\prime})}_{\mathbf{q}} e^{-i\omega t} e^{i q^j x^j} - \hat{a}^{(\lambda^{\prime})\,\dagger}_{\mathbf{q}} e^{i\omega t} e^{-i q^j x^j} \right), \\
    \end{aligned}
\]
since \(\partial_i e^{i p^j x^j} = \frac{\partial}{\partial x^i} e^{i p^j x^j} = i p^i e^{i p^j x^j} = -i p_i e^{i p^j x^j}\).\footnote{When computing the contraction \(\partial_i \hat{A}_{\mu}\partial^i \hat{A}^{\mu}\), we have to pay attention to the other derivative, since \(\partial^i e^{i p^j x^j} = \frac{\partial}{\partial x_i} e^{i p^j x^j} = i p^i \eta_{ij} e^{i p^j x^j} = i p_i e^{i p^j x^j} = -i p^i\); then when contracting \((-i p_i) (-i p^i)\) we need to change sign since \(\vert \mathbf{p} \vert^2 = p^i p^i = - p_i p^i\).} Thus, after integrating in \(\mathrm{d}^3 \mathbf{x}\) (which generates \(\delta(\mathbf{p}+ \mathbf{q})\) and \(\delta(\mathbf{p}-\mathbf{q})\)) and \(\mathrm{d}^3 \mathbf{q}\) (which is responsible for the minus signs in the first two terms of the next equation) in order to resolve the delta functions, we find
\[
    \begin{aligned}
        \frac{1}{2} \int \mathrm{d}^3 \mathbf{x} \, \partial_i A_{\mu} \partial^i A^{\mu} & = \frac{1}{4} \int \frac{\mathrm{d}^3 \mathbf{p}}{(2\pi)^3} \vert \mathbf{p} \vert \sum_{\lambda \lambda^{\prime}} \eta_{\lambda \lambda^{\prime}} \left[ - \hat{a}^{(\lambda)}_{\mathbf{p}} \hat{a}^{(\lambda^{\prime})}_{-\mathbf{p}} - \hat{a}^{(\lambda)\,\dagger}_{\mathbf{p}} \hat{a}^{(\lambda^{\prime})\,\dagger}_{-\mathbf{p}} - \left( \hat{a}^{(\lambda)}_{\mathbf{p}} \hat{a}^{(\lambda^{\prime})\,\dagger}_{\mathbf{p}} + \hat{a}^{(\lambda)\,\dagger}_{\mathbf{p}} \hat{a}^{(\lambda^{\prime})}_{\mathbf{p}} \right) \right], \\
        -\frac{1}{2} \int \mathrm{d}^3 \mathbf{x} \, \pi_{\mu} \pi^{\mu}                  & = \frac{1}{4} \int \frac{\mathrm{d}^3 \mathbf{p}}{(2\pi)^3} \vert \mathbf{p} \vert \sum_{\lambda \lambda^{\prime}} \eta_{\lambda \lambda^{\prime}} \left[ \hat{a}^{(\lambda)}_{\mathbf{p}} \hat{a}^{(\lambda^{\prime})}_{-\mathbf{p}} + \hat{a}^{(\lambda)\,\dagger}_{\mathbf{p}} \hat{a}^{(\lambda^{\prime})\,\dagger}_{-\mathbf{p}} - \left( \hat{a}^{(\lambda)}_{\mathbf{p}} \hat{a}^{(\lambda^{\prime})\,\dagger}_{\mathbf{p}} + \hat{a}^{(\lambda)\,\dagger}_{\mathbf{p}} \hat{a}^{(\lambda^{\prime})}_{\mathbf{p}} \right) \right],
    \end{aligned}
\]
where we used the orthogonality relation for the polarization vectors:
\[
    \epsilon_{\mu}^{(\lambda)}(\mathbf{p}) \epsilon^{(\lambda^{\prime})\,\mu}(\mathbf{p}) = \eta_{\lambda \lambda^{\prime}}.
\]
Summing these two contributions, we find that the terms with two creation or two annihilation operators cancel out, leaving only the mixed terms. Thus, after normal ordering the expression (we will not use the explicit notation for brevity), we find
\[
    \begin{aligned}
        H & = - \frac{1}{2} \int \frac{\mathrm{d}^3 \mathbf{p}}{(2\pi)^3} E_{\mathbf{p}} \sum_{\lambda \lambda^{\prime}} \eta_{\lambda \lambda^{\prime}} \left( \hat{a}^{(\lambda^{\prime})\,\dagger}_{\mathbf{p}} \hat{a}^{(\lambda)}_{\mathbf{p}} + \hat{a}^{(\lambda)\,\dagger}_{\mathbf{p}} \hat{a}^{(\lambda^{\prime})}_{\mathbf{p}} \right) \\
          & = \int \frac{\mathrm{d}^3 \mathbf{p}}{(2\pi)^3} E_{\mathbf{p}} \left( \sum_{\lambda = 1}^3  \hat{a}^{(\lambda)\,\dagger}_{\mathbf{p}}  \hat{a}^{(\lambda)}_{\mathbf{p}} - \hat{a}^{(0)\,\dagger}_{\mathbf{p}}\hat{a}^{(0)}_{\mathbf{p}} \right),
    \end{aligned}
\]
where in the last step we used the metric signature \(-\eta_{\lambda \lambda^{\prime}} = \text{diag}(-1, 1, 1, 1)\) to separate the contributions from the different polarizations.

When acting on physical states \(\ket{\psi} \in \mathcal{F}_{\text{phys}}\), which contain only transverse photons, the expected energy eigenvalues are positive, since each transverse photon contributes positively to the total energy:
\[
    \bra{\psi} \hat{H} \ket{\psi} = \int \frac{\mathrm{d}^3 \mathbf{p}}{(2\pi)^3} E_{\mathbf{p}} \left[ \bra{\psi}\left( \hat{a}^{(1)\,\dagger}_{\mathbf{p}}  \hat{a}^{(1)}_{\mathbf{p}} + \hat{a}^{(2)\,\dagger}_{\mathbf{p}}  \hat{a}^{(2)}_{\mathbf{p}} \right)\ket{\psi} + \bra{\psi} \hat{a}^{(3)\,\dagger}_{\mathbf{p}}\hat{a}^{(3)}_{\mathbf{p}}\ket{\psi} - \bra{\psi} \hat{a}^{(0)\,\dagger}_{\mathbf{p}}\hat{a}^{(0)}_{\mathbf{p}}\ket{\psi} \right],
\]
but since the Gupta-Bleuler condition ensures that the contributions from the timelike and longitudinal photons cancel out in physical states (as shown in \cref{eq:gupta-bleuler-implications}), we have
\[
    \bra{\psi} \hat{H} \ket{\psi} = \int \frac{\mathrm{d}^3 \mathbf{p}}{(2\pi)^3} E_{\mathbf{p}} \bra{\psi}\left( \hat{a}^{(1)\,\dagger}_{\mathbf{p}}  \hat{a}^{(1)}_{\mathbf{p}} + \hat{a}^{(2)\,\dagger}_{\mathbf{p}}  \hat{a}^{(2)}_{\mathbf{p}} \right)\ket{\psi} \geq 0,
\]
confirming that the energy eigenvalues for physical states are indeed positive. This result is crucial for the stability of the vacuum and the overall consistency of the quantum theory of the electromagnetic field.

Notice that the vacuum state \(\ket{0}\) has zero energy, as expected, but so do all the zero norm states created by combinations of timelike and longitudinal photons satisfying the Gupta-Bleuler condition. Thus, the vacuum and all zero norm states have the same eigenvalues when acting with operators on the Fock space, while physical states with transverse photons carry the information about the actual physical excitations of the electromagnetic field. We could say that the vacuum is degenerate, since there are multiple states (the vacuum and all zero norm states) that share the same energy eigenvalue of zero. However, only the vacuum state itself is physically relevant, while the zero norm states do not contribute to physical observables; they could be treated as gauge artifacts or unphysical configurations that do not affect measurable quantities in the theory.