\chapter*{Introduction}

Quantum Field Theory (QFT) provides the modern framework for describing the dynamics of elementary particles and their interactions. It unifies the principles of quantum mechanics with those of special relativity, and it naturally incorporates particle creation and annihilation processes, which are absent in non-relativistic quantum mechanics.

\subsection*{The Poincaré Group and Relativistic Symmetry}
The starting point of any relativistic theory is the invariance under the Poincaré group, the group of isometries of Minkowski spacetime. It consists of Lorentz transformations (rotations and boosts) together with spacetime translations.

The representations of the Poincaré group classify possible relativistic particles, characterized by two invariants: the mass $m$ and the spin $s$.

\subsection*{Classical Field Theory}

Before quantization, fields are introduced as classical dynamical variables defined over spacetime.
Their dynamics are determined by the principle of stationary action, which leads to the \textbf{Euler--Lagrange equations} for fields.

Within this framework, different types of relativistic fields arise naturally:
scalar fields, spinor fields, and vector fields, each described by an appropriate Lagrangian density.

A central result of the Lagrangian formalism is \textbf{Noether’s theorem}, which establishes a direct correspondence between continuous symmetries of the action and conserved physical quantities. For instance, spacetime translation invariance implies conservation of energy and momentum, while internal phase symmetries give rise to conserved charges.


\subsection*{Canonical Quantization of Free Fields}

Quantization promotes classical fields to operator-valued distributions acting on a Fock space.
The canonical approach consists in imposing equal-time commutation or anticommutation relations, in accordance with the spin--statistics theorem.

\begin{itemize}
  \item \textbf{Spin-0 (Klein--Gordon theory):}
        A real scalar field is quantized by expanding it in Fourier modes with creation and annihilation operators obeying bosonic commutation relations. This leads to the description of spinless relativistic particles.

  \item \textbf{Spin-1/2 (Dirac and Weyl theories):}
        Spinor fields are quantized by imposing fermionic anticommutation relations. The Dirac theory provides the framework for massive spin-$\tfrac{1}{2}$ particles such as electrons, while the Weyl theory describes massless chiral fermions.

  \item \textbf{Spin-1 (Maxwell and Proca theories):}
        Vector fields can be quantized only after addressing gauge redundancy. The Maxwell theory describes a massless gauge boson (the photon), whereas the Proca theory provides a consistent formulation for a massive spin-1 particle.
\end{itemize}

These procedures yield the free quantum field theories for the three basic types of relativistic particles: scalars, fermions, and gauge bosons.

\subsection*{Towards Interacting Theories}

Free field theories provide the starting point of quantum field theory, but they describe particles without mutual influence.
To account for the physical world, one must introduce interactions by adding non-linear terms to the Lagrangian density.

The analysis of interacting quantum fields relies on perturbation theory, systematically organized through Feynman diagrams.
Within this framework one computes observable quantities such as decay rates, which measure the probability per unit time that an unstable particle decays, and cross subsections, which quantify the likelihood of scattering processes.
These observables form the bridge between the abstract formalism of quantum field theory and the experimental study of particle physics.

We will not cover exact solutions of interacting theories, which require advanced techniques and mathematical tools beyond the scope of these notes.
Instead, practical approaches rely on approximation schemes.
In particular, perturbative expansions reduce the dynamics of interacting fields to a collection of coupled harmonic oscillators, whose behavior is well understood from quantum mechanics.
This analogy provides the foundation for treating interactions as small corrections to free theories, ultimately leading to the perturbative framework of Feynman diagrams.

